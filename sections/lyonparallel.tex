LyonParallel is a prototype tool, written in OCaml, for the analysis of DPO graph rewriting systems.
It implements the type graph method presented in~\cite{endrullis2024generalized_icgt} and its extension in Chapter~\ref{chap:nwf}, the morphism counting method in Chapter~\ref{chap:subgraph_counting} and its extension in Chapter~\ref{chap:antipattern}. Note that the type graph method presented in~\cite{endrullis2024generalized_icgt} is more powerful compared to the versions presented in~\cite{zantema2014termination,bruggink2014termination,bruggink2015proving}, which are implemented in~\cite{TORPAcyc} and~\cite{grez}, for DPO graph rewriting systems with monic match. It does not have a publicly available implementation. 

% The tool's name starts with the name of the city of Lyon, followed by
% \enquote{Parallel} which reflects the tool's capability to analyze DPO graph rewriting systems in parallel with the type graph method of different semirings. 
 
LyonParallel implements the notions of DPO rewriting systems on finite directed, edge-labelled multigraphs introduced in Chapter~\ref{chap:preliminaries}. Using the \texttt{REPL} (Read-Eval-Print Loop)\textemdash an interactive environment that reads user commands, evaluates them, prints results, and waits for the next input\textemdash users can select a set of rules encoded in the tool, then choose a method to partition the rules of the system into two sets of rules $\mathcal{A}$ and $\mathcal{B}$ such that $\mathcal{A}$ is relative terminating with respect to $\mathcal{B}$. If such a partition is found, the tool replaces the current rule set with \(\mathcal{B}\) and a new iteration begins; otherwise users may try a different method. When the rule set becomes empty, the termination of the original system is proven.
\section{DPO graph rewriting systems on finite directed edge-labeled multigraphs}
\label{lyonparallel:sec:implementation_of_dpo_graph_rewriting_systems}
A finite \textbf{unlabeled graph} is encoded in the tool as a structure with 4 fields: a list of nodes, a list of edges, a mapping that associates to each edge a source node, a mapping that associates to each edge a target node. 
The representation of a labeled graph is constructed using a construction function \colorbox{Ivory2}{$MGraph.fromList$} given 2 parameters. The first parameter is a list of integers representing the nodes of the graph. The second parameter is a list of 4-tuples representing the labeled edges of the graph, where each tuple \colorbox{Ivory2}{$(s,lab,t,id)$} of type \colorbox{Ivory2}{$\textit{int}\times \textit{string}\times \textit{int}\times \textit{int}$} stands for an edge with source node $s$, label $lab$, target node $t$, and identifier $id$.

For example, consider graphs $K$,$L$, and $R$ shown in~\autoref{fig:implemented:grsaa_0}.
\begin{figure}[H]
      \centering 
      \resizebox{0.8\textwidth}{!}{
      \begin{tikzpicture}
          \graphbox{$L$}{0mm}{0mm}{34mm}{15mm}{2mm}{-5mm}{
              \coordinate (o) at (0mm,-3mm); 
              \node[draw,circle] (l1) at ($(o)+(-10mm,0mm)$) {1};
              \node[draw,circle] (l2) at ($(l1)+(2,0)$) {2};
              \node[draw,circle] (l3) at ($(l1) + (1,0)$) {3};
              \draw[->] (l1) -- (l3) node[midway,above] {a};
              \draw[->] (l3) -- (l2) node[midway,above] {a};
          }     
          \graphbox{$K$}{40mm}{0mm}{24mm}{15mm}{2mm}{-5mm}{
              \coordinate (o) at (5mm,-3mm); 
              \node[draw,circle] (l1) at ($(o)+(-10mm,0mm)$) {1};
              \node[draw,circle] (l2) at ($(l1)+(1,0)$) {2};
              % \node[draw,circle] (l3) at ($(l1) + (1,0)$) {$\ $};
              % \draw[->] (l1) -- (l3) node[midway,above] {a};
              % \draw[->] (l3) -- (l2) node[midway,above] {a};
          }    
          \graphbox{$R$}{70mm}{0mm}{45mm}{15mm}{2mm}{-5mm}{
              \coordinate (o) at (-5mm,-3mm); 
              \node[draw,circle] (l1) at ($(o)+(-10mm,0mm)$) {1};
              \node[draw,circle] (l2) at ($(l1)+(3,0)$) {2};
              \node[draw,circle] (l3) at ($(l1) + (1,0)$) {4};
              \node[draw,circle] (l4) at ($(l1) + (2,0)$) {5};
              \draw[->] (l1) -- (l3) node[midway,above] {a};
              \draw[->] (l3) -- (l4) node[midway,above] {b};
              \draw[->] (l4) -- (l2) node[midway,above] {a};
          }    
          \node () at (37mm,-8mm) {$\overset{l}{\leftarrowtail}$};
          \node () at (67mm,-8mm) {$\overset{r}{\rightarrowtail}$};
          % \draw[>->] (51mm,2mm) -- (52mm,3mm);
      \end{tikzpicture}
      }
      \caption{}
      \label{fig:implemented:grsaa_0}
  \end{figure}
Their representation in Ocaml can be constructed by the following code:
\begin{minted}{ocaml}
    let graph_K = MGraph.fromList [1;2] []
    let graph_L = Mgraph.fromList [1;2;3] [(1,"a",3,1);(3,"a",2,2)]
    let graph_R = Mgraph.fromList [1;2;4;5] 
                    [(1,"a",4,1);(4,"b",5,2);(5,"a",2,3)]
\end{minted}

A homomorphism between two labeled graphs is encoded as a mapping from the nodes of the source graph to the nodes of the target graph, which preserves the edge labels and the source and target nodes of each edge. Its representation can be constructed using a labeled graph construction function
\colorbox{Ivory2}{$GraphHomomorphism.fromList$} given 4 arguments:
    \begin{enumerate}
        \item the first argument is the source graph,
        \item the second argument is the target graph,
        \item the third argument is a list of pairs, where each pair $(s,t)$ means that node $s$ of the source graph is mapped to node $t$ of the target graph,
        \item the fourth argument is a list of pairs similar to the third argument, but for the edges.
    \end{enumerate}
For example, consider morphisms $l$ and $r$ shown in~\autoref{fig:implemented:grsaa_0}. Their representation in Ocaml can be constructed using the following code:
\begin{minted}{ocaml}
    let morphism_l = GraphHomomorphism.fromList 
                        graph_K graph_L [(1,1);(2,2)] []
    let morphism_r = GraphHomomorphism.fromList
                        graph_K graph_R [(1,1);(2,2)] []
\end{minted}

A DPO rewriting rule is represented by a structure with 2 fields: a left-hand side homomorphism and a right-hand side homomorphism. The representation of a rewriting rule can be constructed using a rewriting rule construction function \colorbox{Ivory2}{$GraphRewritingSystem.DPOrule.fromHomos$} given the representations of the left- and right-hand side homomorphisms.
For example, the representation of the rewriting rule illustrated in~\autoref{fig:implemented:grsaa_0} is constructed with the following OCaml expression, which creates the representation of a DPO rule from the representations of the left and right morphisms: 
\begin{minted}{ocaml}
    let rl = GraphRewritingSystem.DPOrule.fromHomos 
                morphism_l morphism_r
\end{minted}

A DPO graph rewriting system is represented by a structure with 3 fields: the set of rules, the name of the system, and a boolean indicating whether the system is restricted to monic matches. A representation of the DPO graph rewriting system
can be constructed using the DPO graph rewriting system construction function \colorbox{Ivory2}{$GraphRewritingSystem.fromRulesListAndName$} given a list of rewriting rules and the name of the system (the third parameter is false by defaut).

For example, the DPO graph rewriting system that contains the single rule illustrated in~\autoref{fig:implemented:grsaa_0} is represented by the following OCaml expression. 
\begin{minted}{ocaml}
let bruggink_2014_ex_4 =
  GraphRewritingSystem.fromRulesListAndName 
  [rl] "bruggink_2014_ex_4"
\end{minted}
The function \colorbox{Ivory2}{$GraphRewritingSystem.fromRulesListAndName$} produces a system named \colorbox{Ivory2}{$bruggink\_2014\_ex\_4$} from the singleton list \colorbox{Ivory2}{$[rl]$}.


User-defined DPO graph rewriting systems should be placed in the file \colorbox{Ivory2}{$lib/concretGraphRewritingSystems.ml$}. To expose them, add each instance to the list \colorbox{Ivory2}{$available\_graph\_rewriting\_systems$} in the same file.

For example, for the rewriting rule named \colorbox{Ivory2}{$bruggink\_2014\_ex\_4$}, the following code should be added to the end of the file:
\begin{minted}{ocaml}
    let available_graph_rewriting_systems = 
        bruggink_2014_ex_4 :: available_graph_rewriting_systems 
\end{minted}

\section{Ruler-Graphs}
A ruler-graph with one forbidden context as defined in Chapter~\ref{chap:antipattern} is represented by a structure with a underlying graph $X$ and an injective graph homomorphism from $X$. 
Consider a ruler graph $(X, f)$ where $X$ is graph \raisebox{2pt}{
            \scalebox{0.7}{\tikz[baseline=-0.5ex]{
            \node [draw,circle] (z) at (-1,0) {};
            \node [draw,circle] (x) at (0,0) {};
            \node[draw,circle] (y) at (1,0) {};
            \draw[->] (z)--(x) node[midway, above] {$a$};
            \draw[->] (x)--(y) node[midway, above] {$a$};
        }}} 
and the injective graph homomorphism is shown in~\autoref{fig:intro:graph_transformation_rule_anti_patternsfs}.
 \begin{figure}[H]
    \centering
    \resizebox{0.6\textwidth}{!}{
\begin{tikzpicture}
      \graphbox{$L$}{30mm}{0mm}{34mm}{20mm}{2mm}{-5mm}{
          \coordinate (o) at (0mm,-3mm); 
          \node[draw,circle] (l1) at ($(o)+(-10mm,0mm)$) {1};
          \node[draw,circle] (l2) at ($(l1)+(2,0)$) {2};
          \node[draw,circle] (l3) at ($(l1) + (1,0)$) {3};
          \draw[->] (l1) -- (l3) node[midway,above] {a};
          \draw[->] (l3) -- (l2) node[midway,above] {a};
      }        
      \graphbox{$F$}{70mm}{0mm}{45mm}{20mm}{2mm}{-5mm}{
        \coordinate (o) at (0mm,-3mm); 
        \node[draw,circle] (l1) at ($(o)+(-10mm,0mm)$) {1};
        \node[draw,circle] (l2) at ($(l1)+(2,0)$) {2};
        \node[draw,circle] (l3) at ($(l1) + (1,0)$) {3};
        \draw[->] (l1) -- (l3) node[midway,above] {a};
        \draw[->] (l3) -- (l2) node[midway,above] {a};
        \draw[->] (l3) edge [loop below] node {$c$} (l3);
      }    
    %   \node () at (37mm,-10mm) {$\leftarrowtail$};
      \node () at (67mm,-10mm) {$\rightarrowtail$};
  \end{tikzpicture}
    }
  \caption{}
  \label{fig:intro:graph_transformation_rule_anti_patternsfs}
 \end{figure} 
Its representation in Ocaml
        can be constructed using the following code:
\begin{figure}[H]
\begin{minted}{ocaml} 
    let graph_X = MGraph.fromList [1;2;3] [(1,"a",3,1);(3,"a",2,2)]
    let ruler_graph_aa_notin_aca = 
        let graph_F = MGraph.fromList 
                [1;2;3] [(1,"a",3,1);(3,"c",3,3);(3,"a",2,2)] in
        let f = Homo.fromList graph_X graph_F 
                    [(1,1);(2,2);(3,3)] [(1,1);(2,2)] in
        {x: graph_X; fx = Some f}
\end{minted}
    \caption{}
    \label{fig:implemented:ruler_graph_representation}
\end{figure}

Instances of ruler-graphs can be defined in the file \colorbox{Ivory2}{$lib/ruler\_graph.ml$} of \textbf{LyonParallel}. To expose these instances to REPL, they should be added to the list \colorbox{Ivory2}{$ruler\_graphs$} in the same file. For example, for the instance of ruler-graph \colorbox{Ivory2}{$ruler\_graph\_aa\_notin\_aca$} shown in~\autoref{fig:implemented:ruler_graph_representation}, the following code should be added to the end of the file:
\begin{minted}{ocaml}
    let ruler_graphs = ruler_graph_aa_notin_aca :: ruler_graphs 
\end{minted}

% As another example, the representation of graph \raisebox{2pt}{
%             \scalebox{0.7}{\tikz[baseline=-0.5ex]{
%             \node [draw,circle] (z) at (-1,0) {};
%             \node [draw,circle] (x) at (0,0) {};
%             \node[draw,circle] (y) at (1,0) {};
%             \draw[->] (z)--(x) node[midway, above] {$a$};
%             \draw[->] (x) edge[loop above] node {$c$} (x);
%             \draw[->] (x)--(y) node[midway, above] {$a$};
%         }}}
%    is constructed using the following code:
% \begin{minted}{ocaml}
%     let y = MGraph.fromList [1;2;3] [(1,"a",3,1);(3,"c",3,3);(3,"a",2,2)]
% \end{minted}

% For example, the ruler-graph with underlying graph \raisebox{2pt}{
%             \scalebox{0.7}{\tikz[baseline=-0.5ex]{
%             \node [draw,circle] (z) at (-1,0) {};
%             \node [draw,circle] (x) at (0,0) {};
%             \node[draw,circle] (y) at (1,0) {};
%             \draw[->] (z)--(x) node[midway, above] {$a$};
%             \draw[->] (x)--(y) node[midway, above] {$a$};
%         }}} and forbidden context \raisebox{2pt}{
%             \scalebox{0.7}{\tikz[baseline=-0.5ex]{
%             \node [draw,circle] (z) at (-1,0) {};
%             \node [draw,circle] (x) at (0,0) {};
%             \node[draw,circle] (y) at (1,0) {};
%             \draw[->] (z)--(x) node[midway, above] {$a$};
%             \draw[->] (x) edge[loop above] node {$c$} (x);
%             \draw[->] (x)--(y) node[midway, above] {$a$};
%         }}} is encoded as the following ocaml structure in the file \mintinline{latex}{Parallel/lib/ruler_graph.ml} of \textbf{LyonParallel}:
% \begin{figure}[H]
%     \begin{minted}{ocaml}
%     let aa_not_in_aca =   
%         let x = MGraph.fromList [1;2;3] [(1,"a",3,1);(3,"a",2,2)] in
%         let f = Homo.fromList 
%             [1;2;3] [(1,"a",3,1);(3,"a",2,2)]
%             [1;2;3] [(1,"a",3,1);(3,"c",3,3);(3,"a",2,2)]
%             [(1,1);(2,2);(3,3)] [(1,1);(2,2)]  in
%         {x; fx = Some f}
%     \end{minted}
%     \caption{}
%     \label{fig:implemented:aa_not_in_aca}
% \end{figure}

\section{Installation and Execution of LyonParallel}
\label{lyonparallel:sec:installation}
This section shows the installation process using \colorbox{Ivory2}{$OPAM$}\textemdash a package manager for OCaml. To install LyonParallel, one needs to have \colorbox{Ivory2}{$OPAM$} installed on one's system, then 
run the following command
\begin{minted}[style=bw]{bash}
    >> opam switch create lyonParallel 5.2.1
\end{minted}
    % \colorbox{Ivory2}{opam switch create lyonParallel 5.2.1}
 and follow the instructions to create and switch to a new \colorbox{Ivory2}{$OPAM$} switch. 
 The dependencies of LyonParallel are listed in the file \colorbox{Ivory2}{$LyonParallel.opam$} in the root directory of the project. To install these dependencies, run the following command:
\begin{minted}{bash}
    >> opam install .
\end{minted}
The project can be built and installed by running the following command:
\begin{minted}{bash}
    >> dune build && dune install
\end{minted}
The following command uninstalls the tool:
\begin{minted}{bash}
    >> opam remove LyonParallel
\end{minted}
Finally, the following command launches the interactive REPL of the tool:
\begin{minted}{bash}
    >> LyonParallel
\end{minted}
Upon launching, one should see:
\begin{minted}[style=bw]{bash}
    >> Type "help" for a list of commands.
\end{minted}
\section{Relative Termination Analysis using LyonParallel}
\label{lyonparallel:sec:termination}
To analyze the relative termination of a DPO graph rewriting system using LyonParallel, one interacts with the REPL of LyonParallel. The REPL allows users to select a specific set of rewriting rules and apply the termination analysis methods provided by the tool to determine whether the set of rules can be partitioned into two sets of rules $\mathcal{A}$ and $\mathcal{B}$ such that $\mathcal{A}$ is relative terminating with respect to $\mathcal{B}$. If such a partition is found, the set of rules will be replaced by $\mathcal{B}$, otherwise one can try other methods.

The REPL command \colorbox{Ivory2}{$systems$} lists the available DPO graph rewriting systems; user-defined systems can be added as described in~\autoref{lyonparallel:sec:implementation_of_dpo_graph_rewriting_systems}. The predefined system shown in~\autoref{fig:implemented:grsaa_0} is named \colorbox{Ivory2}{$bruggink\_2014\_ex\_4\_and\_6$} and appears in that list. Select it at the REPL by name:
\begin{minted}[style=bw]{bash}
select_system_by_name bruggink_2014_ex_4_and_6
\end{minted}

The REPL provides a command \colorbox{Ivory2}{$rulergraphs$} to list the available ruler-graphs. User-defined ruler-graphs can be added to this list by using the the method presented in the previous section. The ruler-graph defined in~\autoref{fig:implemented:ruler_graph_representation} is predefined in the tool and can be found in the list the available ruler-graphs where the number \textit{1} is the index associates to the ruler-graph.

\subsection{Termination Analysis using the Morphism Counting Method}
Consider the graph rewriting system presented in~\autoref{subgraph_counting:ex_contrib_variant} whose termination cannot be proven using existing interpretation-based approaches. Its unique rule is illustrated in~\autoref{fig:subgraph_counting:ex_confdkjfakljlfdsfsdfs}.
 
    \begin{figure}[H]
        \centering
        \resizebox{0.6\textwidth}{!}{
            \begin{tikzpicture}
                \graphbox{$L$}{0mm}{0mm}{35mm}{35mm}{2mm}{-5mm}{
                    \coordinate (delta) at (0,-18mm);
                    \node[draw,circle] (l1) at ($(delta) + (-1,1.5)$) {1};
                    \node[draw,circle] (l2) at ($(delta) + (1,1.5)$) {2};
                    \node[draw,circle] (l3) at ($(delta) + (0,0)$) {3};
                    \draw[->] (l1) -- (l3) node[midway,left] {s};
                    \draw[->] (l2) -- (l3) node[midway,right] {s};
                    \draw[->] (l3) edge [loop below] node {0} (l3);
                }
                \graphbox{$K$}{40mm}{0mm}{35mm}{35mm}{2mm}{-5mm}{
                    \coordinate (delta) at (0,-18mm);
                    \coordinate (interfaceorigin) at ($(delta) +(5,0)$);
                    \node[draw,circle] (r1) at ($(delta) +(-1,1.5)$) {1};
                    \node[draw,circle] (r2) at ($(delta) +(0.5,1.5)$) {2};
                    \node[draw,circle] (r3) at ($(delta) + (0,0)$) {3};
                    % \draw[->] (r1) -- (r3) node[midway,left] {s};
                    % \draw[->] (r3) edge [loop below] node {0} (r3);
                }
                \graphbox{$R$}{80mm}{0mm}{50mm}{35mm}{2mm}{-5mm}{
                    \coordinate (delta) at (-10mm,-18mm);
                    \node[draw,circle] (r1) at ($(delta) + (-1,1.5)$) {1};
                    \node[draw,circle] (r2) at ($(delta) + (0.5,1.5)$) {2};
                    \node[draw,circle] (r3) at ($(delta) + (0,0)$) {3};
                    \node[draw,circle] (r4) at ($(delta) + (1,0)$) {4};
                    \draw[->] (r1) -- (r3) node[midway,left] {s};
                    \draw[->] (r2) -- (r4) node[midway,right] {s};
                    \draw[->] (r4) edge [loop below] node {0} (r4);
                    \draw[->] (r3) edge [loop below] node {0} (r3);
                    \node[draw,circle] (r5) at ($(r2) + (1.5,0)$) {};
                    \draw[->] (r5) edge [loop below] node {0} (r5);
                    \draw[->] (r5) edge [loop right] node {0} (r5);
                    \draw[->] (r5) edge [loop left] node {0} (r5);
                }
                % \graphbox{$R_x$}{40mm}{40mm}{35mm}{35mm}{2mm}{-5mm}{
                %     \coordinate (delta) at (0,-18mm);
                %     \coordinate (rxorigin) at ($(interfaceorigin)+(0,6)$);
                %     \node[draw,circle] (r1) at ($(delta) + (-1,1.5)$) {1};
                %     \node[draw,circle] (r2) at ($(delta) +  (0.5,1.5)$) {2};
                %     \node[draw,circle] (r3) at ($(delta) +  (0,0)$) {3};
                %     \draw[->] (r1) -- (r3) node[midway,left] {s};
                %     % \draw[->] (r3) edge [loop below] node {0} (r3);
                % }
                \node () at (38mm,-18mm) {$\leftarrowtail$};
                \node () at (77mm,-18mm) {$\rightarrowtail$};
                % \node () at (57mm,2mm) {$\uparrowtail$};
                % \node () at (38mm,2mm) {$\swarrowtail$};
                % \node () at (79mm,2mm) {$\searrowtail$};
            \end{tikzpicture}
            }
            \caption{}
            \label{fig:subgraph_counting:ex_confdkjfakljlfdsfsdfs}
    \end{figure}

The following sequence of commands select the system, then use the morphism counting method without forbidden patterns presented in Chapter~\ref{chap:subgraph_counting} to analyse the relative termination of the system.
\begin{minted}[style=bw]{bash}
    >> select_system_by_name plump18_ex6_rule_copy_variant
    >> subgraph_counting_no_forbidden_context
\end{minted}
The following command summarizes the result of the termination analysis.
 \begin{minted}[style=bw]{bash}
    >> recap
\end{minted}

The message displayed contains \colorbox{Ivory2}{$Terminating:\ yes$} when the method has proved the rule's relative termination with respect to the empty set; otherwise the message contains \colorbox{Ivory2}{$Terminating:\ unknown$} and \colorbox{Ivory2}{$Remaining\ Rules:$}, followed by a list of rules to be analyzed.


\subsection{Termination Analysis using the Morphism Counting with Antipattern}
Consider the graph rewriting system shown in~\autoref{antipattern:ex:grs_aca}, whose termination cannot be proven either by the morphism counting method without forbidden patterns introduced in~\ref{chap:subgraph_counting} or by the subgraph counting method presented in~\cite{overbeek2024termination_lmcs}. Its unique rule is illustrated in~\autoref{fig:intro:graph_transformation_rule_anti_pattern___}.
 \begin{figure}[H]
    \centering
\begin{tikzpicture}
      \graphbox{$L$}{0mm}{0mm}{34mm}{20mm}{2mm}{-5mm}{
          \coordinate (o) at (0mm,-3mm); 
          \node[draw,circle] (l1) at ($(o)+(-10mm,0mm)$) {1};
          \node[draw,circle] (l2) at ($(l1)+(2,0)$) {2};
          \node[draw,circle] (l3) at ($(l1) + (1,0)$) {3};
          \draw[->] (l1) -- (l3) node[midway,above] {a};
          \draw[->] (l3) -- (l2) node[midway,above] {a};
      }     
      \graphbox{$K$}{40mm}{0mm}{24mm}{20mm}{2mm}{-5mm}{
          \coordinate (o) at (5mm,-3mm); 
          \node[draw,circle] (l1) at ($(o)+(-10mm,0mm)$) {1}; 
          \node[draw,circle] (l2) at ($(l1)+(1,0)$) {2};
      }    
      \graphbox{$R$}{70mm}{0mm}{45mm}{20mm}{2mm}{-5mm}{
        \coordinate (o) at (0mm,-3mm); 
        \node[draw,circle] (l1) at ($(o)+(-10mm,0mm)$) {1};
        \node[draw,circle] (l2) at ($(l1)+(2,0)$) {2};
        \node[draw,circle] (l3) at ($(l1) + (1,0)$) {3};
        \draw[->] (l1) -- (l3) node[midway,above] {a};
        \draw[->] (l3) -- (l2) node[midway,above] {a};
        \draw[->] (l3) edge [loop below] node {$c$} (l3);
      }    

      \node () at (37mm,-10mm) {$\leftarrowtail$};
      \node () at (67mm,-10mm) {$\rightarrowtail$};
  \end{tikzpicture}
  \caption{}
  \label{fig:intro:graph_transformation_rule_anti_pattern___}
 \end{figure}
The rule is predefined in the tool and can be selected with the REPL command:
 \begin{minted}[style=bw]{bash}
   >> select_system_by_name bruggink14_ex1
 \end{minted}
To analyse the system's relative termination using the morphism counting method with a forbidden pattern (see Chapter~\ref{chap:antipattern}) with the ruler-graph shown in~\autoref{fig:implemented:ruler_graph_representation}, run:
 \begin{minted}[style=bw]{bash}
   >> subgraph_counting_one_forbidden_context aa_not_in_aca
\end{minted}
The following command summarizes the result of the termination analysis.
 \begin{minted}[style=bw]{bash}
   >> recap
\end{minted} 
The message displayed
contains the message \colorbox{Ivory2}{$Terminating:\ yes$}, which shows that the method applied successfully proved the relative termination of the rule with respect to the empty set of rules.

\subsection{Termination Analysis using the Type Graph Method}
Consider the rewriting rule shown in~\autoref{fig:implemented:grsaa_0ssssdfss}.
\begin{figure}[H]
      \centering 
      \resizebox{0.8\textwidth}{!}{
      \begin{tikzpicture}
          \graphbox{$L$}{0mm}{0mm}{34mm}{15mm}{2mm}{-5mm}{
              \coordinate (o) at (0mm,-3mm); 
              \node[draw,circle] (l1) at ($(o)+(-10mm,0mm)$) {1};
              \node[draw,circle] (l2) at ($(l1)+(2,0)$) {2};
              \node[draw,circle] (l3) at ($(l1) + (1,0)$) {3};
              \draw[->] (l1) -- (l3) node[midway,above] {a};
              \draw[->] (l3) -- (l2) node[midway,above] {a};
          }     
          \graphbox{$K$}{40mm}{0mm}{24mm}{15mm}{2mm}{-5mm}{
              \coordinate (o) at (5mm,-3mm); 
              \node[draw,circle] (l1) at ($(o)+(-10mm,0mm)$) {1};
              \node[draw,circle] (l2) at ($(l1)+(1,0)$) {2};
              % \node[draw,circle] (l3) at ($(l1) + (1,0)$) {$\ $};
              % \draw[->] (l1) -- (l3) node[midway,above] {a};
              % \draw[->] (l3) -- (l2) node[midway,above] {a};
          }    
          \graphbox{$R$}{70mm}{0mm}{45mm}{15mm}{2mm}{-5mm}{
              \coordinate (o) at (-5mm,-3mm); 
              \node[draw,circle] (l1) at ($(o)+(-10mm,0mm)$) {1};
              \node[draw,circle] (l2) at ($(l1)+(3,0)$) {2};
              \node[draw,circle] (l3) at ($(l1) + (1,0)$) {4};
              \node[draw,circle] (l4) at ($(l1) + (2,0)$) {5};
              \draw[->] (l1) -- (l3) node[midway,above] {a};
              \draw[->] (l3) -- (l4) node[midway,above] {b};
              \draw[->] (l4) -- (l2) node[midway,above] {a};
          }    
          \node () at (37mm,-8mm) {$\overset{l}{\leftarrowtail}$};
          \node () at (67mm,-8mm) {$\overset{r}{\rightarrowtail}$};
          % \draw[>->] (51mm,2mm) -- (52mm,3mm);
      \end{tikzpicture}
      }
      \caption{}
      \label{fig:implemented:grsaa_0ssssdfss}
  \end{figure}
The following command selects this system.
 \begin{minted}[style=bw]{bash}
   >> select 0
\end{minted}
The following command shows relative termination of some non-empty subset of rules of the system with respect to the set containing other rules using the type graph method with the natural arithmetic semiring. The timeout is set to 30 seconds.
 \begin{minted}[style=bw]{bash}
   >> type_graph 30 n
\end{minted}
The command \colorbox{Ivory2}{$recap$} displays a summary of the termination analysis. The summary contains \colorbox{Ivory2}{$Terminating:\ yes$} and \colorbox{Ivory2}{$Remaining\ Rules:\ none$}, showing that the method successfully proved the rule's relative termination with respect to the empty rule set.

This command
\begin{minted}[style=bw]{bash}
   >> type_graph 30 n
\end{minted}
is an instance of the following command pattern
\begin{minted}[style=bw]{bash}
    type_graph [index] [timeout] [semiring_1 semiring_2 ...]
\end{minted}
From the command pattern, we can see that more semirings can be selected. In fact, there are 6 semirings available in the tool, which can be selected by adding flags in the set $\set{a,t,n,A,T,N}$
where $A$, $T$, $N$ stand for the arctic, tropical, and arithmetic semirings over the (extended) natural numbers;  $a$, $t$, $n$ stand for the arctic, tropical, and arithmetic semirings over the (extended) real numbers. For example, we can have the following command which launch the type graph method with the timeout set to 30 seconds and all 6 available semirings.
\begin{minted}[style=bw]{bash} 
    >> type_graph 30 a n t A N T
\end{minted}
Type graph method with different semirings will be launched parallelly and cooperate with each other. When a non-empty subset $A$ of rules is proved to be relative terminating with respect to the set of remaining rules $B$, a new iteration starts with the set of rules $B$.


% The system can be selected by using the command \colorbox{Ivory2}{select 3} which select the system with number \textit{3}. The REPL will display the following message to confirm the selection:
% \begin{center}
%     \colorbox{Ivory2}{$bruggink\_2014\_ex\_4\_and\_6$ selected}
% \end{center}

\section{Availability and license}

The implementation described in this paper is available at \url{https://github.com/Qi-tchi/LyonParallel/tree/thesis}. The project is distributed under the LGPL-2.1 License; see the repository's \texttt{LICENSE} file for the full license text.