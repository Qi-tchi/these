This dissertation focused on automated techniques for proving the relative termination of DPO graph rewriting systems.

We first extended an existing technique, the type graph method, to non-well-founded semirings, enabling more efficient implementation of the method.
A new automated termination technique, called morphism counting, is also proposed for proving relative termination of injective DPO graph rewriting systems on edge-labeled directed multigraphs. It is based on counting monomorphisms from specific pattern-graphs to the graph before and after transformation. This technique can handle cases, such as Example~\ref{subgraph_counting:ex_contrib_variant}, that existing interpretation-based approaches cannot.
Finally, we extended our morphism counting method to count specific subgraphs that are not included in forbidden contexts. This extension successfully proves termination for systems like those presented
in Example~\ref{antipattern:ex:grs_aca} and Example~\ref{antipattern:ex:endrullis:d3:termination} which prior interpretation-based approaches~\cite{zantema2014termination,bruggink2014termination,bruggink2015proving,endrullis2024generalized_arxiv_v2,overbeek2024termination_lmcs} and the morphism counting method presented in Chapter~\ref{chap:subgraph_counting} fail to handle.

An implementation of the type graph method and its extension, along with our morphism counting technique and its extension, was integrated into a unified tool, called LyonParallel, written in OCaml.

For future work, we outline directions that provide a roadmap from immediate improvements to broader theoretical generalizations and increased trustworthiness of the methods and tool presented in this dissertation.

In the short term (months), we will:
    \begin{itemize}
      \item Generalize our morphism counting method (Chapter~\ref{chap:antipattern}) to count subgraphs with multiple forbidden contexts, enhancing its applicability to a broader range of graph rewriting systems.
      \item Investigate the differences between our morphism counting technique and the technique developed by Overbeek and Endrullis for PBPO+~\cite{overbeek2024termination_lmcs}, with the goal of understanding their relationship and moving toward a more general termination criterion for DPO rewriting systems or PBPO+.
    \end{itemize}

In the mid term (2--5 years), we plan to:
    \begin{itemize}
      \item Develop a formal proof of correctness of the extended type graph method and the morphism counting technique using a proof assistant, to increase trustworthiness and prevent potential proof errors~\cite{contejean2004certified}.
      \item Equip our tool with a certificate-generation mechanism that produces artifacts checkable by proof assistants, thereby mitigating implementation bugs and strengthening result reliability~\cite{contejean2007certification,contejean2010a3pat}.
    \end{itemize}

In the long term (5+ years), we aim to:
    \begin{itemize}
      \item Extend the type graph method and the morphism counting technique to other algebraic graph rewriting formalisms.
      \item Extend the morphism counting technique to graph rewriting systems with negative application conditions. This could lead to a more practical termination tool, as it is very convenient to use negative application conditions to express constraints on the application of rules in graph transformation systems.
    %   \item Explore automated techniques for proving confluence of graph rewriting systems. In terminating systems, this property guarantees that the outcome does not depend on the order of rule applications.
    \end{itemize}