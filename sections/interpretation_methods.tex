
Interpretation methods are a fundamental approach in proving the termination of term rewriting systems (TRS). These methods involve mapping terms into well-founded mathematical structures, ensuring that each rewriting step corresponds to a reduction in size according to a specific order. The significance and effectiveness of interpretation methods in termination analysis have been extensively discussed in the literature, including in the comprehensive survey \cite{dersh1987termination}.
In this section, we explore key concepts such as 
$\Sigma$-structures, term interpretations, and homomorphisms. These concepts are crucial for demonstrating that rewriting systems terminate by ensuring that every rewrite leads to a simplification in a well-defined order.

\begin{definition}[$\Sigma$-structure \cite{cori2000mathematical}]
  A \textit{$\Sigma$-structure}, is any structure $\mathcal{M}$ consisting of:
  \begin{itemize}
    \item a non-empty set $M$, called the \textit{underlying set} (or \textit{base set}) of the realization $\mathcal{M}$;
    \item for each constant symbol $c$ of $\Sigma$, an element $c^{\mathcal{M}}$ of $M$, called the \textit{interpretation} of the symbol $c$ in the realization $\mathcal{M}$;
    \item for each natural number $k \mathop{\geq} 1$, and for each function symbol $f$ of arity $k$ in $\Sigma$, a function $f^{\mathcal{M}}$ from $M^k$ to $M$, called the \textit{interpretation} of the symbol $f$ in the realization $\mathcal{M}$;
    % \item for each natural number $k \mathop{\geq} 1$, and for each relation symbol $R$ of arity $k$ in $\Sigma$, a subset $R^{\mathcal{M}}$ of $M^k$ (that is, a $k$-ary relation on the set $M$), called the \textit{interpretation} of the symbol $R$ in the realization $\mathcal{M}$.
  \end{itemize}
\end{definition}

\begin{example}[Term Algebra]
  Give a signature $\Sigma$, the term algebra $(T(\Sigma, \mathcal{X}), \Sigma)$ is a model of $\Sigma$, or $\Sigma$-algebra, where every $f \mathop{\in} \Sigma$ is interpreted by itself.
\end{example}

\begin{definition}[$\Sigma$-Structure homomorphism \cite{cori2000mathematical}]
  The function $\varphi$ is a \textit{homomorphism of} $\Sigma$-\textit{structures} from $\mathcal{M}$ to $\mathcal{N}$ if and only if the following conditions are satisfied:
  \begin{itemize}
      \item For every constant symbol $c$ of $\Sigma$,
      \[
      \varphi(c^{\mathcal{M}}) \mathop{=} c^{\mathcal{N}};
      \]
      \item For every natural number $n \mathop{\geq} 1$, for every $n$-ary function symbol $f$ of $\Sigma$, and for all elements $a_1, a_2, \ldots, a_n$ belonging to $M$,
      \[
      \varphi(f^{\mathcal{M}}(a_1, a_2, \ldots, a_n)) \mathop{=} f^{\mathcal{N}}(\varphi(a_1), \varphi(a_2), \ldots, \varphi(a_n));
      \]
      \item For every natural number $k \mathop{\geq} 1$, for every $k$-ary relation symbol $R$ of $\Sigma$, and for all elements $a_1, a_2, \ldots, a_k$ belonging to $M$,
      \[
      \text{if } (a_1, a_2, \ldots, a_k) \mathop{\in} R^{\mathcal{M}}, \text{ then } (\varphi(a_1), \varphi(a_2), \ldots, \varphi(a_k)) \mathop{\in} R^{\mathcal{N}}.
      \]
  \end{itemize}
  \end{definition}

  \begin{definition}[Term Interpretation \cite{cori2000mathematical}]
    Given a natural number $n$, $n$ variables $X_0, X_1, \ldots, X_{n-1}$ that are distinct from each other, a term $t \mathop{=} t[X_0, X_1, \ldots, X_{n-1}]$ of the language $\Sigma$, an $\Sigma$-structure $\mathcal{M} \mathop{=} \langle M, \ldots \rangle$, and $n$ elements $a_0, a_1, \ldots, a_{n-1}$ of $M$, we call the \textit{interpretation of the term} $t$ in the $\Sigma$-structure $\mathcal{M}$, when the variables $X_0, \ldots, X_{n-1}$ are respectively interpreted by the elements $a_0, a_1, \ldots, a_{n-1}$, the element of $M$ denoted by:
    \[
     t^{\mathcal{M}}[X_0 \mathop{\mapsto} a_0, X_1 \mathop{\mapsto} a_1, \ldots, X_{n-1} \mathop{\mapsto} a_{n-1}]
    \]
    defined as follows, by induction on $t$:
    \begin{itemize}
        \item If $t \mathop{=} X_j$ (where $0 \leq j \leq n-1$),
        \[
        t^{\mathcal{M}}[X_0 \mathop{\mapsto} a_0, X_1 \mathop{\mapsto} a_1, \ldots, X_{n-1} \mathop{\mapsto} a_{n-1}] \mathop{=} a_j;
        \]
        \item If $t \mathop{=} c$ (a constant symbol of $\Sigma$),
        \[
         t^{\mathcal{M}}[X_0 \mathop{\mapsto} a_0, X_1 \mathop{\mapsto} a_1, \ldots, X_{n-1} \mathop{\mapsto} a_{n-1}] \mathop{=}  c^{\mathcal{M}};
        \]
        \item If $t \mathop{=} f t_1 t_2 \ldots t_k$ (where $k \mathop{\in} \mathbb{N}^*$, and $f$ is a $k$-ary function symbol of $\Sigma$, and $t_1, t_2, \ldots, t_k$ are terms of $\Sigma$),
        \[
          t^{\mathcal{M}}[X_0 \mathop{\mapsto} a_0, X_1 \mathop{\mapsto} a_1, \ldots, X_{n-1} \mathop{\mapsto} a_{n-1}] \mathop{=} f^{\mathcal{M}}(t_1^{\mathcal{M}}[X_0 \mathop{\mapsto} a_0, \ldots, X_{n-1} \mathop{\mapsto} a_{n-1}], \ldots, t_k^{\mathcal{M}}[X_0 \mathop{\mapsto} a_0, \ldots, X_{n-1} \mathop{\mapsto} a_{n-1}]).
        \]
    \end{itemize}
    In practice, we will denote more simply:
    \[
    t^{\mathcal{M}}[a_0, a_1, \ldots, a_{n-1}].
    \]
    \end{definition}
    

    \begin{definition}[Monotone Function \cite{nipkow1998term}]
        Let $>$ be a strict order on a set $A$. A function $F : A^n \mathop{\rightarrow} A$ is said to be \textit{monotone} with respect to $>$ if and only if
        \[
        a \mathop{>} b \text{ implies } F(a_1, \ldots, a_{i-1}, a, a_{i+1}, \ldots, a_n) \mathop{>} F(a_1, \ldots, a_{i-1}, b, a_{i+1}, \ldots, a_n)
        \]
        for all $i$ where $1 \leq i \leq n$, and for all elements $a, b, a_1, \ldots, a_{i-1}, a_{i+1}, \ldots, a_n$ in $A$.
    \end{definition}

    \begin{definition}[Interpretation Method \cite{dersh1987termination}]
        Let $\Sigma$ be the rewriting system. 
        Let $\mathcal{A}$ be a $\Sigma$-structure such that for all $1 \leq i \leq n$, the function $f^\mathcal{M}$ is monotone w.r.t. $>_A$. Let $>_A$ be a well founed order on the domain $A$.

        Let $\Pi$ be the set of interpretation of $T(\Sigma,\mathcal{X})$ in $A$.
        Let $B\isdef \{(\pi(t))_{\pi\in \Pi} \mid t \mathop{\in} T(\Sigma,\mathcal{X})\}$, the set of generalized series. 

        Let $>_B$ be the binary relation defined by 
        $$(\pi(s))_{\pi\in \Pi} \mathop{>} (\pi(t))_{\pi\in \Pi} \text{ iff } \forall \pi \mathop{\in} \Pi. \pi(s) >_A \pi(t)$$

        % Let $>$ be the binary relation defined by 
        % $$s \mathop{>} t \text{ iff } \forall \pi:T(\Sigma,\mathcal{X}) \mathop{\to} A. \pi(s) >_A \pi(t)$$

        We have $(B, >_B)$ is a terminating rewriting system. Thus, if $h:(T(\Sigma,\mathcal{X}),R) \mathop{\to} (B,>_B)$ defined by $h(t) \isdef (\pi(t))_{\pi\in \Pi}$ is a homomorphism, then $(T(\Sigma,\mathcal{X}), \mathop{\to} _R)$ terminates.
    \end{definition}