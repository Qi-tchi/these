\begin{itemize}
    \item[bruggink2014] the paper "high-level replacement units and their termination properties 2005" considers high-level replacement units (HLRU), which are transformation systems with externam control expressions. The paper introduces a general framework for proving termination of such HLRUs, but the only concrete termination criteria considered are node and edge counting, which are subsumed by the weighted type graph method.
    \item[bruggink2014] in "termination criteria for model transformation 2005" layered graph transformation systems are considered, which are graph transformation systems where interleaving creation and deletion of edges with the same label is prohibited and creation of nodes is bounded. The paper shows such graph transformation systems are terminating.
    \item[bruggink2014] The paper "Termination analysis of model transformations by petri nets. 2006" simulates a graph transformation system by a Petri-net. Thus, the presence of edges with certain labels and the causal relationships between them are modeled, but no other structural properties of the graph. The paper uses typed graph transformation systems; thus, a type graph is used but, unlike in our weighted type graph method, it is fixed by the graph transformation system. Finally, [3] was one of the inspirations for this paper. 
    \item[bruggink2014] "Towards a systematic method for proving termination of graph transformation systems" is one of the inspirations for the paper of bruggink2014. its termination argument is subsumed by the weighted type graph technique.
\end{itemize}
