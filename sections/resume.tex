Les systèmes de réécriture de graphes sont des modèles formels qui décrivent comment des graphes peuvent être transformés en appliquant un ensemble de règles de transformation. Le domaine de l'analyse automatique de la terminaison des systèmes de réécriture de graphes étudie des techniques permettant de prouver qu'un tel système termine, c'est-à-dire que, pour tout graphe, l'application de ses règles de transformation ne peut pas durer indéfiniment. La contribution de cette thèse est quadruple.

Premièrement, nous étendons une technique existante, appelée méthode par graphes de type, aux systèmes de réécriture de graphes avec double pushout (DPO). La méthode par graphes de type attribue des poids à des morphismes ciblant un graphe de types, et le poids d'un graphe est défini comme la somme des poids de tous les morphismes depuis ce graphe vers le graphe de types. La propriété de terminaison est alors démontrée en montrant que le poids du graphe diminue strictement à chaque application d'une règle de transformation, à condition que l'ensemble des poids soit bien fondé. Cependant, exiger que les poids soient bien fondés peut poser des limitations pratiques, car la complexité de la recherche d'un graphe de types adapté est extrêmement élevée. Notre extension réduit cette complexité en autorisant l'utilisation de poids issus de ensembles non bien fondés, ce qui permet une implémentation plus efficace de la méthode en pratique pour certains cas.

Deuxièmement, nous développons une nouvelle condition suffisante vérifiable par machine, appelée comptage de sous-graphes, pour les systèmes de réécriture de graphes DPO avec des règles injectives sur les multigraphes orientés étiquetés sur les arêtes. Elle repose sur l'idée que la réécriture ne peut pas durer indéfiniment si le nombre d'un sous-graphe spécifique diminue strictement à chaque transformation.

Troisièmement, nous étendons notre méthode de comptage de sous-graphes. Cette extension permet d'établir la propriété de terminaison d'un système de réécriture de graphes en analysant la diminution du nombre d'occurrences de sous-graphes spécifiques qui ne sont pas inclus dans des contextes interdits.

Enfin, nous implémentons un outil automatique de terminaison, nommé LyonParallel, pour la transformation de multigraphes orientés étiquetés sur les arêtes, intégrant à la fois la méthode existante par graphes de type et notre extension de cette méthode, ainsi que notre technique de comptage de sous-graphes et son extension.