Les systèmes de réécriture de graphes sont des modèles formels qui décrivent comment des graphes peuvent être transformés en appliquant un ensemble de règles. Le domaine de l'analyse automatique de la terminaison des systèmes de réécriture de graphes étudie des techniques automatisées pour prouver qu'un tel système termine, c'est-à-dire que l'application de ses règles de transformation ne peut pas durer indéfiniment.

Les contributions de cette thèse sont triples.
Premièrement, nous étendons une technique existante aux systèmes de réécriture de graphes avec double pushout (DPO). La méthode attribue des poids à des morphismes ciblant un graphe de types, et le poids d'un graphe est défini comme la somme des poids de tous les morphismes depuis ce graphe vers le graphe de types. La propriété de terminaison est alors démontrée en montrant que le poids du graphe diminue strictement à chaque application d'une règle de transformation, à condition que l'ensemble des poids soit bien fondé. Cependant, exiger que les poids soient bien fondés peut poser des limitations pratiques, car la complexité de la recherche d'un graphe de types adapté est extrêmement élevée.
Notre extension réduit cette complexité en autorisant l'utilisation de poids issus d'un ensemble non bien fondés, ce qui permet une implémentation plus efficace de la méthode en pratique.

Deuxièmement, nous développons une nouvelle condition suffisante vérifiable par machine pour les systèmes de réécriture de graphes DPO injectives sur les multigraphes orientés étiquetés sur les arêtes. Cette condition repose sur l'idée que si le nombre d'un certain sous-graphe diminue strictement à chaque transformation, alors la réécriture ne peut pas durer indéfiniment. 

Troisièmement, nous implémentons un outil de terminaison nommé LyonParallel pour la transformation de multigraphes orientés étiquetés sur les arêtes, incluant la méthode existante basée sur le graphe de types, notre extension, et notre nouvelle technique.