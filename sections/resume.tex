Les systèmes de réécriture de graphes sont des modèles formels qui décrivent comment des graphes peuvent être transformés en appliquant un ensemble de règles de transformation. Le domaine de l'analyse de la terminaison des systèmes de réécriture de graphes étudie des techniques permettant de prouver qu'un tel système termine, c'est-à-dire que, pour tout graphe, l'application de ses règles de transformation ne peut pas durer indéfiniment. 

Cette thèse s'intéresse au développement de techniques automatiques pour l'analyse de la terminaison des systèmes de réécriture de graphes avec double pushout (DPO).
Elle y contribue sous quatre étapes:

Premièrement, nous étendons une technique existante, appelée méthode par graphes de type. La méthode par graphes de type attribue des poids à des morphismes ciblant un graphe de type, et le poids d'un graphe est défini comme la somme des poids de tous les morphismes depuis ce graphe vers le graphe de type. La propriété de terminaison est alors démontrée en montrant que le poids du graphe diminue strictement à chaque application d'une règle de transformation, à condition que l'ensemble des poids soit bien fondé. Exiger que les poids soient bien fondés peut, cependant, poser des limitations pratiques, car la complexité de la recherche d'un graphe de type adapté est extrêmement élevée. Notre extension réduit cette complexité en autorisant des poids issus d'ensembles non bien fondés.

Deuxièmement, nous développons une nouvelle condition suffisante, appelée méthode de comptage de morphismes. Cette méthode s'applique aux systèmes de réécriture de graphes avec DPO à règles injectives sur des multigraphes orientés dont les arêtes sont étiquetées. Elle repose sur l'idée que la réécriture ne peut pas durer indéfiniment si le nombre de morphismes ayant un sous-graphe spécifique comme domaine diminue strictement à chaque transformation.

Troisièmement, nous étendons notre méthode de comptage de morphismes pour une analyse plus fine. Spécifiquement, nous considérons les morphismes ayant un sous-graphe spécifique comme domaine et dont l'image n'est pas incluse dans un contexte interdit.

Enfin, nous avons implanté un outil automatique de terminaison, appelé LyonParallel, dédié à la transformation avec DPO des multigraphes orientés étiquetés sur les arêtes. Cet outil intègre à la fois la méthode par graphes de type et notre extension de cette méthode, ainsi que la technique de comptage de morphismes et son extension.

% Cette extension permet de démontrer la terminaison d'un système de réécriture de graphes avec DPO en analysant la diminution du nombre de morphismes dont le domaine est un graphe spécifique et dont l'image n'est pas incluse dans un contexte interdit.