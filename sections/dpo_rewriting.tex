\begin{definition}[Rewriting rule and match~\cite{corradini1997algebraic}]
  \label{def:grs:dpo_rule}
A \textbf{DPO rewriting rule} $\rho$ is a span \( L \overset{l}{\leftarrow} K \overset{r}{\rightarrow} R \), where \( K \) is the \textbf{interface}, \( L \) is the \textbf{left-hand-side graph}, denoted \( \operatorname{lhs}(\rho) \), and \( R \) is the \textbf{right-hand-side graph}, denoted \( \operatorname{rhs}(\rho) \). The rule is \textbf{left monic} if \( l \) is monic, and \textbf{right monic} if \( r \) is monic. 
The rule is \textbf{monic} if $l$ and $r$ are both monic.
A match of the rule in an graph \( G \) is a morphism \( m: L \rightarrow G \).   
\end{definition}
In the case of the category \textbf{Graph} of edge-labeled directed multigraphs, the rule is left monic if the morphism \( l \) is injective, and right monic if the morphism \( r \) is injective. The rule is monic if both \( l \) and \( r \) are injective.

Intuitively, $K$ is the common part of $L$ and $R$. To apply the rule to an object $G$, we need to find a match $m:L \to G$. The rule is then applied by constructing completing the diagram with double pushout below and onece the pushout squares are constructed, one says that the object $G$ can be rewritten to $H$ using the rule $\rho$ and the match $m$. (to double check)
\begin{center}
     \resizebox{0.4\textwidth}{!}{
          \begin{tikzpicture}
            % [node distance=11mm]
            \node (I) at (0,0) {$K$};
            \node (L) at (-2,0) {$L$};
            \node (R) at (2,0) {$R$};
            \node (G) at (-2,-2) {$G$};
            \node (C) at (0,-2) {$C$};
            \node (H) at (2,-2) {$H$};
            \draw [->] (I) to  node [midway,below] {$l$} (L);
            \draw [->] (I) to  node [midway,below] {$r$} (R);
            \draw [->] (L) to node [midway,right] {$m$} (G);
            \draw [->] (I) to node [midway,right] {$u$} (C);
            \draw [->] (R) to node [midway,left] {$m'$} (H);
            \draw [->] (C) to node [midway,above] {$l'$} (G);
            \draw [->] (C) to node [midway,above] {$r'$} (H);
            \node [at=($(I)!.5!(G)$)] {\normalfont PO};
            \node [at=($(I)!.5!(H)$)] {\normalfont PO};
          \end{tikzpicture}
        % \end{center}
        }
\end{center}

\begin{example}
  \label{ex:grsaa}
  The follwing DPO rewriting rule on edge-labeled directed multigraphs is from \cite[Example 6]{bruggink2014termination}:
  \begin{center} 
      \resizebox{0.7\textwidth}{!}{
      \begin{tikzpicture}
          \graphbox{$L$}{0mm}{0mm}{34mm}{15mm}{2mm}{-5mm}{
              \coordinate (o) at (0mm,-3mm); 
              \node[draw,circle] (l1) at ($(o)+(-10mm,0mm)$) {1};
              \node[draw,circle] (l2) at ($(l1)+(2,0)$) {2};
              \node[draw,circle] (l3) at ($(l1) + (1,0)$) {3};
              \draw[->] (l1) -- (l3) node[midway,above] {a};
              \draw[->] (l3) -- (l2) node[midway,above] {a};
          }     
          \graphbox{$K$}{40mm}{0mm}{24mm}{15mm}{2mm}{-5mm}{
              \coordinate (o) at (5mm,-3mm); 
              \node[draw,circle] (l1) at ($(o)+(-10mm,0mm)$) {1};
              \node[draw,circle] (l2) at ($(l1)+(1,0)$) {2};
              % \node[draw,circle] (l3) at ($(l1) + (1,0)$) {$\ $};
              % \draw[->] (l1) -- (l3) node[midway,above] {a};
              % \draw[->] (l3) -- (l2) node[midway,above] {a};
          }    
          \graphbox{$R$}{70mm}{0mm}{45mm}{15mm}{2mm}{-5mm}{
              \coordinate (o) at (-5mm,-3mm); 
              \node[draw,circle] (l1) at ($(o)+(-10mm,0mm)$) {1};
              \node[draw,circle] (l2) at ($(l1)+(3,0)$) {2};
              \node[draw,circle] (l3) at ($(l1) + (1,0)$) {4};
              \node[draw,circle] (l4) at ($(l1) + (2,0)$) {5};
              \draw[->] (l1) -- (l3) node[midway,above] {a};
              \draw[->] (l3) -- (l4) node[midway,above] {b};
              \draw[->] (l4) -- (l2) node[midway,above] {a};
          }    
          \node () at (37mm,-8mm) {$\overset{l}{\leftarrowtail}$};
          \node () at (67mm,-8mm) {$\overset{r}{\rightarrowtail}$};
          % \draw[>->] (51mm,2mm) -- (52mm,3mm);
      \end{tikzpicture}
      }
  \end{center}
  This rule has a discrete interface \( K \). It is monic (or injective). It replaces a chain \enquote{aa} whose middle node has no other incident edges by a chain \enquote{aba} with the same endpoints and all other nodes and edges are fresh.
\end{example}

\begin{definition}[DPO Rewriting step~\cite{endrullis2024generalized_arxiv_v2}]
  \label{def:rewriting_step}
    \ \newline
    \noindent
    \begin{minipage}{0.72\textwidth}
      A DPO diagram $\delta$ is a diagram as shown on the right.
      This diagram $\delta$ is a witness for the \textbf{rewriting step} from \( G \) to \( H \) using the rule \( \rho \) and \textbf{match} \( m \), denoted \( G \Rightarrow_\rho^m H \) or \( G \Rightarrow_\rho^\delta H \). We denote $\operatorname{left}(\delta)$ and $\operatorname{right}(\delta)$ the pushout squares $KLGC$ and $KRHC$, respectively.
    \end{minipage}
    \hfill
    \begin{minipage}{0.28\textwidth}
          % \begin{center}
          \hfill
          \resizebox{0.85\textwidth}{!}{
          \begin{tikzpicture}
            % [node distance=11mm]
            \node (I) at (0,0) {$K$};
            \node (L) at (-2,0) {$L$};
            \node (R) at (2,0) {$R$};
            \node (G) at (-2,-2) {$G$};
            \node (C) at (0,-2) {$C$};
            \node (H) at (2,-2) {$H$};
            \draw [->] (I) to  node [midway,below] {$l$} (L);
            \draw [->] (I) to  node [midway,below] {$r$} (R);
            \draw [->] (L) to node [midway,right] {$m$} (G);
            \draw [->] (I) to node [midway,right] {$u$} (C);
            \draw [->] (R) to node [midway,left] {$m'$} (H);
            \draw [->] (C) to node [midway,above] {$l'$} (G);
            \draw [->] (C) to node [midway,above] {$r'$} (H);
            \node [at=($(I)!.5!(G)$)] {\normalfont PO};
            \node [at=($(I)!.5!(H)$)] {\normalfont PO};
          \end{tikzpicture}
        % \end{center}
        }
        \end{minipage}
  \end{definition}
\begin{example}
  \label{ex:rewriting_step_grs_aa}
  The DPO diagram below defines a rewriting step using the rule in \autoref{ex:grsaa}.
  \begin{center} 
      \resizebox{0.7\textwidth}{!}{
      \begin{tikzpicture}
          \graphbox{\( L \)}{0mm}{-3mm}{34mm}{12mm}{2mm}{2mm}{
              \coordinate (o) at (0mm,-8mm); 
              \node[draw,circle] (l1) at ($(o)+(-10mm,0mm)$) {1};
              \node[draw,circle] (l2) at ($(l1)+(2,0)$) {2};
              \node[draw,circle] (l3) at ($(l1) + (1,0)$) {3};
              \draw[] (l1) -- (l3) node[midway,above] {a};
              \draw[] (l3) -- (l2) node[midway,above] {a};
          } 
          \graphbox{\( K \)}{40mm}{-3mm}{34mm}{12mm}{2mm}{2mm}{
              \coordinate (o) at (0mm,-8mm); 
              \node[draw,circle] (l1) at ($(o)+(-10mm,0mm)$) {1};
              \node[draw,circle] (l2) at ($(l1)+(2,0)$) {2};
          }  
          \graphbox{\( R \)}{80mm}{-3mm}{45mm}{12mm}{2mm}{2mm}{
              \coordinate (o) at (-5mm,-8mm); 
              \node[draw,circle] (l1) at ($(o)+(-10mm,0mm)$) {1};
              \node[draw,circle] (l2) at ($(l1)+(3,0)$) {2};
              \node[draw,circle] (l3) at ($(l1) + (1,0)$) {4};
              \node[draw,circle] (l4) at ($(l1) + (2,0)$) {5};
              \draw[ ] (l1) -- (l3) node[midway,above] {a};
              \draw[ ] (l3) -- (l4) node[midway,above] {b};
              \draw[ ] (l4) -- (l2) node[midway,above] {a};
          }    
          \graphbox{\( G \)}{0mm}{-22mm}{34mm}{22mm}{2mm}{-3mm}{
              \coordinate (o) at (0mm,-3mm); 
              \node[draw,circle] (l1) at ($(o)+(-10mm,0mm)$) {1};
              \node[draw,circle] (l2) at ($(l1)+(2,0)$) {2};
              \node[draw,circle] (l3) at ($(l1) + (1,0)$) {3};
              \node[draw,circle] (l4) at ($(l2) + (0,-1)$) {6};
              \draw[] (l1) -- (l3) node[midway,above] {a};
              \draw[] (l3) -- (l2) node[midway,above] {a};
              \draw[ ] (l2) -- (l4) node[midway,right] {a};
              \node[draw,circle] (l6) at ($(l1) + (0,-1)$) {7};
              \draw[] (l1) -- (l6) node[midway,left] {a};
          }    
          \graphbox{\( C  \)}{40mm}{-22mm}{34mm}{22mm}{2mm}{-3mm}{
              \coordinate (o) at (0mm,-3mm); 
              \node[draw,circle] (l1) at ($(o)+(-10mm,0mm)$) {1};
              \node[draw,circle] (l2) at ($(l1)+(2,0)$) {2};
              \node[draw,circle] (l4) at ($(l2) + (0,-1)$) {6};
              \draw[ ] (l2) -- (l4) node[midway,right] {a};
              \node[ draw,circle] (l6) at ($(l1) + (0,-1)$) {7};
              \draw[ ] (l1) -- (l6) node[midway,left] {a};
          }    
          \graphbox{\( H \)}{80mm}{-22mm}{45mm}{22mm}{2mm}{-3mm}{
              \coordinate (o) at (-5mm,-3mm); 
              \node[draw,circle] (l1) at ($(o)+(-10mm,0mm)$) {1};
              \node[draw,circle] (l2) at ($(l1)+(3,0)$) {2};
              \node[draw,circle] (l3) at ($(l1) + (1,0)$) {4};
              \node[draw,circle] (l4) at ($(l1) + (2,0)$) {5};
              \node[ draw,circle] (l5) at ($(l2) + (0,-1)$) {6};
              \node[ draw,circle] (l6) at ($(l1) + (0,-1)$) {7};
              \draw[ ] (l1) -- (l6) node[midway,left] {a};
              \draw[] (l1) -- (l3) node[midway,above] {a};
              \draw[] (l3) -- (l4) node[midway,above] {b};
              \draw[ ] (l4) -- (l2) node[midway,above] {a};
              \draw[ ] (l2) -- (l5) node[midway,right] {a};
          }    
          \node () at (37mm,-8mm) {\( \leftarrowtail \)}; % K -> L
          \node () at (77mm,-8mm) {\( \rightarrowtail \)}; % K -> R
          \node () at (15mm,-18mm) {\( m\ \downarrowtail \)};
          \node () at (37mm,-33mm) {\( \leftarrowtail \)};
          \node () at (58mm,-18mm) {\( u\downarrowtail \)};
          \node () at (102mm,-18mm) {\( \downarrowtail \)};
          \node () at (77mm,-33mm) {\( \rightarrowtail \)}; % C -> H
      \end{tikzpicture}
      }
  \end{center}
\end{example}

In category \textbf{Graph}, the pushout of two arrows always exsits~\cite[p.188]{corradini1997algebraic}. Therefore, once the first pushout square is constructed, the second pushout square can always be constructed and is unique up to isomorphism because of the universal property. However, the first pushout square cannot always be constructed.

\begin{proposition}[Existence of pushout complements~\cite{corradini1997algebraic}]
    Let $b: A\rightarrow B$ and $g : B \rightarrow D$ be two morphisms in the category \textbf{Graph}. There exists a pushout complement $A \overset{u}{\rightarrow} C \overset{l'}{\rightarrow} D$ of $A \overset{l}{\rightarrow} C \overset{m}{\rightarrow} D$ if and only if the following conditions are satisfied:
    \begin{itemize}
        \item{[Dangling edge condition]} No edge $e \in G_1 \setminus m(L_1)$ is incident to any node in $m(L_0 \setminus l(K_0))$
        \item{[Identification condition]} There is no $x,y \in B_V \cup B_E$ such that $x \not = y$, $g(x) = g(y)$ and $y \notin b(A_V \cup A_E)$.
    \end{itemize}
\end{proposition}

Intuitively, an edge $e$ in $G_1 \setminus m(L_1)$ is an edge which is not in the image of the match $m$. If such an edge $e$ is incident to a node in $m(L_0 \setminus l(K_0))$, which are nodes that will be removed in the rewriting step and thus neither contained in $C$ nor in $H$, the edge $e$ will be dangling in $H$ after the rewriting step. A match which does not satisfy the dangling edge condition is shown below. In this example, when node 3 is removed, the edge from node 3 to node 6 will be dangling. (to double check)
    \begin{center}
        \resizebox{0.9\textwidth}{!}{
        \begin{tikzpicture}
            \graphbox{\( L \)}{0mm}{-3mm}{34mm}{12mm}{2mm}{2mm}{
                \coordinate (o) at (0mm,-8mm); 
                \node[draw,circle] (l1) at ($(o)+(-10mm,0mm)$) {1};
                \node[draw,circle] (l2) at ($(l1)+(2,0)$) {2};
                \node[draw,circle] (l3) at ($(l1) + (1,0)$) {3};
                \draw[->] (l1) -- (l3) node[midway,above] {a};
                \draw[->] (l3) -- (l2) node[midway,above] {a};
            } 
    
            \graphbox{\( K \)}{40mm}{-3mm}{34mm}{12mm}{2mm}{2mm}{
                \coordinate (o) at (0mm,-8mm); 
                \node[draw,circle] (l1) at ($(o)+(-10mm,0mm)$) {1};
                \node[draw,circle] (l2) at ($(l1)+(2,0)$) {2};
            }  
    
            \graphbox{\( R \)}{80mm}{-3mm}{45mm}{12mm}{2mm}{2mm}{
                \coordinate (o) at (-5mm,-8mm); 
                \node[draw,circle] (l1) at ($(o)+(-10mm,0mm)$) {1};
                \node[draw,circle] (l2) at ($(l1)+(3,0)$) {2};
                \node[draw,circle] (l3) at ($(l1) + (1,0)$) {4};
                \node[draw,circle] (l4) at ($(l1) + (2,0)$) {5};
                \draw[->] (l1) -- (l3) node[midway,above] {a};
                \draw[->] (l3) -- (l4) node[midway,above] {b};
                \draw[->] (l4) -- (l2) node[midway,above] {a};
            }    
    
            \graphbox{\( G' \)}{0mm}{-22mm}{34mm}{22mm}{2mm}{-3mm}{
                \coordinate (o) at (0mm,-3mm); 
                \node[draw,circle] (l1) at ($(o)+(-10mm,0mm)$) {1};
                \node[draw,circle] (l2) at ($(l1)+(2,0)$) {2};
                \node[draw,circle] (l3) at ($(l1) + (1,0)$) {3};
                \node[draw,circle] (l4) at ($(l2) + (0,-1)$) {6};
                \draw[->,red] (l3) -- (l4) node[midway,above] {a};
                \draw[->] (l1) -- (l3) node[midway,above] {a};
                \draw[->] (l3) -- (l2) node[midway,above] {a};
                \draw[->] (l2) -- (l4) node[midway,right] {a};
                \node[draw,circle] (l6) at ($(l1) + (0,-1)$) {7};
                \draw[<-] (l1) -- (l6) node[midway,left] {a};
                \draw[->] (l2) edge[out=-135,in=-45]node[midway,below] {a} (l1) ;
            }    
    
            \graphbox{\( C'  \)}{40mm}{-22mm}{34mm}{22mm}{2mm}{-3mm}{
                \coordinate (o) at (0mm,-3mm); 
                \node[draw,circle] (l1) at ($(o)+(-10mm,0mm)$) {1};
                \node[draw,circle] (l2) at ($(l1)+(2,0)$) {2};
                \node[draw,circle] (l4) at ($(l2) + (0,-1)$) {6};
                \node[draw,circle,dashed,red] (l3) at ($(l1) + (1,0)$) {3};
                % \draw[->,red] (l3) -- (l4) node[midway,above] {a};
                \draw[->] (l3) -- (l4) node[midway,above] {a};
                \draw[->] (l2) -- (l4) node[midway,right] {a};
                \draw[->] (l2) edge[out=-135,in=-45]node[midway,below] {a} (l1) ;
                \node[draw,circle] (l6) at ($(l1) + (0,-1)$) {7};
                \draw[<-] (l1) -- (l6) node[midway,left] {a};
            }    
     
            % \graphbox{\( H' \)}{80mm}{-22mm}{45mm}{22mm}{2mm}{-3mm}{
            %     \coordinate (o) at (-5mm,-3mm); 
            %     \node[draw,circle] (l1) at ($(o)+(-10mm,0mm)$) {1};
            %     \node[draw,circle] (l2) at ($(l1)+(3,0)$) {2};
            %     \node[draw,circle] (l3) at ($(l1) + (1,0)$) {4};
            %     \node[draw,circle] (l4) at ($(l1) + (2,0)$) {5};
            %     \node[draw,circle] (l5) at ($(l2) + (0,-1)$) {6};
            %     \node[draw,circle] (l6) at ($(l1) + (0,-1)$) {7};
            %     \draw[<-] (l1) -- (l6) node[midway,left] {a};
            %     \draw[->] (l1) -- (l3) node[midway,above] {a};
            %     \draw[->] (l3) -- (l4) node[midway,above] {b};
            %     \draw[->] (l4) -- (l2) node[midway,above] {a};
            %     \draw[->] (l2) -- (l5) node[midway,right] {a};
            %     \draw[->] (l2) edge[out=-135,in=-45]node[midway,below] {a} (l1) ;
            %     \node[draw,circle,dashed,red] (l3) at (-0mm,-7mm) {3};
            %     \draw[->] (l3) -- (l5) node[midway,above] {a};
            % }    
    
            \node () at (37mm,-8mm) {\( \leftarrowtail \)}; % K -> L
            \node () at (77mm,-8mm) {\( \rightarrowtail \)}; % K -> R
            \node () at (15mm,-18mm) {\(\downarrowtail \)};
            \node () at (37mm,-33mm) {\( \leftarrowtail \)};
            % \node () at (37mm,-18mm) {PO};
            \node () at (58mm,-18mm) {\( \downarrowtail \)}; 
            % \node () at (80mm,-18mm) {PO};
            % \node () at (102mm,-18mm) {\( \downarrowtail \)};
            % \node () at (77mm,-33mm) {\( \rightarrowtail \)}; % C -> H
        \end{tikzpicture}
        }         
    \end{center}

TODO TODO: second condition intuition + example
