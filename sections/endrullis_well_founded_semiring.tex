\begin{definition}[Monoid]
    \label{def:monoid}
    \todo{to check and add reference}
    A \textbf{monoid} is a triple \(\langle S, \cdot, e \rangle\) where \(S\) is a set, \(\cdot : S \mathop{\times} S \mathop{\to} S\) is a binary operation and \(e \mathop{\in} S\) is constant
    satisfying for all \(a, b, c \mathop{\in} S\):
    \begin{itemize}
        \item \textbf{Associativity}: \((a \cdot b) \cdot c \mathop{=} a \cdot (b \cdot c)\),
        \item \textbf{Identity}: \(a \cdot e \mathop{=} a \mathop{=} e \cdot a\).
    \end{itemize}
    The monoid is \textbf{commutative} if \(a \cdot b \mathop{=} b \cdot a\) for all \(a, b \mathop{\in} S\).
\end{definition} 

\begin{definition}[Semiring]
    \label{def:semiring}
    \todo{to check and to add some references}
    A \textbf{semiring} is a tuple \(\langle S, \mathop{\oplus}, \mathop{\odot}, 0, 1 \rangle\) where:
    \begin{itemize}
        \item  \(\langle S, \mathop{\oplus}, 0 \rangle\) forms a commutative monoid,
        \item  \(\langle S, \mathop{\odot}, 1 \rangle\) forms a monoid,
        \item  $0$ is an annihilator for $\mathop{\odot}$ : For all \(a \mathop{\in} S\),
              \(
                  0 \mathop{\odot} a \mathop{=} 0 \mathop{=} a \mathop{\odot} 0
              \),
        \item $\mathop{\odot}$ distributes over $\mathop{\oplus}$: For all \(a, b, x \mathop{\in} S\),
              \begin{itemize}
                \item $(a \mathop{\oplus} b) \mathop{\odot} x \mathop{=} (a \mathop{\odot} x) \mathop{\oplus} (b \mathop{\odot} x)$
                \item $x \mathop{\odot} (a \mathop{\oplus} b) \mathop{=} (x \mathop{\odot} a) \mathop{\oplus} (x \mathop{\odot} b)$
              \end{itemize}                 
    \end{itemize}
    The semiring is \textbf{commutative} if its multiplicative monoid \(\langle S, \mathop{\odot}, 1 \rangle\) is commutative.
\end{definition}


\begin{definition}[Well-founded semiring]
    \label{def:well_founded_semiring}
    A \textbf{well-founded semiring} is a tuple \(\langle S, \mathop{\oplus}, \mathop{\odot}, 0, 1, \prec, \leq \rangle\) where:
    \begin{itemize}
        \item  \(\langle S, \mathop{\oplus}, \mathop{\odot}, 0, 1 \rangle\) is a semiring,
        \item \textbf{Order Relations}: 
            \begin{itemize}
                \item \(\prec, \leq \mathop{\subseteq} S \mathop{\times} S\) are non-empty orders,
                \item \(\mathop{\prec} \mathop{\subseteq} \leq\) (i.e., \(x \mathop{\prec} y \implies x \leq y\)),
                \item \(\leq\) is reflexive.
            \end{itemize}
    \end{itemize}
    The structure satisfies:
    \begin{itemize}
        \item \(\text{SN}(\mathop{\succ} / \mathop{\geq})\) (strong normalization modulo \(\geq\)),
        \item \(0 \mathop{\neq} 1\),
        \item For all \(x, x', y, y' \mathop{\in} S\):
            \begin{align*}
                x \leq x' \mathop{\land} y \leq y' &\implies x \mathop{\oplus} y \leq x' \mathop{\oplus} y' \tag{S1} \label{ax:S1} \\
                x \mathop{\prec} x' \mathop{\land} y \mathop{\prec} y' &\implies x \mathop{\oplus} y \mathop{\prec} x' \mathop{\oplus} y' \tag{S2} \label{ax:S2} \\
                x \leq x' \mathop{\land} 1 \leq y &\implies x \mathop{\odot} y \leq x' \mathop{\odot} y \tag{S3} \label{ax:S3} \\
                x \mathop{\prec} x' \mathop{\land} 1 \leq y \mathop{\neq} 0 &\implies x \mathop{\odot} y \mathop{\prec} x' \mathop{\odot} y \tag{S4} \label{ax:S4}
            \end{align*}
    \end{itemize}
    The semiring is \textbf{strictly monotonic} if it additionally satisfies:
    \[
        x \mathop{\prec} x' \mathop{\land} y \leq y' \implies x \mathop{\oplus} y \mathop{\prec} x' \mathop{\oplus} y' \tag{S5} \label{ax:S5}
    \]
\end{definition}

\begin{example}[Semiring examples~\text{\cite[Ex. 2.7]{endrullis2024generalized}}]
    \label{ex:semiring_examples}
    \textbf{Arithmetic semiring}: \(\langle \mathbb{N}, +, \cdot, 0, 1, <, \leq \rangle\) is strictly monotonic and well-founded.

    \textbf{Tropical semiring}: \(\langle \mathbb{N} \mathop{\cup} \{\infty\}, \mathop{\min}, +, \infty, 0, <, \leq \rangle\) is well-founded but not strictly monotone.

    \textbf{Arctic semiring}: \(\langle \mathbb{N} \mathop{\cup} \{\mathop{-\infty}\}, \max, +, \mathop{-\infty}, 0, <, \leq \rangle\) is well-founded but not strictly monotone.
\end{example}