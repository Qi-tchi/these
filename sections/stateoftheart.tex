


 

% ======


% The type graph method, which weighs an object by summing the weights of morphisms from the object to a graph, called type graph, was initially introduced by Zantema, K{\"o}nig and Bruggink~\cite{zantema2014termination} for cycle-rewriting systems. 
% This method has since been generalized to edge-labeled multigraphs by Bruggink et al.~\cite{bruggink2014termination} for DPO rewriting with monic matches and injective rules, later extended to DPO rewriting in general by Bruggink et al.~\cite{bruggink2015proving}, and further adapted to more categories and different DPO variants by Endrullis et al.~\cite{endrullis2024generalized_arxiv_v2}. 


% Existing work requires weighted type graphs to have weights over well-founded semirings. This has practical limitations.



% ====================================
% \paragraph{Existing techniques}
% Few techniques for proving (relative) termination of DPO graph rewriting systems exist. Below, we list some of the existing techniques.

% \begin{itemize}
%     \item Overbeek and Endrullis developed a termination technique for PBPO+ graph rewriting systems~\cite{overbeek2024termination_lmcs}. This technique counts the number of occurrences of certain subgraphs before and after a rewriting step. This technique can be applied to PBPO+ graph rewriting systems on many categories including edge-labeled directed multigraphs. It can also be applied to left-injective DPO graph rewriting systems.
%     \item The type graph method, which weighs an object by summing the weights of morphisms from the object to a type graph, was initially introduced by Zantema, K{\"o}nig and Bruggink~\cite{zantema2014termination} for cycle-rewriting systems. 
%     This method has since been generalized to edge-labeled multigraphs by Bruggink et al.~\cite{bruggink2014termination} for DPO rewriting with monic matches and injective rules, later extended to DPO rewriting in general by Bruggink et al.~\cite{bruggink2015proving}, and further adapted to more categories and different DPO variants by Endrullis et al.~\cite{endrullis2024generalized_arxiv_v2}. 
%     \item Plump~\cite{plump1995ontermination} introduced a necessary and sufficient termination condition for left-injective DPO hypergraph rewriting via forward closure, though verifying this condition is undecidable. 
%     \item Plump~\cite{plump2018modular} later proposed a modular critical pair-based strategy for left-injective DPO hypergraph rewriting with monic matches. It allows to deduce termination of the union of two systems from the termination of their sub-systems.
%     % \item Levendovszky et al.~\cite{levendovszky2007termination} propose a termination criterion for DPO rewriting (monic matches, injective rules, negative application condition), though automated verification is hard as explained in~\cite[\textsection 6]{levendovszky2007termination}. 
%     % \item Bottoni et al.~\cite{bottoni2005termination} propose a termination criterion for DPO/SPO rewriting on high-level replacement units. Their method imposes a strongly constrained measuring function and the only concretes measuring function proposed are node-counting and edge-counting.
%     % \item Bottoni et al.~\cite{bottoni2010atermination} present a criterion for termination of DPO rewriting with monic matches, injective rules and negative application conditions, based on the construction of a labeled transition system. 
% \end{itemize}

% \paragraph{Contribution 1: Termination of Graph Rewriting Using Weighted Type Graphs over Non-well-founded Semirings}
 
% Weighted type graph method is a powerful technique for proving relative termination of double-pushout (DPO) graph rewriting systems~\cite{zantema2014termination,bruggink2014termination,bruggink2015proving,endrullis2024generalized_icgt}. Existing work requires weighted type graphs to have weights over well-founded semirings. This has practical limitations.
% %  when applied to edge-labeled directed multigraph rewriting. 
%  Chapter~\ref{chap:nwf} investigates the use of non-well-founded semirings to overcome these limitations. 
% This work resulted in the following publication: 
% \begin{itemize}
%     \item Qi Qiu. Termination of Graph Rewriting using Weighted Type Graphs over Non-well-founded Semirings. 16th International Workshop on Graph Computation Models, Jun 2025, Koblenz, Germany. 2025. ⟨hal-04954960v3⟩
% \end{itemize}
 
% \paragraph{Contribution 2: Termination of Injective DPO Graph Rewriting Systems Using Subgraph Counting} 

% To resolve some cases that prior interpretation-based methods cannot handle, Chapter~\ref{chap:subgraph_counting} presents a new machine-checkable sufficient condition for relative termination of DPO graph rewriting systems with injective rules on edge-labeled multigraphs. 

% It is based on the idea that if the number of graph homomorphisms from some specific graphs to the targetting graph strictly decreases every time a transformation using a rule from a set of rules \( \mathcal{A} \) is applied, and 
% no transformation using a rule from a set of rules \( \mathcal{B} \) increases the number of graph homomorphisms from the specific graphs to the targetting graph, then \( \mathcal{A} \)


% the number of graph homomorphisms from some specific graphs to the targetting graph  decreases every time a transformation using a rule from a set of rules \( \mathcal{A} \) is applied,


% of a specific subgraph in the graph strictly decreases every time a transformation is performed, then the transformation cannot last indefinitely.
% The method defines a graph's weight as the sum of weights of occurrences of a set of graphs within it. Given two rule sets $A$ and $B$, by ensuring
% (1) every rewriting step using rules in $A$ strictly decreases the host graph's weight and
% (2) every rewriting step using rules in $B$ never increases it, we guarantee that rules in $A$ can be applied only finitely many times in any rewriting sequence using rules from the union of $A$ and $B$.
% This work resulted in the following publication:
% \begin{itemize}
%     \item Qiu, Q. (2025). Termination of Injective DPO Graph Rewriting Systems Using Subgraph Counting. In: Endrullis, J., Tichy, M. (eds) Graph Transformation. ICGT 2025. Lecture Notes in Computer Science, vol 15720. Springer, Cham. \url{https://doi.org/10.1007/978-3-031-94706-3_1}
% \end{itemize}

% \paragraph{Contribution 3: Termination of Injective DPO Graph Rewriting Systems Using Subgraph Counting with Antipattern}

% The termination property of a graph rewriting system can sometimes be established by analyzing a decrease in the number of occurrences of specific subgraphs that are not embedded within forbidden contexts. For example, consider the DPO graph rewriting rule in Figure~\ref{fig:intro:graph_transformation_rule_anti_pattern_}.
%  \begin{figure}[H]
%     \centering
% \begin{tikzpicture}
%       \graphbox{$L$}{0mm}{0mm}{34mm}{20mm}{2mm}{-5mm}{
%           \coordinate (o) at (0mm,-3mm); 
%           \node[draw,circle] (l1) at ($(o)+(-10mm,0mm)$) {1};
%           \node[draw,circle] (l2) at ($(l1)+(2,0)$) {2};
%           \node[draw,circle] (l3) at ($(l1)+(1,0)$) {3};
%           \draw[->] (l1) -- (l3) node[midway,above] {$a$};
%           \draw[->] (l3) -- (l2) node[midway,above] {$a$};
%       }     
%       \graphbox{$K$}{40mm}{0mm}{24mm}{20mm}{2mm}{-5mm}{
%           \coordinate (o) at (5mm,-3mm); 
%           \node[draw,circle] (l1) at ($(o)+(-10mm,0mm)$) {1};
%           \node[draw,circle] (l2) at ($(l1)+(1,0)$) {2};
%       }    
%       \graphbox{$R$}{70mm}{0mm}{45mm}{20mm}{2mm}{-5mm}{
%         \coordinate (o) at (0mm,-3mm); 
%         \node[draw,circle] (l1) at ($(o)+(-10mm,0mm)$) {1};
%         \node[draw,circle] (l2) at ($(l1)+(2,0)$) {2};
%         \node[draw,circle] (l3) at ($(l1)+(1,0)$) {3};
%         \draw[->] (l1) -- (l3) node[midway,above] {$a$};
%         \draw[->] (l3) -- (l2) node[midway,above] {$a$};
%         \draw[->] (l3) edge [loop below] node {$c$} (l3);
%       }    

%       \node () at (37mm,-10mm) {$\leftarrowtail$};
%       \node () at (67mm,-10mm) {$\rightarrowtail$};
%   \end{tikzpicture}
%   \caption{}
%   \label{fig:intro:graph_transformation_rule_anti_pattern_}
%  \end{figure} 
% Each application of this rule reduces the number of occurrences of the left-hand side graph $L$ that are not included in any occurrence of the right-hand side graph $ R $. While this variant suggests termination, existing subgraph counting methods (e.g., those in Chapter~\ref{chap:subgraph_counting} and~\cite{overbeek2024termination_lmcs}) cannot exploit such relationships to infer termination. Chapter~\ref{chap:antipattern} extends the technique introduced in Chapter~\ref{chap:subgraph_counting} to handle such cases.

% The extension successfully proves termination for systems like the ones presented in Example~\ref{antipattern:ex:grs_aca} and Example~\ref{antipattern:ex:endrullis:d3:termination} which prior approaches~\cite{zantema2014termination,bruggink2014termination,bruggink2015proving,endrullis2024generalized_arxiv_v2,overbeek2024termination_lmcs} and the subgraph counting method presented in Chapter~\ref{chap:subgraph_counting} fail to handle. 

% \paragraph{Contribution 4: LyonParallel}
% We developed a termination tool for DPO edge-labeled multigraph rewriting that integrates the existing type graph method, our extension, and our subgraph counting technique and its extension.