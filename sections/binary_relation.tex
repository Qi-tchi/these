This section recalls the definition of binary relations and some related concepts on collections of objects.
A collection is not necessarily a set. A well known example is the collection of all sets which is not a set. For simplicity, we use similar notation for collections as for sets.  
\begin{notation}
    Let $C$ and $D$ be collections of objects. 
    we write $c \mathop{\in} C$ to mean that \enquote{$c$ is an element of C}, 
    and $X \mathop{\subseteq} C$ to mean that $X$ is a collection of elements of $C$,
    $C \mathop{\cup} D$ denotes the smallest collection including $C$ and $D$,
    $C \mathop{\cap} D$ denotes the smallest collection contains all elements in $C$ that are also in $D$,
    $C \mathop{\times} D$ denotes the collection of ordered pairs $(c,d)$ with $c \mathop{\in} C$ and $d \mathop{\in} D$.
  \end{notation} 
  % \begin{definition}[Binary relation]
  %   \label{def:binary_relation:binary_relation}
  %   Let $D$ be a collection of objects. A mathematical structure \( (D, \mathcal{R}) \) where $\mathcal{R}$ is a collection of objects from $D \mathop{\times} D$ is called a \textbf{binary relation} on $D$. 
    
  %   For an object in $(x,y) \mathop{\in} \mathcal{R}$ with first element $x \mathop{\in} D$ and second element $y\in D$, we write $x \mathcal{R} y$, $\mathcal{R}(x,y)$ or $(x,y) \mathop{\in} \mathcal{R}$. 
    
  %   When $D$ is irrelevant, we say simply that $\mathcal{R}$ is a binary relation.
  % \end{definition}
  % Binary relations on collections of objects instead of on sets.
\begin{definition}
    \label{def:binary_relation:binary_relation}
    Let $S$ be a collection of objects. A \textbf{binary relation} on $S$ is a mathematical structure \((S,\to)\) with \( \mathop{\to} \mathop{\subseteq} S\times S\). For \(x,y\in S\) we write \(x \mathop{\to} y\) to mean \((x,y)\) is an element of \(\to\). When the underlying collection \(S\) is clear from context or irrelevant to the discussion, we simply say that \(\to\) is a binary relation.
  \end{definition}
  Examples of binary relations are the empty set, the set $\set{(2,4),(1,3),(3,5)}$, \((\mathbb{N}, \mathop{\geq})\), \((\mathbb{N}, >)\), and \((\mathbb{N}, =)\).
  \begin{definition}
    \label{def:binary_relation:reflexivity_transitivity}
    Let \(\to\) be a binary relation on a collection \(S\). It is said to be
    \begin{itemize}
      \item \textbf{reflexive} if $a \mathop{\to} a$ for all $a \mathop{\in} S$;
      \item \textbf{irreflexive} if for all $a \mathop{\in} S$, $a \mathop{\to} a$ does not hold;
      \item \textbf{symmetric} if for all $a,b \mathop{\in} S$, $a \mathop{\to} b$ implies $b \mathop{\to} a$;
      \item \textbf{transitive} if for all $a,b,c \mathop{\in} S$, $a \mathop{\to} b$ and $b \mathop{\to} c$ implies $a \mathop{\to} c$.
    \end{itemize}
  \end{definition}
Some examples of binary relations with these properties are as follows.
\begin{itemize}
  \item The binary relation $\set{(1,1),(3,3),(5,5),(1,3),(3,5)}$ is reflexive. It is not transitive because $(1,5)$ is not in the relation while $(1,3)$ and $(3,5)$ are in the relation. It is not symmetric because $(3,1)$ is not in the relation while $(1,3)$ is in the relation.
  \item The binary relation $\set{(1,3),(3,1)}$ is irreflexive and symmetric. It is neither reflexive nor transitive, because $(1,1)$ is not in the relation.
  \item The binary relation $\set{(1,3),(3,5),(1,5)}$ is transitive.
   It is not reflexive, because $(3,3)$ is not in the relation. It is not symmetric, because $(3,1)$ is not in the relation while $(1,3)$ is in the relation.
  \item The usual equality relation \(=\) on \(\mathbb{N}\) is reflexive, symmetric, and transitive.
  \item The carnonical order \(>\) on \(\mathbb{N}\) is neither reflexive nor symmetric, but it is irreflexive and transitive.
\end{itemize}

\todo{define order}

\begin{definition} 
    \label{def:binary_relation:closure}
    % The \textbf{transitive closure} of $\to$, denoted $\to^+$, is the smallest transitive relation that includes \( \mathop{\to} \).
    % The \textbf{symmetric closure} of $\to$, denoted $\leftrightarrow$, is the smallest symmetric relation that includes \( \mathop{\to} \).
    % The \textbf{reflexive transitive closure} of $\to$, denoted $\to^*$, is the smallest reflexive and transitive relation that includes \( \mathop{\to} \).
    An equivalence relation is a binary relation that is \textbf{reflexive, symmetric, and transitive}. 
    Let \( \mathop{\to} \) be a binary relation. The \textbf{equivalence relation generated by \(\to\)} is the smallest equivalence relation that includes \(\to\).
\end{definition} 
% For example, 
% the relation \(\geq\) on \(\mathbb{N}\) is the reflexive transitive closure of the binary relation \(\set{(n+1, n) \mid n \mathop{\in} \mathbb{N}}\).
For example, consider the binary relation $\set{(2,4),(1,3),(3,5)}$. The equivalence relation $\sim$ generated by this relation is 
\begin{flalign*}
    \set{&(1,1),(2,2),(3,3),(4,4),(5,5),(1,3),(3,1),\\
         &(2,4),(4,2),(3,5),(5,3),(1,5),(5,1)}.
\end{flalign*}
Because every equivalence relation generated by $\sim$, it includes
\begin{itemize}
  \item $\set{(2,4),(1,3),(3,5)}$;
  \item $\set{(1,1),(2,2),(3,3),(4,4),(5,5)}$, by reflexivity;
  \item $\set{(1,5)}$, by transitivity;
  \item $\set{(4,2),(3,1),(5,3),(5,1)}$, by symmetry.
\end{itemize} 
Given an equivalence relation, one may naturally consider the induced partition of the underlying collection into equivalence classes.
\begin{definition}
   Let \(S\) be a collection of objects and \(\sim\) be an equivalence relation on \(S\). For an element \(x\) in \(S\), the \textbf{equivalence class} of \(x\) in \(S\) with respect to \(\sim\) is the collection that consists of all elements \(y \mathop{\in} S\) such that \(y \sim x\), and is denoted by \([x]_{\sim}\).
      We may simply write \([x]\) instead of \([x]_{\sim}\) when the equivalence relation is clear from context.
   The collection of all equivalence classes in \(S\) with respect to \(\sim\) is called the \textbf{quotient collection}, and is  
    denoted by
   \(S/{\sim}\).

\end{definition}
For example, consider the set $S \isdef \set{1,2,3,4}$ and the binary relation $\sim \isdef \set{(2,4),(1,3)}$. The equivalence class of $[1]$ and $[3]$ are $\set{1,3}$, because $(1,3) \mathop{\in} S$. 
The equivalence class shared by $[2]$ and $[4]$ are $\set{2,4}$, because $(2,4) \mathop{\in} S$.
 Finally, the quotient set $S/{\sim}$ is $\set{[1],[2]}$.

\begin{definition}
    \label{def:binary_relation:chain} 
    Let \(\to\) be a binary relation. 
    A \textbf{\(\to\)-chain} is either
      \begin{itemize}
          \item a finite sequence \((s_i)_{i=0}^n\) with \(n \mathop{\in} \mathbb{N}\) such that \(s_i \mathop{\to} s_{i+1}\) for every \(0 \leq i \leq n-1\); it will be denoted by \(s_0 \mathop{\to} s_1 \mathop{\to} \cdots \mathop{\to} s_n\); or
          \item an infinite sequence \((s_n)_{n \mathop{\in} \mathbb{N}}\) such that \(s_n \mathop{\to} s_{n+1}\) for every \(n \mathop{\in} \mathbb{N}\); it will be denoted by \(s_0 \mathop{\to} s_1 \mathop{\to} \cdots\).
      \end{itemize}
      In a \(\to\)-chain \((s_i)\), each adjacent pair \(s_i \mathop{\to} s_{i+1}\) is a \textbf{\(\to\)-step}.
      A finite \(\to\)-chain \((s_i)_{i=0}^n\) is denoted by \(s_0 \mathop{\to} s_1 \mathop{\to} s_2 \mathop{\to} \cdots \mathop{\to} s_n \). 
      A finite \(\to\)-chain \((s_n)_{n \mathop{\in} \mathbb{N}}\) is denoted by \(s_0 \mathop{\to} s_1 \mathop{\to} s_2 \mathop{\to} \cdots \).

\end{definition}
An example of an infinite \(\geq\)-chain is the sequence \(3 \mathop{\geq} 2 \mathop{\geq} 1 \mathop{\geq} 1 \cdots\). An example of a finite \(>\)-chain is the sequence \(2 \mathop{>} 1 \mathop{>} 0\).
\begin{definition} 
    \label{def:binary_relation:well_founded}
    A binary relation $\to$ is said to be \textbf{well-founded} if there is no infinite $\to$-chain. 
\end{definition}
\todo{todo: $\to$ or $\leftarrow$?}
An example of a well-founded relation is the usual order
\(>\) on \(\mathbb{N}\). An example of a non well-founded relation is \(\geq\) on \(\mathbb{N}\), because there is an infinite \(\geq\)-chain \(3 \mathop{\geq} 2 \mathop{\geq} 1 \mathop{\geq} 1 \cdots\).