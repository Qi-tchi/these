
\begin{itemize}
    \item[bruggink2014]Although all theorems have been stated and proved for (binary) multigraphs, a generalization to hypergraphs would be trivial. On the other hand, transferring the results to other graph transformation formalisms is harder. For example, in the single pushout approach, the graph transformation system corresponding to the one of Ex. 1 is non-terminating, so the result of Ex. 2 (in which it is proved that this system is terminating) shows that the weighted type graph technique cannot be transferred one-to-one to single pushout graph transformation. It is left as future research to find similar arguments for the single pushout approach and other formalisms.
    
    Another direction for further research is to allow for graph transformation systems with negative application conditions or more general application conditions. Note, however, that the implicit negative application condition of double pushout graph transformation, the dangling edge condition, can in some cases already be handled (see Ex. 2). 
    
    Finally, for interesting real-world applications, it would be interesting to generalize the technique to more expressive methods of specifying the initial graph languages, so that we can, for example, restrict to trees or rings of arbitrary size (both graph languages cannot be expressed by a type graph). 
\end{itemize}

\begin{idea}
    example majority A or B\\
    implementation of the algorithms proposed by plump\\
    propose a reinforced version of the algorithms\\
\end{idea}
