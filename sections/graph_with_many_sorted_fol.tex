\begin{definition}[Many sorted (finite) first order language]
    A \textbf{finite many sorted first order language} 
    \( \mathcal{L} \mathop{=} (s_1,\dots,s_m,f_1,\dots, f_n, r_1, \dots, r_w ) \)
    consists of
    \begin{itemize}
        \item sorts $s_1,\dots,s_m$ where $m \mathop{\in} \mathbb{N}$ and $m \mathop{\geq} 1$;
        \item function symbols $f_1,\dots, f_n$ where $n\in \mathbb{N}$. Each function symbol has a signature $s_{i_1} \mathop{\times} s_{i_2} \mathop{\times} \dots \mathop{\times} s_{i_k} \mathop{\to} s$ indicating its $j$-th argument has sort $s_{i_j} \mathop{\in} \mathcal{S}$ and its result has sort $s \mathop{\in} \mathcal{S}$ where $m \mathop{\in} \mathbb{N}$;
        \item relation symbols $r_1,\dots, r_w$ where $ w\in \mathbb{N}$. Each  relation symbol has a signature $s_{i_1} \mathop{\times} s_{i_2} \mathop{\times} \dots \mathop{\times} s_{i_k}$,where $k \mathop{\in} \mathbb{N}$, indicating its $j$-th argument has sort $s_{i_j} \mathop{\in} \mathcal{S}$.
    \end{itemize}  
    For each sort $s \mathop{\in} \mathcal{S}$, there is a countably infinite number of variables $v_1^s, v_2^s, \dots$. The logical symbols are the usual ones: $\neg$, $\vee$, $\wedge$, $\forall$, $\exists$, $=$.
\end{definition}
The above definition is adapted from \cite[Def.~29.26]{monk2012mathematical} by restricting to a finite number of function and relation symbols, a simplification that suffices for our purposes.

\begin{definition}[Structure of a Many Sorted Language]\ \\
    Let \( \mathcal{L} \mathop{=} (s_1, \dots, s_m, f_1, \dots, f_n, r_1, \dots, r_w) \) be a many sorted finite first order language. An \( \mathcal{L} \)-structure $\mathcal{M}$ is a tuple \( \mathfrak{A} \mathop{=} (s_1^\mathcal{M}, \dots, s_m^\mathcal{M}, f_1^\mathcal{M}, \dots, f_n^\mathcal{M}, r_1^\mathcal{M}, \dots, r_w^\mathcal{M}) \) such that:
    
    \begin{enumerate}
        \item[(i)] for $1 \leq i \leq m$, $s_i^\mathcal{M}$ is a nonempty set;
        
        \item[(ii)] For each function symbol \( f_j \), \( 1 \leq j \leq n \), with signature \( s_{i_1} \mathop{\times} \dots \mathop{\times} s_{i_k} \mathop{\to} s_{i_{k+1}} \), $1 \leq j \leq n$, \( f_j^\mathcal{M} \) is a function from \( s_{i_1}^\mathcal{M} \mathop{\times} \dots \mathop{\times} s_{i_k}^\mathcal{M} \) to \( s_{i_{k+1}}^\mathcal{M} \);
        
        \item[(iii)] For each relation symbol \( r_j \) with signature \( s_{i_1} \mathop{\times} \dots \mathop{\times} s_{i_k} \), \( R_{r_j} \) is a subset of \( s_{i_1}^\mathcal{M} \mathop{\times} \dots \mathop{\times} s_{i_k}^\mathcal{M} \).
    \end{enumerate}
\end{definition}

\begin{definition}[Homomorphism of Many-Sorted Structures]\ \\
    Let \( \mathcal{L} \mathop{=} (s_1, \dots, s_m, f_1, \dots, f_n, r_1, \dots, r_w) \) be a many-sorted finite first-order language.  
    Let \( \mathfrak{A} \mathop{=} (s_1^\mathcal{A}, \dots, s_m^\mathcal{A}, f_1^\mathcal{A}, \dots, f_n^\mathcal{A}, r_1^\mathcal{A}, \dots, r_w^\mathcal{A}) \) and \( \mathfrak{B} \mathop{=}  (s_1^\mathcal{B}, \dots, s_m^\mathcal{B}, f_1^\mathcal{B}, \dots, f_n^\mathcal{B}, r_1^\mathcal{B}, \dots, r_w^\mathcal{B}) \) be two \( \mathcal{L} \)-structures, where:

A homomorphism \( h: \mathfrak{A} \mathop{\to} \mathfrak{B} \) of $\mathcal{L}$-structures is a family of functions \( \{h_{s_i} : s_i^\mathcal{A} \mathop{\to} s_i^\mathcal{A}\}_{i=1}^m \) satisfying the following conditions:
\begin{itemize}
    \item For every function symbol \( f_j \) with signature \( s_{i_1} \mathop{\times} \dots \mathop{\times} s_{i_k} \mathop{\to} s \), and for all \( a_1 \mathop{\in} s_{i_1}^\mathcal{A}, \dots, a_k \mathop{\in} s_{i_k}^\mathcal{A} \):  
    \[
    h_s\left(f_j^\mathfrak{A}(a_1, \dots, a_k)\right) \mathop{=} f_j^\mathfrak{B}\left(h_{s_{i_1}}(a_1), \dots, h_{s_{i_k}}(a_k)\right).
    \]
    \item For every relation symbol \( r_j \) with signature \( s_{i_1} \mathop{\times} \dots \mathop{\times} s_{i_k} \), and for all \( a_1 \mathop{\in} s_{i_1}^\mathcal{A}, \dots, a_k \mathop{\in} s_{i_k}^\mathcal{A} \):  
    \[
    (a_1, \dots, a_k) \mathop{\in} R_j^\mathfrak{A} \implies \left(h_{s_{i_1}}(a_1),  \dots, h_{s_{i_k}}(a_k)\right) \mathop{\in} R_j^\mathfrak{B}.
    \]
\end{itemize}
\end{definition}



\begin{definition}[Language of unlabeled directed multigraphs]
    \label{def:l_ulg}
    The many sorted first order \textbf{language of unlabeled directed multigraphs} is $\mathcal{L}_\text{ulg} \mathop{=} (N,E,s,t)$ where $N$ is the sort of nodes, $E$ is the sort of edges, $s$ and $t$ are unary function symbols.
  \end{definition}
\begin{definition}[Unlabeled directed multigraphs]
    \label{def:ulg}
    An unlabeled graph is a $\mathcal{L}_\text{ulg}$-structure. The interpretation of the symbols $s$ and $t$ will be called the domain function and the codomain function, respectively.
\end{definition}
An \textbf{unlabeled graph} \( G \) consists of a collection of \textbf{nodes} (also called \textbf{objects}) and a collection of \textbf{edges} equipped with a \textbf{source} (or \textbf{domain}) node and a \textbf{target} (or \textbf{codomain}) node. 
For an unlabeled graph \( G \), we denote by \( G_0 \) its collection of nodes, \( G_1 \) its collection of edges, \( \operatorname{dom}:G_1{\to}G_0 \) the domain function, and \( \operatorname{cod}:G_1{\to}G_0 \) the codomain function. An unlabeled graph is \textbf{finite} if \( G_0 \) and \( G_1 \) are both finite sets.
We write \( a: s \mathop{\to} t \) to indicate that \( a \) is a directed edge from \( s \) to \( t \).
\begin{definition}[Homomorphisms of unlabeled directed multigraphs]
    \label{def:homomorphism_ulg}
    A homomorphism of unlabeled directed multigraphs is a homomorphism of $\mathcal{L}_{ulg}$-structures.
\end{definition}
Let \( G \) and \( H \) be unlabeled graphs. A \textbf{homomorphism of unlabeled graphs} \( h: G \mathop{\to} H \) is a pair of functions \( h_0: G_0 \mathop{\to} H_0 \) and \( h_1: G_1 \mathop{\to} H_1 \) such that for every edge \( a: s \mathop{\to} t \) in \( G \), we have \( h_1(a) : h_0(s) \mathop{\to} h_0(t) \) in \( H \).
\begin{definition}[Language of edge-labeled directed multigraphs]
    \label{def:l_lg}
    The many sorted first order \textbf{language of edge-labeled directed multigraphs} is $\mathcal{L}_\text{ulg} \mathop{=} (N,E,\Sigma, s,t,l)$ where $N$ is the sort of nodes, $E$ is the sort of edges, $s:E \mathop{\to} N$, $t: E \mathop{\to} N$, $l:E \mathop{\to} \Sigma$ function symbols.
  \end{definition}
\begin{definition}[Edge-labeled directed multigraphs]
    \label{def:lg}
    A labeled graph is a $\mathcal{L}_\text{lg}$-structure. The interpretation of the symbol $l$ will be called the labeling function.
\end{definition}
By $a : s\overset{l}{\rightarrow} t$, we denote the arrow $a$ labeled by $l$ from $s$ to $t$. Unless otherwise specified, the term 'graph' will refer to edge-labeled multigraph.
\begin{definition}[Homomorphisms of graphs]
    \label{def:homomorphism_lg}
    A homomorphism of graphs is a homomorphism of $\mathcal{L}_{lg}$-structures.
\end{definition}
% \begin{definition}[Labeled graph \cite{konig2018tutorial}]
    \label{def:graph}
    Let \(\Sigma\) be a finite set of labels. A \textbf{labeled graph} is an ordered pair \((G,\lambda)\) where \( G \) is an unlabeled graph and \( \lambda : G_1 \rightarrow \Sigma\) is an edge-labeling function. 
    It is called \textbf{finite} if its underlying unlabeled graph is finite.  
\end{definition}
By $a : s\overset{l}{\rightarrow} t$, we denote the arrow $a$ labeled by $l$ from $s$ to $t$. Unless otherwise specified, the term 'graph' will refer to labeled graphs.
% A homomorphism of labeled graphs is a homomorphism of unlabeled graphs that preserves the labels assigned to the edges.
\begin{definition}[Homomorphism of labeled graphs~\cite{konig2018tutorial}]
    \label{def:graph:homomorphism}
    Let \( (G,\lambda) \) and \( (H,\lambda') \) be labeled graphs. A \textbf{homomorphism of labeled graphs} \( h:(G,\lambda) \mathop{\rightarrow} (H,\lambda') \) is a homomorphism of unlabeled graphs such that for each edge \( a \) in \( G \), we have \( \lambda \mathop{=} \lambda' \circ h_1 \).
\end{definition}
% \begin{notation}
    We use the notation from~\cite[Example 9]{overbeek2023pbpo_JLAMP} to visualize edge-labeled graph homomorphisms. Labeled graphs are enclosed in boxes with their names displayed in the top-left corner. Nodes and edges are assigned subsets of \(\mathbb{N}\) as identifiers, and these identifiers are chosen such that: (i) Each node or edge \( y \) in the codomain graph is assigned the union of the identifiers of all nodes or edges in the domain graph that are mapped to \( y \); (ii) The graph homomorphism is uniquely determined by this assignment.
    
    \noindent To further improve readability, we represent sets by listing their elements. Additionally, we omit identifiers when doing so does not cause confusion. This is illustrated in the following representation of a homomorphism \( h: G \mathop{\to} H \).
    
    \begin{center}
        \resizebox{0.45\textwidth}{!}{
        \begin{tikzpicture}
            \graphbox{\( G \)}{00mm}{-20mm}{45mm}{20mm}{2mm}{-5mm}{
                \coordinate (o) at (-5mm,-8mm); 
                \node[draw,circle] (l1) at ($(o)+(-10mm,0mm)$) {1};
                \node[draw,circle] (l2) at ($(l1)+(3,0)$) {2};
                \node[draw,circle] (l3) at ($(l1)+(1,0)$) {3};
                \node[draw,circle] (l4) at ($(l1)+(2,0)$) {4};
                \draw[->] (l1) -- (l3) node[midway,above] {$a$};
                \draw[->] (l3) -- (l4) node[midway,above] {$b$};
                \draw[->] (l4) -- (l2) node[midway,above] {$a$};
            }  
            \graphbox{\( H \)}{50mm}{-20mm}{34mm}{20mm}{2mm}{-5mm}{
                \coordinate (o) at (0mm,-8mm); 
                \node[draw,circle] (l1) at ($(o)+(-10mm,0mm)$) {1};
                \node[draw,circle] (l2) at ($(l1)+(2,0)$) {2};
                \node[draw,circle] (l3) at ($(l1)+(1,0)$) {3\ 4};
                \draw[->] (l1) -- (l3) node[midway,above] {$a$};
                \draw[->] (l3) edge[loop above] (l3) node[midway,above] {$b$};
                \draw[->] (l3) -- (l2) node[midway,above] {$a$};
            }      
            % \node () at (53mm,-30mm) {$\rightarrow$};
        \end{tikzpicture}
    }
    \end{center} 
    In this example, the sets \(\{1\}\), \(\{2\}\), \(\{3\}\), \(\{4\}\), and \(\{3,4\}\) are represented as \(1\), \(2\), \(3\), \(4\), and \(3\ 4\), respectively. Edge identifiers are omitted.
\end{notation} 
    
% \begin{remark}
%     Any labeled graph can be transformed into a graph by adding a self-loop labeled by $\alpha$ to each node, where $\alpha$ is a label not in $\Sigma$. For clarity in visualizations, these self-loops will be omitted.
% \end{remark}
