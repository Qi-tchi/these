The existence of a context closure guarantees the existence of morphisms from any object subject to rewriting to the type graph—a critical requirement as emphasized in \hyperref[remark:semiring_0_unpredictable]{\textsection~\ref{remark:semiring_0_unpredictable}}. Further details can be found in~\cite[\textsection 5.3]{endrullis2024generalized}.
\begin{definition}[Context Closure~\text{\cite[\textdef~5.3]{endrullis2024generalized}}]
    \label{def:context_closure} 
    \ \newline 
\begin{minipage}{0.65\textwidth}
    Let $\mathcal{T} \mathop{=} (T,\mathbb{E}, S, w)$ be a type graph, \(\rho \mathop{=} (L \overset{l}{\leftarrow} K \overset{r}{\rightarrow} R ) \) a DPO rewriting rule and $\mathfrak{F}$ a rewriting framework. 
    A \textbf{context closure} for $\rho$ and $\mathcal{T}$ in $\mathfrak{F}$ is a morphism $c:L \mathop{\rightarrow} T$ such that for every DPO diagram in $\mathfrak{F}(\rho)$ (shown on the right) 
    there exists $\alpha : G \mathop{\rightarrow} T$ such that $m \mathop{\star} \alpha \mathop{=} c$.
\end{minipage}
\begin{minipage}{0.35\textwidth}
    \begin{center}
        \begin{tikzpicture}[rotate=90]
          \node (I) {$K$}; 
          \node (L) [left of=I] {$L$};
          \node (R) [right of=I] {$R$};
          \node (G) [below of=L] {$G$};
          \node (C) [below of=I] {$C$};
          \node (H) [below of=R] {$H$};
          \node (T) [left=of $(L)!0.5!(G)$] {$T$};
          \draw [->] (L) to  node [label, above] {$c$}  (T);
          \draw [->] (G) to  node [label, below] {$\alpha$} (T);
          \draw [->] (I) to node [label, above] {$l$} (L);
          \draw [->] (I) to node [label,above] {$r$} (R);
          \draw [->] (L) to node [label, right] {$m$} (G);
          \draw [->] (I) to (C);
          \draw [->] (R) to (H);
          \draw [->] (C) to (G);
          \draw [->] (C) to (H);
        \end{tikzpicture}
      \end{center}
\end{minipage}
\end{definition}