\begin{example}
    \label{ex:po_in_graph}
    The following diagram
    %  in Figure~\ref{fig:ex:po_in_graph} 

% \begin{figure}[H] 
\begin{center}
    \resizebox{0.5\textwidth}{!}{
        \begin{tikzpicture}
               
                \graphbox{ }{40mm}{5mm}{34mm}{15mm}{2mm}{0mm}{
                    \coordinate (o) at (0mm,-8mm); 
                    \node[draw,circle] (l1) at ($(o)+(-10mm,0mm)$) {1};
                    \node[draw,circle] (l2) at ($(l1)+(2,0)$) {2};
                }  
  
                \graphbox{ }{80mm}{5mm}{45mm}{15mm}{2mm}{-0mm}{
                    \coordinate (o) at (-5mm,-8mm); 
                    \node[draw,circle] (l1) at ($(o)+(-10mm,0mm)$) {1};
                    \node[draw,circle] (l2) at ($(l1)+(3,0)$) {2};
                    \node[red,draw,circle] (l3) at ($(l1)+(1,0)$) {4};
                    \node[red,draw,circle] (l4) at ($(l1)+(2,0)$) {5};
                    \draw[red] (l1) -- (l3) node[midway,above] {$a$};
                    \draw[red] (l3) -- (l4) node[midway,above] {$b$};
                    \draw[red] (l4) -- (l2) node[midway,above] {$a$};
                }    
                \graphbox{ }{40mm}{-17mm}{34mm}{25mm}{2mm}{-5mm}{
                    \coordinate (o) at (0mm,-3mm); 
                    \node[draw,circle] (l1) at ($(o)+(-10mm,0mm)$) {1};
                    \node[draw,circle] (l2) at ($(l1)+(2,0)$) {2};
                    \node[blue,draw,circle] (l4) at ($(l2)+(0,-1)$) {6};
                    \draw[blue] (l2) -- (l4) node[midway,right] {$a$};
                    \node[blue,draw,circle] (l6) at ($(l1)+(0,-1)$) {7};
                    \draw[blue] (l1) -- (l6) node[midway,left] {$a$};
                }    
  
                \graphbox{ }{80mm}{-17mm}{45mm}{25mm}{2mm}{-5mm}{
                    \coordinate (o) at (-5mm,-3mm); 
                    \node[draw,circle] (l1) at ($(o)+(-10mm,0mm)$) {1};
                    \node[draw,circle] (l2) at ($(l1)+(3,0)$) {2};
                    \node[draw,circle,red] (l3) at ($(l1)+(1,0)$) {4};
                    \node[draw,circle,red] (l4) at ($(l1)+(2,0)$) {5};
                    \node[blue,draw,circle] (l5) at ($(l2)+(0,-1)$) {6};
                    \node[blue,draw,circle] (l6) at ($(l1)+(0,-1)$) {7};
                    \draw[blue] (l1) -- (l6) node[midway,left] {$a$};
                    \draw[red] (l1) -- (l3) node[midway,above] {$a$};
                    \draw[red] (l3) -- (l4) node[midway,above] {$b$};
                    \draw[red] (l4) -- (l2) node[midway,above] {$a$};
                    \draw[blue] (l2) -- (l5) node[midway,right] {$a$};
                }    
  
                \node () at (77mm,-3mm) {\( \rightarrowtail \)}; % K -> R
                \node () at (52mm,-13mm) {\( \downarrowtail \)};
                \node () at (92mm,-13mm) {\( \downarrowtail \)};
                \node () at (77mm,-28mm) {\( \rightarrowtail \)}; % C -> H
        \end{tikzpicture}
    }
\end{center}
% \caption{Example of the pushout of span of monomorphism in \(\mathbf{Graph}\)}
% \label{fig:ex:po_in_graph}
% \end{figure}
     is a pushout square in the category \textbf{Graph} where nodes and arrows are colored to visualize the decomposition as in Figure~\ref{fig:po_decomp}.
\end{example}

To demonstrate that the diagram above is a pushout square, consider an object \( X \) and morphisms \( f: A \uplus B' \mathop{\to} X \) and \( g: A \uplus C' \mathop{\to} X \) such that \( \alpha \mathop{\star} f \mathop{=} \beta \mathop{\star} g \). Since \( \alpha \) and \( \beta \) are inclusion maps, it follows that \( f \) and \( g \) agree on \( A \). Therefore, the morphism \( h: A \uplus B' \uplus C' \mathop{\to} X \) defined by $h(x) \mathop{=} f(x)$ if $x\in A\uplus B'$ and $h(x) \mathop{=} g(x)$ otherwise is the unique morphism such that $\beta' \mathop{\star} h \mathop{=} f$ and $\alpha' \mathop{\star} h \mathop{=} g$. Thus, the object \( A \uplus B' \uplus C' \), together with the morphisms \( \beta' \) and \( \alpha' \), satisfies the universal property of the pushout. Furthermore, the compositions \( \alpha \mathop{\star} \beta'\) and \( \beta \mathop{\star} \alpha'\) both map elements from \( A \) to \( A \uplus B' \uplus C' \) via inclusion function.
Since all maps are inclusions and \( A \), \( B' \), and \( C' \) are disjoint, the square commutes $
     \alpha \mathop{\star} \beta' \mathop{=} \beta \mathop{\star} \alpha'
$. Therefore, the diagram is indeed a pushout square.