\begin{definition}[Monicity \cite{barr1990category} and Relative monicity \cite{endrullis2024generalized}]
    A morphism $f : A \mathop{\to} B$ is called 
    \begin{enumerate}[label=(\roman*)] 
        \item 
            \emph{monic for $S \mathop{\subseteq} \operatorname{Hom}(-,A)$} 
            if $g \mathop{\star} \mathop{=} h \mathop{\star} f \implies g \mathop{=} h$ for all $g,h \mathop{\in} S$,
        \item \textbf{monic} if it is monic for $\homset{-}{A}$. A monic morphism is also called a \textbf{monomorphism}.
        \item 
            \emph{$X$-monic}, where $X \mathop{\in} \operatorname{ob}(\mathcal{C})$, if $f$ is monic for $Hom(X,A)$, and
        \item 
            \emph{$X$-monic outside of $u$}, where $X \mathop{\in} \operatorname{ob}(\mathcal{C})$ and $u : C \mathop{\to} A$, if $f$ is monic for $Hom(X,A) -  Hom(X,C) \mathop{\star} u$.
    \end{enumerate}
    For $\Gamma \mathop{\subseteq} \operatorname{ob}(\mathcal{C})$, $f$ is called \emph{$\Gamma$-monic} if $f$ is $X$-monic for every $X \mathop{\in} \Gamma$.
\end{definition}