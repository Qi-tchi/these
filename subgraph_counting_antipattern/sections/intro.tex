Chapter~\ref{chap:antipattern} extends the subgraph counting method presented in Chapter~\ref{chap:subgraph_counting}. 
Consider the DPO rewriting system on edge-labeld directed multi-graphs shown
below.
%  \begin{figure}[H]
%     \centering
\begin{center}
    \begin{tikzpicture}
      \graphbox{$L$}{0mm}{0mm}{34mm}{20mm}{2mm}{-5mm}{
          \coordinate (o) at (0mm,-3mm); 
          \node[draw,circle] (l1) at ($(o)+(-10mm,0mm)$) {1};
          \node[draw,circle] (l2) at ($(l1)+(2,0)$) {2};
          \node[draw,circle] (l3) at ($(l1) + (1,0)$) {3};
          \draw[->] (l1) -- (l3) node[midway,above] {$a$};
          \draw[->] (l3) -- (l2) node[midway,above] {$a$};
      }     
      \graphbox{$K$}{40mm}{0mm}{24mm}{20mm}{2mm}{-5mm}{
          \coordinate (o) at (5mm,-3mm); 
          \node[draw,circle] (l1) at ($(o)+(-10mm,0mm)$) {1};
          \node[draw,circle] (l2) at ($(l1)+(1,0)$) {2};
      }    
      \graphbox{$R$}{70mm}{0mm}{45mm}{20mm}{2mm}{-5mm}{
        \coordinate (o) at (0mm,-3mm); 
        \node[draw,circle] (l1) at ($(o)+(-10mm,0mm)$) {1};
        \node[draw,circle] (l2) at ($(l1)+(2,0)$) {2};
        \node[draw,circle] (l3) at ($(l1) + (1,0)$) {3};
        \draw[->] (l1) -- (l3) node[midway,above] {$a$};
        \draw[->] (l3) -- (l2) node[midway,above] {$a$};
        \draw[->] (l3) edge [loop below] node {$c$} (l3);
      }    

      \node () at (37mm,-10mm) {$\leftarrowtail$};
      \node () at (67mm,-10mm) {$\rightarrowtail$};
  \end{tikzpicture}
%   \caption{}
%   \label{fig:intro:graph_transformation_rule_anti_pattern_}
%  \end{figure} 
\end{center}
%  in Figure~\ref{fig:intro:graph_transformation_rule_anti_pattern_}. 
In this system, the number of injective graph homomorphisms from every pattern graph increases with each application of the rewriting rule. Therefore, the method introduced in Chapter~\ref{chap:subgraph_counting} cannot be applied. However, the number of injective graph homomorphisms from the graph
\raisebox{2pt}{
            \scalebox{0.7}{\tikz[baseline=-0.5ex]{
            \node [draw,circle] (z) at (-1,0) {};
            \node [draw,circle] (x) at (0,0) {};
            \node[draw,circle] (y) at (1,0) {};
            \draw[->] (z)--(x) node[midway, above] {$a$};
            \draw[->] (x)--(y) node[midway, above] {$a$};
        }}} 
    whose images are not included in occurrences of the graph
        \raisebox{2pt}{
            \scalebox{0.7}{\tikz[baseline=-0.5ex]{
            \node [draw,circle] (z) at (-1,0) {};
            \node [draw,circle] (x) at (0,0) {};
            \node[draw,circle] (y) at (1,0) {};
            \draw[->] (z)--(x) node[midway, above] {$a$};
            \draw[->] (x)--(y) node[midway, above] {$a$};
            \draw[->] (x) edge [loop below] node {$c$} (l3);
        }}}
decreases with each application of the rewriting rule.

This extension can handle systems (e.g. Example~\ref{antipattern:ex:grs_aca} and Example~\ref{antipattern:ex:endrullis:d3:termination}) which cannot be handled by the
prior interpretation-based approaches~\cite{zantema2014termination,bruggink2014termination,bruggink2015proving,endrullis2024generalized_arxiv_v2,overbeek2024termination_lmcs} and the morphism counting method presented in Chapter~\ref{chap:subgraph_counting}.
An implementation of this extension is also provided in our tool~\textbf{LyonParallel}.
  
We are going to proceed as follows:~\textsection~\ref{antipattern:sec:termination_criterion} presents the extension.~\textsection~\ref{antipattern:sec:examples} provides some examples,~\textsection~\ref{antipattern:sec:conclusion} concludes. Missing proofs and definitions are in the appendix of this chapter.