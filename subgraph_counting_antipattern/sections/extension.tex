
The technique presented in the preceding section can be extended to non-injective matches. \todo{some intuition and an example for the notion of a Quotient Rule would be good.}

\begin{definition}[Quotient Rule~\cite{habel2001double}] 
    \ \newline 
    \noindent
    \begin{minipage}{0.7\textwidth}
        Given a rule $\rho = L \leftarrow K \rightarrow R$, a rule $\rho' = (L' \leftarrow K' \rightarrow R')$ is called a \textbf{quotient rule} of $\rho$ if there exists a DPO diagram (shown on the right)
    where all vertical graph morphisms are surjective. The set of quotient rules of $\rho$ is denoted by $Q(\rho)$, and 
    for a rule set $\mathcal{A}$, we write $Q(\mathcal{A}) = \bigcup_{\rho\in\mathcal{A}} Q(\rho)$.
    \end{minipage}
    \hfill
    \begin{minipage}{0.29\textwidth}
        \hfill
        \begin{tikzpicture}[scale=0.7]
            \node (k) at (0,0) {K};
            \node (l) at (-2,0) {L};
            \node (r) at (2,0) {R};
            \node (k') at (0,-2) {K'};
            \node (l') at (-2,-2) {L'};
            \node (r') at (2,-2)  {R'};
            \draw[->] (k) -> (l);
            \draw[->] (k) -> (r); 
            \draw[->] (k') -> (l'); 
            \draw[->] (k') -> (r'); 
            \draw[->] (k) -> (k'); 
            \draw[->] (l) -> (l'); 
            \draw[->] (r) -> (r'); 
            \node () [at=($(l)!0.5!(k')$)] {PO};
            \node () [at=($(r)!0.5!(k')$)] {PO};
        \end{tikzpicture}
    \end{minipage}
\end{definition}

\begin{lemma}[Quotient Lemma~\cite{habel2001double}]
    Let $\rho$ be a DPO rewriting rule. For any graphs $G$ and $H$,
    $G \Rightarrow_{p,\mathfrak{F}} H$ if and only if $G \Rightarrow_{Q(p),\mathfrak{M}} H$.
\end{lemma}
\begin{corollary}
    \label{cor:termination}
    Let \(\mathcal{A}\) and \(\mathcal{B}\) be sets of injective DPO rewriting rules. 
    The rewriting relation $\Rightarrow_{\mathcal{A},\mathfrak{F}}$ is terminating relative to $\Rightarrow_{\mathcal{B},\mathfrak{F}}$ 
    if 
    $\Rightarrow_{Q(\mathcal{A}),\mathfrak{M}}$ is terminating relative to $\Rightarrow_{Q(\mathcal{B}),\mathfrak{M}}$.
    % If there is a set $\mathbb{X}$ of subgraphs of left-hand-side graphs of rewriting rules in \(\mathcal{A} \cup \mathcal{B}\) and a weight function $w:\mathbb{X} \to \mathbb{N}$ such that  
    % \begin{itemize}
    %     \item \todo{util?} for every graph $X \in \mathbb{X}$, for every rule \(\rho \in Q(\mathcal{A} \cup \mathcal{B})\), the rule $\rho$ is $X$-non-increasing,
    %     \item For every rule \(\left( L \overset{}{\leftarrowtail} K \overset{}{\rightarrowtail} R \right) \in Q(\mathcal{A})\),
    %     \[\sum_{X\in \mathbb{X}}^{} w(X) * |\operatorname{Mono}(X,L)| > \sum_{X\in \mathbb{X}}^{} w(X) * |\operatorname{Mono}(X,R)|\]
    %     \item For every rule \(\left( L \overset{}{\leftarrowtail} K \overset{}{\rightarrowtail} R \right) \in Q(\mathcal{B})\), 
    %     \[\sum_{X\in \mathbb{X}}^{} w(X) * |\operatorname{Mono}(X,L)| \geq \sum_{X\in \mathbb{X}}^{} w(X) * |\operatorname{Mono}(X,R)|\]
    %     % \item For every rule \(\left( L \overset{l}{\leftarrowtail} K \overset{r}{\rightarrowtail} R \right) \in Q(\mathcal{A})\),
    %     % \[\sum_{X\in S}^{} w(X) * |\operatorname{Mono}(X,L, \lnot l)| > \sum_{X\in S}^{} w(X) * |\operatorname{Mono}(X,R, \lnot r)|\]
    %     % \item For every rule \(\left( L \overset{l}{\leftarrowtail} K \overset{r}{\rightarrowtail} R \right) \in Q(\mathcal{B})\), 
    %     % \[\sum_{X\in S}^{} w(X) * |\operatorname{Mono}(X,L, \lnot l)| \geq \sum_{X\in S}^{} w(X) * |\operatorname{Mono}(X,R, \lnot r)|\]
    % \end{itemize}
    % then $\Rightarrow_{\mathcal{B},\mathfrak{F}}$ is terminating relative to $\Rightarrow_{\mathcal{B},\mathfrak{F}}$.
\end{corollary}

% \begin{corollary}
%     \label{cor:termination}
%     Let \(\mathcal{A}\) and \(\mathcal{B}\) be sets of injective DPO rewriting rules. If there is a set $\mathbb{X}$ of subgraphs of left-hand-side graphs of rewriting rules in \(\mathcal{A} \cup \mathcal{B}\) and a weight function $w:\mathbb{X} \to \mathbb{N}$ such that  
%     \begin{itemize}
%         \item \todo{util?} for every graph $X \in \mathbb{X}$, for every rule \(\rho \in Q(\mathcal{A} \cup \mathcal{B})\), the rule $\rho$ is $X$-non-increasing,
%         \item For every rule \(\left( L \overset{}{\leftarrowtail} K \overset{}{\rightarrowtail} R \right) \in Q(\mathcal{A})\),
%         \[\sum_{X\in \mathbb{X}}^{} w(X) * |\operatorname{Mono}(X,L)| > \sum_{X\in \mathbb{X}}^{} w(X) * |\operatorname{Mono}(X,R)|\]
%         \item For every rule \(\left( L \overset{}{\leftarrowtail} K \overset{}{\rightarrowtail} R \right) \in Q(\mathcal{B})\), 
%         \[\sum_{X\in \mathbb{X}}^{} w(X) * |\operatorname{Mono}(X,L)| \geq \sum_{X\in \mathbb{X}}^{} w(X) * |\operatorname{Mono}(X,R)|\]
%         % \item For every rule \(\left( L \overset{l}{\leftarrowtail} K \overset{r}{\rightarrowtail} R \right) \in Q(\mathcal{A})\),
%         % \[\sum_{X\in S}^{} w(X) * |\operatorname{Mono}(X,L, \lnot l)| > \sum_{X\in S}^{} w(X) * |\operatorname{Mono}(X,R, \lnot r)|\]
%         % \item For every rule \(\left( L \overset{l}{\leftarrowtail} K \overset{r}{\rightarrowtail} R \right) \in Q(\mathcal{B})\), 
%         % \[\sum_{X\in S}^{} w(X) * |\operatorname{Mono}(X,L, \lnot l)| \geq \sum_{X\in S}^{} w(X) * |\operatorname{Mono}(X,R, \lnot r)|\]
%     \end{itemize}
%     then $\Rightarrow_{\mathcal{B},\mathfrak{F}}$ is terminating relative to $\Rightarrow_{\mathcal{B},\mathfrak{F}}$.
% \end{corollary}