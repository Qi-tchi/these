% Like using rulers to measure physical objects, we use ruler-graphs (special reference structures) to measure graphs.
% Each ruler-graph in a set $\mathbb{X}$ provides a measure for a graph
% G, and the total weight of G is determined by a weighted linear
% combination of these measures.
% \todo{explain the difference the concept of ruler-graph presented in this paper and the one in the paper of \cite{qiu2025termination_icgt}.}

We use a definition of ruler-graph more general than the one in~\autoref{chap:subgraph_counting}.

\begin{definition}[Ruler-graph]
    A \textbf{ruler-graph} is either $(X)$ or \( (X, f) \) where $X$ is a graph, called \textbf{underlying graph}, and $f$ is a graph monomorphism with $\operatorname{dom}(f) = X$, called the \textbf{forbidden context}.
\end{definition}
For $\mathcal{X} = (X, f:X \rightarrowtail F)$, we will also called $F$ the forbidden context.

% \begin{notation}
%     $\operatorname{MonoNF}((X,F_X), G) = $
% \end{notation}
\begin{definition}[Measurement]
    % \textcolor{red}{
    Let \( \mathcal{X}\) be a ruler-graph and \( G \) a graph. The \textbf{measurement} of \( G \) with respect to \( \mathcal{X}\), denoted \( m_\mathcal{X}(G) \), is defined as 
    \[
        m_\mathcal{X}(G) =
        \begin{cases}
            \card{\{h \in \operatorname{Mono}(X,G) \mid \nexists g.\,f \star g = h \}}, & \text{if } \mathcal{X} = (X, f), \\
            \card{\operatorname{Mono}(X,G)}, & \text{if } \mathcal{X} = (X).
        \end{cases}
    \]
\end{definition}
The novelty over~\autoref{chap:subgraph_counting} is that, for a given ruler-graph, the occurrences of the underlying graph that overlap with its forbidden context are excluded. The definitions of weight function and graph weights do not change.
% \begin{definition}[Weight function]
%     \label{def:weight_function}
%     A \textbf{weight function} for a set of ruler-graphs \( \mathbb{X} \) is a map \( s_{\mathbb{X}} \colon \mathbb{X} \to \mathbb{N} \).
%     %  assigning a weight in $\mathbb{N}$ to each ruler-graph.
% \end{definition}
% \begin{definition}[Graph weight]
%     \label{def:subgraph_counting_antipattern:graph_weight}  
%     Given a set of ruler-graphs \( \mathbb{X} \), 
%     a weight function \( s_{\mathbb{X}} \), and a graph \( G \), the \textbf{weight of graph} \( G \) relative to \( s_{\mathbb{X}} \), written \( w_{s_{\mathbb{X}}}(G) \), is: 
%     \[
%         w_{s_{\mathbb{X}}}(G) = \sum_{\mathcal{X} \in \mathbb{X}} s_{\mathbb{X}}(\mathcal{X}) \cdot m_\mathcal{X}(G)  
%     \]   
% \end{definition}
