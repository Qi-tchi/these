We generalize the concept of the ruler-graph presented in Chapter~\ref{chap:subgraph_counting}.

\begin{definition}
    \label{antipattern:def:ruler_graph}
    A \textbf{ruler-graph} is either $(X)$ or \( (X, f) \) where $X$ is a graph (the \textbf{underlying graph}) and $f$ is a monomorphism into a graph $F$; $f$ is the \textbf{forbidden context}.
\end{definition}
For $\mathcal{X} \mathop{=} (X, f:X \rightarrowtail F)$, we also call $F$ the forbidden context for simplicity.
The definition of measurement needs some modifications to take the forbidden context into account. When the ruler-graph is just $(X)$, the definition of measurement is the same as that in Definition~\ref{subgraph_counting:def:measurement} in Chapter~\ref{chap:subgraph_counting}. When $\mathcal{X}$ is $(X, f:X \rightarrowtail F)$, we exclude those monomorphisms whose images overlap with the forbidden context. These excluded monomorphisms are those that can be extended to a morphism $g: F \rightarrowtail G$. This leads us to the following definition.
\begin{definition} 
    \label{antipattern:def:measurement}
    Let \( \mathcal{X}\) be a ruler-graph and \( G \) a graph. The \textbf{measurement} of \( G \) with respect to \( \mathcal{X}\), denoted \( m_\mathcal{X}(G) \), is defined as 
    \[
        m_\mathcal{X}(G) \isdef
        \begin{cases}
            \card{\{h \mathop{\in} \operatorname{Mono}(X,G) \mathop{\mid} \nexists g.\,f \mathop{\star} g \mathop{=} h \}}, & \text{if } \mathcal{X} \mathop{=} (X, f), \\
            \card{\operatorname{Mono}(X,G)}, & \text{if } \mathcal{X} \mathop{=} (X).
        \end{cases}
    \]
\end{definition}
The definitions of weight function and graph weights are identical to those in Definition~\ref{subgraph_counting:def:weight_function} and Definition~\ref{subgraph_counting:def:graph_weight} in Chapter~\ref{chap:subgraph_counting}, respectively, except that we use the new definition of measurement.