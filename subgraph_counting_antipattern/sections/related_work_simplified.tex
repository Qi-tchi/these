% \begin{table}[hbtp]
%    \centering
% \small % Reduce font size
% % \setlength{\tabcolsep}{4pt} % Reduce horizontal padding
%    \begin{NiceTabular}{cccccccc}[vlines] % <-- 9 columns now (was 7)
%     \Hline
%     \diagbox{\enskip \textbf{Examples}}{\textbf{Techniques}} &
%     \RowStyle{\rotate}
%      \makecell{Forward closure~\cite{plump1995ontermination}} % NEW column #1
%     & \RowStyle{\rotate}
%      \makecell{Modular criterion~\cite{plump2018modular}} % NEW column #2
%     & \RowStyle{\rotate}
%      \makecell{Type graph~\cite{bruggink2014termination}}  
%     & \RowStyle{\rotate}
%      \makecell{Type graph~\cite{bruggink2015proving}} 
%     & \RowStyle{\rotate}
%      \makecell{Type graph~\cite{endrullis2024generalized_arxiv_v2}} 
%     & \RowStyle{\rotate}
%      \makecell{Subgraph counting~\cite{overbeek2024termination_lmcs}} 
% %     & \RowStyle{\rotate}
% %      \makecell{Subgraph counting~\autoref{chap:subgraph_counting}}
%      & \RowStyle{\rotate}
%      \makecell{Our extension} \\
%     \Hline
%     \Hline
%     % not ok examples
%    ~\cite{plump1995ontermination}~\hyperref[ex:plump95_4d1]{Example 4.1} & \ding{51} & -- & -- & -- & -- & 
%               \ding{51}
              
%               & \ding{55}\\  
%    \Hline
%   ~\cite{plump2018modular}~\hyperref[ex:plump_ex4]{Example 4} &  -- &  \ding{51} &  -- & -- & -- & 
%                --
%                & \ding{55}\\ 
%    \Hline

%   ~\cite{bruggink2014termination} Routing Protocol
%        & -- & -- & \ding{51} & -- & -- & 
%            --
%            &  \ding{55}\\ \Hline

%   ~\cite{bruggink2014termination}\hyperref[ex:plump_ex4]{Example 5}
%    & -- & -- & \ding{51} & -- & -- & -- &  \ding{55}\\ 
% \Hline

%   ~\cite{bruggink2015proving}~\hyperref[ex:bruggink2015_ex5]{Example 5}
%    & -- & -- & -- & \ding{51} & -- &  
%    -- &  \ding{55}\\
%    \Hline

%   ~\cite{bruggink2015proving}~\hyperref[ex:bruggink2015_ex6_endrullis2024_d2]{Example 6} 
%    & -- & -- & -- & \ding{51} & -- &  
%    --&  \ding{55}\\ 
%    \Hline

%   ~\cite{endrullis2024generalized_arxiv_v2}~\hyperref[ex:bruggink2015_ex6_endrullis2024_d2]{Example D.2} 
%    & -- & -- & -- & -- & \ding{51} & -- & \ding{55}\\ 
%    \Hline

%       % not supported examples  
%      ~\cite{plump2018modular}~\hyperref[ex:plump2018_ex6_endrullis_d4]{Example 6} &  -- & \ding{51} & -- & -- & -- & 
%       --
%           & -- \\
%       \Hline

%      ~\cite{endrullis2024generalized_arxiv_v2} Example 6.4  
%       & -- & -- & -- & -- & \ding{51} & -- & -- \\ \Hline

%  ~\cite{endrullis2024generalized_arxiv_v2} Example 6.5  
%       & -- & -- & -- & -- &  \ding{51} & -- & -- \\ \Hline

%      ~\cite{endrullis2024generalized_arxiv_v2}~\hyperref[ex:plump2018_ex6_endrullis_d4]{Example D.4} 
%       & -- & -- & -- & -- & \ding{51} & -- & --\\ \Hline

%    % ----- from overbeek2024termination_lmcs -----
%  ~\cite{overbeek2024termination_lmcs} Example 5.2
%       & -- & -- & -- & -- & -- & \ding{51} & -- \\ \Hline

%      ~\cite{overbeek2024termination_lmcs} Example 5.7 
%       & -- & -- & -- & -- & -- & \ding{51} & -- \\ \Hline
      
%  ~\cite{overbeek2024termination_lmcs} Example 5.9 
%       & -- & -- & -- & -- & -- & \ding{51} & --\\ \Hline
 
%       % ok examples

%    ~\cite{plump1995ontermination}~\hyperref[ex:overbeek_5d8_plump1995_3d8_plump2018_3_overbeek_5d8]{Example 3.8}
%                   & \ding{51} & -- & -- & -- & -- &
%                --
%                   & \ding{51}\\ 
%      \hline
     
%     ~\cite{plump2018modular}~\hyperref[ex:overbeek_5d8_plump1995_3d8_plump2018_3_overbeek_5d8]{Example 3} 
%                & -- & \ding{51} &  -- & -- & -- & 
%                --
%                & \ding{51}\\ 

%     \Hline
%    ~\cite{plump2018modular} Example 5 &  -- &  \ding{51} &   -- & -- & -- &  
%                  --
%                & \ding{51}\\ 
%     \Hline

%   ~\cite{bruggink2014termination} Example 1 
%        & -- & -- & \ding{51} & -- & -- & 
%                      \ding{55}
%                    &  \ding{51}\\ 
%    \Hline

%   ~\cite{bruggink2014termination}~\hyperref[ex:termination:grsaa]{Example 4 and 6}  
%        & -- & -- & \ding{51} & -- & -- & 
%                  --
%                & \ding{51} \\ \Hline

%   ~\cite{bruggink2015proving}~\hyperref[ex:termination:grsaa]{Example 2}  
%        & -- & -- & -- & \ding{51} & -- & 
%        -- &  \ding{51}\\ \Hline
       
%   ~\cite{bruggink2015proving} Example 4 
%        & -- & -- & -- & \ding{51} & -- & 
%        --& \ding{51} \\ \Hline


%     % ----- from endrullis2024generalized_arxiv_v2 -----
%    ~\cite{endrullis2024generalized_arxiv_v2}~\hyperref[ex:endrullis2024_6d2]{Example 6.2}  
%        & -- & -- & -- & -- & \ding{51} & -- & \ding{51}\\ \Hline

%   ~\cite{endrullis2024generalized_arxiv_v2}~\hyperref[ex_endrullis_6d3_endrullis_5d8]{Example 6.3}
%        & -- & -- & -- & -- & \ding{51} &% \ding{55} 
%        \ding{55} & \ding{51}\\ \Hline

%   ~\cite{endrullis2024generalized_arxiv_v2}\hyperref[ex:overbeek_5d8_plump1995_3d8_plump2018_3_overbeek_5d8]{Example D.1}
%        & -- & -- & -- & -- & \ding{51} & -- & \ding{51}\\ \Hline

%   ~\cite{endrullis2024generalized_arxiv_v2}\hyperref[rem:d3_limitation]{Example D.3}
%        & -- & -- & -- & -- & \ding{51} & \ding{55} & \ding{51}\\ \Hline

%   ~\cite{overbeek2024termination_lmcs}~\hyperref[ex:overbeek_5d3]{Example 5.3}
%        & -- & -- & -- & -- & -- & \ding{51} & \ding{51}\\ \Hline

%    %~\cite{overbeek2024termination_lmcs}~\hyperref[ex:overbeek_5d3]{Example 5.3 monic matches}
%    %     & -- & -- & -- & -- & -- & \ding{51} & \ding{51}\\ \Hline
%  ~\cite{overbeek2024termination_lmcs}~\hyperref[ex:overbeek_5d5]{Example 5.5} 
%        & -- & -- & -- & -- & -- & \ding{51} & \ding{51}\\ \Hline

%   ~\cite{overbeek2024termination_lmcs}~\hyperref[ex:overbeek_5d6]{Example 5.6}
%        & -- & -- & -- & -- & -- & \ding{51} & \ding{51} \\ \Hline

%    %~\cite{overbeek2024termination_lmcs}~\hyperref[ex:overbeek_5d6]{Example 5.6 bis}
%    %     & -- & -- & -- & -- & -- & \ding{51} & -- \\ \Hline


%   ~\cite{overbeek2024termination_lmcs}~\hyperref[ex:overbeek_5d8_plump1995_3d8_plump2018_3_overbeek_5d8]{Example 5.8}
%        & -- & -- & -- & -- & -- & \ding{51} & \ding{51}\\ \Hline

% \autoref{ex_contrib_variant}
%        & \ding{51} & \ding{55} & \ding{55} & \ding{55} & \ding{55} & \ding{55} & \ding{51} \\ \Hline
%   \end{NiceTabular}
%   \caption{Applicability of termination techniques to DPO rewriting examples.
%    The symbol \ding{51} indicates termination can be proved by the technique,
%    \ding{55} indicates it cannot be proved, and 
%    $-$ denotes irrelevance or out-of-scope cases.
%          }
%   \label{tab:antipattern:comparison}
%   \end{table}
\begin{table}[hbtp]
   \centering
\small % Reduce font size
% \setlength{\tabcolsep}{4pt} % Reduce horizontal padding
   \begin{NiceTabular}{ccccccccc}[vlines] %
    \Hline
    \diagbox{\enskip \textbf{Examples}}{\textbf{Techniques}} &
    \RowStyle{\rotate}
     \makecell{Forward closure~\cite{plump1995ontermination}} % NEW column #1
    & \RowStyle{\rotate}
     \makecell{Modular criterion~\cite{plump2018modular}} % NEW column #2
    & \RowStyle{\rotate}
     \makecell{Type graph~\cite{bruggink2014termination}}  
    & \RowStyle{\rotate}
     \makecell{Type graph~\cite{bruggink2015proving}} 
    & \RowStyle{\rotate}
     \makecell{Type graph~\cite{endrullis2024generalized_arxiv_v2}} 
    & \RowStyle{\rotate}
     \makecell{Subgraph counting~\cite{overbeek2024termination_lmcs}} 
    & \RowStyle{\rotate}
     \makecell{Subgraph counting(\autoref{chap:subgraph_counting})}
     & \RowStyle{\rotate}
     \makecell{Our extension} \\
    \Hline
    \Hline
    % not ok examples
   ~\cite{plump1995ontermination}~\hyperref[ex:plump95_4d1]{Example 4.1} & \ding{51} & -- & -- & -- & -- & 
              \ding{51}
              
              & \ding{55}& \ding{55}\\  
   \Hline
  ~\cite{plump2018modular}~\hyperref[ex:plump_ex4]{Example 4} &  -- &  \ding{51} &  -- & -- & -- & 
               --
               & \ding{55}& \ding{55}\\ 
   \Hline

  ~\cite{bruggink2014termination} Routing Protocol
       & -- & -- & \ding{51} & -- & -- & 
           --
          & \ding{55} &  \ding{55}\\ \Hline

  ~\cite{bruggink2014termination}\hyperref[ex:plump_ex4]{Example 5}
   & -- & -- & \ding{51} & -- & -- & -- 
   & \ding{55} &  \ding{55}\\ 
\Hline

  ~\cite{bruggink2015proving}~\hyperref[ex:bruggink2015_ex5]{Example 5}
   & -- & -- & -- & \ding{51} & -- &  
   --
   & \ding{55} &  \ding{55}\\
   \Hline

  ~\cite{bruggink2015proving}~\hyperref[ex:bruggink2015_ex6_endrullis2024_d2]{Example 6} 
   & -- & -- & -- & \ding{51} & -- &  
   --
   & \ding{55} &  \ding{55}\\ 
   \Hline

  ~\cite{endrullis2024generalized_arxiv_v2}~\hyperref[ex:bruggink2015_ex6_endrullis2024_d2]{Example D.2} 
   & -- & -- & -- & -- & \ding{51} & -- 
   & \ding{55} & \ding{55}\\ 
   \Hline

      % not supported examples  
     ~\cite{plump2018modular}~\hyperref[ex:plump2018_ex6_endrullis_d4]{Example 6} &  -- & \ding{51} & -- & -- & -- & 
      --
          & -- & -- \\
      \Hline

     ~\cite{endrullis2024generalized_arxiv_v2} Example 6.4  
      & -- & -- & -- & -- & \ding{51} & -- 
       & -- & -- \\ \Hline

 ~\cite{endrullis2024generalized_arxiv_v2} Example 6.5  
      & -- & -- & -- & -- &  \ding{51} & --
       & -- & -- \\ \Hline

     ~\cite{endrullis2024generalized_arxiv_v2}~\hyperref[ex:plump2018_ex6_endrullis_d4]{Example D.4} 
      & -- & -- & -- & -- & \ding{51} & -- 
       & -- & --\\ \Hline

   % ----- from overbeek2024termination_lmcs -----
 ~\cite{overbeek2024termination_lmcs} Example 5.2
      & -- & -- & -- & -- & -- & \ding{51} 
       & -- & -- \\ \Hline

     ~\cite{overbeek2024termination_lmcs} Example 5.7 
      & -- & -- & -- & -- & -- & \ding{51} 
       & -- & -- \\ \Hline
      
 ~\cite{overbeek2024termination_lmcs} Example 5.9 
      & -- & -- & -- & -- & -- & \ding{51} 
       & -- & --\\ \Hline
 
      % ok examples

   ~\cite{plump1995ontermination}~\hyperref[ex:overbeek_5d8_plump1995_3d8_plump2018_3_overbeek_5d8]{Example 3.8}
                  & \ding{51} & -- & -- & -- & -- &
               --
                 & \ding{51}  & \ding{51}\\ 
     \hline
     
    ~\cite{plump2018modular}~\hyperref[ex:overbeek_5d8_plump1995_3d8_plump2018_3_overbeek_5d8]{Example 3} 
               & -- & \ding{51} &  -- & -- & -- & 
               --
                & \ding{51} & \ding{51}\\ 

    \Hline
   ~\cite[Example 5]{plump2018modular}  &  -- &  \ding{51} &   -- & -- & -- &  
                 --
               & \ding{51} & \ding{51}\\ 
    \Hline

  ~\cite[Example 1]{bruggink2014termination} 
       & -- & -- & \ding{51} & -- & -- & 
                     \ding{55}
                  & \ding{55}  &  \ding{51}\\ 
   \Hline

  ~\cite{bruggink2014termination}~\hyperref[ex:termination:grsaa]{Example 4 and 6}  
       & -- & -- & \ding{51} & -- & -- & 
                 --
               & \ding{51} & \ding{51} \\ \Hline

  ~\cite{bruggink2015proving}~\hyperref[ex:termination:grsaa]{Example 2}  
       & -- & -- & -- & \ding{51} & -- & 
       -- 
        & \ding{51} &  \ding{51}\\ \Hline
       
  ~\cite{bruggink2015proving} Example 4 
       & -- & -- & -- & \ding{51} & -- & 
       --
        & \ding{51} & \ding{51} \\ \Hline


    % ----- from endrullis2024generalized_arxiv_v2 -----
   ~\cite{endrullis2024generalized_arxiv_v2}~\hyperref[ex:endrullis2024_6d2]{Example 6.2}  
       & -- & -- & -- & -- & \ding{51} & -- & \ding{51}& \ding{51}\\ \Hline

  ~\cite{endrullis2024generalized_arxiv_v2}~\hyperref[ex_endrullis_6d3_endrullis_5d8]{Example 6.3}
       & -- & -- & -- & -- & \ding{51} &% \ding{55} 
       \ding{55} 
        & \ding{51} & \ding{51}\\ \Hline

  ~\cite{endrullis2024generalized_arxiv_v2}\hyperref[ex:overbeek_5d8_plump1995_3d8_plump2018_3_overbeek_5d8]{Example D.1}
       & -- & -- & -- & -- & \ding{51} & --
        & \ding{51} & \ding{51}\\ \Hline

  ~\cite{endrullis2024generalized_arxiv_v2}\hyperref[rem:d3_limitation]{Example D.3}
       & -- & -- & -- & -- & \ding{51} & \ding{55} 
        & \ding{55} & \ding{51}\\ \Hline

  ~\cite{overbeek2024termination_lmcs}~\hyperref[ex:overbeek_5d3]{Example 5.3}
       & -- & -- & -- & -- & -- & \ding{51} 
        & \ding{51} & \ding{51}\\ \Hline

   %~\cite{overbeek2024termination_lmcs}~\hyperref[ex:overbeek_5d3]{Example 5.3 monic matches}
   %     & -- & -- & -- & -- & -- & \ding{51} & \ding{51}\\ \Hline
 ~\cite{overbeek2024termination_lmcs}~\hyperref[ex:overbeek_5d5]{Example 5.5} 
       & -- & -- & -- & -- & -- & \ding{51} 
        & \ding{51} & \ding{51}\\ \Hline

  ~\cite{overbeek2024termination_lmcs}~\hyperref[ex:overbeek_5d6]{Example 5.6}
       & -- & -- & -- & -- & -- & \ding{51} 
        & \ding{51} & \ding{51} \\ \Hline

   %~\cite{overbeek2024termination_lmcs}~\hyperref[ex:overbeek_5d6]{Example 5.6 bis}
   %     & -- & -- & -- & -- & -- & \ding{51} & -- \\ \Hline


  ~\cite{overbeek2024termination_lmcs}~\hyperref[ex:overbeek_5d8_plump1995_3d8_plump2018_3_overbeek_5d8]{Example 5.8}
       & -- & -- & -- & -- & -- & \ding{51} 
        & \ding{51} & \ding{51}\\ \Hline

\autoref{ex_contrib_variant}
       & \ding{51} & \ding{55} & \ding{55} & \ding{55} & \ding{55} & \ding{55} 
        & \ding{51} & \ding{51} \\ \Hline
  \end{NiceTabular}
  \caption{Applicability of termination techniques to DPO rewriting examples.
   The symbol \ding{51} indicates termination can be proved by the technique,
   \ding{55} indicates it cannot be proved, and 
   $-$ denotes irrelevance or out-of-scope cases.
         }
  \label{tab:antipattern:comparison}
  \end{table}
% The subgraph-counting method in~\cite{overbeek2024termination_lmcs} by Overbeek and Endrullis is designed for the PBPO+~\cite{overbeek2023graph}—a rewriting formalism capable of simulating left-injective DPO rewriting. It can prove termination for systems like~\cite[Examples 5.2, 5.7, 5.9]{overbeek2024termination_lmcs} and~\cite[Example 6]{plump2018modular}, which lie beyond  the scope of our technique. Additionally, this method can be applied to many categories.
 
% In the setting where they are both defined, it is closely related to our work: both methods assign weight to objects by summing weighted morphisms targeting them.

% Their method and our method are not directly comparable for two reasons.
% First, their method suffers from discrete interface, as mentioned in~\cite[Example 5.5]{overbeek2024termination_lmcs}. For instance, it proves termination for~\cite[Example 5.5]{overbeek2024termination_lmcs} but fails for~\cite[Example 6.3]{endrullis2024generalized_arxiv_v2}, where the rules differ only by a discrete interface. 
% Similarly, it fails to prove termination for~\autoref{ex_contrib_variant}, but if an edge labeled \enquote{s} from node $1$ to node $3$ is added, then it succeeds.
% Our method can handle all these examples.
% However, their method handles~\cite[Example 4.1]{plump1995ontermination}, which our method cannot handle.
% Second, their method cannot take antipatterns into account and thus cannot handle~\cite[Example 1]{bruggink2014termination},~\cite[Example D.3]{endrullis2024generalized_arxiv_v2}, while our method can. 

% The type graph method~\cite{zantema2014termination,bruggink2014termination,bruggink2015proving,endrullis2024generalized_arxiv_v2}, which weighs an object by summing the weights of morphisms from the object to a type graph, is related to our method. 

% Each version of the type graph method is not directly comparable with our technique in the setting where they are both defined as shown in~\autoref{tab:comparison}.
 
% Plump~\cite{plump1995ontermination} introduced a necessary and sufficient termination condition for left-injective DPO hypergraph rewriting via forward closure, though verifying this condition is undecidable.

% Plump~\cite{plump2018modular} later proposed a modular critical pair-based strategy for left-injective DPO hypergraph rewriting with monic matches. 
% Our method complements this: while modularity reduces global complexity, each subsystem requires individual termination proofs. 
% For example, the measure based on the indegree proposed in~\cite{plump2018modular} cannot prove the termination of~\autoref{ex_contrib_variant} due to the additional loops. Specifically, if the context graph is identical to the interface graph then we have $3^k < 2^k+2^k+3^k$ for all $k \in \mathbb{N}$. In contrast, our method succeeds.
 

% Levendovszky et al.~\cite{levendovszky2007termination} propose a termination criterion for DPO rewriting (monic matches, injective rules, negative application condition), though automated verification is hard as explained in~\cite[\textsection 6]{levendovszky2007termination}. 

% Bottoni et al.~\cite{bottoni2005termination} propose a termination criterion for DPO/SPO rewriting on high-level replacement units. Their method imposes a strongly constrained measuring function and the only concretes measuring function proposed are node-counting and edge-counting.

% Bottoni et al.~\cite{bottoni2010atermination} presents a criterion for termination of DPO rewriting with monic matches, injective rules and negative application conditions, based on the construction of a labeled transition system. 

A comparative analysis of termination techniques for DPO graph rewriting systems, drawn from prior work~\cite{plump1995ontermination,plump2018modular,bruggink2014termination,bruggink2015proving,endrullis2024generalized_arxiv_v2,
overbeek2024termination_lmcs} and~\autoref{chap:subgraph_counting}, is summarized in~\autoref{tab:antipattern:comparison}. 
Our tool successfully proves termination for 16 systems, including~\cite[Example D.3]{endrullis2024generalized_arxiv_v2} and~\cite[Example 1]{bruggink2014termination}. These cases could not be handled by the subgraph counting methods introduced in~\cite{overbeek2024termination_lmcs} and~\autoref{chap:subgraph_counting}, but are now provable using our extension.
 
