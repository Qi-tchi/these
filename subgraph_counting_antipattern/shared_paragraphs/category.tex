\begin{definition}[Category~\cite{pierce1991basic, barr1990category}]
    \label{def:cat}
    A \textbf{category} is an unlabeled graph \( C \) together with a total function \( u : C_0 \mathop{\to} C_1 \) and a partial function \( \star: C_1 \mathop{\times} C_1 \mathop{\to} C_1 \) such that 
        (i) for all edges \( f:X \mathop{\to} Y \) and \( g:Y \mathop{\to} Z \), the edge \( f \mathop{\star} g :X \mathop{\to} Z \) is defined; 
        (ii) for every node \( X \), \( u(X) \) is an edge from \( X \) to \( X \);
        (iii) for every \( f:X \mathop{\to} Y \), we have \(u(X) \mathop{\star} f \mathop{=} f \mathop{=} f \mathop{\star} u(Y)\);
        (iv) for all edges \( f \), \( g \) and \(h\), we have \( (f \mathop{\star} g) \mathop{\star} h \mathop{=} f \mathop{\star} (g \mathop{\star} h) \) whenever either side is defined.
    Edges are called \textbf{morphisms}. The function $\star$ is called \textbf{composition}. For all \( X \mathop{\in} C_0 \), the edge \( u(X) \) is denoted \( \operatorname{id}_X \) and is called the \textbf{identity} of the object \( X \).
    % \( C \) is called the \textbf{underlying graph} of the category \( \mathcal{C} \).
\end{definition} 