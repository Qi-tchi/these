\begin{notation}
    We use the notation from~\cite[Notation 1]{overbeek2023apbpo+} to visualize edge-labeled graph homomorphisms. Labeled graphs are enclosed in boxes with their names displayed in the top-left corner. Nodes and edges are assigned subsets of \(\mathbb{N}\) as identifiers, and these identifiers are chosen such that: (i) Each node or edge \( y \) in the codomain graph is assigned the union of the identifiers of all nodes or edges in the domain graph that are mapped to \( y \); (ii) The graph homomorphism is uniquely determined by this assignment.
     
    \noindent To further improve readability, we represent sets by listing their elements. Additionally, we omit identifiers when doing so does not cause confusion. This is illustrated in the following representation of a homomorphism \( h: G \to H \).
    
    \begin{center}
        \resizebox{0.45\textwidth}{!}{
        \begin{tikzpicture}
            \graphbox{\( G \)}{00mm}{-20mm}{45mm}{20mm}{2mm}{-5mm}{
                \coordinate (o) at (-5mm,-8mm); 
                \node[draw,circle] (l1) at ($(o)+(-10mm,0mm)$) {1};
                \node[draw,circle] (l2) at ($(l1)+(3,0)$) {2};
                \node[draw,circle] (l3) at ($(l1)+(1,0)$) {3};
                \node[draw,circle] (l4) at ($(l1)+(2,0)$) {4};
                \draw[->] (l1) -- (l3) node[midway,above] {$a$};
                \draw[->] (l3) -- (l4) node[midway,above] {$b$};
                \draw[->] (l4) -- (l2) node[midway,above] {$a$};
            }  
            \graphbox{\( H \)}{50mm}{-20mm}{34mm}{20mm}{2mm}{-5mm}{
                \coordinate (o) at (0mm,-8mm); 
                \node[draw,circle] (l1) at ($(o)+(-10mm,0mm)$) {1};
                \node[draw,circle] (l2) at ($(l1)+(2,0)$) {2};
                \node[draw,circle] (l3) at ($(l1)+(1,0)$) {3\ 4};
                \draw[->] (l1) -- (l3) node[midway,above] {$a$};
                \draw[->] (l3) edge[loop above] (l3) node[midway,above] {$b$};
                \draw[->] (l3) -- (l2) node[midway,above] {$a$};
            }      
            % \node () at (53mm,-30mm) {$\rightarrow$};
        \end{tikzpicture}
    }
    \end{center} 
    In this example, the sets \(\{1\}\), \(\{2\}\), \(\{3\}\), \(\{4\}\), and \(\{3,4\}\) are represented as \(1\), \(2\), \(3\), \(4\), and \(3\ 4\), respectively. Edge identifiers are omitted.
\end{notation} 