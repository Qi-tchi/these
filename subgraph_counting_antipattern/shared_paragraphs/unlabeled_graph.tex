\begin{definition}[\cite{barr1990category}]
    \label{def:graph:unlabeled}
    An \textbf{unlabeled graph} \( G \) consists of a collection of \textbf{nodes} (also called \textbf{objects}) and a collection of \textbf{edges}\trackedtext{, each} equipped with a \textbf{source} (or \textbf{domain}) node and a \textbf{target} (or \textbf{codomain}) node. 
    For an unlabeled graph \( G \), we denote by \( G_0 \) its collection of nodes, \( G_1 \) its collection of edges, \( \operatorname{dom}:G_1{\to}G_0 \) the domain function, and \( \operatorname{cod}:G_1{\to}G_0 \) the codomain function. An unlabeled graph is \textbf{finite} if \( G_0 \) and \( G_1 \) are both finite sets.
    We write \( a: s \to t \) to indicate that \( a \) is a directed edge from \( s \) to \( t \).
\end{definition}   