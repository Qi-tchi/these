\documentclass{book}

\usepackage[colorinlistoftodos,draft
% ,disable
]
{todonotes}

\usepackage{marginnote}
\setlength{\marginparwidth}{2cm}
\usepackage[dvipsnames,x11names]{xcolor}
\usepackage{pgfplots,diagbox,colortbl,tikz,graphicx,
algorithm2e,cancel,verbatim,
listings,float,amsmath,amssymb,enumitem,array,
subfiles,bussproofs,
rotating,MnSymbol,mathtools,subcaption,caption}
\pgfplotsset{compat=1.18}
\usetikzlibrary{automata, positioning,graphs,shapes, arrows, calc}
% CORRECT LOCATION (preamble):
\usepackage{csquotes}
\usepackage[
colorlinks=true,
linkcolor=PineGreen, 
citecolor=PineGreen,
urlcolor=PineGreen,
pdfborder={0 0 0}
]{hyperref}
\hyphenation{ma-chine-check-a-ble} 
\hyphenation{dou-ble-push-out}
%%%%%%% careful : order is important begin
\usepackage{amsthm}
\usepackage{thmtools}
\newtheorem{theorem}{Theorem}[chapter]
\newtheorem{corollary}[theorem]{Corollary}
\newtheorem{lemma}[theorem]{Lemma}
\newtheorem{idea}[theorem]{Idea}
\newtheorem{definition}[theorem]{Definition}
\newtheorem{proposition}[theorem]{Proposition}
\newtheorem{remark}[theorem]{Remark}
\newtheorem{example}[theorem]{Example}
\newtheorem{Plan}[theorem]{Plan}
\newtheorem{claim}[theorem]{Claim}
\newtheorem{notation}[theorem]{Notation}
% \usepackage[section]{placeins}
\usepackage{minted}



\newcommand{\set}[1]{\{#1\}} 
\newcommand{\graphbox}[8]{
  \begin{scope}[xshift=#2,yshift=#3]
    \draw [rounded corners=2mm] (0,0) rectangle (#4,-#5);
    \node at (0,0mm) [anchor=north west,inner sep=1mm] {#1};
    \begin{scope}[xshift=#4/2+#6,yshift=#7] 
    #8
    \end{scope}
  \end{scope}
}
\newcommand{\card}[1]{|#1|}
\newcommand{\homset}[2]{\operatorname{Hom}(#1,#2)} 
\newcommand{\isdef}{\overset{\operatorname{def}}{=}}
\newcommand{\cat}[1]{\mathcal{#1}}
\newcommand{\f}[1]{\mathfrak{#1}}
\newcommand{\rs}[1]{\mathcal{#1}}
\newcommand{\opn}[1]{\operatorname{#1}}
%arrow
\newcommand{\itrs}{\overset{i}{\to}} 

\usepackage{tocloft}

% % Customizing the table of contents
% % \setlength{\cftpartindent}{0pt} %  space before part titles
\setlength{\cftchapindent}{15pt} %  space before chapter titles
\setlength{\cftsecindent}{30pt} %  space before sec titles
\setlength{\cftsubsecindent}{40pt} %  space before subsec titles
\setlength{\cftsubsubsecindent}{50pt} %  space before subsubsec titles


\newcommand{\textdef}{\text{$\operatorname{Def.}$}}
\newcommand{\pbpop}{$\operatorname{PBPO}^+$}
\addtolength{\cftbeforechapskip}{-10pt} % Reduce spacing by 10pt
\addtolength{\cftbeforepartskip}{-20pt} % Reduce spacing by 10pt
\newif\iflongversion
\longversiontrue
\newif\iftrackChange
\trackChangetrue
\newcommand{\trackedtext}[1]{%
    \iftrackChange
        \textcolor{red}{#1}% Apply red color if trackChange is true
    \else
        #1% No color change if trackChange is false
    \fi
} 


% \usepackage[style=alphabetic, maxbibnames=999]{biblatex}
\usepackage[maxbibnames=999]{biblatex}
\addbibresource{bib/these.bib} % Add your bibliography file here

\usepackage{titlesec} 
 
\usepackage{tikz-cd}
\usepackage{pifont}      
\usepackage{makecell}    
\usepackage{nicematrix}  
\usepackage{wrapfig}     
\usepackage{lipsum}      

% 

% \usepackage{diagbox,colortbl,tikz,graphicx,algorithm2e,cancel,verbatim,
% listings,float,amsthm,amsmath,amssymb,enumitem,array,subfiles,bussproofs,
% rotating,MnSymbol,hyperref,mathtools,subcaption,caption}
\begin{document}  
  

\begin{titlepage}

    \unitlength 1cm
    \begin{center}
    
    \vspace*{-2.5cm}
    \begin{figure}[h]
        \centering
        \includegraphics[width=0.8\textwidth]{LogoLyon1Off_CoulCmjn300dpi.jpg}
    \end{figure}
    
    {\large\bf THESE de DOCTORAT DE\\L'UNIVERSITE CLAUDE BERNARD LYON 1\\}
    \vspace{12pt}
               {\large \textbf{Ecole Doctorale} N\textsuperscript{o} 512 \\
               Informatique Mathématiques}

    \vspace{0.5cm}
    
               {Soutenue publiquement le 16/12/2025, par~:\\}
              %  {À soutenir publiquement le 16/12/2025, par~:\\}
    \vspace{0.1cm}
               {\Large\bf {Qi QIU}}
    \vspace{0.5cm}           
    
    
    \rule{\textwidth}{1pt}
    
    \vspace{0.1cm}
               
               {\LARGE \bf Automated Termination Proving: Contributions to Graph Rewriting\\via Extended Weighted Type Graphs\\and Morphism Counting}\\
              %  Preuve automatique de terminaison : contributions à la réécriture de graphes par graphes de type pondérés étendus et dénombrement de morphismes
               \vspace{0.3cm}
               {\Large \bf }
               
    \vspace{0.1cm}
    \rule{\textwidth}{1pt}
    
    \vspace{0.1cm}
    
    \end{center}
    
    Devant le jury composé de~:\\ 
       
    \begin{center}
    \small { 
    \begin{tabular}{lll}
    \textbf{Mme} \hspace{3cm}&\textbf{Clara Bertolissi}  &\textbf{Examinatrice}
    \\
      \multicolumn{3}{l}{Professeure des universités, INSA Centre Val de Loire}\\
    \textbf{Mme} &\textbf{Évelyne Contejean}           &\textbf{Examinatrice}
    \\
      \multicolumn{3}{l}{Directrice de recherche, CNRS Gif sur Yvette}\\
    \textbf{M.} &\textbf{Pierre Hyvernat}           &\textbf{Examinateur}\\
      \multicolumn{3}{l}{Maître de conférences, Université Savoie Mont Blanc}\\
    \textbf{M.} &\textbf{Jean Krivine}           &\textbf{Rapporteur}
    \\
      \multicolumn{3}{l}{Chargé de recherche, CNRS Paris}
      \\
    \textbf{Mme} &\textbf{Pascale Le Gall}           &\textbf{Rapporteure}\\
      \multicolumn{3}{l}{Professeure des universités, CentraleSupélec Gif-sur-Yvette}\\
    \textbf{M.} &\textbf{Philippe Malbos}           &\textbf{Président, Examinateur}\\
    \multicolumn{3}{l}{Professeur des universités, Université Claude Bernard Lyon 1}\\
    \textbf{M.} &\textbf{Frédéric Prost}           &\textbf{Examinateur}\\
    \multicolumn{3}{l}{Maître de conférences, INSA Lyon}\\
    \textbf{M.} &\textbf{Xavier URBAIN}           &\textbf{Directeur de thèse}\\
      \multicolumn{3}{l}{Professeur des universités, Université Claude Bernard Lyon 1}\\
    \end{tabular}
    }
  \end{center}
    
    \vfill \clearpage{\pagestyle{empty}\cleardoublepage}
    
    %-------------------------------- Remerciement  -------------------
    \end{titlepage}  

\section*{Acknowledgements}


First of all, I thank my supervisor, Xavier Urbain, for providing me with the opportunity to work on this subject and for his guidance not only in my research but, more importantly, in my personal development: his integrity is a great example for me to follow. At the midpoint of my PhD, we had different opinions regarding the research direction, and he generously allowed me to pursue the path which inspired me the most. This freedom has been invaluable to me, and I am grateful for his kindness.

I extend my sincere thanks to all the faculty members who contributed to my academic development during my studies in France, for their generosity in sharing knowledge, and for their kindness and patience towards their students, all of which greatly influenced my personal growth.

I thank the anonymous reviewers of my work for their time and valuable feedback; Zhilin Wu for his suggestion on the general structure of this dissertation; and Pierre Courtieu for his help in the early stages of my PhD.

I thank my parents, Shuangluan Ye and Guowen Qiu, for their unconditional love and sacrifices that have allowed me to pursue my dreams, and my wife, Wang Lu, for her patience and understanding during the challenging times of my PhD.

I thank my friends Pierre, Thomas, Auday, Medhi, Yacine, Zhongyun, Wei, Chukun, Xin, and all those who have come into my life, as each person has played a role in shaping who I am today.

I'm also grateful to the large language model that helped me in writing on a daily basis. Thanks to its tireless proofreading, I was able to avoid burdening the people closest to me with endless revisions.

% Of course, I thank myself for my perseverance and determination during my adventure in France. I failed many times, but I never gave up. I learned from my mistakes and kept moving forward. I finally achieved my goal of becoming someone like Émile as described in the book \enquote{Émile ou De l'éducation} by Jean-Jacques Rousseau who is a person of integrity and is useful to the society.

Finally, I would like to express my gratitude to the French National Research Agency (ANR) for funding the project Safe, Adaptive, and Provable Protocols for Oblivious Robots Operation (SAPPORO) under Grant 2019-CE25-0005, which has provided essential resources and support for my research endeavors.  

\newpage      


\tableofcontents  
\newpage      
 
\chapter{Introduction}
\subsection{Graphs}
 The concepts and notation in this section follow the treatments in~\cite{barr1990category}.
% \begin{definition}
%     \label{def:graph:unlabeled}
%     An \textbf{directed unlabeled multigraph} \( G \) consists of 
%     \begin{itemize}
%         \item a collection $V(G)$ of elements, which are called \textbf{nodes} (or \textbf{objects}),
%         \item a collection $E(G)$ of elements different from nodes, which are called \textbf{edges} (or \textbf{arrows}),
%         \item a function $\opn{dom}: E \to V$ assigning to each edge its \textbf{source} (or \textbf{domain}) node,
%         \item a function $\opn{cod}: E \to V$ assigning to each edge its \textbf{target} (or \textbf{codomain}) node.
%     \end{itemize} 
%     An directed unlabeled multigraph $G$ is \textbf{finite} if $V(G)$ and $E(G)$ have both finite elements.
%     Throughout, the term \enquote{unlabeled graph} denotes a directed unlabeled multigraph.
%     We write \( a: s \to t \) to indicate that \( a \) is an edge from \( s \) to \( t \).
% \end{definition}   

% Note that an unlabeled graph can have multiple edges from one node to another, as well as loops (edges from a node to itself), because different edges $u,v$ of the unlabeled graph can have the same source and target nodes, i.e. $\opn{dom}(u) = \opn{dom}(v)$ and $\opn{cod}(u) = \opn{cod}(v)$.


% \begin{example}
%     An unlabeled graph is shown in Figure~\ref{fig:preliminaries:unlabeled_graph}.

%     In an unlabeled graph, edges are directed and will be visualized as arrows; between any two nodes, there can be multiple edges, and loops are allowed. An unlabeled graph is depicted in Figure~\ref{fig:preliminaries:unlabeled_graph} where nodes are marked with numbers to facilitate discussion, but in general, nodes are not labeled.
%     \begin{figure}[H]
%         \centering
%         \begin{tikzpicture}
%             \graphbox{}{0mm}{0mm}{32mm}{28mm}{-10mm}{-14mm}{
%                 \node[draw,circle] (1) at (0,0) {1};
%                 \node[draw,circle] (2) at (2,0) {2};
%                 \draw[->] (1) edge[loop above] (1) ;
%                 \draw[->] (1) edge[loop below] (1) ;
%                 \draw[->] (1) edge[bend left] (2)  ;
%                 \draw[->] (2) edge[bend left] (1)   ;
%             }
%         \end{tikzpicture}
%     \caption{Unlabeled graph}
%     \label{fig:preliminaries:unlabeled_graph}
%     \end{figure}
% \end{example}

% \begin{example}
%     In an unlabeled graph, edges are directed and will be visualized as arrows; between any two nodes, there can be multiple edges, and loops are allowed. An unlabeled graph is depicted in Figure~\ref{fig:preliminaries:unlabeled_graph} where nodes are marked with numbers to facilitate discussion, but in general, nodes are not labeled.
%     \begin{figure}[H]
%         \centering
%         \begin{tikzpicture}
%             \graphbox{}{0mm}{0mm}{32mm}{28mm}{-10mm}{-14mm}{
%                 \node[draw,circle] (1) at (0,0) {1};
%                 \node[draw,circle] (2) at (2,0) {2};
%                 \draw[->] (1) edge[loop above] (1) ;
%                 \draw[->] (1) edge[loop below] (1) ;
%                 \draw[->] (1) edge[bend left] (2)  ;
%                 \draw[->] (2) edge[bend left] (1)   ;
%             }
%         \end{tikzpicture}
%     \caption{Unlabeled graph}
%     \label{fig:preliminaries:unlabeled_graph}
%     \end{figure}
% \end{example}
% A homomorphism of unlabeled graphs is a mapping between the nodes and edges of two graphs that preserves the graph structure.
\begin{definition}
    \label{def:unlabeled_graph:homomorphism}
    Let \( G \) and \( H \) be unlabeled graphs. A \textbf{homomorphism of unlabeled graphs} $h: G \to H$ is a pair of functions $h_0: G_0 \to H_0 $ and $h_1: G_1 \to H_1$ such that for every edge \( a: s \to t \) in \( G \), we have \( h_1(a) : h_0(s) \to h_0(t) \) in \( H \).
\end{definition}


\begin{notation}
    We use the notation from~\cite[Notation 1]{overbeek2023apbpotutorial} to visualize edge-labeled graph homomorphisms. Labeled graphs are enclosed in boxes with their names displayed in the top-left corner. Nodes
     and edges are assigned subsets of \(\mathbb{N}\) which are chosen such that: (i) each node or edge \( y \) in the codomain graph is assigned the union of subsets of all nodes or edges in the domain graph that are mapped to \( y \); (ii) the graph homomorphism is uniquely determined by this assignment. To further improve readability, we represent sets by listing their elements. Additionally, we omit identifiers when doing so does not cause confusion. This is illustrated in the following representation of a homomorphism \( h: G \to H \).
    
   \begin{center}
        \resizebox{0.5\textwidth}{!}{
        \begin{tikzpicture}
            \graphbox{\( G \)}{00mm}{-20mm}{45mm}{25mm}{2mm}{-10mm}{
                \coordinate (o) at (-5mm,-8mm); 
                \node[draw,circle] (l1) at ($(o)+(-10mm,0mm)$) {1};
                \node[draw,circle] (l2) at ($(l1)+(3,0)$) {2};
                \node[draw,circle] (l3) at ($(l1)+(1,0)$) {3};
                \node[draw,circle] (l4) at ($(l1)+(2,0)$) {4};
                \draw[->] (l1) -- (l3) node[midway,above] {a};
                \draw[->] (l3) -- (l4) node[midway,above] {b};
                \draw[->] (l4) -- (l2) node[midway,above] {a};
            }  
            \graphbox{\( H \)}{52mm}{-20mm}{50mm}{25mm}{2mm}{-10mm}{
                \coordinate (o) at (-5mm,-8mm); 
                \node[draw,circle] (l1) at ($(o)+(-1,0mm)$) {1};
                \node[draw,circle] (l2) at ($(l1)+(3,0)$) {2};
                \node[draw,circle] (l3) at ($(l1)+(1.5,0)$) {3\ 4};
                \draw[->] (l1) edge node[midway,above] {a} (l3);
                \draw[->] (l3) edge [loop above] node[midway,above] {b} (l3) ;
                \draw[->] (l3) -- (l2) node[midway,above] {a};
            }      
            \node () at (48mm,-30mm) {$\rightarrow$};
        \end{tikzpicture}
    }
    \end{center}  
    In this example, the sets \(\{1\}\), \(\{2\}\), \(\{3\}\), \(\{4\}\), and \(\{3,4\}\) are represented as \(1\), \(2\), \(3\), \(4\), and \(3\ 4\), respectively. Edge identifiers are omitted.
\end{notation}
\begin{definition} 
    \label{def:graph}
     Let \(\Sigma\) be a finite set of labels. A \textbf{directed edge-labeled multigraph} is a structure \((V, E, \opn{dom}, \opn{cod},\lambda)\) where
    \begin{itemize}
        \item $V$ is a collection of distinct elements called \textbf{nodes} (or \textbf{objects}),
        \item $E$ is a collection of distinct elements, disjoint from $V$, called \textbf{edges} (or \textbf{arrows}),
        \item $\opn{dom}: E \to V$ is the \textbf{domain} (or \textbf{source}) function assigning to each edge its source node,
        \item $\opn{cod}: E \to V$ is the \textbf{codomain} (or \textbf{target}) function assigning to each edge its target node,
        \item $\lambda: E \to \Sigma$ is the \textbf{labeling} function assigning to each edge a label from $\Sigma$.
    \end{itemize}
    A directed edge-labeled multigraph is said to be \textbf{finite} if $V$ and $E$ are finite, and \textbf{discrete} if its edge set \(E\) is empty.
    For a directed edge-labeled multigraph \( G \), we write \( V(G) \) for its set of nodes, \( E(G) \) for its set of edges, \( \opn{dom}_G \) for its domain function, \( \opn{cod}_G \) for its codomain function, and \( \lambda_G \) for its labeling function. $a : s\overset{l}{\rightarrow} t$ denotes an arrow $a$ labeled by $l$ from $s$ to $t$.
\end{definition}
A directed edge-labeled multigraph may have multiple edges from one node to another, as well as loops (edges from a node to itself), because different edges $u,v$ can have the same source and target nodes, i.e. $\opn{dom}(u) = \opn{dom}(v)$ and $\opn{cod}(u) = \opn{cod}(v)$.

    Throughout, a directed edge-labeled multigraph will be simply referred to as a \enquote{labeled graph} or \enquote{graph} when the context makes it clear, and we fix a set $\Sigma$ of labels.

% \begin{definition} 
%     \label{def:graph}
%     Let \(\Sigma\) be a finite set of labels. A \textbf{directed edge-labeled multigraph} is an ordered pair \((G,\lambda)\) where \( G \) is an unlabeled graph and \( \lambda : E(G) \rightarrow \Sigma\) is an edge-labeling function. 
%     It is called \textbf{finite} if its underlying unlabeled graph is finite.
%     Throughout, a directed edge-labeled multigraph will be simply referred to as a \enquote{labeled graph} when the context makes it clear.
%     An arrow $a$ labeled by $l$ from $s$ to $t$ is denoted by $a : s\overset{l}{\rightarrow} t$.
% \end{definition}
%  The definition of labeled graphs extends the definition of unlabeled graphs, as unlabeled graphs can be seen as labeled graphs with a single label for all edges, i.e., the label function is a constant function.


\begin{example}
    Consider the graph shown in~\autoref{fig:preliminaries:labeled_graph}.
    It has two nodes which are marked with numbers to facilitate discussion, but in general, nodes are not labeled.
    There are five edges: two loops on node $1$ labeled \(a\) and \(b\), two edges from node $1$ to node $2$ both labeled \(a\), and one edge from node $2$ to node $1$ labeled \(a\). Note that the two edges from node $1$ to node $2$ are distinct edges, even though they share the same source, target, and label.

    \begin{figure}[H]
       \centering
       \resizebox{0.4\textwidth}{!}{
        \begin{tikzpicture}
            \graphbox{}{0mm}{0mm}{32mm}{28mm}{-10mm}{-14mm}{
                \node[draw,circle] (1) at (0,0) {1};
                \node[draw,circle] (2) at (2,0) {2};
                \draw[->] (1) edge[loop above] node[midway, above] {$a$} (1) ;
                \draw[->] (1) edge[loop below] node[midway, below] {$b$} (1) ;
                \draw[->] (1) edge[bend left] node[midway, above] {$a$}  (2)  ;
                \draw[->] (1) edge node[midway, above] {$a$}  (2)  ;
                \draw[->] (2) edge[bend left] node[midway, below] {$a$} (1)   ;
            }
        \end{tikzpicture}
       }
        \caption{Labeled graph.}
        \label{fig:preliminaries:labeled_graph}
    \end{figure}

\end{example}
  A homomorphism of labeled graphs is a homomorphism of unlabeled graphs that preserves the labels assigned to the edges. 
\begin{definition}
    \label{def:graph:homomorphism}
    Let \( (G,\lambda) \) and \( (H,\lambda') \) be labeled graphs. A \textbf{homomorphism of labeled graphs} $h:(G,\lambda) \rightarrow (H,\lambda')$ is a homomorphism of unlabeled graphs such that for each edge \( a \) in \( G \), we have \( \lambda (a) = \lambda' (h_E (a)) \).
\end{definition}

\begin{example}
    An example of a homomorphism of labeled graphs is shown below where nodes 3 and 4 of the graph $G$ are mapped to the middle node of the graph $H$.
    \begin{center}
        \resizebox{0.5\textwidth}{!}{
        \begin{tikzpicture}
            \graphbox{\( G \)}{00mm}{-20mm}{45mm}{25mm}{2mm}{-10mm}{
                \coordinate (o) at (-5mm,-8mm); 
                \node[draw,circle] (l1) at ($(o)+(-10mm,0mm)$) {1};
                \node[draw,circle] (l2) at ($(l1)+(3,0)$) {2};
                \node[draw,circle] (l3) at ($(l1)+(1,0)$) {3};
                \node[draw,circle] (l4) at ($(l1)+(2,0)$) {4};
                \draw[->] (l1) -- (l3) node[midway,above] {a};
                \draw[->] (l3) -- (l4) node[midway,above] {b};
                \draw[->] (l4) -- (l2) node[midway,above] {a};
            }  
            \graphbox{\( H \)}{52mm}{-20mm}{50mm}{25mm}{2mm}{-10mm}{
                \coordinate (o) at (-5mm,-8mm); 
                \node[draw,circle] (l1) at ($(o)+(-1,0mm)$) {1};
                \node[draw,circle] (l2) at ($(l1)+(3,0)$) {2};
                \node[draw,circle] (l3) at ($(l1)+(1.5,0)$) {3\ 4};
                \draw[->] (l1) edge node[midway,above] {a} (l3);
                \draw[->] (l3) edge [loop above] node[midway,above] {b} (l3) ;
                \draw[->] (l3) -- (l2) node[midway,above] {a};
            }      
            \node () at (48mm,-30mm) {$\overset{h}{\rightarrow}$};
        \end{tikzpicture}
    }
    \end{center} 
\end{example}

 
\subsection{Pushout and pullback}  
Pushouts play a central role in the double-pushout approach to graph rewriting considered in this thesis. The concepts and notation in this section follow the treatments of Pierce~\cite{pierce1991basic} and Barr and Wells~\cite{barr1990category}.
\begin{definition}
    \label{def:cat}
    A \textbf{category}\index{Category} is an unlabeled graph \( C \) together with a total function \( u : V(C)  \mathop{\to} E(C) \) and a partial function \( \star: E(C) \mathop{\times} E(C)  \mathop{\to} E(C) \) such that 
        \begin{itemize}
            \item for all edges \( f:X  \mathop{\to} Y \) and \( g:Y  \mathop{\to} Z \), the edge \( f \mathop{\star} g :X  \mathop{\to} Z \) is defined; 
            \item  for every node \( X \), \( u(X) \) is an edge from \( X \) to \( X \); 
            \item for every \( f:X  \mathop{\to} Y \), we have \(u(X) \mathop{\star} f \mathop{=} f \mathop{=} f \mathop{\star} u(Y)\);
            \item for all edges \( f \), \( g \) and \(h\), we have \( (f \mathop{\star} g) \mathop{\star} h \mathop{=} f \mathop{\star} (g \mathop{\star} h) \) whenever either side is defined.
        \end{itemize}
    Edges are called \textbf{morphisms}\index{Morphism}. The function $\star$ is called \textbf{composition}\index{Composition}. For all \( X \mathop{\in} V(C) \), the edge \( u(X) \) is denoted by \( \operatorname{id}_X \) and is called the \textbf{identity}~\index{Indentity} of the object \( X \).
    % \( C \) is called the \textbf{underlying graph} of the category \( \mathcal{C} \).
\end{definition}    
\begin{definition}
    A category \(\mathcal{C}\) is said to be \textbf{locally small}\index{Category!locally small} if for all objects \(X,Y\) in \(\mathcal{C}\), the collection $\opn{Hom}(X,Y)$\index{hom(@$\opn{Hom}(X,Y)$} of morphisms from \(X\) to \(Y\) is a set (called a \textbf{hom-set})~\index{Hom-set}. For a locally small category, $\opn{Mono}(X,Y)$\index{mono(@$\opn{Mono}(X,Y)$} denotes the set of all monomorphisms from $X$ to $Y$.
\end{definition}
\textbf{Throughout this section, fix a locally small category \( \mathcal{C} \).}
\begin{example}
    Consider the unlabeled graph shown below.
    %  in Figure~\ref{fig:preliminaries:category}. 
     It can be considered as a category where the objects are the nodes and the morphisms are the paths between nodes; composition is path concatenation. The identity of a node is the self-loop of the node. There are at least three morphisms from the left node to itself: the identity morphism (the self-loop), the path that traverses the self-loop twice, and the path that goes to the right node and back. 
        % \begin{figure}[H]
        % \centering
        \begin{center}
            \resizebox{0.4\textwidth}{!}{
        \begin{tikzpicture}
            \graphbox{}{0mm}{0mm}{32mm}{16mm}{-10mm}{-9mm}{
                \node[draw,circle] (1) at (0,0) {};
                \node[draw,circle] (2) at (2,0) {};
                \draw[->] (1) edge[loop above] node[midway, above] { } (1) ;
                \draw[->] (2) edge[loop above] node[midway, above] { } (2) ;
                \draw[->] (1) edge[bend left] node[midway, above] {}  (2)  ;
                \draw[->] (2) edge[bend left] node[midway, below] {} (1)   ;
            }
        \end{tikzpicture}
            }
    %     \caption{}
    %     \label{fig:preliminaries:category}
    % \end{figure}
        \end{center}
\end{example}

% cat notation * 
\begin{notation}
    The composition of morphisms \( f : X  \mathop{\to} Y \) and \( g : Y  \mathop{\to} Z \) is written in diagrammatic order as \( f \mathop{\star} g \), rather than in functional order \( g \circ f \) as is common in litterature. The advantage is that, when reading from left to right, the morphisms appear in the same order as in the corresponding diagram, making the reasoning accompanying diagrams more intuitive. 
\end{notation}  

\begin{definition} 
    \label{def:cat:homo}
    A morphism \( f : X  \mathop{\to} Y \) is said to be a \textbf{monomorphism}\index{Monomorphism} (is \textbf{monic}\index{Monic}) if given any morphisms \( g,h: Z  \mathop{\to} X  \), \( g \mathop{\star} f \mathop{=} h \mathop{\star} f \) implies \( g \mathop{=} h \). 
    In this case, we write $f : X \rightarrowtail Y$\index{a@$\rightarrowtail$} to indicate that $f$ is a monomorphism.
\end{definition} 

When visualizing a monomorphism, we often use $\rightarrowtail$ instead of $\to$ to emphasize that it is monic. For example, a monomorphism of labeled graphs can be represented as follows:
\begin{center}
        \resizebox{0.7\textwidth}{!}{
        \begin{tikzpicture}
            \graphbox{$K$}{40mm}{0mm}{24mm}{15mm}{2mm}{-5mm}{
                \coordinate (o) at (5mm,-3mm); 
                \node[draw,circle] (l1) at ($(o)+(-10mm,0mm)$) {1};
                \node[draw,circle] (l2) at ($(l1)+(1,0)$) {2};
            }    
            \graphbox{$R$}{70mm}{0mm}{45mm}{15mm}{2mm}{-5mm}{
                \coordinate (o) at (-5mm,-3mm); 
                \node[draw,circle] (l1) at ($(o)+(-10mm,0mm)$) {1};
                \node[draw,circle] (l2) at ($(l1)+(3,0)$) {2};
                \node[draw,circle] (l3) at ($(l1)+(1,0)$) {4};
                \node[draw,circle] (l4) at ($(l1)+(2,0)$) {5};
                \draw[->] (l1) -- (l3) node[midway,above] {$a$};
                \draw[->] (l3) -- (l4) node[midway,above] {$b$};
                \draw[->] (l4) -- (l2) node[midway,above] {$a$};
            }    
            \node () at (67mm,-8mm) {$\rightarrowtail$};
        \end{tikzpicture}
        }
    \end{center}

\begin{example}
    \index{set@\(\mathbf{Set}\)}\index{Category!set}
The category \(\mathbf{Set}\) has sets as objects and total functions between them as morphisms. For \(f\mathop{\colon} A \mathop{\to} B\) and \(g\mathop{\colon} B \mathop{\to} C\), composition is given by \(g\circ f\), and the identity morphism on a set \(A\) is the identity function \(\mathrm{id}_A\).
\end{example}

\begin{example} 
    Finite labeled graphs and their homomorphisms form a category, hereafter denoted by \textbf{Graph}\index{graph@\textbf{Graph}}\index{Category!graph}. Its objects are labeled graphs, its morphisms are graph homomorphisms, and the monomorphisms are homomorphisms. 
    $\textbf{Graph}$ is locally small. 
\end{example}

% \begin{definition}[Span \cite{lowe2010graph}]
%     A pair \( (\alpha : A  \mathop{\to} B,~\beta : A  \mathop{\to} C) \) of morphisms with a common domain is called a \textbf{span}, denoted by \( B \overset{\alpha}{\leftarrow} A \overset{\beta}{\rightarrow} C \).
% \end{definition}
 
% \begin{definition}[Cospan]
%     A pair \( (\beta' : B  \mathop{\to} D,~\alpha' : C  \mathop{\to} D) \) of morphisms with a common codomain is called a \textbf{cospan}, denoted by \( B \overset{\beta'}{\rightarrow} D \overset{\alpha'}{\leftarrow} C \). 
% \end{definition} 
A span (resp. cospan) is a couple of morphisms with a common domain (resp. codomain).
\begin{definition}
An ordered pair \((\alpha : A  \mathop{\to} B,\, \beta : A  \mathop{\to} C)\) of morphisms with a common domain is called a \textbf{span}\index{Span} \cite{lowe2010graph}, denoted by
\(
B \overset{\alpha}{\leftarrow} A \overset{\beta}{\rightarrow} C
\). 
% An example of a span $(\alpha, \beta)$ is shown below.
Likewise, an ordered pair \((\beta' : B  \mathop{\to} D,\, \alpha' : C  \mathop{\to} D)\) of morphisms with a common codomain is called a \textbf{cospan}\index{Cospan}, denoted by
\(
B \overset{\beta'}{\rightarrow} D \overset{\alpha'}{\leftarrow} C
\). 
\end{definition}
\begin{example}
Consider the diagram below in the category \textbf{Graph}, where the numbers inside nodes and the subgraphs in different colors illustrate how the morphisms map nodes and edges. $(\alpha, \beta)$ is a span, and $(\beta', \alpha')$ is a cospan.


\begin{center}
        \resizebox{0.8\textwidth}{!}{
        \begin{tikzpicture} 
            \graphbox{\( L \)}{40mm}{20mm}{34mm}{12mm}{2mm}{2mm}{
                \coordinate (o) at (0mm,-8mm); 
                \node[draw,circle] (l1) at ($(o)+(-10mm,0mm)$) {1};
                \node[draw,circle] (l2) at ($(l1)+(2,0)$) {2};
                \node[draw,circle,red] (l3) at ($(l1)+(1,0)$) {3};
                \draw[->,red] (l1) -- (l3) node[midway,above] {$a$};
                \draw[->,red] (l3) -- (l2) node[midway,above] {$a$};
            } 
    
            \graphbox{\( K \)}{0mm}{0mm}{34mm}{12mm}{2mm}{2mm}{
                \coordinate (o) at (0mm,-8mm); 
                \node[draw,circle] (l1) at ($(o)+(-10mm,0mm)$) {1};
                \node[draw,circle] (l2) at ($(l1)+(2,0)$) {2};
            }  
            \graphbox{\(G\)}{90mm}{5mm}{34mm}{24mm}{2mm}{-3mm}{
                \coordinate (o) at (0mm,-5mm); 
                \node[draw,circle] (l1) at ($(o)+(-10mm,0mm)$) {1};
                \node[draw,circle] (l2) at ($(l1)+(2,0)$) {2};
                \node[draw,circle,red] (l3) at ($(l1)+(1,0)$) {3};
                \node[draw,circle,blue] (l4) at ($(l2)+(0,-1)$) {6};
                \draw[->,red] (l1) -- (l3) node[midway,above] {$a$};
                \draw[->,red] (l3) -- (l2) node[midway,above] {$a$};
                \draw[->,blue] (l2) -- (l4) node[midway,right] {$a$};
                \node[draw,circle,blue] (l6) at ($(l1)+(0,-1)$) {7};
                \draw[<-,blue] (l1) -- (l6) node[midway,left] {$a$};
                \draw[->,blue] (l2) edge[out=-135,in=-45]node[midway,below] {$a$} (l1) ;
            }   
     
            \graphbox{\( C \)}{40mm}{-20mm}{34mm}{24mm}{2mm}{-3mm}{
                \coordinate (o) at (0mm,-5mm); 
                \node[draw,circle] (l1) at ($(o)+(-10mm,0mm)$) {1};
                \node[draw,circle] (l2) at ($(l1)+(2,0)$) {2};
                \node[draw,circle,blue] (l4) at ($(l2)+(0,-1)$) {6};
                \draw[->,blue] (l2) -- (l4) node[midway,right] {$a$};
                \draw[->,blue] (l2) edge[out=-135,in=-45]node[midway,below] {$a$} (l1) ;
                \node[ draw,circle,blue] (l6) at ($(l1)+(0,-1)$) {7};
                \draw[<-,blue] (l1) -- (l6) node[midway,left] {$a$};
            }      
            % K to L
            \draw[->] (17mm,5mm) -- node[above] {$\alpha$} (37mm,15mm);
            % C to G
            \draw[->] (76mm,-28mm)-- node[below] {$\alpha'$} (104mm,-21mm) ;
            % K to C
            \draw[->] (17mm,-17mm) -- node[below] {$\beta$} (37mm,-28mm);
            % L to G
            \draw[->] (76mm,16mm) -- node[above] {$\beta'$} (104mm,7mm);
            % \node () at (57mm,-6mm) {$PO$};
        \end{tikzpicture}
        }
    \end{center}
\end{example}

\begin{definition}[\cite{barr1990category}]
    \label{def:cat:diagram}
    Let $\mathcal{C}$ be a category, and \( G \) an unlabeled graph. A \textbf{diagram}\index{Diagram} (of shape \( G \)) is a homomorphism of unlabeled graphs \( h : G  \mathop{\to} \mathcal{C} \) where \( \mathcal{C} \) is considered as an unlabeled graph. A diagram is said to be \textbf{commutative}\index{Diagram!commutative} if, for all nodes \( u \), \( v \), and any two paths from \( u \) to \( v \) in the unlabeled graph \( G \):

    \begin{center}
    \resizebox{12cm}{!}{
        \begin{tikzpicture}
        \node (u) at (0,0) {\( u \)};
        \node (k1) at (2,0.5) {\( k_1 \)};
        \node (k2) at (4,0.5) {\( k_2 \)};
        \node (ketc) at (6,0.5) {\( \dots \)};
        \node (knm2) at (8,0.5) {\( k_{n-2} \)};
        \node (knm1) at (10,0.5) {\( k_{n-1} \)};
        \node (v) at (12,0) {\( v \)};
        \node (l1) at (2,-0.5) {\( l_1 \)};
        \node (l2) at (4,-0.5) {\( l_2 \)};
        \node (letc) at (6,-0.5) {\( \dots \)};
        \node (lnm2) at (8,-0.5) {\( l_{m-2} \)};
        \node (lnm1) at (10,-0.5) {\( l_{m-1} \)};
        \draw[->] (u) -- (k1) node [midway,above] {\( s_1 \)};
        \draw[->] (k1) -- (k2) node [midway,above] {\( s_2 \)};
        \draw[->] (k2) -- (ketc);
        \draw[->] (ketc) -- (knm2); 
        \draw[->] (knm2) -- (knm1) node[midway,above] {\( s_{n-2} \)}; 
        \draw[->] (knm1) -- (v) node[midway,above] {\( s_{n-1} \)}; 
        \draw[->] (u) -- (l1) node[midway,below] {\( t_1 \)};
        \draw[->] (l1) -- (l2) node[midway,below] {\( t_2 \)};
        \draw[->] (l2) -- (letc);
        \draw[->] (letc) -- (lnm2); 
        \draw[->] (lnm2) -- (lnm1) node[midway,below] {\( t_{m-2} \)}; 
        \draw[->] (lnm1) -- (v) node[midway,below] {\( t_{m-1} \)}; 
        \end{tikzpicture}
    }
    \end{center}
    \noindent
    the equality \( h(s_1) \mathop{\star} h(s_2) \mathop{\star} \dots  \mathop{\star} h(s_{n-1}) \mathop{=} h(t_1) \mathop{\star} h(t_2) \mathop{\star} \dots  \mathop{\star} h(t_{m-1}) \) holds.
\end{definition}

\begin{example}
    A commutative diagram in the category \textbf{Graph} of finite, directed, edge-labeled multigraphs is illustrated below. The numbers inside nodes and the subgraphs in different colors illustrate how the morphisms map nodes and edges. 
     The symbol $\mathop{=}$ in the center of the diagram
    is used to indicate that the diagram is commutative,
    i.e. the composition of morphisms along every path from node 1 to node 2 is the same.
    \begin{center}
        \resizebox{0.8\textwidth}{!}{
        \begin{tikzpicture} 
            \graphbox{\( L \)}{40mm}{20mm}{34mm}{12mm}{2mm}{2mm}{
                \coordinate (o) at (0mm,-8mm); 
                \node[draw,circle] (l1) at ($(o)+(-10mm,0mm)$) {1};
                \node[draw,circle] (l2) at ($(l1)+(2,0)$) {2};
                \node[draw,circle,red] (l3) at ($(l1)+(1,0)$) {3};
                \draw[->,red] (l1) -- (l3) node[midway,above] {$a$};
                \draw[->,red] (l3) -- (l2) node[midway,above] {$a$};
            } 
    
            \graphbox{\( K \)}{0mm}{0mm}{34mm}{12mm}{2mm}{2mm}{
                \coordinate (o) at (0mm,-8mm); 
                \node[draw,circle] (l1) at ($(o)+(-10mm,0mm)$) {1};
                \node[draw,circle] (l2) at ($(l1)+(2,0)$) {2};
            }  
            \graphbox{\(G\)}{90mm}{5mm}{34mm}{24mm}{2mm}{-3mm}{
                \coordinate (o) at (0mm,-5mm); 
                \node[draw,circle] (l1) at ($(o)+(-10mm,0mm)$) {1};
                \node[draw,circle] (l2) at ($(l1)+(2,0)$) {2};
                \node[draw,circle,red] (l3) at ($(l1)+(1,0)$) {3};
                \node[draw,circle,blue] (l4) at ($(l2)+(0,-1)$) {6};
                \draw[->,red] (l1) -- (l3) node[midway,above] {$a$};
                \draw[->,red] (l3) -- (l2) node[midway,above] {$a$};
                \draw[->,blue] (l2) -- (l4) node[midway,right] {$a$};
                \node[draw,circle,blue] (l6) at ($(l1)+(0,-1)$) {7};
                \draw[<-,blue] (l1) -- (l6) node[midway,left] {$a$};
                \draw[->,blue] (l2) edge[out=-135,in=-45]node[midway,below] {$a$} (l1) ;
            }   
     
            \graphbox{\( C \)}{40mm}{-20mm}{34mm}{24mm}{2mm}{-3mm}{
                \coordinate (o) at (0mm,-5mm); 
                \node[draw,circle] (l1) at ($(o)+(-10mm,0mm)$) {1};
                \node[draw,circle] (l2) at ($(l1)+(2,0)$) {2};
                \node[draw,circle,blue] (l4) at ($(l2)+(0,-1)$) {6};
                \draw[->,blue] (l2) -- (l4) node[midway,right] {$a$};
                \draw[->,blue] (l2) edge[out=-135,in=-45]node[midway,below] {$a$} (l1) ;
                \node[ draw,circle,blue] (l6) at ($(l1)+(0,-1)$) {7};
                \draw[<-,blue] (l1) -- (l6) node[midway,left] {$a$};
            }      
            % K to L
            \draw[->] (17mm,5mm) -- node[above] {$\alpha$} (37mm,15mm);
            % C to G
            \draw[->] (76mm,-28mm)-- node[below] {$\alpha'$} (104mm,-21mm) ;
            % K to C
            \draw[->] (17mm,-17mm) -- node[below] {$\beta$} (37mm,-28mm);
            % L to G
            \draw[->] (76mm,16mm) -- node[above] {$\beta'$} (104mm,7mm);
            \node () at (57mm,-6mm) {$\mathop{=}$};
        \end{tikzpicture}
        }
    \end{center}
\end{example}

\begin{notation}   
    When the context makes it clear, \( h_{AB} \) denotes a morphism \( h : A  \mathop{\to} B \), and we refer to diagrams by listing their nodes, as is standard in geometry. 
    % For example, the diagram shown in Definition~\ref{def:cat:po} is denoted by \( ACDB \) or \( ABDC \).
\end{notation}   

The pushout is a construction in category theory that can often be thought of as the construction of a new structure from two given structures by gluing them along a common interface structure.
\begin{definition}
    \label{def:cat:po} 
    A \textbf{pushout}\index{Pushout} of a span \( B \overset{\alpha}{\leftarrow} A \overset{\beta}{\rightarrow} C \), shown in the following diagram,
    %  in Figure~\ref{fig:preliminaries:pushout_sdfkjasdlgjfl}
    is defined as a cospan \( B \overset{\beta'}{\rightarrow} D \overset{\alpha'}{\leftarrow} C \) such that the following conditions hold:
    \begin{itemize}
        \item \( \alpha \mathop{\star} \beta' \mathop{=} \beta \mathop{\star} \alpha' \),
        \item for every cospan \( B \overset{\gamma'}{\rightarrow} E \overset{\gamma}{\leftarrow} C \), if \( \alpha \mathop{\star} \gamma' \mathop{=} \beta \mathop{\star} \gamma \) holds, then there is a unique morphism \(\delta : D  \mathop{\to} E\) such that \( \gamma' \mathop{=} \beta' \mathop{\star} \delta \) and \( \gamma \mathop{=} \alpha' \mathop{\star} \delta \).
    \end{itemize} 
    \begin{center}
        \resizebox{0.45\textwidth}{!}{
            \begin{tikzpicture}
                    \node (i) at (0,0) {A};
                    \node (r) at (1,1) {B};
                    \node (c) at (1,-1) {C};
                    \node (h) at (2,0) {D};
                    % \node () at (1,-1) {\( \Delta \)};
                    \draw[->]  (i) -- (r) node [midway,left] {$ \alpha $};
                    \draw[->] (c) -- (h) node [midway,left] {$ \alpha' $};
                    \draw[->] (r) -- (h) node[midway, left] {$ \beta' $};
                    \draw[->] (i) -- (c) node[midway, left] {$ \beta $};
                    \node (d') at (4,0) {E};
                    \draw[->] (c) -- (d') node [midway,below]{$ \gamma $};
                    \draw[->] (r) -- (d') node [midway,above]{$ \gamma' $};
                    \draw[->,dashed] (h) -- (d') node [midway]{$ \delta $};
                \end{tikzpicture}
        }
            \end{center}
The diagram involving \( (\alpha, \beta, \alpha', \beta') \) is called a \textbf{pushout square}\index{Pushout!square}, or simply a \textbf{pushout}, with \(D\) as the \textbf{pushout object}\index{Pushout!object}. The existence of a unique morphism is known as the \textbf{universal mapping property of the pushout}\index{Pushout!universal mapping property}.
\end{definition} 

\begin{example}
    \label{ex:cat:posfjsdlkgja}
     Pushouts of a span always exist in \(\mathbf{Set}\), and (up to isomorphism) can be described as follows. Let
    \( B \overset{\alpha}{\leftarrow} A \overset{\beta}{\rightarrow} C \) be a span. Its pushout is the cospan \( B \overset{\beta'}{\rightarrow} D \overset{\alpha'}{\leftarrow} C \) where the pushout object of \((\alpha,\beta)\) is the quotient set
    \[
    D \;=\; (B\mathop{+}C)/{\sim}
    \]
    where \(B\mathop{+}C\) denotes the disjoint union of $B$ and $C$ and \(\sim\) is the smallest equivalence relation that includes \(\set{(\alpha(a),\beta(b))\mid a \mathop{\in} A }\). The maps
    \(\beta' \mathop{\colon} B \mathop{\to} D\) and \(\alpha' \mathop{\colon} C \mathop{\to} D\) send each element to its equivalence class.

    For example, consider the functions \(\alpha\) and \(\beta\) in the category \(\mathbf{Set}\) in the diagram, illustrated in Figure~\ref{fig:preliminaries:a_rewriting_step_dfjalsdkjflg}.
    In this diagram, each set is drawn as a box and its elements are represented by circles. The numbers inside circles indicate how the functions map those elements.
    \begin{figure}[H]
      \centering 
      \resizebox{0.6\textwidth}{!}{
      \begin{tikzpicture}
          \graphbox{\( A\)}{40mm}{-3mm}{34mm}{12mm}{2mm}{2mm}{
              \coordinate (o) at (0mm,-8mm); 
              \node[draw,circle] (l1) at ($(o)+(-10mm,0mm)$) {1};
              \node[draw,circle] (l2) at ($(l1)+(2,0)$) {2};
          }  
          \graphbox{\( B \)}{80mm}{-3mm}{45mm}{12mm}{2mm}{2mm}{
              \coordinate (o) at (-5mm,-8mm); 
              \node[draw,circle] (l1) at ($(o)+(-10mm,0mm)$) {1};
              \node[draw,circle] (l2) at ($(l1)+(3,0)$) {2};
              \node[draw,circle] (l3) at ($(l1)+(1,0)$) {4};
          }     
          \graphbox{\( C  \)}{40mm}{-22mm}{34mm}{22mm}{2mm}{-3mm}{
              \coordinate (o) at (0mm,-3mm); 
              \node[draw,circle] (l1) at ($(o)+(-10mm,0mm)$) {1};
              \node[draw,circle] (l2) at ($(l1)+(2,0)$) {2};
              \node[ draw,circle] (l6) at ($(l1)+(0,-1)$) {3};
          }    
          \graphbox{\( D \)}{80mm}{-22mm}{45mm}{22mm}{2mm}{-3mm}{
              \coordinate (o) at (-5mm,-3mm); 
              \node[draw,circle] (l1) at ($(o)+(-10mm,0mm)$) {1};
              \node[draw,circle] (l2) at ($(l1)+(3,0)$) {2};
              \node[draw,circle] (l3) at ($(l1)+(1,0)$) {4};
              \node[ draw,circle] (l6) at ($(l1)+(0,-1)$) {3};
          }    
          \node () at (77mm,-8mm) {\( \overset{\alpha}{\rightarrow} \)}; % K -> R
          \node () at (58mm,-18mm) {\( \beta\downarrow \)};
          \node () at (102mm,-18mm) {\( \beta'\downarrow \)};
          \node () at (77mm,-33mm) {\( \overset{\alpha'}{\rightarrow} \)}; % C -> H
      \end{tikzpicture}
      }
      \caption{}
      \label{fig:preliminaries:a_rewriting_step_dfjalsdkjflg}
  \end{figure}
    Throughout this example, an element labeled \(n\) in a set \(X\) is denoted by \(n_X\) to avoid ambiguity.        
    The binary relation \(\sim\) is the reflexive, symmetric and transitive closure of the binary relation $\{(1_B,1_C),(2_B,2_C)\}$.
 
    The disjoint union of \(B\) and \(C\) is
    \[ 
    D' \mathop{=} \{1_B,2_B,4_B,1_C,2_C,3_C\},
    \]
    and the quotient set is
    \[
    D'/\sim \mathop{=} \{[1_B],[2_B],[3_C],[4_B]\},
    \]
    where $[x]$ denotes the equivalence class of the element \(x\).
    We have \([1_C]=[1_B]\) and \([2_C]=[2_B]\), and the maps
    \(\beta'' \mathop{\colon} B \mathop{\to} D'\) and \(\alpha'' \mathop{\colon} C \mathop{\to} D'\) send each element to its equivalence class. Note that \(\{[1_B],[2_B],[3_C],[4_B]\}\) is isomorphic to \(D\), shown in Figure~\ref{fig:preliminaries:a_rewriting_step_dfjalsdkjflg}, which is expected because the pushout of a span is unique up to isomorphism.
\end{example}

\begin{proposition}{\cite[p.188]{corradini1997algebraic}}
    \label{prop:pushout_graph_always_exists}
    In category \textbf{Graph}, the pushout of two arrows always exists: It can be computed componentwise (as a pushout in \textbf{Set}) for the nodes and for the edges, and the source, target, and labeling mappings are uniquely determined.
\end{proposition}

\begin{example}
    The diagram in the category \textbf{Graph} shown below
    %  in Figure~\ref{fig:preliminaries:pushout_injective} 
     is a pushout square. The numbers inside nodes and the subgraphs in different colors illustrate how the morphisms map nodes and edges. In this example, both $\alpha$ and $\beta$ are injective morphisms. Therefore, the pushout object $G$ can be constructed easily by taking the interface graph $K$ and adding elements from $L$ and $C$ which are not present in $K$.
    % \begin{figure}[H]
    %     \centering
    \begin{center}
        \resizebox{0.8\textwidth}{!}{
        \begin{tikzpicture} 
            \graphbox{\( L \)}{40mm}{15mm}{34mm}{12mm}{2mm}{2mm}{
                \coordinate (o) at (0mm,-8mm); 
                \node[draw,circle] (l1) at ($(o)+(-10mm,0mm)$) {1};
                \node[draw,circle] (l2) at ($(l1)+(2,0)$) {2};
                \node[draw,circle,red] (l3) at ($(l1)+(1,0)$) {3};
                \draw[->,red] (l1) -- (l3) node[midway,above] {$a$};
                \draw[->,red] (l3) -- (l2) node[midway,above] {$a$};
            } 
    
            \graphbox{\( K \)}{0mm}{0mm}{34mm}{12mm}{2mm}{2mm}{
                \coordinate (o) at (0mm,-8mm); 
                \node[draw,circle] (l1) at ($(o)+(-10mm,0mm)$) {1};
                \node[draw,circle] (l2) at ($(l1)+(2,0)$) {2};
            }  
            \graphbox{\(G  \)}{90mm}{5mm}{34mm}{24mm}{2mm}{-3mm}{
                \coordinate (o) at (0mm,-5mm); 
                \node[draw,circle] (l1) at ($(o)+(-10mm,0mm)$) {1};
                \node[draw,circle] (l2) at ($(l1)+(2,0)$) {2};
                \node[draw,circle,red] (l3) at ($(l1)+(1,0)$) {3};
                \node[draw,circle,blue] (l4) at ($(l2)+(0,-1)$) {6};
                \draw[->,red] (l1) -- (l3) node[midway,above] {$a$};
                \draw[->,red] (l3) -- (l2) node[midway,above] {$a$};
                \draw[->,blue] (l2) -- (l4) node[midway,right] {$a$};
                \node[draw,circle,blue] (l6) at ($(l1)+(0,-1)$) {7};
                \draw[<-,blue] (l1) -- (l6) node[midway,left] {$a$};
                \draw[->,blue] (l2) edge[out=-135,in=-45]node[midway,below] {$a$} (l1) ;
            }   
     
            \graphbox{\( C \)}{40mm}{-15mm}{34mm}{24mm}{2mm}{-3mm}{
                \coordinate (o) at (0mm,-5mm); 
                \node[draw,circle] (l1) at ($(o)+(-10mm,0mm)$) {1};
                \node[draw,circle] (l2) at ($(l1)+(2,0)$) {2};
                \node[draw,circle,blue] (l4) at ($(l2)+(0,-1)$) {6};
                \draw[->,blue] (l2) -- (l4) node[midway,right] {$a$};
                \draw[->,blue] (l2) edge[out=-135,in=-45]node[midway,below] {$a$} (l1) ;
                \node[ draw,circle,blue] (l6) at ($(l1)+(0,-1)$) {7};
                \draw[<-,blue] (l1) -- (l6) node[midway,left] {$a$};
            }      
            % K to L
            \draw[>->] (17mm,5mm) -- node[above] {$\alpha$} (37mm,10mm);
            % C to G
            \draw[>->] (76mm,-28mm)-- node[below] {$\alpha'$} (104mm,-21mm) ;
            % K to C
            \draw[>->] (17mm,-17mm) -- node[below] {$\beta$} (37mm,-28mm);
            % L to G
            \draw[>->] (76mm,10mm) -- node[above] {$\beta'$} (104mm,7mm);
            \node () at (57mm,-6mm) {$PO$};
        \end{tikzpicture}
        }
    \end{center}
    %     \caption{Pushout square with injective morphisms.}
    %     \label{fig:preliminaries:pushout_injective}
    % \end{figure}
\end{example}

\begin{example}
    \label{ex:cat:pushout_non_injective_ssss}
    Consider the diagram in the category \textbf{Graph} shown below, where the numbers inside nodes and the subgraphs in different colors illustrate how the morphisms map nodes and edges. 
    % The diagram 
    % in Figure~\ref{fig:preliminaries:pushout_non_injective} is a pushout square in the category \textbf{Graph}. 
    In this example, $\beta$ is not injective. Therefore, some elements are merged in the pushout object $G$.

    % \begin{figure}[H]
    %     \centering 
    \begin{center}
        \resizebox{0.8\textwidth}{!}{
        \begin{tikzpicture} 
            \graphbox{\( L \)}{40mm}{20mm}{34mm}{20mm}{2mm}{2mm}{
                \coordinate (o) at (0mm,-11mm); 
                \node[draw,circle] (l1) at ($(o)+(-10mm,0mm)$) {1};
                \node[draw,circle] (l2) at ($(l1)+(2,0)$) {2};
                \draw[->,red] (l2) edge[out=-135,in=-45]node[midway,below] {$a$} (l1) ;
                \node[draw,circle,red] (l3) at ($(l1)+(1,0)$) {3};
                \draw[->,red] (l1) -- (l3) node[midway,above] {$a$};
                \draw[->,red] (l3) -- (l2) node[midway,above] {$a$};
            } 
    
            \graphbox{\( K \)}{0mm}{0mm}{34mm}{12mm}{2mm}{2mm}{
                \coordinate (o) at (0mm,-8mm); 
                \node[draw,circle] (l1) at ($(o)+(-10mm,0mm)$) {1};
                \node[draw,circle] (l2) at ($(l1)+(2,0)$) {2};
            }  
            \graphbox{\(G  \)}{90mm}{10mm}{34mm}{40mm}{2mm}{-3mm}{
                \coordinate (o) at (0mm,-20mm); 
                 \node[draw,circle] (l1) at ($(o)+(0,0)$) {1\ 2};
                % \node[draw,circle] (l1) at ($(o)+(-10mm,0mm)$) {1};
                % \node[draw,circle] (l2) at ($(l1)+(2,0)$) {2};
                \draw[->,red] (l1) edge[loop below] node[midway, below] {$a$} (l1) ;
                \node[draw,circle,red] (l3) at ($(l1)+(0,1.4)$) {3};
                \node[draw,circle,blue] (l4) at ($(l1)+(1,-1)$) {6};
                \draw[->,red] (l1) edge[bend left] node[midway,left] {$a$} (l3);
                \draw[->,red] (l3) edge[bend left] node[midway,right] {$a$} (l1);
                \draw[->,blue] (l1) edge  node[midway,right] {$a$} (l4);
                \node[draw,circle,blue] (l6) at ($(l1)+(-1,-1)$) {7};
                \draw[<-,blue] (l1) edge node[midway,left] {$a$} (l6) ;
                % \draw[->,blue] (l1) edge[out=-135,in=-45]node[midway,below] {$a$} (l1) ;
            }   
     
            \graphbox{\( C \)}{40mm}{-13mm}{34mm}{25mm}{2mm}{-3mm}{
                \coordinate (o) at (-2mm,-6mm); 
                \node[draw,circle] (l1) at ($(o)+(0,0)$) {1\ 2};
                % \node[draw,circle] (l2) at ($(l1)+(2,0)$) {2};
                \node[draw,circle,blue] (l4) at ($(l1)+(1,-1)$) {6};
                \draw[->,blue] (l1) -- (l4) node[midway,right] {$a$};
                % \draw[->,blue] (l1) edge[loop above] node[midway, above] {$a$} (l1) ;
                \node[ draw,circle,blue] (l6) at ($(l1)+(-1,-1)$) {7};
                \draw[<-,blue] (l1) -- (l6) node[midway,left] {$a$};
            }
            % K to L  
            \draw[>->] (17mm,5mm) -- node[above] {$\alpha$} (37mm,15mm);
            % C to G
            \draw[>->] (76mm,-28mm)-- node[below] {$\alpha'$} (88mm,-24mm) ;
            % K to C
            \draw[->] (17mm,-17mm) -- node[below] {$\beta$} (37mm,-28mm);
            % L to G
            \draw[->] (76mm,16mm) -- node[above] {$\beta'$} (88mm,7mm);
            \node () at (57mm,-6mm) {$PO$};
        \end{tikzpicture}
        } 
    \end{center}
\end{example}
The \emph{pullback} is the dual construction of the pushout, and it can be thought of as construction of the interface structure along which two structures are glued together. 
\begin{definition} 
    \label{def:cat:pb}
   A \textbf{pullback}\index{Pullback} of a cospan \(B \overset{\beta'}{\rightarrow} D \overset{\alpha'}{\leftarrow} C \) 
%    , shown below,
%    in Figure~\ref{fig:preliminaries:pullback_ssdsfd},
is a span \( B \overset{\alpha}{\leftarrow} A \overset{\beta}{\rightarrow} C \) such that the following conditions hold:
\begin{itemize}
    \item  \( \alpha \mathop{\star} \beta' \mathop{=} \beta \mathop{\star} \alpha' \),
    \item for every span \( B \overset{\gamma'}{\leftarrow} E \overset{\gamma}{\rightarrow} C \) if \(\gamma' \mathop{\star} \beta' \mathop{=} \gamma \mathop{\star} \alpha'\) holds, then there is a unique morphism \(\delta: E  \mathop{\to} A\) such that $\gamma' \mathop{=} \delta \mathop{\star} \alpha$ and $\gamma \mathop{=} \delta \mathop{\star} \beta$.
\end{itemize}  
    % \begin{figure}[H]
    %     \centering
    \begin{center}
        \resizebox{0.4\textwidth}{!}{
                    \begin{tikzpicture}
                        \node (i) at (0,0) {A};
                        \node (r) at (1,1) {B};
                        \node (c) at (1,-1) {C};
                        \node (h) at (2,0) {D}; 
                        % \node () at (1,-1) {\( \Delta \)};
                        \draw[->]  (i) -- (r) node [midway,left] {$\alpha$};
                        \draw[->] (c) -- (h) node [midway,left] {$\alpha'$};
                        \draw[->] (r) -- (h) node[midway, left] {$\beta'$};
                        \draw[->] (i) -- (c) node[midway, left] {$\beta$};
                        \node (d') at (-2,0) {E};
                        \draw[<-] (c) -- (d') node [midway,below]{$\gamma$};
                        \draw[<-] (r) -- (d') node [midway,above]{$\gamma'$};
                        \draw[->, dashed] (d') -- (i) node [midway]{$\delta$};
                    \end{tikzpicture}
        }
    %     \caption{}
    %     \label{fig:preliminaries:pullback_ssdsfd}
    % \end{figure}
                \end{center}
The diagram involving \( (\alpha, \beta, \alpha', \beta') \) is called a \textbf{pullback square}, or simply a \textbf{pullback}\index{Pullback!square}, with \(A\) as the \textbf{pullback object}\index{Pullback!object}. The existence of a unique morphism is known as the \textbf{universal mapping property of the pullback}\index{Pullback!universal mapping property}.
\end{definition} 

\begin{example} 
    \label{ex:cat:pbfsdljkgjasssss}
    In the category \textbf{Set},
    the pullback 
    of a cospan \(B \overset{\beta'}{\rightarrow} D \overset{\alpha'}{\leftarrow} C \) is the span \( B \overset{\alpha}{\leftarrow} A \overset{\beta}{\rightarrow} C \) where
    the pullback object is the set $A \mathop{=} \set{(b,c)\in B \mathop{\times} C \mathop{\mid} \beta (c) \mathop{=} \alpha (b)}$; $\alpha$ and $\beta$ are defined as the corresponding projections, e.g., $\alpha((b, c)) \mathop{=} b$ and $\beta((b, c)) \mathop{=} c$.
    Consider the following diagram in the category \textbf{Set}, where 
    sets are drawn as boxes,
    circles represent elements of sets, and numbers inside circles indicate how the functions map those elements.
    \begin{center}
      \resizebox{0.7\textwidth}{!}{
      \begin{tikzpicture}
          \graphbox{\( A\)}{40mm}{-3mm}{34mm}{12mm}{2mm}{2mm}{
              \coordinate (o) at (0mm,-8mm); 
              \node[draw,circle] (l1) at ($(o)+(-10mm,0mm)$) {1};
              \node[draw,circle] (l2) at ($(l1)+(2,0)$) {2};
          }  
          \graphbox{\( B \)}{80mm}{-3mm}{45mm}{12mm}{2mm}{2mm}{
              \coordinate (o) at (-5mm,-8mm); 
              \node[draw,circle] (l1) at ($(o)+(-10mm,0mm)$) {1};
              \node[draw,circle] (l2) at ($(l1)+(3,0)$) {2};
              \node[draw,circle] (l3) at ($(l1)+(1,0)$) {4};
            %   \node[draw,circle] (l4) at ($(l1)+(2,0)$) {5};
            %   \draw[ ] (l1) -- (l3) node[midway,above] {$a$};
            %   \draw[ ] (l3) -- (l4) node[midway,above] {$b$};
            %   \draw[ ] (l4) -- (l2) node[midway,above] {$a$};
          }     
          \graphbox{\( C  \)}{40mm}{-22mm}{34mm}{22mm}{2mm}{-3mm}{
              \coordinate (o) at (0mm,-3mm); 
              \node[draw,circle] (l1) at ($(o)+(-10mm,0mm)$) {1};
              \node[draw,circle] (l2) at ($(l1)+(2,0)$) {2};
            %   \node[draw,circle] (l4) at ($(l2)+(0,-1)$) {6};
              \node[ draw,circle] (l6) at ($(l1)+(0,-1)$) {3};
            %   \draw[ ] (l1) -- (l6) node[midway,left] {$a$};
            %   \draw[ ] (l2) -- (l4) node[midway,right] {$a$};
          }    
          \graphbox{\( D \)}{80mm}{-22mm}{45mm}{22mm}{2mm}{-3mm}{
              \coordinate (o) at (-5mm,-3mm); 
              \node[draw,circle] (l1) at ($(o)+(-10mm,0mm)$) {1};
              \node[draw,circle] (l2) at ($(l1)+(3,0)$) {2};
              \node[draw,circle] (l3) at ($(l1)+(1,0)$) {4};
            %   \node[draw,circle] (l4) at ($(l1)+(2,0)$) {5};
            %   \node[ draw,circle] (l5) at ($(l2)+(0,-1)$) {6};
              \node[ draw,circle] (l6) at ($(l1)+(0,-1)$) {3};
            %   \draw[ ] (l1) -- (l6) node[midway,left] {$a$};
            %   \draw[] (l1) -- (l3) node[midway,above] {$a$};
            %   \draw[] (l3) -- (l4) node[midway,above] {$b$};
            %   \draw[ ] (l4) -- (l2) node[midway,above] {$a$};
            %   \draw[ ] (l2) -- (l5) node[midway,right] {$a$};
          }    
          \node () at (77mm,-8mm) {\( \overset{\alpha}{\rightarrow} \)};
          \node () at (58mm,-18mm) {\( \beta\downarrow \)};
          \node () at (102mm,-18mm) {\( \beta'\downarrow \)};
          \node () at (77mm,-33mm) {\( \overset{\alpha'}{\rightarrow} \)};
      \end{tikzpicture}
      }
    \end{center} 
    The span \( B \overset{\alpha}{\leftarrow} A \overset{\beta}{\rightarrow} C \)
    % shown in Figure~\ref{fig:preliminaries:a_rewriting_step_dfjalsdkdfsdfjflg} 
    is the pullback of the cospan \(B \overset{\beta'}{\rightarrow} D \overset{\alpha'}{\leftarrow} C \).
    Indeed, the pullback object can be taken as the set $A' \mathop{=} \set{(1_B,1_C),(2_B,2_C)} \mathop{\subseteq} B \mathop{\times} C$, and the morphisms $\alpha'' \mathop{\colon} A'  \mathop{\to} B$ and $\beta'' \mathop{\colon} A'  \mathop{\to} C$ are defined as the corresponding projections, e.g., $\alpha''((x,y)) \mathop{=} x$ and $\beta''((x,y)) \mathop{=} y$. Note that $A$ is isomorphic to $A'$, which is expected because the pullback of a cospan is unique up to isomorphism.
\end{example}
In the category \textbf{Graph} pullbacks always exist and are computed pointwise: take the pullback in \textbf{Set} of the node sets and of the edge sets, and equip the resulting graph with source, target and labeling maps induced componentwise; the projection graph homomorphisms are the corresponding componentwise projections and they satisfy the universal property.
\begin{example}
    \label{ex:cat:pbfsdljkgjasssss2222ssss}
    Consider the diagram in the category \textbf{Graph}, shown below. The numbers inside nodes and the subgraphs in different colors illustrate how the morphisms map nodes and edges. 
    The span $L \overset{\alpha}{\leftarrow} K \overset{\beta}{\rightarrow} C$ is a pullback of the cospan $L \overset{\beta'}{\rightarrow} G \overset{\alpha'}{\leftarrow} C$.
    % \begin{figure}[H]
    %     \centering
    \begin{center}
        \resizebox{0.8\textwidth}{!}{
        \begin{tikzpicture} 
            \graphbox{\( L \)}{40mm}{20mm}{34mm}{20mm}{2mm}{2mm}{
                \coordinate (o) at (0mm,-10mm); 
                \node[draw,circle] (l1) at ($(o)+(-10mm,0mm)$) {1};
                \node[draw,circle] (l2) at ($(l1)+(2,0)$) {2};
                \draw[->,red] (l2) edge[out=-135,in=-45]node[midway,below] {$a$} (l1) ;
                \node[draw,circle,red] (l3) at ($(l1)+(1,0)$) {3};
                \draw[->,red] (l1) -- (l3) node[midway,above] {$a$};
                \draw[->,red] (l3) -- (l2) node[midway,above] {$a$};
            } 
    
            \graphbox{\( K \)}{0mm}{0mm}{34mm}{12mm}{2mm}{2mm}{
                \coordinate (o) at (0mm,-8mm); 
                \node[draw,circle] (l1) at ($(o)+(-10mm,0mm)$) {1};
                \node[draw,circle] (l2) at ($(l1)+(2,0)$) {2};
            }  
            \graphbox{\(G  \)}{90mm}{10mm}{34mm}{40mm}{2mm}{-3mm}{
                \coordinate (o) at (0mm,-20mm); 
                 \node[draw,circle] (l1) at ($(o)+(0,0)$) {1\ 2};
                % \node[draw,circle] (l1) at ($(o)+(-10mm,0mm)$) {1};
                % \node[draw,circle] (l2) at ($(l1)+(2,0)$) {2};
                \draw[->,red] (l1) edge[loop below] node[midway, below] {$a$} (l1) ;
                \node[draw,circle,red] (l3) at ($(l1)+(0,1.4)$) {3};
                \node[draw,circle,blue] (l4) at ($(l1)+(1,-1)$) {6};
                \draw[->,red] (l1) edge[bend left] node[midway,left] {$a$} (l3);
                \draw[->,red] (l3) edge[bend left] node[midway,right] {$a$} (l1);
                \draw[->,blue] (l1) edge  node[midway,right] {$a$} (l4);
                \node[draw,circle,blue] (l6) at ($(l1)+(-1,-1)$) {7};
                \draw[<-,blue] (l1) edge node[midway,left] {$a$} (l6) ;
                % \draw[->,blue] (l1) edge[out=-135,in=-45]node[midway,below] {$a$} (l1) ;
            }   
     
            \graphbox{\( C \)}{40mm}{-12mm}{34mm}{24mm}{2mm}{-3mm}{
                \coordinate (o) at (0mm,-5mm); 
                \node[draw,circle] (l1) at ($(o)+(0,0)$) {1\ 2};
                % \node[draw,circle] (l2) at ($(l1)+(2,0)$) {2};
                \node[draw,circle,blue] (l4) at ($(l1)+(1,-1)$) {6};
                \draw[->,blue] (l1) -- (l4) node[midway,right] {$a$};
                % \draw[->,blue] (l1) edge[loop above] node[midway, above] {$a$} (l1) ;
                \node[ draw,circle,blue] (l6) at ($(l1)+(-1,-1)$) {7};
                \draw[<-,blue] (l1) -- (l6) node[midway,left] {$a$};
            }
            % K to L  
            \draw[>->] (17mm,5mm) -- node[above] {$\alpha$} (37mm,15mm);
            % C to G
            \draw[>->] (76mm,-28mm)-- node[below] {$\alpha'$} (88mm,-26mm) ;
            % K to C
            \draw[->] (17mm,-17mm) -- node[below] {$\beta$} (37mm,-28mm);
            % L to G 
            \draw[->] (76mm,16mm) -- node[above] {$\beta'$} (88mm,10mm);
            \node () at (57mm,-6mm) {$PB$};
        \end{tikzpicture}
        }
    \end{center}
\end{example} 
Consider the following diagram in the category \textbf{Graph}:\begin{center}
        \resizebox{0.8\textwidth}{!}{
        \begin{tikzpicture} 
            \graphbox{\( L \)}{40mm}{15mm}{34mm}{12mm}{2mm}{2mm}{
                \coordinate (o) at (0mm,-8mm); 
                \node[draw,circle] (l1) at ($(o)+(-10mm,0mm)$) {1};
                \node[draw,circle] (l2) at ($(l1)+(2,0)$) {2};
                \node[draw,circle,red] (l3) at ($(l1)+(1,0)$) {3};
                \draw[->,red] (l1) -- (l3) node[midway,above] {$a$};
                \draw[->,red] (l3) -- (l2) node[midway,above] {$a$};
            } 
    
            \graphbox{\( K \)}{0mm}{0mm}{34mm}{12mm}{2mm}{2mm}{
                \coordinate (o) at (0mm,-8mm); 
                \node[draw,circle] (l1) at ($(o)+(-10mm,0mm)$) {1};
                \node[draw,circle] (l2) at ($(l1)+(2,0)$) {2};
            }  
            \graphbox{\(G  \)}{90mm}{5mm}{34mm}{24mm}{2mm}{-3mm}{
                \coordinate (o) at (0mm,-5mm); 
                \node[draw,circle] (l1) at ($(o)+(-10mm,0mm)$) {1};
                \node[draw,circle] (l2) at ($(l1)+(2,0)$) {2};
                \node[draw,circle,red] (l3) at ($(l1)+(1,0)$) {3};
                \node[draw,circle,blue] (l4) at ($(l2)+(0,-1)$) {6};
                \draw[->,red] (l1) -- (l3) node[midway,above] {$a$};
                \draw[->,red] (l3) -- (l2) node[midway,above] {$a$};
                \draw[->,blue] (l2) -- (l4) node[midway,right] {$a$};
                \node[draw,circle,blue] (l6) at ($(l1)+(0,-1)$) {7};
                \draw[<-,blue] (l1) -- (l6) node[midway,left] {$a$};
                \draw[->,blue] (l2) edge[out=-135,in=-45]node[midway,below] {$a$} (l1) ;
            }   
     
            \graphbox{\( C \)}{40mm}{-15mm}{34mm}{24mm}{2mm}{-3mm}{
                \coordinate (o) at (0mm,-5mm); 
                \node[draw,circle] (l1) at ($(o)+(-10mm,0mm)$) {1};
                \node[draw,circle] (l2) at ($(l1)+(2,0)$) {2};
                \node[draw,circle,blue] (l4) at ($(l2)+(0,-1)$) {6};
                \draw[->,blue] (l2) -- (l4) node[midway,right] {$a$};
                \draw[->,blue] (l2) edge[out=-135,in=-45]node[midway,below] {$a$} (l1) ;
                \node[ draw,circle,blue] (l6) at ($(l1)+(0,-1)$) {7};
                \draw[<-,blue] (l1) -- (l6) node[midway,left] {$a$};
            }      
            % K to L
            \draw[>->] (17mm,5mm) -- node[above] {$\alpha$} (37mm,10mm);
            % C to G
            \draw[>->] (76mm,-28mm)-- node[below] {$\alpha'$} (104mm,-21mm) ;
            % K to C
            \draw[>->] (17mm,-17mm) -- node[below] {$\beta$} (37mm,-28mm);
            % L to G
            \draw[>->] (76mm,10mm) -- node[above] {$\beta'$} (104mm,7mm);
            \node () at (57mm,-6mm) {$PO$};
        \end{tikzpicture}
        }
    \end{center}
We observe that it is a pushout square as well as a pullback square. This is not a coincidence, as stated in the following proposition.
\begin{proposition}[\text{\cite[Lemma 13]{lack2004adhesive}}]
    \label{prop:pb_eq_po}
    In the category \textbf{Graph}, pushouts along monomorphisms are also pullbacks. 
\end{proposition}
This proposition is illustrated in Example~\ref{ex:cat:pushout_non_injective_ssss} and Example~\ref{ex:cat:pbfsdljkgjasssss2222ssss}.



% \chapter{Termination of Graph Rewriting
% using Weighted Type Graphs
% over Non-Well-Founded Semirings} 

% \section{to be integrated}
% \begin{remark}
%     We work within the category \textbf{Graph}. The terms homomorphism and morphism will be used interchangeably; the same applies to monic and injective, and to injective homomorphism and monomorphism.
% \end{remark} 

% \section{abstract}
% The weighted type graph method is a technique for proving termination of double-pushout (DPO) graph rewriting systems. 
% Existing approaches use weighted type graphs over well-founded semirings, but these have practical limitations when applied to edge-labeled directed multigraph rewriting.
% We investigate the use of non-well-founded semirings to overcome these limitations.
% We have implemented both our method and a prior approach for edge-labeled graph rewriting within a unified tool.
% \section{Introduction}

% \label{sec:type_graph:introduction}
% % This chapter extends the weighted type graph method to non-well-founded semirings. This extension allows for more efficient searching for weighted type graphs that witness relative termination of DPO rewriting systems on edge-labeled directed multigraphs.

% The method was first introduced in~\cite{zantema2014termination} for cycle rewriting. Subsequent work generalized it for DPO rewriting on edge-labeled directed multigraphs with injective rules and injective matches~\cite{bruggink2014termination}; later, it was extended to general DPO rewriting on edge-labeled multigraphs~\cite{bruggink2015proving}; and it was adapted to broader categories and DPO variants~\cite{endrullis2024generalized_icgt}.
% The method assigns weights to morphisms targeting a weighted type graph over a well-founded semiring. The weight of a graph is defined
%  as the sum of the weights of all morphisms from that graph to the type graph. Relative termination of $\mathcal{A}$ with respect to $\mathcal{B}$ is proven by ensuring that rewriting steps using rules in \( \mathcal{A} \) strictly decrease the weights of the host graphs, while rewriting steps using rules in \( \mathcal{B} \) do not increase them.
    
%     In the previous work~\cite{zantema2014termination,bruggink2014termination,bruggink2015proving}, three concrete semirings on natural numbers were proposed: the natural tropical semiring $\mathfrak{T}$, the natural arctic semiring $\mathfrak{A}$, and the natural arithmetic semiring $\mathfrak{N}$.
%     However, constructing weighted type graphs over these concrete semirings for DPO rewriting on edge-labeled directed multigraphs is undecidable in general. This is because it requires quantifying over all edge-labeled directed multigraphs with weighted edges over $\mathbb{N}$. 

%     In response to this challenge,~\cite{zantema2014termination,bruggink2014termination,bruggink2015proving} proposed reducing the search space by fixing the number of nodes \( k \in \mathbb{N} \) and the maximum weight of edges of the weighted type graph, and then constructing a weighted type graph as follows: (1) construct a graph with \( k \) nodes, with a directed edge per ordered pair of nodes and label from the finite set of edge labels $\Sigma$ of the rewriting system; (2) for each edge, decide if it exists in the weighted type graph; (3) if an edge exists, assign a weight (a natural number) to it; (4) check if the weighted type graph satisfies the required conditions.
    
%     Despite these constraints, searching for weighted type graphs that witness termination of a DPO rewriting system on edge-labeled directed multigraphs
%     remains challenging. 
%     Let $n = k^2 \cdot | \Sigma |$.
%     For a weighted type graph over the natural tropical semiring $\mathfrak{T}$ or the natural arctic semiring $\mathfrak{A}$, the problem can be reduced to checking the satisfiability of an existential Presburger arithmetic formula with $n$ binary variables and $n$ integer variables.
%     While modern SMT solvers, such as Z3~\cite{arithmetic2024z3} (which incorporates the dedicated CutSat solver~\cite{z3ilp_cutsat}), can solve practical instances of this problem, its worst-case complexity remains exponential \( O(2^{2n}) \).
%      For a weighted type graph over the natural arithmetic semiring $\mathfrak{N}$, the task can be reduced to checking the satisfiability of an existential Peano arithmetic formula with addition and multiplication, involving $n$ binary variables and $n$ integer variables. Though modern solvers like Z3 can tackle practical instances, it is a semi-decidable problem~\cite{matiyasevivc2003enumerable}.

%     There are automated tools that implement the weighted type graph method for proving termination of DPO rewriting systems on edge-labeled directed multigraphs: 
%     \texttt{TORPAcyc}~\cite{TORPAcyc} implements the weighted type graph method for cycle rewriting and uses the SMT solver \texttt{Yices}~\cite{yices} to solve constraint systems;
%     \texttt{Grez}~\cite{grez} implements the weighted type graph method for DPO rewriting on edge-labeled directed multigraphs; \texttt{GraphTT-wtg}~\footnote{To the best of our knowledge, no implementation is publicly available}~\cite{endrullis2024generalized_arxiv_v3} implements the weighted type graph method for DPO rewriting on many categories.
%     \texttt{Grez} and \texttt{GraphTT-wtg}  
%     use  the SMT solver \texttt{Z3}~\cite{de2008z3} to solve constraint systems.


%     In spite of the theoretical power of the weighted type graph method, its practical applicability is limited by the high computational complexity of searching for suitable weighted type graphs over well-founded semirings on 
%     natural numbers. In fact, to reduce the search space, users of the above-mentioned tools are required
%      to fix the size of the weighted type graph, as well as the maximum weight of edges in the weighted type graph. These are challenging tasks if not impossible to determine a priori even for experts in the field. 
%     Furthermore, our experiments with the two publicly available ones \texttt{TORPAcyc} and \texttt{Grez} show that when the number of edge labels is 2, they struggle to search for weighted type graphs with 4 nodes and maximum edge weight of 2, or to search for weighted type graphs with 3 nodes when the maximum edge weight is larger than 3. 
%     This article addresses this limitation to make the weighted type graph method more accessible to non-expert users and to enhance its practical applicability.

% To reduce the complexity of searching for weighted type graphs that witness termination of a DPO rewriting system on edge-labeled directed multigraphs, we propose extending the weighted type graph method to non-well-founded semirings and introduce three concrete semirings over the real numbers: the real tropical semiring $\mathfrak{T}'$, the real arctic semiring $\mathfrak{A}'$, and the real arithmetic semiring $\mathfrak{N}'$. 


% Under the same constraints, searching for weighted type graphs over these semirings on real numbers is computationally more tractable. Specifically,
% for the real tropical semiring $\mathfrak{T}'$ or the real arctic semiring $\mathfrak{A}'$, the problem reduces 
% to checking the satisfiability of an 
% existential linear real arithmetic formula with $n$ binary variables and $n$ real variables for the real tropical semiring $\mathfrak{T}'$ or the real arctic semiring $\mathfrak{A}'$. 
% The SMT solver Z3 can solve practical instances of this problem more easily using CutSat solver, because there are only $n$ integer variables. The worst-case complexity for solving this problem is $O(2^n)$.
% For a weighted type graph over the real arithmetic semiring $\mathfrak{N}'$, the task reduces to checking the satisfiability of 
% an existential non-linear real arithmetic formula\textemdash a task that is decidable~\cite{collins1974quantifier,z3realarithmetic}.


% We have implemented both our approach and the approach proposed by Endrullis et al.~\cite{endrullis2024generalized_arxiv_v2} for DPO rewriting on edge-labeled directed multigraphs in a unified tool, written in OCaml, and employs Z3~\cite{de2008z3} to solve the constraint systems.

% In the remainder of this chapter, we will first introduce the type graph method presented in~\cite{endrullis2024generalized_arxiv_v2}, then we will present our extension of the type graph method to non-well-founded semirings.
% this paper is structured as follows.  
% \autoref{sec:type_graph:preliminaries} reviews essential preliminary definitions. 
% \autoref{sec:strongly_monotonic_measurable_semiring} introduces the non-well-founded semirings. 
% \autoref{sec:type_graph:measuring_graphs} introduces weighted type graphs.  
% \autoref{sec:type_graph:weighing_pushout} introduces definitions used to estimate pushout-object weights.   
% \autoref{sec:type_graph:termination} presents a termination criterion. 
% \autoref{sec:type_graph:implementation} presents the implementation of our tool. 
% \autoref{sec:type_graph:result} presents the experimental results.
%  \autoref{sec:type_graph:related_work} discusses empirical results obtained using the tool. 
% \autoref{sec:type_graph:conclusion} presents brief remarks and outlines future research directions. The missing proofs can be found in the Appendix.

Chapter~\ref{chap:nwf} extends a existing method for termination of DPO graph rewriting systems, called type graph method. Throughout this chapter, we fix a finite set $\Sigma$ of edge labels.

The weighted type graph method was first introduced by Hans, Konig and Bruggink in~\cite{zantema2014termination} for cycle rewriting. Their subsequent work in~\cite{bruggink2014termination} generalized it for DPO rewriting on edge-labeled directed multigraphs with injective rules and injective matches; later, it was extended to general DPO rewriting on edge-labeled multigraphs by Bruggink et al. in~\cite{bruggink2015proving}; and it was adapted to broader categories and DPO variants by Endrullis and Overbeek~\cite{endrullis2024generalized_icgt}. 

The method assigns weights to morphisms targeting a weighted type graph over a well-founded semiring. The weight of a graph is defined
 as the sum of the weights of all morphisms from that graph to the type graph. Relative termination of $\mathcal{A}$ with respect to $\mathcal{B}$ is proven by ensuring that rewriting steps using rules in \( \mathcal{A} \) strictly decrease the weights of the host graphs, while rewriting steps using rules in \( \mathcal{B} \) do not increase them.
    
   Previous work proposed three concrete semirings on natural numbers: the natural tropical semiring $\mathfrak{T}$, the natural arctic semiring $\mathfrak{A}$, and the natural arithmetic semiring $\mathfrak{N}$.
    However, constructing weighted type graphs over these concrete semirings for DPO rewriting on edge-labeled directed multigraphs is difficult in general, because it requires quantifying over all edge-labeled directed multigraphs with weighted edges over $\mathbb{N}$.

    In response to this challenge, the previous works~\cite{zantema2014termination,bruggink2014termination,bruggink2015proving} proposed reducing the search space by fixing the number of nodes \( k \in \mathbb{N} \) and the maximum weight of edges of the weighted type graph, and then constructing a weighted type graph as follows: 
    \begin{enumerate}
      \item Construct a graph with \( k \) nodes, with a directed edge per ordered pair of nodes and label from the finite set of edge labels $\Sigma$ of the rewriting system. 
      \item For each edge, decide if it exists in the weighted type graph;  if an edge exists, assign a weight (a natural number) to it.
      \item Check if the weighted type graph satisfies the required conditions.
    \end{enumerate}
    Despite these constraints, searching for weighted type graphs that witness termination of a DPO rewriting system on edge-labeled directed multigraphs
    remains challenging. 
    Let $n = k^2 \cdot | \Sigma |$.
    For a weighted type graph over the natural tropical semiring $\mathfrak{T}$ or the natural arctic semiring $\mathfrak{A}$, the problem amounts to
    checking the satisfiability of an existential Presburger arithmetic formula with $n$ binary variables and $n$ integer variables.
    While modern SMT solvers, such as Z3~\cite{arithmetic2024z3} (which incorporates the dedicated CutSat solver~\cite{z3ilp_cutsat}), can solve practical instances of this problem, its worst-case complexity remains exponential \( O(2^{2n}) \).
     For a weighted type graph over the natural arithmetic semiring $\mathfrak{N}$, the task amounts to
     checking the satisfiability of an existential Peano arithmetic formula with addition and multiplication, involving $n$ binary variables and $n$ integer variables. Though modern solvers like Z3 can tackle practical instances, it is a semi-decidable problem~\cite{matiyasevivc2003enumerable}.

    There are automated tools that implement the weighted type graph method for proving termination of DPO rewriting systems on edge-labeled directed multigraphs:  
    \texttt{TORPAcyc}~\cite{TORPAcyc} implements the weighted type graph method for cycle rewriting and uses the SMT solver \texttt{Yices}~\cite{yices} to solve constraint systems;
    \texttt{Grez}~\cite{grez} implements the weighted type graph method for DPO rewriting on edge-labeled directed multigraphs; \texttt{GraphTT-wtg}~\footnote{To the best of our knowledge, no implementation is publicly available}~\cite{endrullis2024generalized_arxiv_v3} implements the weighted type graph method for DPO rewriting on many categories.
    \texttt{Grez} and \texttt{GraphTT-wtg}  
    use  the SMT solver \texttt{Z3}~\cite{de2008z3} to solve constraint systems.

    In spite of the theoretical power of the weighted type graph method, its practical applicability is limited by the high computational complexity of searching for suitable weighted type graphs over well-founded semirings on 
    natural numbers. In fact, to reduce the search space, users of the above-mentioned tools are required
     to fix the size of the weighted type graph, as well as the maximum weight of edges in the weighted type graph. These are challenging tasks if not impossible to determine a priori even for experts in the field. 
    Furthermore, our experiments with the two publicly available ones \texttt{TORPAcyc} and \texttt{Grez} show that when the number of edge labels is $2$, they struggle to search for weighted type graphs with $4$ nodes and maximum edge weight of $2$, or to search for weighted type graphs with $3$ nodes when the maximum edge weight is larger than $3$. 
   Chapter~\ref{chap:nwf} addresses this limitation to make the weighted type graph method more accessible to non-expert users and to enhance its practical applicability.

To reduce the complexity of searching for weighted type graphs that witness termination of a DPO rewriting system on edge-labeled directed multigraphs, we propose extending the weighted type graph method to non-well-founded semirings and introduce three concrete semirings over the real numbers: the real tropical semiring $\mathfrak{T}'$, the real arctic semiring $\mathfrak{A}'$, and the real arithmetic semiring $\mathfrak{N}'$. The idea of using weights from non-well-founded domains to prove termination was used in the context of term rewriting in~\cite{lucas2006relative}.

Under the same constraints, searching for weighted type graphs over these semirings on real numbers is computationally more tractable. Specifically,
for the real tropical semiring $\mathfrak{T}'$ or the real arctic semiring $\mathfrak{A}'$, the problem amounts to checking the satisfiability of an
existential linear real arithmetic formula with $n$ binary variables and $n$ real variables for the real tropical semiring $\mathfrak{T}'$ or the real arctic semiring $\mathfrak{A}'$.
The SMT solver Z3 can solve practical instances of this problem more easily using CutSat solver, because there are only $n$ integer variables. The worst-case complexity for solving this problem is $O(2^n)$.
For a weighted type graph over the real arithmetic semiring $\mathfrak{N}'$, the task reduces to checking the satisfiability of 
an existential non-linear real arithmetic formula\textemdash a task that is decidable~\cite{collins1974quantifier,z3realarithmetic}.

We have implemented both our approach and the approach proposed by Endrullis et al.~\cite{endrullis2024generalized_arxiv_v2} for DPO rewriting on edge-labeled directed multigraphs in a unified tool, written in OCaml, and employs Z3~\cite{de2008z3} to solve the constraint systems.

we will first introduce the type graph method presented in~\cite{endrullis2024generalized_icgt}, then we will present our extension of the type graph method to non-well-founded semirings.
 
% \section{Preliminaries}
% \label{sec:type_graph:preliminaries}
% %  We first introduce the concept of pushout from category theory in \hyperref[preliminaries:pushout]{Section~\ref*{preliminaries:pushout}}. 
%   Next, we discuss the double-pushout (DPO) rewriting in \hyperref[preliminaries:grs]{Section~\ref*{preliminaries:grs}}. 
%   Finally, we explore the concept of relative termination in \hyperref[preliminaries:relative_termination]{Section~\ref*{preliminaries:relative_termination}}. 
%   Additionally, we introduce the notion of a strongly monotonic measurable semiring in \hyperref[sec:strongly_monotonic_measurable_semiring]{Section~\ref*{sec:strongly_monotonic_measurable_semiring}}. 
%   The type graph, which is presented in \hyperref[sec:weighted_type_graph]{Section~\ref*{sec:weighted_type_graph}}, is weighted over this semiring.
  
In this section, we recall definitions related to the double-pushout approach to graph rewriting and relative termination, and we introduce strongly monotonic measurable semirings. For further details, we refer the reader to~\cite{konig2018atutorial,corradini1997algebraic,habel2001double} for the DPO graph rewriting,~\cite{pierce1991basic, barr1990category} for category theory concepts, and~\cite{geser1990relative} for relative termination. Basic concepts in the category theory and on edge-labeled graphs needed in this paper are collected in the preliminaries section of \cite{qiu2025termination}.

\subsection{DPO rewriting}
\label{preliminaries:grs}
\begin{definition}[Rewriting rule and match~\cite{corradini1997algebraic}]
  \label{def:grs:dpo_rule}
A \textbf{DPO rewriting rule} $\rho$ is a span \( L \overset{l}{\leftarrow} K \overset{r}{\rightarrow} R \), where \( K \) is the \textbf{interface}, \( L \) is the \textbf{left-hand-side graph}, denoted \( \operatorname{lhs}(\rho) \), and \( R \) is the \textbf{right-hand-side graph}, denoted \( \operatorname{rhs}(\rho) \). The rule is \textbf{monic} if $l$ and $r$ are both monic.
A match of the rule in an graph \( G \) is a morphism \( m: L \rightarrow G \).   
\end{definition}
   In this paper, we use examples from the category \textbf{Graph} of edge-labeled directed multigraphs (see~\cite{konig2018atutorial}) to illustrate the discussed concepts. To facilitate this, we introduce the following notation for visualizing graph homomorphisms.
\begin{notation}[\cite{qiu2025termination}]
    We use the notation from~\cite[Notation 1]{overbeek2023apbpotutorial} to visualize edge-labeled graph homomorphisms. Labeled graphs are enclosed in boxes with their names displayed in the top-left corner. Nodes and edges are assigned subsets of \(\mathbb{N}\) as identifiers, and these identifiers are chosen such that: (i) Each node or edge \( y \) in the codomain graph is assigned the union of the identifiers of all nodes or edges in the domain graph that are mapped to \( y \); (ii) The graph homomorphism is uniquely determined by this assignment. To further improve readability, we represent sets by listing their elements. Additionally, we omit identifiers when doing so does not cause confusion. This is illustrated in the following representation of a homomorphism \( h: G \to H \).
    
   \begin{center}
        \resizebox{0.5\textwidth}{!}{
        \begin{tikzpicture}
            \graphbox{\( G \)}{00mm}{-20mm}{45mm}{25mm}{2mm}{-10mm}{
                \coordinate (o) at (-5mm,-8mm); 
                \node[draw,circle] (l1) at ($(o)+(-10mm,0mm)$) {1};
                \node[draw,circle] (l2) at ($(l1)+(3,0)$) {2};
                \node[draw,circle] (l3) at ($(l1)+(1,0)$) {3};
                \node[draw,circle] (l4) at ($(l1)+(2,0)$) {4};
                \draw[->] (l1) -- (l3) node[midway,above] {a};
                \draw[->] (l3) -- (l4) node[midway,above] {b};
                \draw[->] (l4) -- (l2) node[midway,above] {a};
            }  
            \graphbox{\( H \)}{52mm}{-20mm}{50mm}{25mm}{2mm}{-10mm}{
                \coordinate (o) at (-5mm,-8mm); 
                \node[draw,circle] (l1) at ($(o)+(-1,0mm)$) {1};
                \node[draw,circle] (l2) at ($(l1)+(3,0)$) {2};
                \node[draw,circle] (l3) at ($(l1)+(1.5,0)$) {3\ 4};
                \draw[->] (l1) edge node[midway,above] {a} (l3);
                \draw[->] (l3) edge [loop above] node[midway,above] {b} (l3) ;
                \draw[->] (l3) -- (l2) node[midway,above] {a};
            }      
            \node () at (48mm,-30mm) {$\rightarrow$};
        \end{tikzpicture}
    }
    \end{center}  
    In this example, the sets \(\{1\}\), \(\{2\}\), \(\{3\}\), \(\{4\}\), and \(\{3,4\}\) are represented as \(1\), \(2\), \(3\), \(4\), and \(3\ 4\), respectively. Edge identifiers are omitted.
\end{notation}
\begin{example}
  \label{ex:grsaa}
  The injective DPO rule from \cite[Example 6]{bruggink2014termination} will be used to illustrate the concepts discussed throughout this paper.
  The rule can be visualized as follows:
  \begin{center} 
      \resizebox{0.7\textwidth}{!}{
      \begin{tikzpicture}
          \graphbox{$L$}{0mm}{0mm}{34mm}{15mm}{2mm}{-5mm}{
              \coordinate (o) at (0mm,-3mm); 
              \node[draw,circle] (l1) at ($(o)+(-10mm,0mm)$) {1};
              \node[draw,circle] (l2) at ($(l1)+(2,0)$) {2};
              \node[draw,circle] (l3) at ($(l1) + (1,0)$) {3};
              \draw[->] (l1) -- (l3) node[midway,above] {a};
              \draw[->] (l3) -- (l2) node[midway,above] {a};
          }     
          \graphbox{$K$}{40mm}{0mm}{24mm}{15mm}{2mm}{-5mm}{
              \coordinate (o) at (5mm,-3mm); 
              \node[draw,circle] (l1) at ($(o)+(-10mm,0mm)$) {1};
              \node[draw,circle] (l2) at ($(l1)+(1,0)$) {2};
              % \node[draw,circle] (l3) at ($(l1) + (1,0)$) {$\ $};
              % \draw[->] (l1) -- (l3) node[midway,above] {a};
              % \draw[->] (l3) -- (l2) node[midway,above] {a};
          }    
          \graphbox{$R$}{70mm}{0mm}{45mm}{15mm}{2mm}{-5mm}{
              \coordinate (o) at (-5mm,-3mm); 
              \node[draw,circle] (l1) at ($(o)+(-10mm,0mm)$) {1};
              \node[draw,circle] (l2) at ($(l1)+(3,0)$) {2};
              \node[draw,circle] (l3) at ($(l1) + (1,0)$) {4};
              \node[draw,circle] (l4) at ($(l1) + (2,0)$) {5};
              \draw[->] (l1) -- (l3) node[midway,above] {a};
              \draw[->] (l3) -- (l4) node[midway,above] {b};
              \draw[->] (l4) -- (l2) node[midway,above] {a};
          }    
          \node () at (37mm,-8mm) {$\overset{l}{\leftarrowtail}$};
          \node () at (67mm,-8mm) {$\overset{r}{\rightarrowtail}$};
          % \draw[>->] (51mm,2mm) -- (52mm,3mm);
      \end{tikzpicture}
      }
  \end{center}
\end{example}
\begin{definition}[DPO Rewriting step \cite{endrullis2024generalized_arxiv_v2}]
  \label{def:rewriting_step}
    \ \newline
    \noindent
    \begin{minipage}{0.72\textwidth}
      A DPO diagram $\delta$ is a diagram as shown on the right.
      This diagram $\delta$ is a witness for the \textbf{rewriting step} from \( G \) to \( H \) using the rule \( \rho \) and \textbf{match} \( m \), denoted \( G \mathop{\Rightarrow}_\rho^m H \) or \( G \mathop{\Rightarrow}_\rho^\delta H \). We denote $\operatorname{left}(\delta)$ and $\operatorname{right}(\delta)$ the pushout squares $KLGC$ and $KRHC$, respectively.
    \end{minipage}
    \hfill
    \begin{minipage}{0.28\textwidth}
          % \begin{center}
          \hfill
          \resizebox{0.85\textwidth}{!}{
          \begin{tikzpicture}
            % [node distance=11mm]
            \node (I) at (0,0) {$K$};
            \node (L) at (-2,0) {$L$};
            \node (R) at (2,0) {$R$};
            \node (G) at (-2,-2) {$G$};
            \node (C) at (0,-2) {$C$};
            \node (H) at (2,-2) {$H$};
            \draw [->] (I) to  node [midway,below] {$l$} (L);
            \draw [->] (I) to  node [midway,below] {$r$} (R);
            \draw [->] (L) to node [midway,right] {$m$} (G);
            \draw [->] (I) to node [midway,right] {$u$} (C);
            \draw [->] (R) to node [midway,left] {$m'$} (H);
            \draw [->] (C) to node [midway,above] {$l'$} (G);
            \draw [->] (C) to node [midway,above] {$r'$} (H);
            \node [at=($(I)!.5!(G)$)] {\normalfont PO};
            \node [at=($(I)!.5!(H)$)] {\normalfont PO};
          \end{tikzpicture}
        % \end{center}
        }
        \end{minipage}
  \end{definition}
\begin{example}
  \label{ex:rewriting_step_grs_aa}
  The DPO diagram below defines a rewriting step using the rule from Example~\ref{ex:grsaa}.
  \begin{center} 
      \resizebox{0.7\textwidth}{!}{
      \begin{tikzpicture}
          \graphbox{\( L \)}{0mm}{-3mm}{34mm}{12mm}{2mm}{2mm}{
              \coordinate (o) at (0mm,-8mm); 
              \node[draw,circle] (l1) at ($(o)+(-10mm,0mm)$) {1};
              \node[draw,circle] (l2) at ($(l1)+(2,0)$) {2};
              \node[draw,circle] (l3) at ($(l1)+(1,0)$) {3};
              \draw[] (l1) -- (l3) node[midway,above] {$a$};
              \draw[] (l3) -- (l2) node[midway,above] {$a$};
          } 
          \graphbox{\( K \)}{40mm}{-3mm}{34mm}{12mm}{2mm}{2mm}{
              \coordinate (o) at (0mm,-8mm); 
              \node[draw,circle] (l1) at ($(o)+(-10mm,0mm)$) {1};
              \node[draw,circle] (l2) at ($(l1)+(2,0)$) {2};
          }  
          \graphbox{\( R \)}{80mm}{-3mm}{45mm}{12mm}{2mm}{2mm}{
              \coordinate (o) at (-5mm,-8mm); 
              \node[draw,circle] (l1) at ($(o)+(-10mm,0mm)$) {1};
              \node[draw,circle] (l2) at ($(l1)+(3,0)$) {2};
              \node[draw,circle] (l3) at ($(l1)+(1,0)$) {4};
              \node[draw,circle] (l4) at ($(l1)+(2,0)$) {5};
              \draw[ ] (l1) -- (l3) node[midway,above] {$a$};
              \draw[ ] (l3) -- (l4) node[midway,above] {$b$};
              \draw[ ] (l4) -- (l2) node[midway,above] {$a$};
          }    
          \graphbox{\( G \)}{0mm}{-22mm}{34mm}{22mm}{2mm}{-3mm}{
              \coordinate (o) at (0mm,-3mm); 
              \node[draw,circle] (l1) at ($(o)+(-10mm,0mm)$) {1};
              \node[draw,circle] (l2) at ($(l1)+(2,0)$) {2};
              \node[draw,circle] (l3) at ($(l1)+(1,0)$) {3};
              \node[draw,circle] (l4) at ($(l2)+(0,-1)$) {6};
              \draw[] (l1) -- (l3) node[midway,above] {$a$};
              \draw[] (l3) -- (l2) node[midway,above] {$a$};
              \draw[ ] (l2) -- (l4) node[midway,right] {$a$};
              \node[draw,circle] (l6) at ($(l1)+(0,-1)$) {7};
              \draw[] (l1) -- (l6) node[midway,left] {$a$};
          }    
          \graphbox{\( C  \)}{40mm}{-22mm}{34mm}{22mm}{2mm}{-3mm}{
              \coordinate (o) at (0mm,-3mm); 
              \node[draw,circle] (l1) at ($(o)+(-10mm,0mm)$) {1};
              \node[draw,circle] (l2) at ($(l1)+(2,0)$) {2};
              \node[draw,circle] (l4) at ($(l2)+(0,-1)$) {6};
              \draw[ ] (l2) -- (l4) node[midway,right] {$a$};
              \node[ draw,circle] (l6) at ($(l1)+(0,-1)$) {7};
              \draw[ ] (l1) -- (l6) node[midway,left] {$a$};
          }    
          \graphbox{\( H \)}{80mm}{-22mm}{45mm}{22mm}{2mm}{-3mm}{
              \coordinate (o) at (-5mm,-3mm); 
              \node[draw,circle] (l1) at ($(o)+(-10mm,0mm)$) {1};
              \node[draw,circle] (l2) at ($(l1)+(3,0)$) {2};
              \node[draw,circle] (l3) at ($(l1)+(1,0)$) {4};
              \node[draw,circle] (l4) at ($(l1)+(2,0)$) {5};
              \node[ draw,circle] (l5) at ($(l2)+(0,-1)$) {6};
              \node[ draw,circle] (l6) at ($(l1)+(0,-1)$) {7};
              \draw[ ] (l1) -- (l6) node[midway,left] {$a$};
              \draw[] (l1) -- (l3) node[midway,above] {$a$};
              \draw[] (l3) -- (l4) node[midway,above] {$b$};
              \draw[ ] (l4) -- (l2) node[midway,above] {$a$};
              \draw[ ] (l2) -- (l5) node[midway,right] {$a$};
          }    
          \node () at (37mm,-8mm) {\( \leftarrowtail \)}; % K -> L
          \node () at (77mm,-8mm) {\( \rightarrowtail \)}; % K -> R
          \node () at (15mm,-18mm) {\( m\ \downarrowtail \)};
          \node () at (37mm,-33mm) {\( \leftarrowtail \)};
          \node () at (58mm,-18mm) {\( u\downarrowtail \)};
          \node () at (102mm,-18mm) {\( \downarrowtail \)};
          \node () at (77mm,-33mm) {\( \rightarrowtail \)}; % C -> H
      \end{tikzpicture}
      }
  \end{center}
\end{example}
DPO rewriting has several variants, depending on factors such as whether the matching morphism, the left-hand morphism, or the right-hand morphism is monic~\cite{habel2001double}, and whether the left pushout square is restricted~\cite{behr2021concurrency,behr2023fundamentals}. Since these restrictions can influence the termination properties of a DPO rewrite system,  we use the definition from~\cite{endrullis2024generalized_arxiv_v2} to encompass different DPO variants.
% \begin{definition}[DPO rewriting framework~\cite{endrullis2024generalized_arxiv_v2}]
%     A \emph{DPO rewriting framework} $\mathfrak{F}$ is a mapping of DPO rewriting rules to classes of DPO diagrams such that, for every DPO rule $\rho$, $\mathfrak{F}(\rho)$ is a class of DPO diagrams with top-span $\rho$.
    
%     The DPO rewriting relation $\Rightarrow_{\rho,\mathfrak{F}}$ induced by a DPO rewriting rule $\rho$ in $\mathfrak{F}$ is defined as follows: $G \Rightarrow_{\rho,\mathfrak{F}} H$ iff $G \Rightarrow_\rho^\delta H$ for some $\delta \in \mathfrak{F}(\rho)$. 
%     % for some $\delta \in \mathfrak{F}(\rho)$
%     The rewriting relation $\Rightarrow_{\mathcal{R},\mathfrak{F}}$ induced by a set $\mathcal{R}$ of DPO rewriting rules in $\mathfrak{F}$ is given by: $G \Rightarrow_{\mathcal{R},\mathfrak{F}} H$ iff $G \Rightarrow_{\rho,\mathfrak{F}} H$ for some $\rho \in \mathcal{R}$. When $\mathfrak{F}$ is clear from the context, we 
%     suppress $\mathfrak{F}$ and 
%     write $\Rightarrow_{\rho}$ and $\Rightarrow_{\mathcal{R}}$.
% \end{definition} 
\begin{definition}[Rewriting framework~\cite{endrullis2024generalized_arxiv_v2}]
    A \textbf{DPO rewriting framework} $\mathfrak{F}$ is a mapping of DPO rewriting rules to collections of DPO diagrams. Specifically, for every rule \( \rho = (L \overset{l}{\leftarrow} K \overset{r}{\rightarrow} R) \), the collection $\mathfrak{F}(\rho)$ consists of DPO diagrams of the form shown in Definition~\ref{def:rewriting_step}.

    The \textbf{DPO rewriting relation $\Rightarrow_{\rho,\mathfrak{F}}$ induced by a DPO rewriting rule $\rho$ in $\mathfrak{F}$} is defined as follows: $G \Rightarrow_{\rho,\mathfrak{F}} H$ iff $G \Rightarrow_\rho^\delta H$ for some $\delta \in \mathfrak{F}(\rho)$. 
    % for some $\delta \in \mathfrak{F}(\rho)$
     The \textbf{DPO rewriting relation $\Rightarrow_{\mathcal{R},\mathfrak{F}}$ induced by a set $\mathcal{R}$ of DPO rewriting rules in $\mathfrak{F}$} is given by: $G \Rightarrow_{\mathcal{R},\mathfrak{F}} H$ iff $G \Rightarrow_{\rho,\mathfrak{F}} H$ for some $\rho \in \mathcal{R}$. When $\mathfrak{F}$ is clear from the context, we 
    suppress $\mathfrak{F}$ and 
    write $\Rightarrow_{\rho}$ and $\Rightarrow_{\mathcal{R}}$.
  \end{definition}
\subsection{Relative termination}
  \label{preliminaries:relative_termination}
\begin{definition}[Rewriting sequence]
    Let \(\mathcal{R}\) be a set of rewriting rules. Let $\mathfrak{F}$ be a DPO rewriting framework.
    A \textbf{$(\mathcal{R},\mathfrak{F})$-rewriting sequence} is a finite sequence \(s_0,s_1,\hdots, s_m\) of objects such that \(s_n \Rightarrow_{\mathcal{R},\mathfrak{F}} s_{n+1}\) for each \( 0 \leq n \leq m-1\), or an infinite sequence \(s_0,s_1,\hdots\) of objects such that \(s_n \Rightarrow_{\mathcal{R},\mathfrak{F}} s_{n+1}\) for each \(n \in \mathbb{N}\).
    A $(\mathcal{R},\mathfrak{F})$-rewriting sequence from \( s_0 \) will be denoted as:
    \(
    s_0 \Rightarrow_{\mathcal{R},\mathfrak{F}} s_1 \Rightarrow_{\mathcal{R},\mathfrak{F}} s_2 \Rightarrow_{\mathcal{R},\mathfrak{F}} \cdots 
    \)
\end{definition}
% \todo{plus generalement, c'est defini sur des relations, pas forcement de reecriture}
% \begin{definition} 
%     Let \(\mathcal{R}\) be a set of rewriting rules and let $\mathfrak{F}$ be a DPO rewriting framework.
%     An \textbf{$(\mathcal{R},\mathfrak{F})$-rewriting sequence} is either  
%     \begin{itemize}
%         \item a finite sequence \(s_0,s_1,\hdots, s_m\) of objects such that \( s_n \Rightarrow_{\mathcal{R},\mathfrak{F}} s_{n+1}\text{ for each } 0 \leq n \leq m-1\), or
%         \item an infinite sequence \(s_0,s_1,\hdots\) of objects such that \(s_n \Rightarrow_{\mathcal{R},\mathfrak{F}} s_{n+1}\) for each \(n \in \mathbb{N}\).
%     \end{itemize}
%     An $(\mathcal{R},\mathfrak{F})$-rewriting sequence from \( s_0 \) will be denoted by:
%     \[
%     s_0 \Rightarrow_{\mathcal{R},\mathfrak{F}} s_1 \Rightarrow_{\mathcal{R},\mathfrak{F}} s_2 \Rightarrow_{\mathcal{R},\mathfrak{F}} \cdots 
%     \]
% \end{definition}
Given a DPO rewriting framework \(\mathfrak{F}\), a rule set \(\mathcal{R}\) defines the binary relation \enquote{object $X$ can be rewritten to object $Y$ using rules from the rule set} on the objects of the category $\mathcal{C}$.
\begin{definition}\label{def:rewriting-chain}
Let \(\mathcal{R}\) be a rule set and let \(\mathfrak{F}\) be a DPO rewriting framework.
An \textbf{\((\mathcal{R},\mathfrak{F})\)-rewriting chain} is a \(\Rightarrow_{\mathcal{R},\mathfrak{F}}\)-chain. Hence, when the context is clear, we simply call it a \textbf{rewriting chain}.
\end{definition}

\begin{example}
    Consider the rewriting rules in~\autoref{fig:preliminaries:graph_transformation_rule_nonterminating}, which
    replaces an occurrence of the graph 
\raisebox{2pt}{
            \scalebox{0.7}{\tikz[baseline=-0.5ex]{
            \node [draw,circle] (z) at (-1,0) {};
            \node [draw,circle] (x) at (0,0) {};
            \node[draw,circle] (y) at (1,0) {};
            \draw[->] (z)--(x) node[midway, above] {$a$};
            \draw[->] (x)--(y) node[midway, above] {$b$};
        }}} with an occurrence of the graph \raisebox{2pt}{
            \scalebox{0.7}{\tikz[baseline=-0.5ex]{
            \node [draw,circle] (z) at (-1,0) {};
            \node [draw,circle] (x) at (0,0) {};
            \node[draw,circle] (y) at (1,0) {};
            \draw[->] (z)--(x) node[midway, above] {$b$};
            \draw[->] (x)--(y) node[midway, above] {$a$};
        }}}, keeping the extreme nodes unchanged.
    \begin{figure}[H]
        \centering
            \resizebox{0.85\textwidth}{!}{
                \begin{tikzpicture}[baseline=-3ex]
                    \graphbox{\( L \)}{0mm}{-3mm}{34mm}{15mm}{2mm}{2mm}{
                        \coordinate (o) at (0mm,-11mm); 
                        \node[draw,circle] (l1) at ($(o)+(-10mm,0mm)$) {1};
                        \node[draw,circle] (l2) at ($(l1)+(2,0)$) {2};
                        \node[draw,circle] (l3) at ($(l1) + (1,0)$) {3};
                        \draw[->] (l1) -- (l3) node[midway,above] {$a$};
                        \draw[->] (l3) -- (l2) node[midway,above] {$b$};
                    } 
            
                    \graphbox{\( K \)}{40mm}{-3mm}{34mm}{15mm}{2mm}{2mm}{
                        \coordinate (o) at (0mm,-11mm); 
                        \node[draw,circle] (l1) at ($(o)+(-10mm,0mm)$) {1};
                        \node[draw,circle] (l2) at ($(l1)+(2,0)$) {2};
                    }  
            
                    \graphbox{\( R \)}{80mm}{-3mm}{35mm}{15mm}{2mm}{2mm}{
                        \coordinate (o) at (-5mm,-11mm); 
                        \node[draw,circle] (l1) at ($(o)+(-10mm,0mm)$) {1};
                        % \node[draw,circle] (l2) at ($(l1)+(3,0)$) {2};
                        \node[draw,circle] (l3) at ($(l1) + (1,0)$) {4};
                        \node[draw,circle] (l4) at ($(l1) + (2,0)$) {2};
                        \draw[->] (l1) -- (l3) node[midway,above] {$b$};
                        \draw[->] (l3) -- (l4) node[midway,above] {$a$};
                        % \draw[->] (l4) -- (l2) node[midway,above] {$a$};
                    }    
                    \node () at (37mm,-10mm) {\( \leftarrowtail \)}; % K -> L
                    \node () at (77mm,-10mm) {\( \rightarrowtail \)}; % K -> R
                \end{tikzpicture}
                }
        \caption{}
        \label{fig:preliminaries:graph_transformation_rule_nonterminating}
    \end{figure} 
  
    A looping rewriting chain using this rule is shown in~\autoref{fig:preliminaries:sequence_of_transformation_infinite}, in which the subgraph to be replaced at each transformation step is highlighted in red.
       
        \begin{figure}[H]
           \centering
          \resizebox{0.85\textwidth}{!}{
            \tikz
            [baseline=-0.5ex]
            { 
                \node[draw,circle] (x) at (0,0) {};
                \node[draw,circle] (y) at (1,0) {};
                \node[draw,circle] (z) at (0.5,0.86) {};
                \draw[->,red] (x) -- node[midway,below] {$a$} (y) ;
                \draw[->,red] (y) -- node[midway,right] {$b$} (z) ;
                \draw[->] (z) -- node[midway,left] {$b$} (x) ;
            } 
            $\Rightarrow$ 
            \tikz[baseline=-0.5ex]{ 
                \node[draw,circle] (x) at (0,0) {};  
                \node[draw,circle] (y) at (1,0) {};
                \node[draw,circle] (z) at (0.5,0.86) {};
                \draw[->] (x) -- node[midway,below] {$b$} (y) ;
                \draw[->,red] (y) -- node[midway,right] {$a$} (z) ;
                \draw[->,red] (z) -- node[midway,left] {$b$} (x) ;
            }
            $\Rightarrow$ 
            \tikz[baseline=-0.5ex]{ 
                \node[draw,circle] (x) at (0,0) {};  
                \node[draw,circle] (y) at (1,0) {};
                \node[draw,circle] (z) at (0.5,0.86) {};
                \draw[->,red] (x) -- node[midway,below] {$b$} (y) ;
                \draw[->] (y) -- node[midway,right] {$b$} (z) ;
                \draw[->,red] (z) -- node[midway,left] {$a$} (x) ;
            }
            $\Rightarrow$
            \tikz[baseline=-0.5ex]{ 
                \node[draw,circle] (x) at (0,0) {};   
                \node[draw,circle] (y) at (1,0) {};
                \node[draw,circle] (z) at (0.5,0.86) {};
                \draw[->,red] (x) -- node[midway,below] {$a$} (y) ;
                \draw[->,red] (y) -- node[midway,right] {$b$} (z) ;
                \draw[->] (z) -- node[midway,left] {$b$} (x) ;
            }
          }
          \caption{}
          \label{fig:preliminaries:sequence_of_transformation_infinite}
        \end{figure}
\end{example}

For a set of rewriting rules \(\mathcal{R}\) (and a DPO rewriting framework \(\mathfrak{F}\)), the impossibility of transforming any object infinitely with the non-deterministic strategy \enquote{apply rules as long as possible} using rules from \(\mathcal{R}\) (in the framework \(\mathfrak{F}\)) is called \emph{termination} in the literature~\cite{middeldorp1997simple}. This property corresponds to program termination on all inputs in conventional programming languages, and is undecidable in general~\cite{plump1998terminationundecidable}.

For example of a non-terminating rule set, consider the set with the unique rule shown in~\autoref{fig:preliminaries:graph_transformation_rule_nonterminating} and the infinite rewriting chain shown in~\autoref{fig:preliminaries:sequence_of_transformation_infinite}. 
For example of a terminating rule set, consider the set with the unique rule shown in~\autoref{fig:intro:edge_deletion_andffsfjsssdkdsglkadjl}, which deletes an edge. Any rewriting chain using this rule is finite, because each rewriting step decreases the number of edges by one and a graph has only finitely many edges by definition.
 \begin{figure}[H]
    \centering
     \resizebox{0.85\textwidth}{!}{ 
                \begin{tikzpicture}[baseline=-3ex]
                    \graphbox{\( L \)}{0mm}{-3mm}{34mm}{15mm}{2mm}{2mm}{
                        \coordinate (o) at (0mm,-11mm); 
                        \node[draw,circle] (l1) at ($(o)+(-10mm,0mm)$) {1};
                        \node[draw,circle] (l2) at ($(l1)+(2,0)$) {2};
                        % \node[draw,circle] (l3) at ($(l1) + (1,0)$) {3};
                        \draw[->] (l1) -- (l2) node[midway,above] {$a$};
                    } 
            
                    \graphbox{\( K \)}{40mm}{-3mm}{34mm}{15mm}{2mm}{2mm}{
                        \coordinate (o) at (0mm,-11mm); 
                        \node[draw,circle] (l1) at ($(o)+(-10mm,0mm)$) {1};
                        \node[draw,circle] (l2) at ($(l1)+(2,0)$) {2};
                    }  
            
                    \graphbox{\( R \)}{80mm}{-3mm}{35mm}{15mm}{2mm}{2mm}{
                        \coordinate (o) at (-5mm,-11mm); 
                        \node[draw,circle] (l1) at ($(o)+(-10mm,0mm)$) {1};
                        % \node[draw,circle] (l2) at ($(l1)+(3,0)$) {2};
                        % \node[draw,circle] (l3) at ($(l1) + (1,0)$) {4};
                        \node[draw,circle] (l4) at ($(l1) + (2,0)$) {2};
                        \draw[->] (l1) -- (l4) node[midway,above] {$b$};
                        % \draw[->] (l4) -- (l2) node[midway,above] {$a$};
                    }    
                    \node () at (37mm,-10mm) {\( \leftarrowtail \)}; % K -> L
                    \node () at (77mm,-10mm) {\( \rightarrowtail \)}; % K -> R
                \end{tikzpicture}
                }
                \caption{}
                \label{fig:intro:edge_deletion_andffsfjsssdkdsglkadjl}
        \end{figure}

However, in many cases, an interesting property can be proved even if the whole rule set is not terminating. For example, consider the graph transformation system with two rules shown in~\autoref{fig:intro:edge_deletion_and_node_addition_ruledfakdsjflsdaj}: rule $\alpha$ deletes an arbitrary edge labeled by $A$, and rule $\beta$ introduces a fresh node.
  \begin{figure}[H]
        \centering
%   \begin{subfigure}{0.3\textwidth}
%         % \centering
        $\alpha$ = {
             \resizebox{0.7\textwidth}{!}{
             \begin{tikzpicture}[baseline=-7ex]
                    \graphbox{\( \mathcal{L} \)}{0mm}{-3mm}{34mm}{15mm}{2mm}{2mm}{
                        \coordinate (o) at (0mm,-11mm); 
                        \node[draw,circle] (l1) at ($(o)+(-10mm,0mm)$) {$1$};
                        \node[draw,circle] (l2) at ($(l1)+(2,0)$) {$2$};
                        \draw[->] (l1) -- (l2) node[midway,above] {$A$};
                    } 
            
                    \graphbox{\( \mathcal{K} \)}{40mm}{-3mm}{34mm}{15mm}{2mm}{2mm}{
                        \coordinate (o) at (0mm,-11mm); 
                        \node[draw,circle] (l1) at ($(o)+(-10mm,0mm)$) {$1$};
                        \node[draw,circle] (l2) at ($(l1)+(2,0)$) {$2$};
                    }  
            
                    \graphbox{\( \mathcal{R} \)}{80mm}{-3mm}{35mm}{15mm}{5mm}{2mm}{
                        \coordinate (o) at (-5mm,-11mm); 
                        \node[draw,circle] (l1) at ($(o)+(-10mm,0mm)$) {$1$};
                        \node[draw,circle] (l4) at ($(l1) + (2,0)$) {$2$};
                    }    
                    \node () at (37mm,-10mm) {\( \leftarrowtail \)}; % K -> L
                    \node () at (77mm,-10mm) {\( \rightarrowtail \)}; % K -> R
                \end{tikzpicture}
            }
        }
    %     \caption{A graph transformation rule for edge deletion}
    % \label{fig:intro:edge_deletion_rule}
    % \end{subfigure}
    
    % \begin{subfigure}{0.3\textwidth}
    %     % \centering

        $\beta$ ={
             \resizebox{0.7\textwidth}{!}{
             \begin{tikzpicture}[baseline=-7ex]
                    \graphbox{\( \mathcal{L} \)}{0mm}{-3mm}{34mm}{15mm}{2mm}{2mm}{
                        
                    } 
            
                    \graphbox{\( \mathcal{K} \)}{40mm}{-3mm}{34mm}{15mm}{2mm}{2mm}{
                       
                    }  
            
                    \graphbox{\( \mathcal{R} \)}{80mm}{-3mm}{35mm}{15mm}{5mm}{2mm}{
                        \coordinate (o) at (0mm,-11mm); 
                        \node[draw,circle] (l1) at ($(o)+(-10mm,0mm)$) {};
                    }    
                    \node () at (37mm,-10mm) {\( \leftarrowtail \)}; % K -> L
                    \node () at (77mm,-10mm) {\( \rightarrowtail \)}; % K -> R
                \end{tikzpicture}
            }
        }
    %     \caption{A graph transformation rule for node addition}
    %     \label{fig:intro:node_addition_rule}
    % \end{subfigure} 
    \caption{}
    \label{fig:intro:edge_deletion_and_node_addition_ruledfakdsjflsdaj}
  \end{figure}
The system does not terminate because the node-adding rule $\beta$ can be applied indefinitely. However, the edge-deleting rule $\alpha$ can be applied a finite number of times only: it deletes an edge on each application, and since no rule increases the edge count and the initial graph is finite, only finitely many deletions are possible. Therefore, termination of the full system depends solely on the node-adding rule $\beta$. This observation motivates the more general property called \emph{relative termination}, which was originally introduced by Klop in~\cite{klop1987term} for binary relations, and has been studied and employed in the context of rewriting systems~\cite{geser1990relative,kassing2024dependency,endrullis2024generalized_icgt,zantema2014termination,bruggink2014termination,bruggink2015proving}. 

\begin{definition}
Let $A$ be a collection of objects and let $R$ and $S$ be binary relations on $A$. 
We say that $R$ is \textbf{terminating relative to} $S$ (or that $R$ \textbf{terminates relative to} $S$) if 
any $(R \cup S)$-chain contains only finitely many $R$-steps.
In particular, $R$ is \textbf{terminating} (or terminates) if it is terminating relative to the empty relation $\emptyset$.
\end{definition}

For example, the relation $>$ on $\mathbb{N}$ is terminating relative to the empty relation, since any $(> \cup \emptyset)$-chain contains only finitely many $>$-steps. The relation $>$ is also terminating relative to $\geq$ on $\mathbb{N}$, since any $(> \cup \geq)$-chain can only contain finitely many $>$-steps.

Note that termination and well-foundedness are the same property of a binary relation $\to$ described from two perspectives. Operationally, we read $x \to y$ as \enquote{$x$ can be transformed to $y$}; the absence of infinite $\rightarrow$-chains therefore means that any transformation sequence starting from an initial object $x$ is finite, i.e. every computation or rewrite sequence eventually terminates. Structurally, we read $x \to y$ as \enquote{$x$ is constructed from $y$}; the absence of infinite $\to$-chains then implies that every object $x$ can be traced back along a finite chain to an element that is not constructed from any other, so the structure is well-founded. In this thesis we adopt the operational viewpoint and use the term termination.

Relative termination 
carries over to rewriting systems in a straightforward way via their associated rewriting relations.
\begin{definition}
    \label{termination:def:relative_termination}
     Let $\mathcal{R}$ and $\mathcal{S}$ be sets of rewriting rules and let $\mathfrak{F}$ be a DPO rewriting framework. 
     We say that $\Rightarrow_{\mathcal{R},\mathfrak{F}}$ is \textbf{terminating relative to} $\Rightarrow_{\mathcal{S}, \mathfrak{F}}$ if 
     $\Rightarrow_{\mathcal{R},\mathfrak{F}}$ is \textbf{terminating relative to} $\Rightarrow_{\mathcal{S}, \mathfrak{F}}$.
\end{definition}
In practice, to prove termination of a rewriting system $\mathcal{R}$, one partitions the set of rules into two disjoint subsets \( \mathcal{B} \) and \( \mathcal{A} \) with non-empty $\mathcal{A}$ such that \( \mathcal{A} \) terminates relative to \( \mathcal{B} \), if $\mathcal{B}$ is not empty then the termination of $\mathcal{R}$ is established, otherwise, a new iteration starts with the strictly smaller rule set $\mathcal{B}$.
 
 
\section{Strongly monotonic measurable semiring}
\label{sec:strongly_monotonic_measurable_semiring}
We introduce the notion of a strongly monotonic measurable semiring.
The key difference with well-founded semirings (Definition~\ref{def:well_founded_semiring}) is that the semiring is required to be equipped with a homomorphism to the extended real numbers instead of being well-founded.
Hereafter, $<$ and $\leq$ denote the canonical irreflexive and reflexive orders on the set of extended real numbers $\overline{\mathbb{R}} \mathop{=} \mathbb{R} \mathop{\cup} \{\mathop{-\infty}, \mathop{+\infty}\}$.
\begin{definition} 
    \label{def:nwf:real_strongly_monotonic_semiring}
    A \textbf{strongly monotonic measurable semiring} $(S, \mathop{\oplus}, \mathop{\odot}, 0, 1, \prec, \mu)$ consists of
    \begin{itemize} 
        \item A commutative semiring $(S, \mathop{\oplus}, \mathop{\odot}, 0, 1)$,
        \item A non-empty irreflexive order $\prec$ on $S$,
        \item A homomorphism $\mu : (S, \prec) \mathop{\to} ( \overline{\mathbb{R}}, <_{\overline{\mathbb{R}}} )$,
    \end{itemize}
    such that $0 \mathop{\neq} 1$ and for all $x,y,z,w \mathop{\in} S$, for all $\delta \mathop{\in} \mathbb{R}$ and $\delta>0$, we have
        \begin{align*}
            1 \mathop{\preceq} x \mathop{\land} 1 \mathop{\preceq} y 
            &\mathop{\Rightarrow}
            1 \mathop{\preceq} x \mathop{\oplus} y,
            &\tag{S0} \label{ax:s0} 
            \\ 
            x \mathop{\preceq} x' \mathop{\land} y \mathop{\preceq} y' 
            &\mathop{\Rightarrow}
            x \mathop{\oplus} y \mathop{\preceq} x' \mathop{\oplus} y',
            &\tag{S1} \label{ax:s1} 
            \\   
            % x < y  
            % &\mathop{\Rightarrow}
            % x \mathop{\oplus} z \leq y \mathop{\oplus} z 
            % \tag{S1} \label{eq:ordered_semiring_plus_monotonic} 
            % \\ w
            x \mathop{\prec} x' \mathop{\land} y \mathop{\prec} y'  
            &\mathop{\Rightarrow}
            x \mathop{\oplus} y \mathop{\prec} x' \mathop{\oplus} y',
            &\tag{S2} \label{ax:s2} 
            \\
            \delta\mathop{+}\mu(x) <_{\overline{\mathbb{R}}} \mu(y) \mathop{\land} \delta\mathop{+}\mu(z) <_{\overline{\mathbb{R}}} \mu(w)
            &\mathop{\Rightarrow}
            \delta\mathop{+}\mu(x \mathop{\oplus} z) <_{\overline{\mathbb{R}}} \mu(y \mathop{\oplus} w),
            &\tag{S3} \label{ax:s2'}
            \\
            x \mathop{\preceq} x'
            &\mathop{\Rightarrow} 
            x \mathop{\odot} y \mathop{\preceq} x' \mathop{\odot} y,
            &\tag{S4} \label{ax:s3} 
            \\
            x \mathop{\prec} x' \mathop{\land} y \mathop{\neq} 0 
            &\mathop{\Rightarrow}
            x \mathop{\odot} y \mathop{\prec} x' \mathop{\odot} y,
            &\tag{S5} \label{ax:s4}
            \\ 
            \delta\mathop{+}\mu(x) <_{\overline{\mathbb{R}}} \mu(y) \mathop{\land} 1 \mathop{\preceq} z \mathop{\land} z \mathop{\neq} 0
            &\mathop{\Rightarrow}
            \delta\mathop{+}\mu(x \mathop{\odot} z) <_{\overline{\mathbb{R}}} \mu(y \mathop{\odot} z),
            &\tag{S6} \label{ax:s4'}
            \\
            \delta+ \mu(x) <_{\overline{\mathbb{R}}} \mu(x') \mathop{\land} y \mathop{\neq} 0
            &\mathop{\Rightarrow}
            \mu(x \mathop{\odot} y) <_{\overline{\mathbb{R}}} \mu(x' \mathop{\odot} y).
            &\tag{S7} \label{ax:s4''}
        %    \\
            % \\
            % 1 \leq z \mathop{\neq} 0 \mathop{\land} X < Y  
            % &\mathop{\Rightarrow}
            % \exists \mu(x * z) < \mu( y * z)
            % \tag{S101} \label{eq:strongly_ordered_measurable_semiring_lt_preserved_neq0_geq1}  
        %      \\     
        %     a\mathop{+}X < Y \mathop{\land} z \mathop{\neq} 0 
        %    &\mathop{\Rightarrow}
        %    \exists b> 0. b\mathop{+}\mu(x* z) < \mu(y * z) 
        %    \tag{S3} \label{eq:ordered_semiring_times_stable_under_mesure} 
        \end{align*}
        where $\mathop{\preceq}$ denotes the reflexive closure of $\prec$. The semiring is a \textbf{strictly monotonic measurable semiring} if it additionally satisfies 
    \begin{flalign*}
        \hspace{4.5cm} x \mathop{\prec} x' 
        &\mathop{\Rightarrow}
        x \mathop{\oplus} y \mathop{\prec} x' \mathop{\oplus} y,
        &\tag{S8} \label{ax:s5} 
        \\
        \delta\mathop{+}\mu(x) <_{\overline{\mathbb{R}}} \mu(x')
        &\mathop{\Rightarrow}
        \delta\mathop{+}\mu(x \mathop{\oplus} y) <_{\overline{\mathbb{R}}} \mu(x' \mathop{\oplus} y).
        &\tag{S9} \label{ax:s5'}
    \end{flalign*}
\end{definition} 
\begin{example} 
    % The real tropical semiring $\mathfrak{T}' \mathop{=} (\mathbb{R} \mathop{\cup} \{\mathop{+\infty}\}, \mathop{\min},+,\mathop{+\infty}, 0_\mathbb{R},<,\operatorname{id}_{\mathbb{R} \mathop{\cup} \{\mathop{+\infty}\}})$ has domain $\mathbb{R} \mathop{\cup} \{\mathop{+\infty}\}$, the binary function symbol $\mathop{\oplus}$ interpreted by $\min$ and the binary function symbol $\mathop{\odot}$ interpreted by $+$, the constant symbols $0_s$ and $1_s$ interpreted by $\mathop{+\infty}$ and $0_\mathbb{R}$, respectively, the binary relation symbol $\prec$ interpreted by the canonical order $<$ on $\mathbb{R} \mathop{\cup} \{\mathop{+\infty}\}$, and the unary function symbol $\mu$ interpreted by the identity function on $\mathbb{R} \mathop{\cup} \{\mathop{+\infty}\}$. The real tropical semiring is a strongly monotonic measurable semiring. It is not strictly monotonic measurable because $2 < 3$ but $2 \mathop{\oplus} 2 \mathop{=} \min(2,2) \mathop{=} 2 \not < 2 \mathop{=} \min(3,2) \mathop{=} 3 \mathop{\oplus} 2$.

     The real tropical semiring $$\mathfrak{T}' \isdef (
        \mathbb{R} \mathop{\cup} \{\mathop{+\infty}\}, 
        \opn{\min}_{\mathbb{R} \mathop{\cup} \{\mathop{+\infty}\}},
        +_{\mathbb{R} \mathop{\cup} \{\mathop{+\infty}\}},\\
        \mathop{\mathop{+\infty}},
        0_{\mathbb{R} \mathop{\cup} \{\mathop{+\infty}\}},
        <_{\mathbb{R} \mathop{\cup} \{\mathop{+\infty}\}},
        \operatorname{id}_{\mathbb{R} \mathop{\cup} \{\mathop{+\infty}\}})$$ is an instance of the strongly monotonic measurable semiring where
     \begin{flalign*}
         S & \mathop{\longmapsto} \mathbb{R} \mathop{\cup} \{\mathop{+\infty}\},
         \\
         \mathop{\oplus} & \mathop{\longmapsto} \operatorname{min}_{\mathbb{R} \mathop{\cup} \{\mathop{+\infty}\}},
         \\
         \mathop{\odot} & \mathop{\longmapsto} +_{\mathbb{R} \mathop{\cup} \{\mathop{+\infty}\}},
         \\
         0_s & \mathop{\longmapsto} \mathop{+\infty},
         \\
         1_s & \mathop{\longmapsto} 0_{\mathbb{R} \mathop{\cup} \{\mathop{+\infty}\}},
         \\
         \mathop{\prec} & \mathop{\longmapsto} <_{\mathbb{R} \mathop{\cup} \{\mathop{+\infty}\}},
         \\
         \mu & \mathop{\longmapsto} \operatorname{id}_{\mathbb{R} \mathop{\cup} \{\mathop{+\infty}\}}.
     \end{flalign*}
    It is a strongly monotonic measurable semiring but not strictly monotonic measurable, because we have $2 <_{\mathbb{R} \mathop{\cup} \{\mathop{+\infty}\}} 3$ but $$2 \mathop{\oplus} 2 \isdef \operatorname{min}_{\mathbb{R} \mathop{\cup} \{\mathop{+\infty}\}}(2,2) \mathop{=} 2 \not <_{\mathbb{R} \mathop{\cup} \{\mathop{+\infty}\}} 2 \mathop{=} \operatorname{min}_{\mathbb{R} \mathop{\cup} \{\mathop{+\infty}\}}(3,2) \mathop{=} 3 \isdef 2.$$
\end{example}
\begin{example}
    % The real arctic semiring $\mathfrak{A}' \mathop{=} (\mathbb{R} \mathop{\cup} \{\mathop{-\infty}\},\max,+,\mathop{-\infty}, 0_\mathbb{R},<,\operatorname{id}_{\mathbb{R} \mathop{\cup} \{\mathop{-\infty}\}})$ has domain $\mathbb{R} \mathop{\cup} \{\mathop{-\infty}\}$, the binary function symbol $\mathop{\oplus}$ interpreted by $\max$ and the binary function symbol $\mathop{\odot}$ interpreted by $+$, the constant symbols $0_s$ and $1_s$ interpreted by $\mathop{-\infty}$ and $0_\mathbb{R}$, respectively, the binary relation symbol $\prec$ interpreted by the canonical order $<$ on $\mathbb{R} \mathop{\cup} \{\mathop{-\infty}\}$, and the unary function symbol $\mu$ interpreted by the identity function on $\mathbb{R} \mathop{\cup} \{\mathop{-\infty}\}$. The real arctic semiring is a strongly monotonic measurable semiring. It is not strictly monotonic measurable because $2 < 3$ but $2 \mathop{\oplus} 3 \mathop{=} \max(2,3) \mathop{=} 3 \not < 3 \mathop{=} \max(3,3) \mathop{=} 3 \mathop{\oplus} 3$.
    The real arctic semiring $$\mathfrak{A}' \isdef (\mathbb{R} \mathop{\cup} \{\mathop{-\infty}\},\operatorname{max}_{\mathbb{R} \mathop{\cup} \{\mathop{-\infty}\}}, +_{\mathbb{R} \mathop{\cup} \{\mathop{-\infty}\}}, 
    \\
      \mathop{\mathop{-\infty}}, 0_{\mathbb{R} \mathop{\cup} \{\mathop{-\infty}\}}, <_{\mathbb{R} \mathop{\cup} \{\mathop{-\infty}\}}, \operatorname{id}_{\mathbb{R} \mathop{\cup} \{\mathop{-\infty}\}})$$ is an instance of the strongly monotonic measurable semiring where
    \begin{flalign*}
        S & \mathop{\longmapsto} \mathbb{R} \mathop{\cup} \{\mathop{-\infty}\},
        \\
        \mathop{\oplus} & \mathop{\longmapsto} \operatorname{\max}_{\mathbb{R} \mathop{\cup} \{\mathop{-\infty}\}},
        \\
        \mathop{\odot} & \mathop{\longmapsto} +_{\mathbb{R} \mathop{\cup} \{\mathop{-\infty}\}},
        \\
        0_s & \mathop{\longmapsto} \mathop{-\infty},
        \\
        1_s & \mathop{\longmapsto} 0_{\mathbb{R} \mathop{\cup} \{\mathop{-\infty}\}},
        \\
        \mathop{\prec} & \mathop{\longmapsto} <_{\mathbb{R} \mathop{\cup} \{\mathop{-\infty}\}},
        \\
        \mu & \mathop{\longmapsto} \operatorname{id}_{\mathbb{R} \mathop{\cup} \{\mathop{-\infty}\}}.
    \end{flalign*}
    It is a strongly monotonic measurable semiring but not strictly monotonic measurable, because we have $2 <_{\mathbb{R} \mathop{\cup} \{\mathop{-\infty}\}} 3$ but $$2 \mathop{\oplus} 3 \isdef \operatorname{\max}_{\mathbb{R} \mathop{\cup} \{\mathop{-\infty}\}}(2,3) \mathop{=} 3 \not <_{\mathbb{R} \mathop{\cup} \{\mathop{-\infty}\}} 3 \mathop{=} \operatorname{\max}_{\mathbb{R} \mathop{\cup} \{\mathop{-\infty}\}}(3,3) \isdef 3 \mathop{\oplus} 3.$$
%    The real arctic semiring: $\mathfrak{A}' \mathop{=} (\mathbb{R} \mathop{\cup} \{\mathop{-\infty}\},\max,+,\mathop{-\infty}, 0,<,\operatorname{id}_{\mathbb{R} \mathop{\cup} \{\mathop{-\infty}\}})$.
\end{example}
\begin{example}
    The real arithmetic semiring $$\mathfrak{N}' \isdef (\mathbb{R}^+,+_{\mathbb{R}^+},*_{\mathbb{R}^+},0_{\mathbb{R}^+},1_{\mathbb{R}^+},<_{\mathbb{R}^+},\operatorname{id}_{\mathbb{R}^+})$$ is an instance of the strongly monotonic measurable semiring where
    \begin{flalign*}
        S & \mathop{\longmapsto} \mathbb{R}^+,
        \\
        \mathop{\oplus} & \mathop{\longmapsto} +_{\mathbb{R}^+},
        \\
        \mathop{\odot} & \mathop{\longmapsto} *_{\mathbb{R}^+},
        \\
        0_s & \mathop{\longmapsto} 0_{\mathbb{R}^+},
        \\
        1_s & \mathop{\longmapsto} 1_{\mathbb{R}^+},
        \\
        \mathop{\prec} & \mathop{\longmapsto} <_{\mathbb{R}^+},
        \\
        \mu & \mathop{\longmapsto} \operatorname{id}_{\mathbb{R}^+}.
    \end{flalign*} 
    It is a strictly monotonic measurable semiring. 
    % The real arithmetic semiring $\mathfrak{N}' \mathop{=} (\mathbb{R}^+,+,*,0_\mathbb{R},1_\mathbb{R},<,\operatorname{id}_{\mathbb{R}^+})$ has as domain the set $\mathbb{R}^+$ of positive real numbers, the binary function symbol $\mathop{\oplus}$ interpreted by $+$ and the binary function symbol $\mathop{\odot}$ interpreted by $*$, the constant symbols $0_s$ and $1_s$ interpreted by $0_\mathbb{R}$ and $1_\mathbb{R}$, respectively, the binary relation symbol $\prec$ interpreted by the canonical order $<$ on $\mathbb{R}^+$, and the unary function symbol $\mu$ interpreted by the identity function on $\mathbb{R}^+$. The real arithmetic semiring is a strictly monotonic measurable semiring. 
\end{example}
\begin{example} 
    \label{example:real_semirings}
    The natural tropical semiring $$\mathfrak{T} \isdef (\mathbb{N} \mathop{\cup} \{\mathop{+\infty}\},\operatorname{min}_\mathbb{N},+_{\mathbb{N}},\mathop{+\infty}, 0_\mathbb{N}, <_{\mathbb{N}} , \operatorname{id}_{\mathbb{N} \mathop{\cup} \{\mathop{+\infty}\}})$$ is an instance of the strongly monotonic measurable semiring where
    \begin{flalign*}
        S & \mathop{\longmapsto} \mathbb{N} \mathop{\cup} \{\mathop{+\infty}\},
        \\
        \mathop{\oplus} & \mathop{\longmapsto} \operatorname{min}_\mathbb{N},
        \\
        \mathop{\odot} & \mathop{\longmapsto} +_\mathbb{N},
        \\
        0_s & \mathop{\longmapsto} \mathop{+\infty},
        \\
        1_s & \mathop{\longmapsto} 0_\mathbb{N},
        \\
        \mathop{\prec} & \mathop{\longmapsto} <_\mathbb{N},
        \\
        \mu & \mathop{\longmapsto} \operatorname{id}_\mathbb{N}.
    \end{flalign*}
    The natural arctic semiring $$\mathfrak{A} \isdef (\mathbb{N} \mathop{\cup} \{\mathop{-\infty}\},\operatorname{max}_\mathbb{N},+_{\mathbb{N}},\mathop{-\infty}, 0_\mathbb{N},<_{\mathbb{N}}, \operatorname{id}_{\mathbb{N} \mathop{\cup} \{\mathop{-\infty}\}})$$ is an instance of the strongly monotonic measurable semiring where
    \begin{flalign*}
        S & \mathop{\longmapsto} \mathbb{N} \mathop{\cup} \{\mathop{-\infty}\},
        \\
        \mathop{\oplus} & \mathop{\longmapsto} \operatorname{max}_\mathbb{N},
        \\
        \mathop{\odot} & \mathop{\longmapsto} +_\mathbb{N},
        \\
        0_s & \mathop{\longmapsto} \mathop{-\infty},
        \\ 
        1_s & \mathop{\longmapsto} 0_\mathbb{N},
        \\
        \mathop{\prec} & \mathop{\longmapsto} <_\mathbb{N},
        \\
        \mu & \mathop{\longmapsto} \operatorname{id}_\mathbb{N}.
    \end{flalign*}  
    The natural arithmetic semiring $$\mathfrak{N} \isdef (\mathbb{N},+_\mathbb{N},*_\mathbb{N},+_{\mathbb{N}},0_\mathbb{N},1_\mathbb{N},<_\mathbb{N},\operatorname{id}_\mathbb{N})$$ is an instance of the strictly monotonic measurable semiring where
    \begin{flalign*}
        S & \mathop{\longmapsto} \mathbb{N},
        \\
        \mathop{\oplus} & \mathop{\longmapsto} +_\mathbb{N},
        \\
        \mathop{\odot} & \mathop{\longmapsto} *_\mathbb{N},
        \\
        0_s & \mathop{\longmapsto} 0_\mathbb{N},
        \\
        1_s & \mathop{\longmapsto} 1_\mathbb{N},
        \\
        \mathop{\prec} & \mathop{\longmapsto} <_\mathbb{N},
        \\
        \mu & \mathop{\longmapsto} \operatorname{id}_\mathbb{N}.
    \end{flalign*}
\end{example}

% \begin{notation} 
%     \label{def:bigodot}
% Let $(S, \mathop{\oplus}, \mathop{\odot}, 0_s, 1_s)$ be a semiring. We extend naturally the binary operations $\mathop{\oplus}$ and $\mathop{\odot}$ to finite sets $E \mathop{\subseteq} S$ by letting
%     \begin{itemize}
%         \item $\mathop{\bigodot} \emptyset \overset{\operatorname{def}}{=} 1_s$ and $\mathop{\bigodot} \left( E \mathop{\cup} \{x\} \right) \overset{\operatorname{def}}{=} \left( \mathop{\bigodot} E \right) \mathop{\odot} x$;
%         \item $\mathop{\bigoplus} \emptyset \overset{\operatorname{def}}{=} 0_s$ and $\mathop{\bigoplus} \left( E \mathop{\cup} \{x\} \right) \overset{\operatorname{def}}{=} \left( \mathop{\bigoplus} E \right) \mathop{\oplus} x$.
%     \end{itemize}
% %  \begin{flalign*}
% %     \mathop{\bigodot} \emptyset &\overset{\operatorname{def}}{=} 1_s
% % \\
% %     \mathop{\bigodot} \left( E \mathop{\cup} \{x\} \right) &\overset{\operatorname{def}}{=} \left( \mathop{\bigodot} E \right) \mathop{\odot} x
% %     \\
% %     \mathop{\bigoplus} \emptyset &\overset{\operatorname{def}}{=} 0_s
% %     \\
% %         \mathop{\bigoplus} \left( E \mathop{\cup} \{x\} \right) &\overset{\operatorname{def}}{=} \left( \mathop{\bigoplus} E \right) \mathop{\oplus} x
% % \end{flalign*}
% \end{notation}

% \textcolor{blue}{\begin{remark}
%     \label{remark:diff_measurable_semiring}
% A strongly monotonic measurable semiring differs from a well-founded strongly monotonic semiring in~\cite{endrullis2024generalized} in four ways:
% \begin{enumerate}[label=(\arabic*),noitemsep]
%     \item Replacing well-foundedness with the homomorphism~$\mu$ and introducing Axioms \eqref{ax:s2'}, \eqref{ax:s4'}, and \eqref{ax:s4''}. This modification enables the use of non-well-founded semirings (e.g., as in Example~\ref{example:real_semirings}).
%     % \item Removing the condition $1_S \mathop{\preceq} y$ from the original Axioms~(S3) and~(S4)\todo{This is not helpful}, resulting in our Axioms~(S4) and~(S5). This relaxation allows inclusion of elements smaller than $1_S$, as motivated in~\autoref{remark:greater_than_1}.
%     \item Adding Axiom Equation~\eqref{ax:s0}. This technical adjustment ensures that if every $\mathcal{T}$-valued element of a type graph (see Definition~\ref{def:weighted_type_graph}) has a weight greater than $1_S$, then all objects subject to rewriting inherit this property.
%     \item Defining $\mathop{\preceq}$ as the reflexive closure of $\prec$ to simplify the theory. This is motivated by the fact that concrete semirings proposed in prior work and our paper satisfy this property.
% \end{enumerate} 
% \end{remark}} 
% \section{Measuring objects by counting morphisms}
% \label{sec:type_graph:measuring_graphs}
% % Let $\mathcal{C}$ be an arbitrary category. 
% The type graph method is parameterized by a tuple \(\mathcal{T} = (T, \mathbb{E}, \mathcal{S}, w)\), called a \textbf{weighted type graph}~\cite{endrullis2024generalized_arxiv_v2}, consists of:
%     \begin{itemize} 
%         \item an object \(T\) in $\mathcal{C}$, called \textbf{type graph},
%         \item a strongly monotonic, measurable semiring \(\mathcal{S}=(S, \oplus, \odot, 0_\mathcal{S}, 1_\mathcal{S}, \prec, \mu)\),
%         \item a set \(\mathbb{E}\) of morphisms with codomain $T$ in $\mathcal{C}$, called \textbf{morphism-rulers}, 
%         \item a weight function \(w : \mathbb{E} \to S \setminus \{0_\mathcal{S}\}\),
%         \item for every \( (e :X \to T) \in \mathbb{E}\) and every object \(G\), the sets \(\operatorname{Hom}(X, G)\) and \(\operatorname{Hom}(G, T)\) are finite.
%     \end{itemize}

% \begin{example}
%     \label{example:weighted_type_graph}
%      In \textbf{Graph}, a weighted type graph can be visualized as a graph with weighted labels and weights given as superscripts as proposed in \cite{bruggink2015proving} if $\operatorname{dom}(\mathbb{E})$ consists of graphs with two vertices and one labeled edge between them. For example, the weighted type graph $\mathcal{T} = (T, \mathbb{E}, \mathcal{S}, w)$ with $\mathcal{S}$ as the real arithmetic semiring $(\mathbb{R}^+, +, *, 0_\mathbb{R}, 1_\mathbb{R}, <, \operatorname{id}_{\mathbb{R}^+})$,
%      $T$ as the graph illustrated below (without superscripts), $\mathbb{E}=\{e_{11a},e_{12a},e_{21a},e_{11b}\}$ as the set of morphism-rulers where 
%      $e_{uvl}$ has domain 
%      \tikz[baseline=-0.5ex]{
%         \node (x) at (0,0) {$\bullet$};
%         \node (y) at (1,0) {$\bullet$};
%         \draw[->] (x) -- (y) node[midway, above] {$l$};
%     } and image 
%     \begin{tikzpicture}
%         \node[draw, circle] (x) at (0,0) {$\mathrm{u}$};
%         \node[draw, circle] (y) at (1,0) {$\mathrm{v}$};
%         \draw[->]  (x) -- (y) node [midway,above] {$l$};
%     \end{tikzpicture} in the graph $T$,
%     and $w(e) = 1$ for all $e \in \mathbb{E}$, can be visualized as follows:
%     \begin{center}
%         \begin{tikzpicture}
%             \graphbox{}{0mm}{0mm}{32mm}{28mm}{-10mm}{-14mm}{
%                 \node[draw,circle] (1) at (0,0) {1};
%                 \node[draw,circle] (2) at (2,0) {2};
%                 \draw[->] (1) edge[loop above] node[midway, above] {$a^{1}$} (1) ;
%                 \draw[->] (1) edge[loop below] node[midway, below] {$b^{1}$} (1) ;
%                 \draw[->] (1) edge[bend left] node[midway, above] {$a^{1}$}  (2)  ;
%                 \draw[->] (2) edge[bend left] node[midway, below] {$a^{1}$} (1)   ;
%             }
%         \end{tikzpicture}
%     \end{center}
% \end{example}

% Let \(\mathcal{T} = (T, \mathbb{E}, \mathcal{S}, w)\) be a weighted type graph and \(G\) an object of the category $\mathcal{C}$.

% Analogous to measuring a physical object, a morphism ruler measures a morphism.
% Let \(\mathcal{T} = (T, \mathbb{E}, \mathcal{S}, w)\) be a weighted type graph and \(G\) an object of the category $\mathcal{C}$. Consider the morphism-ruler \(e\colon X\to T\) and the morphism \(h\colon G\to T\) in~\ref{fig:wf:measurement_of_a_morphism_relative_to_a_morphism_ruler}. The measurement of \(h\) relative to \(e\) is the number of morphisms \(\iota\colon X\to G\) with \(h\circ\iota = e\). 
In this section, we recall the definition of morphism weight and object weight relative to a type graph from~\cite{endrullis2024generalized_icgt}.
% Analogous to measuring a physical object, a morphism ruler measures a morphism by counting the maps \(\iota\) that make the triangle in Figure~\ref{fig:wf:measurement_of_a_morphism_relative_to_a_morphism_ruler} commute. Concretely, given a morphism-ruler \(e\colon X\to T\) and a morphism \(h\colon G\to T\), the measurement of \(h\) relative to \(e\) is the number of morphisms \(\iota\colon X\to G\) with \(h\circ\iota = e\).

    % Analogous to the measurement of a physical object, a morphism-ruler measures a morphism, and provides a measurement, which is the number of $\iota$ that make the commutative triangle shown in~\autoref{fig:wf:measurement_of_a_morphism_relative_to_a_morphism_ruler}.
    \begin{figure}[H]
        \centering
            \begin{tikzpicture}
                \node (a) at (0,0) {$X$};
                \node (c) at (4,0) {$G$};
                \node (d) at (2,-1) {$T$};
                \draw[->] (a) -- (d) node [midway,below] {$e$};
                \draw[->] (a) -- (c) node [midway,above] {$g$};
                \draw[->] (c) -- (d) node[midway, below] {$h$};
                % \node (d) at (2,-0.5) {=};
            \end{tikzpicture}
        \caption{}
        \label{fig:wf:measurement_of_a_morphism_relative_to_a_morphism_ruler}
    \end{figure}  
\begin{notation}[\cite{endrullis2024generalized_arxiv_v2}]
Let \(A,B,C\) be objects. For morphism $\alpha\colon A\to B$, we introduce the notation 
          $$\set{ \alpha \star - = \gamma } \overset{\operatorname{def}}{=} \{ \beta \in \operatorname{Hom}(B, C) \mid \alpha \star \beta = \gamma \}$$ 
For morphism $\beta\colon B\to C$, we introduce the notation
          $$\set{ - \star \beta = \gamma }  \overset{\operatorname{def}}{=} \{ \alpha \in \operatorname{Hom}(A, B) \mid \alpha \star \beta = \gamma \}$$
For morphism set $E \subseteq \operatorname{Hom}(A,B)$ and morphism $\beta\colon B\to C$, we introduce the notation
          $$E \star \beta \overset{\operatorname{def}}{=} \set{ \alpha \star \beta \mid \alpha \in E }$$
\end{notation}
Analogous to measuring a physical object, a morphism ruler measures a morphism (\autoref{def:measurement_of_a_morphism_relative_to_a_morphism_ruler}).
Let \(\mathcal{T}=(T,\mathbb{E},\mathcal{S},w)\) be a weighted type graph and let \(G\) be an object of the category \(\mathcal{C}\).
Given a morphism-ruler \(e\colon X\to T\) and a morphism \(h\colon G\to T\) as shown in~\autoref{fig:wf:measurement_of_a_morphism_relative_to_a_morphism_ruler},
the measurement of \(h\) relative to \(e\) is the number of morphisms \(g \colon X\to G\) with \(h\circ g = e\). 
\begin{definition}[Morphism measurement]
    \label{def:measurement_of_a_morphism_relative_to_a_morphism_ruler}
    The \emph{measurement} of a morphism \( h:G \to T \) relative to a morphism-ruler \( e: X \to T \), denoted by $m_e(h)$, is defined as:
                \(
                m_e(h) 
                    \overset{\operatorname{def}}{=}
                \card{\{- \star h = e\}}
                \)
\end{definition}
Combining the measurements of a morphism provided by all morphism-rulers with the help of a weight function $w$ gives the weight of the morphism (\autoref{def:weight_of_a_morphism_relative_to_a_type_graph}).
 We define the exponentiation operation for elements of the semiring before defining formally the morphism weight.
\begin{notation}[Exponentiation operation] 
    \label{wfs:def:exponentiation}
Let $(S, \oplus, \odot, 0_S, 1_S)$ be a semiring. We define the exponentiation operation for all $x \in S$ and $n \in \mathbb{N}$ by
   \begin{itemize}
        \item $ x^0 \isdef 1_S$,
        \item $x^{n+1} \isdef x^n \odot x$
   \end{itemize}
\end{notation}
\begin{definition}[Morphism weight]
    \label{def:weight_of_a_morphism_relative_to_a_type_graph}
        Let $\mathcal{T}=(T,\mathbb{E},\mathcal{S},w)$ be a finitary weighted type graph.
         The \textbf{weight of a morphism $h: G \rightarrow T$ relative to a type graph $\mathcal{T}$} is defined as the semiring product of $w(e)^{m_e(h)}$ for all $e \in \mathbb{E}$:
        \[  w_{\mathcal{T}}(h) \overset{\operatorname{def}}{=} \underset{e \in \mathbb{E}}{\bigodot} 
                w(e)^{m_e(h)} \]
\end{definition}
Finally, the weight of an object is defined as the semiring sum of the weights of all morphisms from the object to the underlying type graph $T$ of the weighted type graph $\mathcal{T}$.
\begin{definition}[Object weight]
    \label{def:weight_of_an_object_relative_to_a_type_graph}
       Let $\mathcal{T}=(T,\mathbb{E},\mathcal{S},w)$ be a finitary weighted type graph. The \textbf{weight of an object \( G \)} is defined as the semiring sum of $w_\mathcal{T}(h)$, for all \( h \in \operatorname{Hom}(G,T) \):
        \[ w_\mathcal{T}(G) \overset{\operatorname{def}}{=} \underset{h \in \operatorname{Hom}(G,T)}{\bigoplus}  w_\mathcal{T}(h) \]
\end{definition}

\begin{example}
    Consider the two morphisms shown in~\autoref{fig:example:two_weighted_type_graph_morphisms}
    \begin{figure}[H]
        \centering
        \resizebox{0.49\textwidth}{!}{
        \begin{tikzpicture}
          \graphbox{\( L \)}{-50mm}{0mm}{40mm}{39mm}{2mm}{-6mm}{
            \coordinate (o) at (0mm,-10mm); 
            \node[draw,circle] (l1) at ($(o)+(-10mm,0mm)$) {1};
            \node[draw,circle] (l2) at ($(l1)+(2,0)$) {2};
            \node[draw,circle] (l3) at ($(l1) + (1,0)$) {3};
            \draw[] (l1) -- (l3) node[midway,above] {a};
            \draw[] (l3) -- (l2) node[midway,above] {a};
        } 
            \graphbox{$T$}{0mm}{0mm}{40mm}{39mm}{-10mm}{-17mm}{
                \node[draw,circle] (1) at (0,0) {$1\ 2$};
                \node[draw,circle] (2) at (2,0) {3};
                \draw[->] (1) edge[loop above] node[midway, above] {$a^{1}$} (1) ;
                \draw[->] (1) edge[loop below] node[midway, below] {$b^{1}$} (1) ;
                \draw[->] (1) edge[bend left] node[midway, above] {$a^{1}$}  (2)  ;
                \draw[->] (2) edge[bend left] node[midway, below] {$a^{1}$} (1)   ;
            }
            \node () at (-5mm,-15mm) {$\overset{h_{11}^1}{\to}$};
        \end{tikzpicture}
        }
        \resizebox{0.49\textwidth}{!}{
            \begin{tikzpicture}
              \graphbox{\(L\)}{-50mm}{0mm}{40mm}{39mm}{2mm}{-6mm}{
                \coordinate (o) at (0mm,-10mm); 
                \node[draw,circle] (l1) at ($(o)+(-10mm,0mm)$) {1};
                \node[draw,circle] (l2) at ($(l1)+(2,0)$) {2};
                \node[draw,circle] (l3) at ($(l1) + (1,0)$) {3};
                \draw[] (l1) -- (l3) node[midway,above] {a};
                \draw[] (l3) -- (l2) node[midway,above] {a};
            } 
                \graphbox{$T$}{0mm}{0mm}{40mm}{39mm}{-10mm}{-19mm}{
                    \node[draw,circle] (1) at (0,0) {$1\ 2\ 3$};
                    \node[draw,circle] (2) at (2,0) {};
                    \draw[->] (1) edge[loop above] node[midway, above] {$a^{1}$} (1) ;
                    \draw[->] (1) edge[loop below] node[midway, below] {$b^{1}$} (1) ;
                    \draw[->] (1) edge[bend left] node[midway, above] {$a^{1}$}  (2)  ;
                    \draw[->] (2) edge[bend left] node[midway, below] {$a^{1}$} (1)   ;(1)   ;
                }
                \node () at (-5mm,-15mm) {$\overset{h_{11}^2}{\to}$};
            \end{tikzpicture}
            }
            \caption{}
            \label{fig:example:two_weighted_type_graph_morphisms}
      \end{figure}
       and the weighted type graph shown in~\autoref{fig:weighted_type_graph_instance_sfsdsfs}.
    \begin{figure}[H]
        \centering
        \begin{tikzpicture}
            \graphbox{}{0mm}{0mm}{32mm}{28mm}{-10mm}{-14mm}{
                \node[draw,circle] (1) at (0,0) {1};
                \node[draw,circle] (2) at (2,0) {2};
                \draw[->] (1) edge[loop above] node[midway, above] {$a^{1}$} (1) ;
                \draw[->] (1) edge[loop below] node[midway, below] {$b^{1}$} (1) ;
                \draw[->] (1) edge[bend left] node[midway, above] {$a^{1}$}  (2)  ;
                \draw[->] (2) edge[bend left] node[midway, below] {$a^{1}$} (1)   ;
            }
        \end{tikzpicture}
        \caption{}
        \label{fig:weighted_type_graph_instance_sfsdsfs}
    \end{figure}
    We have \begin{itemize}
        \item $ w_\mathcal{T}(h_{11}^1) = w(e_{13a})^{m_{e_{13a}}(h_{11}^1)} \odot w(e_{31a})^{m_{e_{31a}}(h_{11}^1)} =
     1^1 * 1^1 = 1$
        \item $
        w_\mathcal{T}(h_{11}^2) 
        % = 1^1 * 1^1 
        = 1$
    \end{itemize}
\end{example}   

% % \section{Weighted Type Graph} 
% % \label{sec:weighted_type_graph}
% % \begin{definition}[Weighted Type Graph~\cite{endrullis2024generalized_arxiv_v2}]
    \label{wf:def:weighted_type_graph}
    A \textbf{weighted type graph}~\(\mathcal{T} = (T, \mathbb{E}, \mathcal{S}, w)\) consists of:
    \begin{itemize} 
        \item an object \(T\) in $\mathcal{C}$, called \textbf{type graph},
        \item a commutative semiring \(\mathcal{S}=(S, \oplus, \odot, 0, 1)\),
        \item a set \(\mathbb{E}\) of morphisms with codomain $T$ in $\mathcal{C}$, called \textbf{morphism-rulers}
        \item a weight function \(w : \mathbb{E} \to S \setminus \{0\}\) such that for all $e \in \mathbb{E}, w(e) \geq 1$.
    \end{itemize}
    \(\mathcal{T}\) is \textbf{finitary} if for every \( (e :X \to T) \in \mathbb{E}\) and every object \(G\), the sets \(\operatorname{Hom}(X, G)\) and \(\operatorname{Hom}(G, T)\) are finite.
\end{definition}

\begin{example}
    \label{wf:example:weighted_type_graph}
     In \textbf{Graph}, a weighted type graph 
     whose $\operatorname{dom}(\mathbb{E})$ consists of graphs with two vertices and one labeled edge between them
     can be visualized as a graph with weighted labels and weights given as superscripts. For example, the finitary weighted type graph $\mathcal{T} = (T, \mathbb{E}, \mathcal{S}, w)$ with $\mathcal{S}$ as the natural arithmetic semiring $(\mathbb{N}, +_\mathbb{N}, *_\mathbb{N}, 0_\mathbb{N}, 1_\mathbb{N}, <_\mathbb{N},\leq_\mathbb{N})$ defined in~\autoref{def:weight_of_an_object_relative_to_a_type_graph},
     $T$ as the graph illustrated below (without superscripts), $\mathbb{E}=\{e_{11a},e_{12a},e_{21a},e_{11b}\}$ as the set of morphism-rulers where 
     $e_{uvl}$, for all edge from $u$ to $v$ labeled by $l$, has domain 
     \tikz[baseline=-0.5ex]{
        \node[draw,circle] (x) at (0,0) {};
        \node[draw,circle] (y) at (1,0) {};
        \draw[->] (x) -- (y) node[midway, above] {$l$};
    } and image 
    \begin{tikzpicture}[baseline=-2mm]
        \node[draw, circle] (x) at (0,0) {$\mathrm{u}$};
        \node[draw, circle] (y) at (1,0) {$\mathrm{v}$};
        \draw[->]  (x) -- (y) node [midway,above] {$l$};
    \end{tikzpicture} in the graph $T$,
    and $w(e) = 1_\mathbb{N}$ for all $e \in \mathbb{E}$, is visualized in~\autoref{fig:weighted_type_graph_instance}. To improve readability, we omit the subscripts.
    \begin{figure}[!ht]
        \centering
        \begin{tikzpicture}
            \graphbox{}{0mm}{0mm}{32mm}{28mm}{-10mm}{-14mm}{
                \node[draw,circle] (1) at (0,0) {1};
                \node[draw,circle] (2) at (2,0) {2};
                \draw[->] (1) edge[loop above] node[midway, above] {$a^{1}$} (1) ;
                \draw[->] (1) edge[loop below] node[midway, below] {$b^{1}$} (1) ;
                \draw[->] (1) edge[bend left] node[midway, above] {$a^{1}$}  (2)  ;
                \draw[->] (2) edge[bend left] node[midway, below] {$a^{1}$} (1)   ;
            }
        \end{tikzpicture}
        \caption{Weighted type graph instance}
        \label{fig:weighted_type_graph_instance}
    \end{figure}
\end{example}

% % \section{Weighing Morphisms and Objects}   
% % \label{sec:weigh_morphisms_and_objects}
% % In this section, we explain how the type graph method assigns weights to both morphisms targeting the type graph and to objects.
\begin{definition}[Morphism weight \cite{endrullis2024generalized}]
    \label{def:weight}
    Let $\mathcal{T} = (T,\mathbb{E},S, w)$ be a finitary type graph.
    \newline
    \noindent
    \begin{minipage}{0.6\textwidth}
        The \textbf{weight of a morphism $h: G \rightarrow T$ relative to $(e:X \to T) \in \mathbb{E}$} is defined as the weight $w(e)$ raised to the power of the number of $\iota$ making the diagram (shown on the right) commutative.
    \end{minipage}
    \begin{minipage}{0.29\textwidth}
        \begin{center}
                \begin{tikzpicture}
                    \node (a) at (0,0) {X};
                    \node (c) at (4,0) {G};
                    \node (d) at (2,-1) {T};
                    \draw[->] (a) -- (d) node [midway,below] {e};
                    \draw[->] (a) -- (c) node [midway,above] {$\iota$};
                    \draw[->] (c) -- (d) node[midway, below] {h};
                    % \node (d) at (2,-0.5) {=};
                \end{tikzpicture}
        \end{center} 
    \end{minipage}
                \[
                w_e(h) 
                    \overset{\operatorname{def}}{=}
                \underset{\alpha \in \{- \star h = e\}}{\bigodot}w(e) 
                \]
        % which can be visualized as $w(e)$ raised to the power of the number of possible morphisms $g$ which make the following commutative diagram hold

        \noindent
        The \textbf{weight of a morphism $h: G \rightarrow T$ relative to \(\mathcal{T}\)} is defined as the semiring product of $w(e)$ for $e \in \mathbb{E}$:
        \[  w_\mathcal{T}(h) \overset{\operatorname{def}}{=} \underset{e \in \mathbb{E}}{\bigodot} 
                w_e(h) \]

        \noindent
       The \textbf{weight of an object \( G \in \mathcal{C}_0 \) relative to \( \mathcal{T}\)} is defined as the semiring sum of $w_\mathcal{T}(h)$ for $h \in \operatorname{Hom}(G,T)$:
        \[w_\mathcal{T}(G) \overset{\operatorname{def}}{=} \underset{h \in \operatorname{Hom}(G,T)}{\bigoplus}  w_\mathcal{T}(h) \]
\end{definition}


% In certain scenarios, we want to exclude specific morphisms when calculating morphism weight relative to a give T-valued element $e:X \to T$. For this reason, for all set \( \Gamma \subseteq \operatorname{Hom}(A, G) \), we define 



% The weight of the morphism \( \phi : G \to T\) excluding morphisms in \( \Gamma' \) is defined as the number of morphisms from \( X \) to \( G \) that do not factor through any \( \alpha \in \Gamma \).
\todo{This starts to be quite hard to follow without intuition or example}
\begin{definition}[Weight excluding specific morphisms \cite{endrullis2024generalized}]
    \label{def:weight_excluding}
    Let \( A \in \mathcal{C}_0 \) and $\Gamma \subseteq \operatorname{Hom}(A,G)$.
    \newline  
    \noindent
    \begin{minipage}{0.6\textwidth}
        We define $\Gamma'$ as the set consisting of all morphisms \( \iota : X \to G \) admitting morphisms \( \zeta \colon X \to A \) and \( \alpha \in \Gamma \) such that the diagram illustrated on the right is commutative. Formally, 
    \end{minipage}
    \begin{minipage}{0.4\textwidth}
        \hfill 
        % \begin{center}
            \begin{tikzpicture}[scale=0.8]
                \node (X) at (0,0) {\(X \)};
                \node (A) at (2,1) {\( A \)};
                \node (G) at (4,0) {\( G \)}; 
                \draw[->] (X) -- (A) node[midway, above] {\( \zeta \)};
                \draw[ ->] (X) -- (G) node[midway, below] {\( \iota \)};
                \draw[->] (A) -- (G) node[midway, above] {\( \alpha \)};
                % \node at (2,0.5) {\( = \)};
                % \node (T) at (2,-1) {\( T \)};
                % \draw[ ->] (X) -- (T) node[midway, below] {e};
                % \draw[<-] (T) -- (G) node[midway, above] {};
            \end{tikzpicture}
        % \end{center} 
    \end{minipage}

    \[
    \Gamma' \overset{\operatorname{def}}{=} \left\{ \iota \in \operatorname{Hom}(X, G)~\middle|~\exists \alpha \in \Gamma,~\exists \zeta:X \to A,~\zeta \star \alpha = \iota \right\}.
    \]

    \noindent
    The \textbf{weight of a morphism \(h : G \to T\) excluding morphisms in \( \Gamma' \) relative to $(e:X \to T) \in \mathbb{E}$} is defined as $w(e)$ raised to the power of the number of morphisms \( (\iota : X \to G) \notin \Gamma' \).
        \[
        w_e(h - \Gamma) \overset{\operatorname{def}}{=} \underset{
            \substack{\alpha \in \set{- \star h = e} \\
                        \alpha \notin \Gamma'}}{\bigodot} w(e)\] 
        The \textbf{weight of a morphism $h: G \to T$ excluding morphisms in \( \Gamma' \) relative to \(\mathcal{T}\)} is defined as the semiring product of $w_e(h-\Gamma)$ for $e \in \mathbb{E}$:
        \[ 
            w_\mathcal{T}(h-\Gamma) \overset{\operatorname{def}}{=} \underset{e \in \mathbb{E}}{\bigodot} 
        w_e(h-\Gamma)
                \]
\end{definition} 
% \begin{definition}[Weight excluding morphisms \cite{endrullis2024generalized}]
%     \label{def:weight_excluding}
%     Let \(\Gamma \subseteq \operatorname{Hom}(A, G)\). Define:
%     \[
%     \Gamma' \overset{\text{def}}{=} \left\{ \iota \in \operatorname{Hom}(X, G) \,\middle|\, \exists \alpha \in \Gamma, \exists \zeta : X \to A, \, \zeta \star \alpha = \iota \right\}.
%     \]
%     The \textbf{weight of \(h : G \to T\) excluding \(\Gamma'\)} is:
%     \[
%     w_e(h - \Gamma) \overset{\text{def}}{=} \bigodot_{\substack{\iota \in \{- \star h = e\} \\ \iota \notin \Gamma'}} w(e), \quad
%     w_\mathcal{T}(h - \Gamma) \overset{\text{def}}{=} \bigodot_{e \in \mathbb{E}} w_e(h - \Gamma).
%     \]
% \end{definition}
    If \( \Gamma \) is a singleton \( \{ \alpha \} \), we denote \( w_{\mathcal{T}_\Sigma^X}(h - \alpha) \) instead of \( w_{\mathcal{T}_\Sigma^X}(h - \{ \alpha \}) \). 
  
% \section{Estimating weights of pushout objects} 
% \label{sec:type_graph:weighing_pushout} 
% % Let \( \mathcal{R} = \mathcal{A} \cup \mathcal{B} \) be a set of DPO rewriting rules and let $\mathcal{T}=(T,\mathbb{E},\mathcal{S},w)$ be a weighted type graph over a well-founded commutative semiring such that for every graph $G$ subject to rewriting, we have $\operatorname{Hom}(G,T)\neq \emptyset$. By~\autoref{rem:wf:weight_of_object_geq_1}, for all objects \( G \) subject to rewriting, \(w_\mathcal{T}(G) \succeq_S 1_\mathcal{S} \).
% The rewriting relation \( \Rightarrow_{\mathcal{A},\mathfrak{F}} \) is terminating relative to $\Rightarrow_{\mathcal{B},\mathfrak{F}}$ if (i) for all \(G \Rightarrow_{\mathcal{A},\mathfrak{F}} H\), \( w_\mathcal{T}(G) \succ_S w_\mathcal{T}(H)\), and (ii) for all \(G \Rightarrow_{\mathcal{B},\mathfrak{F}} H\), \( w_\mathcal{T}(G) \succeq_S w_\mathcal{T}(H) \). However, directly verifying all rewriting steps is infeasible due to the potentially infinite number of rewriting steps.

We first introduce the notions of \emph{morphism measurement excluding morphisms in a set}(Definition~\ref{def:weight_excluding_pre}) and \emph{weight of a morphism relative to a morphism-ruler set excluding some specific morphisms}(Definition~\ref{def:weight_excluding}),
to reduce notational overhead (following Endrullis and Overbeek~\cite{endrullis2024generalized_icgt}).
We then recall the notions of traceability of objects (Definition~\ref{def:traceability}) and relative monicity of morphisms (Definition~\ref{def:relative_monicity}),
which are used to define the notion of weighable pushout squares (Definition~\ref{def:weighable}). With the notion of a weighable pushout square, we recall a result from Endrullis and Overbeek~\cite{endrullis2024generalized_icgt} which shows that the weight of the pushout object of a weighable pushout square can be bounded by the weights of the other objects in the pushout square.

For any set \( \Gamma \subseteq \operatorname{Hom}(A, G) \),
    we define $\Gamma'$ as the set consisting of all morphisms \( \iota : X \to G \) admitting morphisms \( \zeta \colon X \to A \) and \( \alpha \in \Gamma \) such that $\zeta \star \alpha = \iota$ holds, i.e., such that
     the diagram $XAG$ shown below
    %  in Figure~\ref{fig:wf:measurement_of_a_morphism_relative_to_a_morphism_ruler_dfsdfsdsa} 
     is commutative.  Formally,  
    \(
    \Gamma' \overset{\operatorname{def}}{=} \left\{ \iota \in \operatorname{Hom}(X, G)~\middle|~\exists \alpha \in \Gamma,~\exists \zeta:X \to A,~\zeta \star \alpha = \iota \right\}. 
    \)
    % \begin{figure}[H]
    % \centering
    \begin{center}
        \begin{tikzpicture}[scale=0.8]
            \node (X) at (0,0) {\(X \)};
            \node (A) at (2,1) {\( A \)};
            \node (G) at (4,0) {\( G \)}; 
            \draw[->] (X) -- (A) node[midway, above] {\( \zeta \)};
            \draw[ ->] (X) -- (G) node[midway, below] {\( \iota \)};
            \draw[->] (A) -- (G) node[midway, above] {\( \alpha \)};
            % \node at (2,0.5) {\( = \)};
            \node (T) at (2,-1) {\( T \)};
            \draw[ ->] (X) -- (T) node[midway, below ] {$e$};
            \draw[<-] (T) -- (G) node[midway, below] {$h$};
        \end{tikzpicture}
%     \caption{}
%     \label{fig:wf:measurement_of_a_morphism_relative_to_a_morphism_ruler_dfsdfsdsa}
% \end{figure} 
    \end{center}
    

\begin{definition} 
    \label{def:weight_excluding_pre}
    Let \( \Gamma \subseteq \operatorname{Hom}(A, G) \).
    The \textbf{measurement of a morphism \( h:G \to T \) relative to a morphism-ruler \( e: X \to T \), excluding morphisms in \( \Gamma' \)}, denoted by $m_e(h-\Gamma)$, is defined as:
    \[
        m_e(h-\Gamma) \overset{\operatorname{def}}{=} 
            \card{\set{- \star h = e} \setminus \Gamma'}
    \]
\end{definition}
\begin{definition}
    \label{def:weight_excluding}
    weight of a morphism $h: G \to T$ relative to a set $\mathbb{E}$ of morphism-rulers excluding morphisms
    Let $\mathcal{T}=(T,\mathbb{E},\mathcal{S},w)$ be a weighted type graph. The \textbf{weight of a morphism $h: G \to T$ relative to a set $\mathbb{E}$ of morphism-rulers excluding morphisms in \( \Gamma' \)} is defined as the semiring product of $w(e)^{w_e(h-\Gamma)}$ for $e \in \mathbb{E}$:
    \[ 
        w_\mathcal{T}(h-\Gamma) \overset{\operatorname{def}}{=} \underset{e \in \mathbb{E}}{\bigodot} 
    w(e)^{m_e(h-\Gamma)}
            \]
\end{definition}
% Consider the pushout square shown below.
% %  in Figure~\ref{fig:preliminaries:pushout_square_traceable_sdldfsfsfsdfkfjsladkj}. Informally, 
%  Any subgraph of the pushout object \(D\) is either the image of \(B\) or the image of \(C\), or is obtained by gluing together parts of the images of \(B\) and \(C\). This intuition is made precise in Definition~\ref{def:traceability}. \todo{todo to do: ????}
% \begin{figure}[H]
%     \centering
%      \resizebox{0.4\textwidth}{!}{
%     \begin{tikzpicture}
%       \node (A) at (0,0) {$A$};
%       \node (B) at (2,0) {$B$}; 
%       \node  (C) at (0,-2) {$C$}; 
%       \node  (D) at (2,-2) {$D$}; 
%       \node  (X) at (4,-2) {$X$};
%       \begin{scope}[nodes=rectangle]          
%       \draw [->] (A) to node [above,label,pos=0.5] {$\alpha$} (B);
%       \draw [->] (A) to node [left,label,pos=0.5] {$\beta$} (C);
%       \draw [->] (B) to node [right,label,pos=0.45] {$\beta'$} (D); 
%       \draw [->] (C) to node [below,label,pos=0.45] {$\alpha'$} (D);
%     %   \draw [->] (X) to node [below,label,pos=0.4] {$h$} (D);
%       \end{scope}
%       \node at ($(A)!.5!(D)$) {$\delta$};
%     \end{tikzpicture}
%     }
%     \caption{}
%     \label{fig:preliminaries:pushout_square_traceable_sdldfsfsfsdfkfjsladkj}
% \end{figure}

\begin{definition}[\cite{endrullis2024generalized_arxiv_v2}]
    \label{def:traceability}
Let $\Delta$ be a class of pushout squares. 
An object $X$ is said to be \textbf{traceable along $\Delta$} if for every diagram $\delta$ in $\Delta$, as shown below:
% in Figure~\ref{fig:preliminaries:pushout_square_traceable_sdlkfjsladkj}, and $h:X \to D$,      
% \begin{figure}[H]
%     \centering
\begin{center}
     \resizebox{0.4\textwidth}{!}{
    \begin{tikzpicture}
      \node (A) at (0,0) {$A$};
      \node (B) at (2,0) {$B$}; 
      \node  (C) at (0,-2) {$C$}; 
      \node  (D) at (2,-2) {$D$}; 
      \node  (X) at (4,-2) {$X$};
      \begin{scope}[nodes=rectangle]          
      \draw [->] (A) to node [above,label,pos=0.5] {$\alpha$} (B);
      \draw [->] (A) to node [left,label,pos=0.5] {$\beta$} (C);
      \draw [->] (B) to node [right,label,pos=0.45] {$\beta'$} (D); 
      \draw [->] (C) to node [below,label,pos=0.45] {$\alpha'$} (D);
      \draw [->] (X) to node [below,label,pos=0.4] {$h$} (D);
      \end{scope}
      \node at ($(A)!.5!(D)$) {$\delta$};
    \end{tikzpicture}
    }
\end{center}
%     \caption{}
%     \label{fig:preliminaries:pushout_square_traceable_sdlkfjsladkj}
% \end{figure} 
 the following hold
    \begin{enumerate}[label=(\alph*)]
        \item\label{traceable:a} there exists a morphism $f : X \to B$ such that $h = f \star \beta'$, or
        \item\label{traceable:b} there exists a morphism $g : X \to C$ such that $h = g \star \alpha'$.
    \end{enumerate}
    If additionally, 
    whenever \ref{traceable:a} and \ref {traceable:b} hold,
    \begin{enumerate}[label=(\alph*),resume]
        \item 
        there exists a morphism $k : X \to A$ such that $h = k \star \alpha \star \beta' $,
    \end{enumerate}
    then we say that $X$ is \emph{strongly traceable} along $\Delta$.
\end{definition}
Intuitively, in the category \textbf{Graph}, a graph $X$ is $\beta$-strongly traceable along a pushout square as illustrated above, if whenever $X$ occurs in $D$, it occurs either in $B$ or in $C$, and if it occurs in both, then it occurs in $A$ as well.

\begin{remark}[\cite{endrullis2024generalized_arxiv_v2}]
    \label{remark:traceability_graph}
    In \textbf{Graph}, the objects \tikz[baseline=-0.5ex]{
        \node[draw,circle] (x) at (0,0) {};
    } and
    %  \tikz[baseline=-0.5ex]{
    %     \node (x) at (0,0) {$\bullet$};
    %     \node (y) at (1,0) {$\bullet$};
    %     \draw[->] (x) -- (y) node[midway, above] {$x$};
    % } 
    \raisebox{2pt}{
            \scalebox{0.7}{\tikz[baseline=-0.5ex]{
            \node[draw,circle] (x) at (0,0) {};
            \node[draw,circle] (y) at (1,0) {};
            \draw[->] (x)--(y) node[midway, above] {$x$};
        }}}
    (for edge labels $x$) are the only (non-initial) objects that are (strongly) traceable along all pushout squares. 
    Other objects, such as loops~\tikz[baseline=-0.5ex]{
        \node[draw,circle] (x) at (0,0) {};
        \draw[->] (x) edge[loop right]  node[midway, right] {$x$}  (x)
    }, are strongly traceable if all morphisms in the square are monomorphisms.
\end{remark}

\begin{example}
    Consider the following DPO diagram.
    %  shown in Figure~\ref{fig:nwf:two_pushout_squares}.
    \begin{center}
    % \begin{figure}[H]
    %     \centering 
      \resizebox{0.7\textwidth}{!}{
      \begin{tikzpicture}
          \graphbox{\( L \)}{0mm}{-3mm}{34mm}{12mm}{2mm}{2mm}{
              \coordinate (o) at (0mm,-8mm); 
              \node[draw,circle] (l1) at ($(o)+(-10mm,0mm)$) {1};
              \node[draw,circle] (l2) at ($(l1)+(2,0)$) {2};
              \node[draw,circle] (l3) at ($(l1) + (1,0)$) {3};
              \draw[] (l1) -- (l3) node[midway,above] {$a$};
              \draw[] (l3) -- (l2) node[midway,above] {$a$};
          } 
          \graphbox{\( K \)}{40mm}{-3mm}{34mm}{12mm}{2mm}{2mm}{
              \coordinate (o) at (0mm,-8mm); 
              \node[draw,circle] (l1) at ($(o)+(-10mm,0mm)$) {1};
              \node[draw,circle] (l2) at ($(l1)+(2,0)$) {2};
          }  
          \graphbox{\( R \)}{80mm}{-3mm}{45mm}{12mm}{2mm}{2mm}{
              \coordinate (o) at (-5mm,-8mm); 
              \node[draw,circle] (l1) at ($(o)+(-10mm,0mm)$) {1};
              \node[draw,circle] (l2) at ($(l1)+(3,0)$) {2};
              \node[draw,circle] (l3) at ($(l1) + (1,0)$) {4};
              \node[draw,circle] (l4) at ($(l1) + (2,0)$) {5};
              \draw[ ] (l1) -- (l3) node[midway,above] {$a$};
              \draw[ ] (l3) -- (l4) node[midway,above] {$a$};
              \draw[ ] (l4) -- (l2) node[midway,above] {$a$};
          }    
          \graphbox{\( G \)}{0mm}{-22mm}{34mm}{22mm}{2mm}{-3mm}{
              \coordinate (o) at (0mm,-3mm); 
              \node[draw,circle] (l1) at ($(o)+(-10mm,0mm)$) {1};
              \node[draw,circle] (l2) at ($(l1)+(2,0)$) {2};
              \node[draw,circle] (l3) at ($(l1) + (1,0)$) {3};
              \node[draw,circle] (l4) at ($(l2) + (0,-1)$) {6};
              \draw[] (l1) -- (l3) node[midway,above] {$a$};
              \draw[] (l3) -- (l2) node[midway,above] {$a$};
              \draw[ ] (l2) -- (l4) node[midway,right] {$a$};
              \node[draw,circle] (l6) at ($(l1) + (0,-1)$) {7};
              \draw[] (l1) -- (l6) node[midway,left] {$a$};
          }    
          \graphbox{\( C  \)}{40mm}{-22mm}{34mm}{22mm}{2mm}{-3mm}{
              \coordinate (o) at (0mm,-3mm); 
              \node[draw,circle] (l1) at ($(o)+(-10mm,0mm)$) {1};
              \node[draw,circle] (l2) at ($(l1)+(2,0)$) {2};
              \node[draw,circle] (l4) at ($(l2) + (0,-1)$) {6};
              \draw[ ] (l2) -- (l4) node[midway,right] {$a$};
              \node[ draw,circle] (l6) at ($(l1) + (0,-1)$) {7};
              \draw[ ] (l1) -- (l6) node[midway,left] {$a$};
          }    
          \graphbox{\( H \)}{80mm}{-22mm}{45mm}{22mm}{2mm}{-3mm}{
              \coordinate (o) at (-5mm,-3mm); 
              \node[draw,circle] (l1) at ($(o)+(-10mm,0mm)$) {1};
              \node[draw,circle] (l2) at ($(l1)+(3,0)$) {2};
              \node[draw,circle] (l3) at ($(l1) + (1,0)$) {4};
              \node[draw,circle] (l4) at ($(l1) + (2,0)$) {5};
              \node[ draw,circle] (l5) at ($(l2) + (0,-1)$) {6};
              \node[ draw,circle] (l6) at ($(l1) + (0,-1)$) {7};
              \draw[ ] (l1) -- (l6) node[midway,left] {$a$};
              \draw[] (l1) -- (l3) node[midway,above] {$a$};
              \draw[] (l3) -- (l4) node[midway,above] {$a$};
              \draw[ ] (l4) -- (l2) node[midway,above] {$a$};
              \draw[ ] (l2) -- (l5) node[midway,right] {$a$};
          }    
          \node () at (37mm,-8mm) {\( \leftarrowtail \)}; % K -> L
          \node () at (77mm,-8mm) {\( \rightarrowtail \)}; % K -> R
          \node () at (15mm,-18mm) {\( m\ \downarrowtail \)};
          \node () at (37mm,-33mm) {\( \leftarrowtail \)};
          \node () at (58mm,-18mm) {\( u\downarrowtail \)};
          \node () at (102mm,-18mm) {\( \downarrowtail \)};
          \node () at (77mm,-33mm) {\( \rightarrowtail \)}; % C -> H
      \end{tikzpicture}
      }
    \end{center}

   The graph 
   \tikz[baseline=-0.5ex]{
        \node[draw,circle] (x) at (0,0) {};
        \node[draw,circle] (y) at (1,0) {};
        \node[draw,circle] (z) at (2,0) {};
        \draw[->] (x) -- (y) node[midway, above] {$a$};
        \draw[<-] (z) -- (y) node[midway, above] {$a$};
    } is not traceable in both pushout squares. The graph \tikz[baseline=-0.5ex]{
        \node[draw,circle] (x) at (0,0) {};
        \node[draw,circle] (y) at (1,0) {};
        \draw[->] (x) -- (y) node[midway, above] {$a$};
    } is traceable (and strongly traceable) along both pushout squares.
\end{example}

% \begin{example} 
%     Consider the rewriting step shown in Figure~\ref{fig:nwf:rewriting_step_traceability}.
%     The graph \begin{tikzpicture}
%         \node[draw, circle] (x) at (0,0) {};
%         \node[draw, circle] (y) at (1,0) {};
%         \draw[->]  (x) -- (y) node [midway,above] {$c$};
%     \end{tikzpicture} 
%     is traceable along both pushout squares because for every morphism $h$ from this graph to $G$ (resp. to $H$), $h$ factors through $m$ or $l$ (resp. through $u$ or $r$).

%      It is $u$-strongly traceable along the left pushout square, because the only morphism $h:X \to G$ factorizable through $m$ and $l$ has as image the graph \begin{tikzpicture}
%         \node[draw, circle] (x) at (0,0) {z};
%         \node[draw, circle] (y) at (1,0) {y};
%         \draw[->]  (x) -- (y) node [midway,above] {$c$};
%      \end{tikzpicture} and there is a unique morphism $k:X \to A$ such that $h = k \star l \star m$.

%     It is not $u$-strongly traceable along the right pushout square
%     The graph \begin{tikzpicture}
%         \node[draw, circle] (x) at (0,0) {};
%         \node[draw, circle] (y) at (1,0) {};
%         \node[draw, circle] (z) at (2,0) {};
%         \draw[->]  (x) -- (y) node [midway,above] {$c$};
%         \draw[->]  (y) -- (z) node [midway,above] {$c$};
%     \end{tikzpicture} is not traceable along both pushout squares. \todo{todo: why}
% %     because    
% %     \begin{tikzpicture}
% %        \node[draw, circle] (x) at (0,0) {z};
% %        \node[draw, circle] (y) at (1,0) {x};
% %        \node[draw, circle] (z) at (2,0) {y};
% %        \draw[->]  (x) -- (y) node [midway,above] {$c$};
% %        \draw[->]  (y) -- (z) node [midway,above] {$c$};
% %    \end{tikzpicture} is in the pushout object $G$ but in neither $L$ nor $C$.
%    \begin{figure}[H]
%     \centering
%     \resizebox{0.7\textwidth}{!}{
%     \begin{tikzpicture}
%         \graphbox{$L$}{0mm}{0mm}{35mm}{35mm}{2mm}{-7mm}{
%         \node[draw,circle] (1) at (0,0) {x};
%         \node[draw,circle] (2) at (1,-2) {y};
%         \node[draw,circle] (3) at (-1,-2) {z};
%         \draw[->] (1) edge node[midway,right]  {$c$}  node[midway,left]{} (2);
%         \draw[->] (2) edge node[midway,above]  {$c$}  node[midway,below]{}  (3);
%         }
%         \graphbox{$K$}{45mm}{0mm}{35mm}{35mm}{2mm}{-7mm}{
%         % I
%             % \def\x{5};
%             % \def\y{0};
%             \node[draw,circle] (1) at (0,0) {x};
%             \node[draw,circle] (2) at (1,-2) {y};
%             \node[draw,circle] (3) at (-1,-2) {z};
%             \draw[->] (2) edge node[midway,above]  {$c$}  node[midway,below]{}  (3);
%         }
%         \node () at (40mm,-15mm)  {$\overset{l}{\leftarrowtail}$};

%         % \node () at (2.5,-2.5) {$PO$};
%         \graphbox{$R$}{90mm}{0mm}{35mm}{35mm}{2mm}{-7mm}{
%         % R
%             % \def\x{10};
%             % \def\y{-1.1};
%             \node[draw,circle] (1) at (0,-1) {x y z};
%             \draw[->] (1) edge[loop below]  node[midway, below] {$c$}  (1);
%         }
%         \node () at (85mm,-15mm)  {$\overset{r}{\rightarrow}$};

%         \graphbox{$G$}{0mm}{-45mm}{35mm}{35mm}{2mm}{-7mm}{
%         % \node () at (7.5,-2.5) {$PO$};
%         % G
%             % \def\x{0};
%             % \def\y{-5};
%             \node[draw,circle] (1) at (0,0) {x};
%             \node[draw,circle] (2) at (1,-2) {y};
%             \node[draw,circle] (3) at (-1,-2) {z};
%             \draw[->] (1) edge node[midway,right]  {$c$}  node[midway,left]{} (2);
%             \draw[->] (2) edge node[midway,above]  {$c$}  node[midway,below]{}  (3);
%             \draw[->] (3) edge node[midway,above]  {$c$}  node[midway,below]{}  (1);
%         }
%         % C
%         \graphbox{$C$}{45mm}{-45mm}{35mm}{35mm}{2mm}{-7mm}
%         {    
%             \node[draw,circle] (1) at (0,0) {x};
%             \node[draw,circle] (2) at (1,-2) {y};
%             \node[draw,circle] (3) at (-1,-2) {z};
%             % \draw[->] (1) edge node[midway,right]  {$c$}  node[midway,left]{} (2);
%             \draw[->] (3) edge node[midway,above]  {$c$}  node[midway,below]{} (1);
%             \draw[->] (2) edge node[midway,above]  {$c$}  node[midway,below]{} (3);
%             }
%         \node () at (40mm,-55mm)  {$\overset{l'}{\leftarrowtail}$};
%         % H
%         \graphbox{$H$}{90mm}{-45mm}{35mm}{35mm}{2mm}{-7mm}{
%         % R
%             % \def\x{10};
%             % \def\y{-1.1};
%             \node[draw,circle] (1) at (0,-1) {x y z};
%             \draw[->] (1) edge[loop below]  node[midway, below] {} node[midway, below] {$c$} (1);
%             \draw[->] (1) edge[loop above]  node[midway, below] {} node[midway, above] {$c$} (1);
%         }

%         \node () at (85mm,-55mm)  {$\overset{r'}{\rightarrow}$};
%         \node () at (17mm,-40mm) {$m\downarrowtail$};
%         \node () at (62mm,-40mm) {$u\downarrowtail$};
%         \node () at (107mm,-40mm) {$\downarrow m'$};
%     \end{tikzpicture}
%     }
%     \caption{}
%     \label{fig:nwf:rewriting_step_traceability}
%    \end{figure}
% \end{example} 
The following concepts are restricted versions of the concept of monicity.
\begin{definition}[\cite{endrullis2024generalized_arxiv_v2}]
    \label{def:relative_monicity}
    Let $A,B,C$ and $X$ be objects, $\Gamma$ a set of objects, $S$ a set of morphisms to $A$ and $u:C\to A$ a morphism. A morphism $f : A \to B$ is said to be
    \begin{itemize} 
        \item 
            \textbf{monic for $S$} 
            if $g \star f = h \star f$ implies $g = h$ for all $g, h \in S$;
        \item 
            \textbf{$X$-monic} if $f$ is monic for $\operatorname{Hom}(X, A)$.
        \item \textbf{$X$-monic outside of $u$}, if $f$ is monic for \( \operatorname{Hom}(X,A) \setminus \left ( \operatorname{Hom}(X,C) \star u \right ) \).
        \item  \textbf{$\Gamma$-monic} if $f$ is $X$-monic for every $X \in \Gamma$.
    \end{itemize}
\end{definition} 
\begin{example}[Edge-Monicity \cite{endrullis2024generalized_arxiv_v2}]
 
 Let \(f\colon G\to H\) be a morphism in \(\mathbf{Graph}\) that does not identify distinct edges, but may identify distinct nodes. Then \(f\) need not be monic. Indeed, if \(u\neq v\) are nodes of \(G\) with \(f(u)=f(v)\), let \(K\) be the graph consisting of a single node and no edges, and let \(g,h\colon K\to G\) send that node to \(u\) and \(v\), respectively. Then \(g\neq h\) while \(f\circ g=f\circ h\), so \(f\) is not monic.

Nevertheless \(f\) is \(\Gamma\)-monic, where
\[
\Gamma = \left\{ \vcenter{\hbox{
\begin{tikzpicture}[baseline=-0.5ex]
\node[draw,circle] (A) at (0,0) {};
\node[draw,circle] (B) at (1,0) {};
\draw[->] (A) -- (B) node[midway, above] {$x$};
\end{tikzpicture}
}} \mid x \text{ is an edge label} \right\}.
\]
To see this, let \(X\in\Gamma\) and let \(g,h\colon X\to G\) satisfy \(g \star f = h \star f\). Let \(e\) denote the unique edge of \(X\). From \(f(g(e))=f(h(e))\) and the hypothesis that \(f\) is injective on edges we obtain \(g(e)=h(e)\). Since we are working with directed graphs, the images of the two nodes of \(X\) are determined by the image of \(e\), so \(g\) and \(h\) agree on both nodes. Hence \(g=h\), and \(f\) is \(\Gamma\)-monic.

\end{example}
\begin{definition}[\text{\cite[\textdef~4.9]{endrullis2024generalized_arxiv_v2}}]
    \label{def:weighable}
    \ \newline
    \noindent
    \begin{minipage}{0.7\textwidth}
        Let  $\mathcal{T} = (T,\mathbb{E}, S, w)$ be a weighted type graph.
        Consider the pushout square $\delta$ shown on the right. We say that $\delta$ is said to be
         \begin{enumerate}[label=(\alph*)]
        \item \textbf{weighable} with $\mathcal{T}$ if the following hold:
            \begin{enumerate}[label=(\roman*)]
                \item $dom(\mathbb{E})$ is strongly traceable along $\delta$,
                \item $\beta'$ is $dom(\mathbb{E})$-monic,
                \item $\alpha'$ is $dom(\mathbb{E})$-monic outside of $\beta$.
            \end{enumerate}
        \item \textbf{bounded-above} by $\mathcal{T}$ if $dom(\mathbb{E})$ is traceable along $\delta$.
    \end{enumerate}
    \end{minipage}
    \begin{minipage}{0.3\textwidth}
        \begin{center}
            \begin{tikzpicture}[node distance=12mm]
                \node (A) {$A$};
                \node (B) [right of=A] {$B$};
                \node (C) [below of=A] {$C$};
                \node (D) [right of=C] {$D$};
                \draw [->] (A) to node [above, label] {$\alpha$} (B);
                \draw [->] (A) to node [left, label] {$\beta$} (C);
                \draw [->] (B) to node [right, label] {$\beta'$} (D);
                \draw [->] (C) to node [below, label] {$\alpha'$} (D);
                \node [at=($(A)!.5!(D)$)] {$\delta$};
            \end{tikzpicture}
        \end{center}
    \end{minipage}
   
\end{definition}

    Consider the following DPO diagram
    %  shown in Figure~\ref{fig:nwf:dpo_diagram_sdfslafgjsl} 
     that defines a rewriting step \( G \Rightarrow_{\rho,\mathfrak{F}} H \). 
    \begin{center}
        \resizebox{0.4\textwidth}{!}{
        \begin{tikzpicture}
            % [node distance=11mm]
            \node (I) at (0,0) {$K$};
            \node (L) at (-2,0) {$L$};
            \node (R) at (2,0) {$R$};
            \node (G) at (-2,-2) {$G$};
            \node (C) at (0,-2) {$C$};
            \node (H) at (2,-2) {$H$};
            \draw [->] (I) to  node [midway,below] {$l$} (L);
            \draw [->] (I) to  node [midway,below] {$r$} (R);
            \draw [->] (L) to node [midway,right] {$m$} (G);
            \draw [->] (I) to node [midway,right] {$u$} (C);
            \draw [->] (R) to node [midway,left] {$m'$} (H);
            \draw [->] (C) to node [midway,above] {$l'$} (G);
            \draw [->] (C) to node [midway,above] {$r'$} (H);
            \node [at=($(I)!.5!(G)$)] {\normalfont PO};
            \node [at=($(I)!.5!(H)$)] {\normalfont PO};
        \end{tikzpicture}
    } 
    \end{center}
    If the left pushout square is weighable with $\mathcal{T}$ and the right pushout square is bounded above by $\mathcal{T}$ (concepts defined in Definition~\ref{def:weighable}), then, by \cite[Lemma 4.13]{endrullis2024generalized_arxiv_v2}, we obtain:

\begin{flalign*}
    w_\mathcal{T}(G) 
        & = \bigoplus_{t_K: K \rightarrow T} 
        \left ( \bigoplus_{\substack{t_C: C \rightarrow T\\ t_K = u \star t_C}}
          w_\mathcal{T}(t_C - u) \right ) 
          \odot
        \left (\bigoplus_{\substack{t_L: L \rightarrow T\\ t_K = l \star t_L}}
        w_\mathcal{T}(t_L) \right )
         \\
    w_\mathcal{T}(H) 
        &  \preceq \bigoplus_{t_K: K \rightarrow T} 
        \left ( \bigoplus_{\substack{t_C: C \rightarrow T\\ t_K = u \star t_C}}
         w_\mathcal{T}(t_C - u) \right ) 
         \odot 
         \left ( \bigoplus_{\substack{t_R: R \rightarrow T\\ t_K = r \star t_R}}
            w_\mathcal{T}(t_R) \right ) \\
\end{flalign*}
According to these results, in \textsection\ref{wf:sec:termination}, we establish a termination criterion by, for every $t_K: K \rightarrow T$, comparing
$\bigoplus_{\substack{t_L: L \rightarrow T\\ t_K = l \star t_L}}
        w_\mathcal{T}(t_L)$ and 
$\bigoplus_{\substack{t_R: R \rightarrow T\\ t_K = r \star t_R}} 
        w_\mathcal{T}(t_R)$.
Before doing so, we introduce the concept of context closure in the following section. 
  

% \section{A Termination Criterion}
% \label{sec:type_graph:termination}
% \subsection{Context Closure}
\label{sec:context_closure}
% It is crucial to ensure that the weight of any object subject to rewriting is greater than or equal to $1_\mathcal{S}$.
% This issue is addressed by the existence of a context closure. Specifically, it ensures that every graph that can be rewritten using that rule admits some morphism into the weighted type graph, and thus the weight of the morphism, which is the sum of the weights of the morphisms into the type graph, is greater than or equal to \(1_S\) by the following~\autoref{lem:wf:morphism_weight_geq_1_neq_0}.

It is crucial to ensure that the weight of any object subject to rewriting is not $0_S$, because \(0_S\) behaves unpredictably in strongly monotonic measurable semirings. For instance, in the natural tropical semiring \((\mathbb{N} \mathop{\cup} \{\mathop{+\infty}\}, \mathop{\min}, +)\), the element \(0_S\) is the greatest element \(\mathop{+\infty}\), while in the natural arctic semirings \((\mathbb{N} \mathop{\cup} \{\mathop{-\infty}\}, \max, +)\), the element \(0_S\) is the smallest element \(\mathop{-\infty}\).
This issue is addressed by the existence of a context closure defined as follows:

% \begin{lemma}[Lemma 4.5~\cite{endrullis2024generalized_arxiv_v4}]
% \label{lem:wf:morphism_weight_geq_1_neq_0}
% Let $\mathcal{T} \mathop{=} (T, \mathbb{E}, S, \mathbf{w})$ be a finitary weighted type graph. Then
% \begin{itemize}
%     \item for every $\phi : G \mathop{\to} T$: $1_S \mathop{\preceq} \mathbf{w}_\mathcal{T}(\phi) \mathop{\neq} 0_S$;
%     \item for every $\phi : G \mathop{\to} T$ and $\alpha : A \mathop{\to} G$: $1_S \mathop{\preceq} \mathbf{w}_\mathcal{T}(\phi - (\alpha \circ -)) \mathop{\neq} 0_S$;
% \end{itemize}
% \end{lemma}
% \begin{remark} 
%   \label{remark:semiring_0_unpredictable}
%   The requirement \textquote{for all \(e \mathop{\in} \mathbb{E}, w(e) \mathop{\neq} 0_S\)} is necessary because \(0_S\) behaves unpredictably in strongly monotonic measurable semirings. For instance, in the natural and real tropical semirings \((\mathbb{N} \mathop{\cup} \{\mathop{+\infty}\}, \mathop{\min}, +)\), \(0_S\) is the greatest element \(\mathop{+\infty}\), while in the natural and real arctic semirings \((\mathbb{N} \mathop{\cup} \{\mathop{-\infty}\}, \max, +)\), \(0_S\) is the smallest element \(\mathop{-\infty}\).
% \end{remark} 

\begin{definition}[\cite{endrullis2024generalized_arxiv_v2}]
    \label{def:context_closure}  
    Let $\mathcal{T}=(T,\mathbb{E},\mathcal{S},w)$ be a finitary weighted type graph, \(\rho \mathop{=} (L \overset{l}{\leftarrow} K \overset{r}{\rightarrow} R ) \) a DPO rewriting rule and $\mathfrak{F}$ a rewriting framework. 
    A \textbf{context closure} for $\rho$ and $\mathcal{T}$ in $\mathfrak{F}$ is a morphism $c:L \mathop{\rightarrow} T$ such that for every DPO diagram in $\mathfrak{F}(\rho)$ shown below,
    there is $\alpha : G \mathop{\rightarrow} T$ such that $m \mathop{\star} \alpha \mathop{=} c$.
    \begin{center}
        \begin{tikzpicture}
            % [scale=1.5]
          \node (I) at (0,0) {$K$};
          \node (L) at (-2,0) {$L$};
          \node (R) at (2,0) {$R$};
          \node (G) at (-2,-2) {$G$};
          \node (C) at (0,-2) {$C$};
          \node (H) at (2,-2) {$H$};
          \draw [->] (I) to node [label, above] {$l$} (L);
          \draw [->] (I) to node [label,above] {$r$} (R);
          \draw [->] (L) to node [label, right] {} (G);
          \draw [->] (I) to (C);
          \draw [->] (R) to (H);
          \draw [->] (C) to (G);
          \draw [->] (C) to (H);
        \end{tikzpicture}
      \end{center}
\end{definition}
\begin{example}
    \label{wf:example:context_closure}
    Let the set of edge labels be $\Sigma \mathop{=} \{a,b\}$.
    Consider the following DPO rule:
    % in Example~\ref{ex:nwf:grsaa_rule}.
    % \begin{figure}[H]
    %  \centering 
    \begin{center}
    %   \resizebox{\textwidth}{!}{
      \begin{tikzpicture}
          \graphbox{$L$}{0mm}{0mm}{34mm}{15mm}{2mm}{-5mm}{
              \coordinate (o) at (0mm,-3mm); 
              \node[draw,circle] (l1) at ($(o)+(-10mm,0mm)$) {1};
              \node[draw,circle] (l2) at ($(l1)+(2,0)$) {2};
              \node[draw,circle] (l3) at ($(l1)+(1,0)$) {3};
              \draw[->] (l1) -- (l3) node[midway,above] {$a$};
              \draw[->] (l3) -- (l2) node[midway,above] {$a$};
          }     
          \graphbox{$K$}{40mm}{0mm}{24mm}{15mm}{2mm}{-5mm}{
              \coordinate (o) at (5mm,-3mm); 
              \node[draw,circle] (l1) at ($(o)+(-10mm,0mm)$) {1};
              \node[draw,circle] (l2) at ($(l1)+(1,0)$) {2};
              % \node[draw,circle] (l3) at ($(l1)+(1,0)$) {$\ $};
              % \draw[->] (l1) -- (l3) node[midway,above] {$a$};
              % \draw[->] (l3) -- (l2) node[midway,above] {$a$};
          }    
          \graphbox{$R$}{70mm}{0mm}{45mm}{15mm}{2mm}{-5mm}{
              \coordinate (o) at (-5mm,-3mm); 
              \node[draw,circle] (l1) at ($(o)+(-10mm,0mm)$) {1};
              \node[draw,circle] (l2) at ($(l1)+(3,0)$) {2};
              \node[draw,circle] (l3) at ($(l1)+(1,0)$) {4};
              \node[draw,circle] (l4) at ($(l1)+(2,0)$) {5};
              \draw[->] (l1) -- (l3) node[midway,above] {$a$};
              \draw[->] (l3) -- (l4) node[midway,above] {$b$};
              \draw[->] (l4) -- (l2) node[midway,above] {$a$};
          }    
          \node () at (37mm,-8mm) {$\overset{l}{\leftarrowtail}$};
          \node () at (67mm,-8mm) {$\overset{r}{\rightarrowtail}$};
          % \draw[>->] (51mm,2mm) -- (52mm,3mm);
      \end{tikzpicture}
    %   }
    \end{center}
  \noindent and the following weighted type graph:
\begin{center}
        \begin{tikzpicture}
            [scale=1.3]
            \graphbox{}{0mm}{0mm}{32mm}{28mm}{-10mm}{-14mm}{
                \node[draw,circle] (1) at (0,0) {1};
                \node[draw,circle] (2) at (2,0) {2};
                \draw[->] (1) edge[loop above] node[midway, above] {$a^{1}$} (1) ;
                \draw[->] (1) edge[loop below] node[midway, below] {$b^{1}$} (1) ;
                \draw[->] (1) edge[bend left] node[midway, above] {$a^{1}$}  (2)  ;
                \draw[->] (2) edge[bend left] node[midway, below] {$a^{1}$} (1)   ;
            }
        \end{tikzpicture}
    %     \caption{}
    %     \label{fig:nwf:weighted_type_graph_grsaa}
    % \end{figure}
\end{center}
   Let $c$ be the morphism shown below.
%    in Figure~\ref{fig:nwf:context_closure_grsaa}. 
   Since for every match $m : L \mathop{\to} G$, there is a morphism $\alpha : G \mathop{\to} T$ such that $m \mathop{\star} \alpha \mathop{=} c$, the morphism $c$ is a context closure for the DPO rule in the type graph.
%   \begin{figure}[H]
%     \centering
    \begin{center}
    
    % \resizebox{0.7\textwidth}{!}{
    \begin{tikzpicture}
      \graphbox{\( L \)}{-50mm}{0mm}{40mm}{40mm}{2mm}{-8mm}{
        \coordinate (o) at (0mm,-10mm); 
        \node[draw,circle] (l1) at ($(o)+(-10mm,0mm)$) {1};
        \node[draw,circle] (l2) at ($(l1)+(2,0)$) {2};
        \node[draw,circle] (l3) at ($(l1)+(1,0)$) {3};
        \draw[] (l1) -- (l3) node[midway,above] {$a$}; 
        \draw[] (l3) -- (l2) node[midway,above] {$a$};
    } 
        \graphbox{$T$}{0mm}{0mm}{40mm}{40mm}{-10mm}{-17mm}{
            % \node[draw,circle] (1) at (0,0) {$1\ 2\ 3$};
            % \node[draw,circle] (2) at (2,0) {};
            \coordinate (o) at (2mm,-3mm); 
            \node[draw,circle] (1) at ($(o)+(0,0mm)$) {$1\ 2\ 3$};
            \node[draw,circle] (2) at ($(o)+(2,0)$) {};
            \draw[->] (1) edge[loop above] node[midway, above] {$a^{1}$} (1) ;
            \draw[->] (1) edge[loop below] node[midway, below] {$b^{1}$} (1) ;
            \draw[->] (1) edge[bend left] node[midway, above] {$a^{1}$}  (2)  ;
            \draw[->] (2) edge[bend left] node[midway, below] {$a^{1}$} (1)   ;
        }
        \node () at (-5mm,-15mm) {$\overset{c}{\to}$};
    \end{tikzpicture}
    % }
\end{center}  
\end{example} 

\subsection{Decreasing rules}
\label{sec:decreasing_rules}
%  This difference must exceed a fixed positive constant $\delta \mathop{\in} \mathbb{R}_{>0}$.
The following definition of decreasing rules from~\cite{endrullis2024generalized_arxiv_v2} classifies DPO rewriting rules.

\begin{definition}[\cite{endrullis2024generalized_arxiv_v2}]
    \label{wf:def:decreasing_rule}
    Let $\mathcal{T} \mathop{=} (T,\mathbb{E}, (S, \mathop{\oplus}, \mathop{\odot}, 0_S, 1_S, \prec, \mathop{\preceq}),w)$ be a finitary weighted type graph, \(\mathfrak{F}\) a DPO rewriting framework, $\rho \mathop{=} (L \overset{l}{\leftarrow} K \overset{r}{\rightarrow} R)$ a DPO rewriting rule.
 
    \noindent
    The rule $\rho$ is said to be \textbf{weakly decreasing} with respect to $\mathcal{T}$ in $\mathfrak{F}$ if 
            for every $t_K : K \mathop{\to} T$, the following inequality holds:
                $$ 
                  w_\mathcal{T}(\{l \mathop{\star} - \mathop{=} t_K\}) \mathop{\succeq} w_\mathcal{T}(\{r\star - \mathop{=} t_K\}).$$
           
    \noindent
    The rule $\rho$ is said to be \textbf{uniformly decreasing} with respect to $\mathcal{T}$ in $\mathfrak{F}$ if the following conditions holds:
        \begin{itemize}
            \item[]- there is a context closure $c_\rho$ for $\rho$ and $\mathcal{T}$ in $\mathfrak{F}$, 
            \item[]- for every $t_K : K \mathop{\to} T$,
            \begin{itemize}
                \item[] $\bullet$ $\{l \mathop{\star} - \mathop{=} t_K\} \mathop{=} \emptyset \mathop{=} \{r \mathop{\star} - \mathop{=} t_K\}$, or
                \item[] $\bullet$ $w_\mathcal{T}(\{l \mathop{\star} - \mathop{=} t_K\}) 
                        \mathop{\succ}   w_\mathcal{T}(\{r \mathop{\star} - \mathop{=} t_K\}) $.
            \end{itemize}
        \end{itemize}  
         
    \noindent
   If $S$ is moreover strictly monotonic, we say the rule $\rho$ is
            \textbf{closure decreasing} with respect to $\mathcal{T}$ in $\mathfrak{F}$ if the following holds:
            \begin{itemize}
                \item[]- $\rho$ is weakly decreasing,
                \item[]- there is a context closure $c_\rho$ for $\rho$ and $\mathcal{T}$ in $\mathfrak{F}$,
                \item[]- $w_\mathcal{T}(\{l \mathop{\star} - \mathop{=} t_K\})  
                \mathop{\succ}  w_\mathcal{T}(\{r \mathop{\star} - \mathop{=} t_K\})$ for $t_K \mathop{=} l \mathop{\star} c_\rho$.
            \end{itemize}
\end{definition}

\begin{example}
    \label{wf:example:decreasing_rule}
    Consider 
    the weighted type graph
%    in Figure~\ref{fig:nwf:weighted_type_graph_grsaa_2} 
   over the natural arithmetic semiring $(\mathbb{N},+,*,0,1,<,\leq)$:
%   \begin{figure}[H]
%     \centering 
\begin{center}
        \begin{tikzpicture}[scale=1.3]
            \graphbox{}{0mm}{0mm}{32mm}{26mm}{-10mm}{-14mm}{
                \node[draw,circle] (1) at (0,0) {};
                \node[draw,circle] (2) at (2,0) {};
                \draw[->] (1) edge[loop above] node[midway, above] {$a^{1}$} (1) ;
                \draw[->] (1) edge[loop below] node[midway, below] {$b^{1}$} (1) ;
                \draw[->] (1) edge[bend left] node[midway, above] {$a^{1}$}  (2)  ;
                \draw[->] (2) edge[bend left] node[midway, below] {$a^{1}$} (1)   ;
            }
        \end{tikzpicture}
    %     \caption{}
    %     \label{fig:nwf:weighted_type_graph_grsaa_2}
    % \end{figure}
    \end{center}
    and
    the DPO rule:
    % in Example~\ref{ex:nwf:grsaa_rule_2}
    % \begin{figure}[H]
    %   \centering
    \begin{center}
    %   \resizebox{\textwidth}{!}{
      \begin{tikzpicture}
          \graphbox{$L$}{0mm}{0mm}{34mm}{15mm}{2mm}{-5mm}{
              \coordinate (o) at (0mm,-3mm); 
              \node[draw,circle] (l1) at ($(o)+(-10mm,0mm)$) {1};
              \node[draw,circle] (l2) at ($(l1)+(2,0)$) {2};
              \node[draw,circle] (l3) at ($(l1)+(1,0)$) {3};
              \draw[->] (l1) -- (l3) node[midway,above] {$a$};
              \draw[->] (l3) -- (l2) node[midway,above] {$a$};
          }     
          \graphbox{$K$}{40mm}{0mm}{24mm}{15mm}{2mm}{-5mm}{
              \coordinate (o) at (5mm,-3mm); 
              \node[draw,circle] (l1) at ($(o)+(-10mm,0mm)$) {1};
              \node[draw,circle] (l2) at ($(l1)+(1,0)$) {2};
              % \node[draw,circle] (l3) at ($(l1)+(1,0)$) {$\ $};
              % \draw[->] (l1) -- (l3) node[midway,above] {$a$};
              % \draw[->] (l3) -- (l2) node[midway,above] {$a$};
          }    
          \graphbox{$R$}{70mm}{0mm}{45mm}{15mm}{2mm}{-5mm}{
              \coordinate (o) at (-5mm,-3mm); 
              \node[draw,circle] (l1) at ($(o)+(-10mm,0mm)$) {1};
              \node[draw,circle] (l2) at ($(l1)+(3,0)$) {2};
              \node[draw,circle] (l3) at ($(l1)+(1,0)$) {4};
              \node[draw,circle] (l4) at ($(l1)+(2,0)$) {5};
              \draw[->] (l1) -- (l3) node[midway,above] {$a$};
              \draw[->] (l3) -- (l4) node[midway,above] {$b$};
              \draw[->] (l4) -- (l2) node[midway,above] {$a$};
          }    
          \node () at (37mm,-8mm) {$\overset{l}{\leftarrowtail}$};
          \node () at (67mm,-8mm) {$\overset{r}{\rightarrowtail}$};
          % \draw[>->] (51mm,2mm) -- (52mm,3mm);
      \end{tikzpicture}
    %   }
%       \caption{}
%       \label{ex:nwf:grsaa_rule_2}
%   \end{figure} 
    \end{center}
     There are 4 morphisms $t_K^{11}, t_K^{12}, t_K^{21}, t_K^{22}$ from $K$ to $T$:
    %   in Figure~\ref{fig:nwf:grsaa_rule_morphisms}.

    % \begin{figure}[H]
    %     \centering
    \begin{center}
        % \resizebox{0.7\textwidth}{!}{
            \begin{tikzpicture}
            \graphbox{\( K \)}{-50mm}{0mm}{40mm}{30mm}{2mm}{-6mm}{
                \coordinate (o) at (0mm,-10mm); 
                \node[draw,circle] (l1) at ($(o)+(-10mm,0mm)$) {1};
                \node[draw,circle] (l2) at ($(l1)+(2,0)$) {2};
                % \node[draw,circle] (l3) at ($(l1)+(1,0)$) {3};
                % \draw[] (l1) -- (l3) node[midway,above] {$a$};
                % \draw[] (l3) -- (l2) node[midway,above] {$a$};
            } 
                \graphbox{$T$}{0mm}{0mm}{40mm}{30mm}{-10mm}{-15mm}{
                    \node[draw,circle] (1) at (0,0) {$1\ 2$};
                    \node[draw,circle] (2) at (2,0) {};
                    \draw[->] (1) edge[loop above] node[midway, above] {$a$} (1) ;
                    \draw[->] (1) edge[loop below] node[midway, below] {$b$} (1) ;
                    \draw[->] (1) edge[bend left] node[midway, above] {$a$}  (2)  ;
                    \draw[->] (2) edge[bend left] node[midway, below] {$a$} (1)   ;
                }
                \node () at (-5mm,-15mm) {$\overset{t_K^{11}}{\to}$};
            \end{tikzpicture}
            % } 

            \vspace{2mm}

            % \resizebox{0.7\textwidth}{!}{
            \begin{tikzpicture}
                \graphbox{\( K \)}{-50mm}{0mm}{40mm}{28mm}{2mm}{-6mm}{
                \coordinate (o) at (0mm,-10mm); 
                \node[draw,circle] (l1) at ($(o)+(-10mm,0mm)$) {1};
                \node[draw,circle] (l2) at ($(l1)+(2,0)$) {2};
                % \node[draw,circle] (l3) at ($(l1)+(1,0)$) {3};
                % \draw[] (l1) -- (l3) node[midway,above] {$a$};
                % \draw[] (l3) -- (l2) node[midway,above] {$a$};
            } 
                \graphbox{$T$}{0mm}{0mm}{40mm}{28mm}{-10mm}{-15mm}{
                    \node[draw,circle] (1) at (0,0) {$1$};
                    \node[draw,circle] (2) at (2,0) {2};
                    \draw[->] (1) edge[loop above] node[midway, above] {$a$} (1) ;
                    \draw[->] (1) edge[loop below] node[midway, below] {$b$} (1) ;
                    \draw[->] (1) edge[bend left] node[midway, above] {$a$}  (2)  ;
                    \draw[->] (2) edge[bend left] node[midway, below] {$a$} (1)   ;
                }
                \node () at (-5mm,-15mm) {$\overset{t_K^{12}}{\to}$};
            \end{tikzpicture}
            % }
            
            \vspace{2mm}

            % \resizebox{0.7\textwidth}{!}{
            \begin{tikzpicture}
                \graphbox{\( K \)}{-50mm}{0mm}{40mm}{28mm}{2mm}{-6mm}{
                \coordinate (o) at (0mm,-10mm); 
                \node[draw,circle] (l1) at ($(o)+(-10mm,0mm)$) {1};
                \node[draw,circle] (l2) at ($(l1)+(2,0)$) {2};
                % \node[draw,circle] (l3) at ($(l1)+(1,0)$) {3};
                % \draw[] (l1) -- (l3) node[midway,above] {$a$};
                % \draw[] (l3) -- (l2) node[midway,above] {$a$};
            } 
                \graphbox{$T$}{0mm}{0mm}{40mm}{28mm}{-10mm}{-15mm}{
                    \node[draw,circle] (1) at (0,0) {2};
                    \node[draw,circle] (2) at (2,0) {1};
                    \draw[->] (1) edge[loop above] node[midway, above] {$a$} (1) ;
                    \draw[->] (1) edge[loop below] node[midway, below] {$b$} (1) ;
                    \draw[->] (1) edge[bend left] node[midway, above] {$a$}  (2)  ;
                    \draw[->] (2) edge[bend left] node[midway, below] {$a$} (1)   ;
                }
                \node () at (-5mm,-15mm) {$\overset{t_K^{21}}{\to}$};
            \end{tikzpicture}
            % }

            \vspace{2mm}

            % \resizebox{0.7\textwidth}{!}{
            \begin{tikzpicture}
                \graphbox{\( K \)}{-50mm}{0mm}{40mm}{28mm}{2mm}{-6mm}{
                \coordinate (o) at (0mm,-10mm); 
                \node[draw,circle] (l1) at ($(o)+(-10mm,0mm)$) {1};
                \node[draw,circle] (l2) at ($(l1)+(2,0)$) {2};
                % \node[draw,circle] (l3) at ($(l1)+(1,0)$) {3};
                % \draw[] (l1) -- (l3) node[midway,above] {$a$};
                % \draw[] (l3) -- (l2) node[midway,above] {$a$};
            } 
                \graphbox{$T$}{0mm}{0mm}{40mm}{26mm}{-10mm}{-15mm}{
                    \node[draw,circle] (1) at (0,0) {};
                    \node[draw,circle] (2) at (2,0) {$1\ 2$};
                    \draw[->] (1) edge[loop above] node[midway, above] {$a$} (1) ;
                    \draw[->] (1) edge[loop below] node[midway, below] {$b$} (1) ;
                    \draw[->] (1) edge[bend left] node[midway, above] {$a$}  (2)  ;
                    \draw[->] (2) edge[bend left] node[midway, below] {$a$} (1)   ;
                }
                \node () at (-5mm,-15mm) {$\overset{t_K^{22}}{\to}$};
            \end{tikzpicture}
            % }
    %     \caption{}
    %     \label{fig:nwf:grsaa_rule_morphisms}
    %   \end{figure}
    \end{center}
    The set $\{l \mathop{\star} - \mathop{=} t_K^{11}\}$ consists of two morphisms $h_{11}^1$ and $h_{11}^2$:
    % shown in Figure~\ref{fig:nwf:grsaa_rule_morphisms_22}.
    % \begin{figure}[H]
    %     \centering
    \begin{center}
        \begin{tikzpicture}
          \graphbox{\( L \)}{-50mm}{0mm}{40mm}{35mm}{2mm}{-6mm}{
            \coordinate (o) at (0mm,-10mm); 
            \node[draw,circle] (l1) at ($(o)+(-10mm,0mm)$) {1};
            \node[draw,circle] (l2) at ($(l1)+(2,0)$) {2};
            \node[draw,circle] (l3) at ($(l1)+(1,0)$) {3};
            \draw[] (l1) -- (l3) node[midway,above] {$a$};
            \draw[] (l3) -- (l2) node[midway,above] {$a$};
        } 
            \graphbox{$T$}{0mm}{0mm}{40mm}{35mm}{-8mm}{-17mm}{
                \node[draw,circle] (1) at (0,0) {$1\ 2$};
                \node[draw,circle] (2) at (2,0) {3};
                \draw[->] (1) edge[loop above] node[midway, above] {$a^{1}$} (1) ;
                \draw[->] (1) edge[loop below] node[midway, below] {$b^{1}$} (1) ;
                \draw[->] (1) edge[bend left] node[midway, above] {$a^{1}$}  (2)  ;
                \draw[->] (2) edge[bend left] node[midway, below] {$a^{1}$} (1)   ;
            }
            \node () at (-5mm,-15mm) {$\overset{h_{11}^1}{\to}$};
        \end{tikzpicture}

        \vspace{2mm}

            \begin{tikzpicture}
              \graphbox{\(L\)}{-50mm}{0mm}{40mm}{37mm}{2mm}{-10mm}{
                \coordinate (o) at (0mm,-10mm); 
                \node[draw,circle] (l1) at ($(o)+(-10mm,0mm)$) {1};
                \node[draw,circle] (l2) at ($(l1)+(2,0)$) {2};
                \node[draw,circle] (l3) at ($(l1)+(1,0)$) {3};
                \draw[] (l1) -- (l3) node[midway,above] {$a$};
                \draw[] (l3) -- (l2) node[midway,above] {$a$};
            } 
                \graphbox{$T$}{0mm}{0mm}{40mm}{37mm}{-8mm}{-20mm}{
                    \node[draw,circle] (1) at (0,0) {$1\ 2\ 3$};
                    \node[draw,circle] (2) at (2,0) {};
                    \draw[->] (1) edge[loop above] node[midway, above] {$a^{1}$} (1) ;
                    \draw[->] (1) edge[loop below] node[midway, below] {$b^{1}$} (1) ;
                    \draw[->] (1) edge[bend left] node[midway, above] {$a^{1}$}  (2)  ;
                    \draw[->] (2) edge[bend left] node[midway, below] {$a^{1}$} (1)   ;(1)   ;
                }
                \node () at (-5mm,-15mm) {$\overset{h_{11}^2}{\to}$};
            \end{tikzpicture}
    \end{center}
    Therefore, we have \begin{flalign*}
        w_\mathcal{T}(\{l \mathop{\star} - \mathop{=} t_K^{11}\})
        =&w_\mathcal{T}(\{h_{11}^1, h_{11}^2\})\\
        =&w_\mathcal{T}(h_{11}^1)\mathop{+}w_\mathcal{T}(h_{11}^2) \\
        =&(1^1 * 1^1)+(1^1 * 1^1)\\
        =&2.
    \end{flalign*}
    The set $\{r \mathop{\star} - \mathop{=} t_K^{11}\}$ has one morphism $h_{11}^3$:
    \begin{center}
        % \resizebox{0.7\textwidth}{!}{
        \begin{tikzpicture}
          \graphbox{\( R \)}{-55mm}{0mm}{45mm}{44mm}{1mm}{-22mm}{
            \coordinate (o) at (-5mm,-3mm); 
            \node[draw,circle] (l1) at ($(o)+(-10mm,0mm)$) {1};
            \node[draw,circle] (l2) at ($(l1)+(3,0)$) {2};
            \node[draw,circle] (l3) at ($(l1)+(1,0)$) {4};
            \node[draw,circle] (l4) at ($(l1)+(2,0)$) {5};
            \draw[->] (l1) -- (l3) node[midway,above] {$a$};
            \draw[->] (l3) -- (l4) node[midway,above] {$b$};
            \draw[->] (l4) -- (l2) node[midway,above] {$a$};
        } 
            \graphbox{$T$}{0mm}{0mm}{40mm}{44mm}{-10mm}{-22mm}{
                \node[draw,circle] (1) at (0,0) {$1\ 2\ 4\ 5$};
                \node[draw,circle] (2) at (2,0) {};
                \draw[->] (1) edge[loop above] node[midway, above] {$a^{1}$} (1) ;
                \draw[->] (1) edge[loop below] node[midway, below] {$b^{1}$} (1) ;
                \draw[->] (1) edge[bend left] node[midway, above] {$a^{1}$}  (2)  ;
                \draw[->] (2) edge[bend left] node[midway, below] {$a^{1}$} (1)   ;
            }
            \node () at (-5mm,-19mm) {$\overset{h_{11}^3}{\to}$};
        \end{tikzpicture}
        % }
    \end{center}
        %     \caption{}
    %     \label{fig:example:context_closure_grs_aa}
    %   \end{figure}
    Therefore, we have: 
        $$w_\mathcal{T}(\{r \mathop{\star} - \mathop{=} t_K^{11}\}) \mathop{=} w_\mathcal{T}(h_{11}^3) \mathop{=} 1^1 * 1^1 * 1 ^ 1 \mathop{=} 1.$$ 
    The following inequality follows
     $$w_\mathcal{T}(\{l \mathop{\star} - \mathop{=} t_K^{11}\}) \mathop{=} 2 \mathop{\geq} 1 \mathop{=} w_\mathcal{T}(\{r \mathop{\star} - \mathop{=} t_K^{11}\}).$$

    Similarly, we can check that
        \begin{itemize}
            \item $w_\mathcal{T}(\{l \mathop{\star} - \mathop{=} t_K^{12}\}) \mathop{=} 1 \mathop{\geq} 1 \mathop{=} w_\mathcal{T}(\{r \mathop{\star} - \mathop{=} t_K^{12}\})$, and
            \item $w_\mathcal{T}(\{l \mathop{\star} - \mathop{=} t_K^{21}\}) \mathop{=} 1 \mathop{\geq} 1 \mathop{=} w_\mathcal{T}(\{r \mathop{\star} - \mathop{=} t_K^{21}\})$, and
            \item $w_\mathcal{T}(\{l \mathop{\star} - \mathop{=} t_K^{22}\}) \mathop{=} 1 \mathop{\geq} 1 \mathop{=} w_\mathcal{T}(\{r \mathop{\star} - \mathop{=} t_K^{22}\})$.
        \end{itemize}  
     The rule is therefore weakly decreasing.

    The morphism $c$ illustrated below is a context closure for the DPO rule and the weighted type graph shown above, as explained in Example~\ref{wf:example:decreasing_rule}.
    % \begin{figure}[H]
    % \centering    
    \begin{center}
    \begin{tikzpicture}
      \graphbox{\( L \)}{-50mm}{0mm}{40mm}{40mm}{2mm}{-8mm}{
        \coordinate (o) at (0mm,-10mm); 
        \node[draw,circle] (l1) at ($(o)+(-10mm,0mm)$) {1};
        \node[draw,circle] (l2) at ($(l1)+(2,0)$) {2};
        \node[draw,circle] (l3) at ($(l1)+(1,0)$) {3};
        \draw[] (l1) -- (l3) node[midway,above] {$a$}; 
        \draw[] (l3) -- (l2) node[midway,above] {$a$};
    } 
        \graphbox{$T$}{0mm}{0mm}{40mm}{40mm}{-10mm}{-17mm}{
            % \node[draw,circle] (1) at (0,0) {$1\ 2\ 3$};
            % \node[draw,circle] (2) at (2,0) {};
            \coordinate (o) at (2mm,-3mm); 
            \node[draw,circle] (1) at ($(o)+(0,0mm)$) {$1\ 2\ 3$};
            \node[draw,circle] (2) at ($(o)+(2,0)$) {};
            \draw[->] (1) edge[loop above] node[midway, above] {$a^{1}$} (1) ;
            \draw[->] (1) edge[loop below] node[midway, below] {$b^{1}$} (1) ;
            \draw[->] (1) edge[bend left] node[midway, above] {$a^{1}$}  (2)  ;
            \draw[->] (2) edge[bend left] node[midway, below] {$a^{1}$} (1)   ;
        }
        \node () at (-5mm,-15mm) {$\overset{c}{\to}$};
    \end{tikzpicture}
\end{center}
    Since the following conditions hold:
        \begin{itemize}
            \item $t_K^{11} \mathop{=} l \mathop{\star} c$, and
            \item $w_\mathcal{T}(\{l \mathop{\star} - \mathop{=} t_K^{11}\}) \mathop{=} 2 \mathop{>} 1 \mathop{=} w_\mathcal{T}(\{r \mathop{\star} - \mathop{=} t_K^{11}\})$.
        \end{itemize}
       we conclude that the rule is closure decreasing since the semiring is strictly monotonic.
\end{example} 
 
\subsection{Termination Criterion}
\label{wf:sec:termination}
Finally, we can state the main result from~\cite{endrullis2024generalized_icgt}.
\begin{theorem}[Termination of DPO rewriting system] 
    \label{wf:thm:termination_grs}
    Let $\mathcal{A}$ and $\mathcal{B}$ be sets of DPO rewriting rules, $\mathcal{T} \mathop{=} (T,\mathbb{E}, (S, \mathop{\oplus}, \mathop{\odot}, 0_S, 1_S, \prec, \leq), w)$ a finitary weighted type graph and $\mathfrak{F}$ a DPO rewriting framework such that

        \begin{itemize} 
            \item \(\operatorname{left}(\Delta)\) is weighable with \(\mathcal{T}\),
            \item \(\operatorname{right}(\Delta)\) is bounded above by \(\mathcal{T}\). 
        \end{itemize}
    for every rule $\rho \mathop{\in} (\mathcal{A }\mathop{\cup} \mathcal{B })$ and every double pushout diagram  
        $\Delta \mathop{\in} \mathfrak{F}(\rho)$. If the following conditions hold:
    \begin{enumerate}
        \item either every $\rho \mathop{\in} \mathcal{A}$ is uniformly decreasing or every $\rho \mathop{\in} \mathcal{A}$ is closure decreasing, and
        \item every rule $\rho \mathop{\in} \mathcal{B}$ is weakly decreasing,
    \end{enumerate}
    then $\mathop{\Rightarrow}_{\mathcal{A},\mathfrak{F}}$ is \textbf{terminating} relative to $\mathop{\Rightarrow}_{\mathcal{B},\mathfrak{F}}$.
\end{theorem} 
\begin{example} 
    \label{wf:example:termination}
    Termination of the DPO rule in Example~\ref{ex:grsaa} can be established using Theorem~\ref{wf:thm:termination_grs} together with the weighted type graph in Example~\ref{wf:example:weighted_type_graph} over the natural arithmetic semiring $\mathfrak{N} \mathop{=} (\mathbb{N},+,*,0_\mathbb{R},1_\mathbb{N},<,\leq)$. It is closure decreasing as explained in Example~\ref{wf:example:decreasing_rule}.
\end{example}
   
  
% % \section{Examples} 
% % \label{sec:examples}
% % \begin{notation}
    We use the notation from~\cite[Notation 1]{overbeek2023apbpo+} to visualize edge-labeled graph homomorphisms. Labeled graphs are enclosed in boxes with their names displayed in the top-left corner. Nodes and edges are assigned subsets of \(\mathbb{N}\) as identifiers, and these identifiers are chosen such that: (i) Each node or edge \( y \) in the codomain graph is assigned the union of the identifiers of all nodes or edges in the domain graph that are mapped to \( y \); (ii) The graph homomorphism is uniquely determined by this assignment.
    
    \noindent To further improve readability, we represent sets by listing their elements. Additionally, we omit identifiers when doing so does not cause confusion. This is illustrated in the following representation of a homomorphism \( h: G \to H \).
    
    \begin{center}
        \resizebox{0.45\textwidth}{!}{
        \begin{tikzpicture}
            \graphbox{\( G \)}{00mm}{-20mm}{45mm}{20mm}{2mm}{-5mm}{
                \coordinate (o) at (-5mm,-8mm); 
                \node[draw,circle] (l1) at ($(o)+(-10mm,0mm)$) {1};
                \node[draw,circle] (l2) at ($(l1)+(3,0)$) {2};
                \node[draw,circle] (l3) at ($(l1)+(1,0)$) {3};
                \node[draw,circle] (l4) at ($(l1)+(2,0)$) {4};
                \draw[->] (l1) -- (l3) node[midway,above] {a};
                \draw[->] (l3) -- (l4) node[midway,above] {b};
                \draw[->] (l4) -- (l2) node[midway,above] {a};
            }  
            \graphbox{\( H \)}{50mm}{-20mm}{34mm}{20mm}{2mm}{-5mm}{
                \coordinate (o) at (0mm,-8mm); 
                \node[draw,circle] (l1) at ($(o)+(-10mm,0mm)$) {1};
                \node[draw,circle] (l2) at ($(l1)+(2,0)$) {2};
                \node[draw,circle] (l3) at ($(l1)+(1,0)$) {3\ 4};
                \draw[->] (l1) -- (l3) node[midway,above] {a};
                \draw[->] (l3) edge[loop above] (l3) node[midway,above] {b};
                \draw[->] (l3) -- (l2) node[midway,above] {a};
            }      
            % \node () at (53mm,-30mm) {$\rightarrow$};
        \end{tikzpicture}
    }
    \end{center} 
    In this example, the sets \(\{1\}\), \(\{2\}\), \(\{3\}\), \(\{4\}\), and \(\{3,4\}\) are represented as \(1\), \(2\), \(3\), \(4\), and \(3\ 4\), respectively. Edge identifiers are omitted.
\end{notation} 
\begin{example}
    \label{nonwf}
    Let $n \in \mathbb{N}_{\geq 2}$. Consider the graph rewriting system with three rules shown below. The right-hand-side graph of the first rule contains \( n \) arrows labeled \( b \) from node $1$ to node $2$, and the right-hand-side graph of the second rule contains \( n \) arrows labeled \( c \) from node $1$ to node $2$. The left-hand-side graph for the third rule contains \( n^2+1\) arrows labeled \( c \) from node $1$ to node $2$.

    \begin{center}
        % \resizebox{0.7\textwidth}{!}{
            \begin{tikzpicture}
                    \graphbox{\( L  \)}{0mm}{5mm}{34mm}{19mm}{2mm}{-5mm}{
                        \coordinate (o) at (0mm,-7mm); 
                        \node[draw,circle] (l1) at ($(o)+(-10mm,2mm)$) {1};
                        \node[draw,circle] (l2) at ($(l1)+(2,0)$) {2};
                        \draw[->] (l1) -- (l2) node[midway,above] {a};
                    } 
      
                    \graphbox{\( K \)}{40mm}{5mm}{34mm}{19mm}{2mm}{-5mm}{
                        \coordinate (o) at (0mm,-7mm); 
                        \node[draw,circle] (l1) at ($(o)+(-10mm,2mm)$) {1};
                        \node[draw,circle] (l2) at ($(l1)+(2,0)$) {2};
                    }  
      
                    \graphbox{\( R  \)}{80mm}{5mm}{45mm}{19mm}{2mm}{-5mm}{
                        \coordinate (o) at (-5mm,-7mm); 
                        \node[draw,circle] (l1) at ($(o)+(-10mm,2mm)$) {1};
                        \node[draw,circle] (l2) at ($(l1)+(2,0)$) {2};
                        \node () at ($(l1)+(1,0)$) {$\vdots$};
                        \draw[->] (l1) to[bend left] node[midway,above] {b} (l2) ;
                        \draw[->] (l1) to[bend right]  node[midway,below] {b}(l2) ;
                    }    
                    \node () at (37mm,-8mm) {\( \leftarrowtail \)}; % K -> L
                    \node () at (77mm,-8mm) {\( \rightarrowtail \)}; % K -> R
            \end{tikzpicture}
    
            \begin{tikzpicture}
                \graphbox{\( L  \)}{0mm}{5mm}{34mm}{19mm}{2mm}{-5mm}{
                    \coordinate (o) at (0mm,-7mm); 
                    \node[draw,circle] (l1) at ($(o)+(-10mm,2mm)$) {1};
                    \node[draw,circle] (l2) at ($(l1)+(2,0)$) {2};
                    \draw[->] (l1) -- (l2) node[midway,above] {b};
                } 
  
                \graphbox{\( K \)}{40mm}{5mm}{34mm}{19mm}{2mm}{-5mm}{
                    \coordinate (o) at (0mm,-7mm); 
                    \node[draw,circle] (l1) at ($(o)+(-10mm,2mm)$) {1};
                    \node[draw,circle] (l2) at ($(l1)+(2,0)$) {2};
                }  
  
                \graphbox{\( R  \)}{80mm}{5mm}{45mm}{19mm}{2mm}{-5mm}{
                    \coordinate (o) at (-5mm,-7mm); 
                    \node[draw,circle] (l1) at ($(o)+(-10mm,2mm)$) {1};
                    \node[draw,circle] (l2) at ($(l1)+(2,0)$) {2};
                    \node () at ($(l1)+(1,0)$) {$\vdots$};
                    \draw[->] (l1) to[bend left] node[midway,above] {c} (l2) ;
                    \draw[->] (l1) to[bend right]  node[midway,below] {c}(l2) ;
                }    
                \node () at (37mm,-8mm) {\( \leftarrowtail \)}; % K -> L
                \node () at (77mm,-8mm) {\( \rightarrowtail \)}; % K -> R
        \end{tikzpicture}


        \begin{tikzpicture}
            \graphbox{\( L  \)}{0mm}{5mm}{34mm}{19mm}{2mm}{-5mm}{
                \coordinate (o) at (0mm,-7mm); 
                \node[draw,circle] (l1) at ($(o)+(-10mm,2mm)$) {1};
                \node[draw,circle] (l2) at ($(l1)+(2,0)$) {2};
                \draw[->] (l1) to[bend left] node[midway,above] {c} (l2);
                % \draw[->] (l1) to[out = 60, in=120] node[midway,above] {c} (l2);
                % \draw[->] (l1) to[out = 45, in=135] node[midway,above] {c} (l2);
                % \draw[->] (l1) to[out = 30, in=150] node[midway,above] {c} (l2);
                \draw[->] (l1) to[bend right] node[midway,below] {c} (l2);
                \node () at ($(l1)+(1,0)$) {$\vdots$};
            } 

            \graphbox{\( K \)}{40mm}{5mm}{34mm}{19mm}{2mm}{-5mm}{
                \coordinate (o) at (0mm,-7mm); 
                \node[draw,circle] (l1) at ($(o)+(-10mm,2mm)$) {1};
                \node[draw,circle] (l2) at ($(l1)+(2,0)$) {2};
            }  

            \graphbox{\( R  \)}{80mm}{5mm}{45mm}{19mm}{2mm}{-5mm}{
                \coordinate (o) at (-5mm,-7mm); 
                \node[draw,circle] (l1) at ($(o)+(-10mm,2mm)$) {1};
                \node[draw,circle] (l2) at ($(l1)+(2,0)$) {2};
                \draw[->] (l1) -- (l2) node[midway,above] {a};
            }    
            \node () at (37mm,-8mm) {\( \leftarrowtail \)}; % K -> L
            \node () at (77mm,-8mm) {\( \rightarrowtail \)}; % K -> R
    \end{tikzpicture}
        \end{center}

    The termination of this rewriting system can be proved using weighted type graphs of size $1$ over the natural tropical semiring, the natural arctic semiring or the arithmetic semiring. Let $w_a$,$w_c$ and $w_c$ be the weights of edges labeled by $a$, $b$ and $c$ respectively.
    For any feasible weighted type graph over the natural tropical semiring or the natural arctic semiring, we have $w_a \geq n*2$ the following conditions must be satisfied
    \begin{flalign}
        w_a \geq n*w_b\\
        w_b \geq n*w_c\\
        (n^2+1) * w_c \geq w_a \\
        w_a > n*w_b \lor 
        w_b > n*w_c \lor 
        (n^2+1) * w_c > w_a
    \end{flalign}
    For any feasible weighted type graph over the natural arithmetic semiring, we have $w_a \geq n*2$ as the following conditions must be satisfied
    \begin{flalign}
        w_a \geq w_b^n\\
        w_b \geq w_c^n\\
        w_c^{n^2+1} \geq w_a \\
        w_a \geq w_b^n \lor 
        w_b \geq w_c^n \lor
        w_c^{n^2+1} \geq w_a
    \end{flalign}

    Let $\mathcal{R}$ be a systems including this system. The constraints on weights will included too. If proving termination of $R$ requires a type graph with dimension greater than $1$, then constructing a suitable weighted type graph over the well-founded semiring on extended natural numbers proposed in the prior work becomes extremely difficult in practice, due the high computational complexity.
\end{example}

% \section{Implementation}
% \label{sec:type_graph:implementation}
% % We have implemented both our approach and the one proposed by Endrullis and Overbeek~\cite{endrullis2024generalized_arxiv_v2} for edge-labeled DPO rewriting systems into a unified tool. These two approaches can be launched in parallel and collaborate to establish termination of a DPO rewriting system.
Chapter~\ref{chap:lyonparallel} provides a description of the implementation of our approach and the one proposed by Endrullis and Overbeek~\cite{endrullis2024generalized_arxiv_v2} in our tool, called LyonParallel and described in Section~\ref{chap:lyonparallel}.

\paragraph{Searching Strategy.}
Consider a DPO rewriting system $\mathcal{R}$. Given a fixed semiring $S$, a processor is spawned to iteratively search for a suitable weighted type graph over $S$.
The search begins with a weighted type graph containing $1$ node and increases the node count by $1$ until it reaches a maximum of 4 nodes.

For a fixed number of nodes,
 if $S$ is a semiring over the natural numbers, the maximum edge weight is initialized to 1 and incremented by 1 (up to a limit of 3) if no suitable weighted type graph is found;
if $S$ is a semiring over the real numbers, weights are constrained to be positive real numbers, and no upper bound is imposed.
Processors targeting different semirings can run in parallel to analyze the same DPO rewriting system.
When a processor discovers a weighted type graph that witnesses relative termination of a subset of rules $\mathcal{A}$ with respect to another subset of rules $\mathcal{B}$ such that $\mathcal{R} \mathop{=} \mathcal{A} \mathop{\cup} \mathcal{B}$, it broadcasts this result to all processors and waits for them to terminate. If $\mathcal{B} \mathop{=} \emptyset$, the system's termination is proven. Otherwise, we initiate a new search for the rules in $\mathcal{B}$ unless a timeout is reached.
 
For a fixed number of nodes $k$, we adopt the approach proposed in \cite{bruggink2015proving, bruggink2014termination,zantema2014termination} to reduce the search space. Specifically, we construct a weighted type graph with
% \( \mathcal{T}=(T, \mathbb{E}, S, w) \) where: (1) the type graph $T$ has 
$k$ nodes and no parallel edges of the same label.
%  and (2) the set $\mathbb{E}$ of morphism-rulers consists of identity morphisms from subgraphs of the type graph $T$ (each is a labeled edge with its incident nodes) to the type graph $T$.
The search proceeds as follows: 
\begin{enumerate}
    \item decide if $s \overset{l}{\to} t$ exists for every pair of nodes \( s, t\) and label \( l\);
    \item assign a weight to every existing edge;
    \item verify the existence of a constant $\delta >0$ and a partition of the rule set $\mathcal{R}$ into a non-empty subset $\mathcal{A}$ and a subset $\mathcal{B}$
 such that:
    \begin{itemize}
        \item either all rules in $\mathcal{A}$ are $\delta$-uniformly decreasing or all rules in $\mathcal{A}$ are $\delta$-closure decreasing, and
        \item all rules in $\mathcal{B}$ are weakly decreasing.
    \end{itemize}
\end{enumerate}

%  Check if the weighted type graph satisfy requirements
% (1) construct a graph with \( k \) nodes and no parallel edges of the same label, which has a directed edge from node $s$ to node $t$ labeled by $l$ for each ordered pair $(s,t)$ of nodes and label $l$ from the finite set of edge labels $\Sigma$ of the rewriting system; (2) decide whether each edge exists in the weighted type graph; (3) assign a weight (a natural number or a real number according to $S$) to every existing edge; (4) 

This procedure amounts to checking the satisfiability of an existential Presburger arithmetic, Peano arithmetic, linear real arithmetic, or non-linear real arithmetic formula depending on the semiring $S$ considered.

\paragraph{Z3 Modeling.}
The type graph $T$ is modeled in Z3 by defining 
a boolean variable $x_{u,v,l} \mathop{\in} \mathbb{B}$ for every directed labeled edge $u\overset{l}{\to} v$ in the type graph, where $u,v\in\{1,...,k\}$ are nodes and $l \mathop{\in} \Sigma$ is an edge label. 
The variable $x_{u,v,l}$ has the value \textit{true} if the directed edge $u\overset{l}{\to} v$ exists in the result type graph, \textit{false} otherwise. 

The weight function $w$ and the set $\mathbb{E}$ of morphism-rulers are modeled by defining 
a variable $y_{u,v,l}$ of Z3's \textit{Real} sort or \textit{Int} sort (the solver's default theory for real numbers and integers), depending on the semiring $\mathcal{S}$ considered,
 for every pair $u,v\in\{1,...,k\}$ of nodes and edge label $l \mathop{\in} \Sigma$. 
The variable $y_{u,v,l}$ represents the weight of the directed labeled edge $u\overset{l}{\to} v$ in the resulting weighted type graph, but only if edge $u\overset{l}{\to} v$ exists (i.e. $x_{u,v,l}$ has the value \textit{true}). 
They are constrained to be non-negative.

A variable $\delta \mathop{\in} \mathbb{R}_{>0}$ is defined to ensure that there is
a partition of the rule set $\mathcal{R}$ into a non-empty subset $\mathcal{A}$ and a subset $\mathcal{B}$ such that all rules in $\mathcal{A}$ are either $\delta$-uniformly decreasing or $\delta$-closure decreasing, and all rules in $\mathcal{B}$ are weakly decreasing.

For convenience, the following auxiliary variables are also defined:
\begin{itemize}
    \item a boolean variable $v_h \mathop{\in} \mathbb{B}$ for every morphism $h$ from $L$ or $R$ to $T$;
    \item a real-valued variable $v_{h'} \mathop{\in} \mathbb{R}_{\geq 0}$ for every morphism $h$ from $L$ or $R$ to $T$;
    \item a boolean variable $v_c \mathop{\in} \mathbb{B}$ for every morphism $c$ from $L$ to $T$.
\end{itemize}  
The value of $v_h$ has the value \textit{true} if the morphism $h$ exists in the result type graph (i.e. all edges in its image exist), and \textit{false} otherwise.
The variable $v_{h'}$ holds the weight of the morphism $h$, provided the morphism $h$ exists (i.e., $v_h$ has the value \textit{true}).
The variable $v_c$ has the value \textit{true} if the morphism $c$ exists in the result type graph and can serve as a context closure for the rule, and \textit{false} otherwise.

If there is an assignment of values to 
    \begin{itemize}
        \item $(x_{u,v,l})$, for all $u,v \mathop{\in} \{1,...,k\}$ and for all $l \mathop{\in} \Sigma$, and
        \item $(y_{u,v,l})$, for all $u,v \mathop{\in} \{1,...,k\}$ and for all $l \mathop{\in} \Sigma$, and
        \item $\delta$,
    \end{itemize} 
    such that all conditions of Theorem~\ref{nwf:thm:termination_grs} are satisfied, then a suitable weighted type graph that witnesses termination of the DPO rewriting system can be constructed, because
    \begin{itemize}
        \item the values of \( (x_{u,v,l})_{u,v \mathop{\in} \{1,...,k\}, l \mathop{\in} \Sigma} \) define the type graph $T$, and
        \item the values of \( (y_{u,v,l})_{u,v \mathop{\in} \{1,...,k\}, l \mathop{\in} \Sigma} \) define the weight function $w$ and the set $\mathbb{E}$ of morphism-rulers.
    \end{itemize}  


% % \section{Implementation}
% % \label{sec:implementation}
% % To the best of our knowledge, no method exists to decide whether a weighted type graph can be constructed over the existing concrete semirings in  Example~\ref{example:real_semirings} in general. 
This problem is inherently hard because it requires quantifying over all edge-labeled multigraphs with weighted edges.

Existing implementations presented in \cite[\textsection 6]{bruggink2015proving}, ~\cite[\textsection 6]{zantema2014termination}, \cite[\textsection E]{endrullis2024generalized} employ Z3, a satisfiable modulo theories (SMT) solver that can solve first order theories over the nautural numbers, to searching a suitable type graph.
Furthermore, they constrain the search space by fixing a natural number
\( k \mathop{\in} \mathbb{N} \) and constructing a weighted type graph \((T, \mathbb{E}, S, w)\) where the underlying graph \( T \) is a simple graph with \( k \) nodes.

We adopt the same method, employing the Z3 and constraining the search space by fixing a natural number \( k \mathop{\in} \mathbb{N} \). Specifically, we construct a weighted type graph \((T, \mathbb{E}, S, w)\), where the underlying graph \( T \) is a simple graph with \( k \) nodes.

To model a type graph in Z3, we adopt an approach similar to that proposed by Bruggink in \cite[\textsection 6]{bruggink2015proving}.
Specifically, we fix a complete simple graph $T$ with $k$ nodes, i.e. a graph with an edge for every pair $i,j\in\{1,...,k\}$ of nodes and every edge label $l \mathop{\in} \Sigma$. 
Every edge $e$ in this graph is associated with two variables:
a binary variable \( x_e \mathop{\in} \{0,1\} \) indicating the presence/absence of \( e \) in the type graph;
a numeric variable \( y_e \) (integer or real, depending on the semiring) representing the edge weight when \( e \) exists.  
We fix additionally a real variable $\delta \mathop{\in} \mathbb{R}_{>0}$.
The objective is to assign values to $\delta$, \( (x_e)_{e \mathop{\in} E(T)} \) and \( (y_e)_{e \mathop{\in} E(T)} \) such that all rules are either: decreasing, $\delta$-uniformly decreasing (for tropical or arctic semirings) or \(\delta\)-closure decreasing (for arithmetic semirings).  
These constraints are formalized as first-order formulas over the natural numbers (or real numbers) and encoded in the Z3. The resulting satisfiability problem is then resolved using Z3.

% Existing implementations of the type graph method~\cite{TORPAcyc,grez}~\footnote{There exists an implementation in GraphTT~\cite{endrullis2024generalized}, but to the best of our knowledge, this implementation is not publicly available} 
% constrain the search space by fixing a natural number
% \( k \mathop{\in} \mathbb{N} \) and constructing a weighted type graph \((T, \mathbb{E}, S, w)\) where the underlying graph \( T \) is a simple graph with \( k \) nodes (see~\cite[\textsection 6]{bruggink2015proving}, ~\cite[\textsection 6]{zantema2014termination}, \cite[\textsection E]{endrullis2024generalized}).
However, even under these constraints, \textcolor{red}{constructing weighted type graphs over the natural semirings remains challenging.}\todo{It is (still) not clear that you mean by "constructing". Are you searching for a type graph with specific properties?} Specifically, constructing a weighted type graph over the natural tropical semiring $\mathfrak{T}$ or the natural arctic semiring $\mathfrak{A}$ requires solving first-order formulas in Presburger arithmetic. Any algorithm that decides this problem has at least double-exponential time complexity with respect to $k^2 * | \Sigma |$ where \( \Sigma \) denotes the set of edge labels~\cite{fischer1998super}. For the natural arithmetic semiring $\mathfrak{N}$, the task reduces to solving first-order formulas in Peano arithmetic\textemdash a task that is semi-decidable~\cite{matiyasevivc2003enumerable}.

Consequently, existing tools rely on user-specified sets of weights to constrain the size of the search space. However, determining these sets in advance is hard if not impossible, limiting the tools' accessibility and practical applicability.
 
However, constructing weighted type graphs over the real semirings is significantly easier under these constraints.
Specifically, constructing a weighted type graph whose underlying graph is a \textcolor{red}{simple}\todo{what makes a graph simple? Do you mean that here you are restricting to a specific category? Which one?} graph of size \( k \) over $\mathfrak{T}'$ and $\mathfrak{A}'$ involves solving a mixed-integer linear programming problem with $k^2 * | \Sigma |$ binary variables and $k^2 * | \Sigma |$ real variables.
State-of-the-art SMT solvers such as Z3~\cite{de2008z3} can efficiently handle this task.
 \todo{complexity}
For the real arithmetic semiring $\mathfrak{N}'$, the task reduces to solving first-order formulas of the real numbers. This problem is decidable in double-exponential time
relative to $k^2 * | \Sigma |$, where
\( \Sigma \) denotes the set of edge labels~\cite{collins1974quantifier}, in contrast to the undecidable problem for the natural arithmetic semiring $\mathfrak{N}$.

By avoiding undecidable Peano arithmetic, manual weight selection, and reducing computational complexity, a more automated, accessible, practical and efficient tool can be implemented. Besides, it would be interesting to
combine the complementary strengths of our approach and prior methods, as constructing weighted type graphs over the real arithmetic semiring remains computationally challenging.

We implemented both our approach and the approach proposed by Endrullis et al. \cite{endrullis2024generalized} for edge-labeled directed graph rewriting in a tool in OCaml to leverage their complementary strengths.

% Following prior work, we constrain the search space by fixing a natural number
% \( k \mathop{\in} \mathbb{N} \) and constructing a weighted type graph \((T, \mathbb{E}, S, w)\) where the underlying graph \( T \) is a simple graph with \( k \) nodes. 
Given a graph rewriting system, our tool searches in parallel weighted type graphs over the six existing concrete semirings using an automated strategy: the graph size incrementally increases from 1 to 4; for the natural semirings, the max weight incrementally increases from 1 to 3. Decision problems are first translated into first-order theories in Z3 and then solved using Z3~\cite{de2008z3}.
This fully automated approach requires only the problem definition and a timeout, removing the need for users to specify details such as the semiring to be used or the size of the type graph. 
It accurately reflects real-world scenarios, as it is generally hard to determine parameters such as graph size, semiring, and weight bounds in advance.

Finally, if our tool terminates before timeout and reports that no suitable weighted type graph of size \( k \mathop{\in} \mathbb{N} \) has been found, this implies that no such weighted type graph. Specifically, no weighted type graph with a simple underlying graph of size \( k \) exists for any of the existing six concrete semirings analyzed. \todo{how; why}
For the tropical and arctic semirings, current tools are constrained to constructing type graphs with user-specified finite weight sets, a restriction that inherently precludes them from proving the absence of solutions for all possible weights. In the case of the natural arithmetic semiring, where the decision problem is semi-decidable, existing tools are fundamentally \textcolor{red}{incapable of providing negative results.}\todo{Not true; but you can't be complete in this respect} 

% \section{Empirical Results}
% \label{sec:type_graph:result}
% We evaluated our OCaml implementation using 13 examples from~\cite{endrullis2024generalized_arxiv_v2,plump1995ontermination,plump2018modular,bruggink2015proving,bruggink2014termination}, excluding~\cite[Example 6.4]{endrullis2024generalized_arxiv_v2} which is a DPO rewriting system in an unsupported category, and~\cite[Example 6]{bruggink2014termination} because it duplicates~\cite[Example 4]{bruggink2015proving}. Our approach with semirings over the real numbers proves termination for 8 out of 13 systems tested. Table~\ref{tabular:benchmarks} summarizes runtime performance 
across benchmark examples under different semiring configurations~\footnote{Experiments were conducted on a laptop equipped with an i5-1038NG7 CPU, which features 4 cores, a base clock speed of 2 GHz, a boost speed of 3.8 GHz, and 16 GB RAM.}.


% try_type_graph 1 200 
\begin{table}[htb]   
    \renewcommand{\arraystretch}{1.2}
    \centering
\begin{tabular}{|c|c|c|c|c|c|c |}
    \hline
     &\;\;A\;\;&\;\;a\;\;&\;\;T\;\;&\;\;t\;\;&\;\; N\;\;&\;\;n\;\; \\
    \hline
   ~\cite[Example 6.2]{endrullis2024generalized_arxiv_v2} & & & & & 2.68 &1.15   \\
   ~\cite[Example 6.3]{endrullis2024generalized_arxiv_v2} & & & & & 2.74 &1.16   \\
    %~\cite[Example 6.4]{endrullis2024generalized_arxiv_v2}&  && & &  &   &  &  & our only Graph\\
   ~\cite[Example D.3]{endrullis2024generalized_arxiv_v2} &2.25 
    % (2.30+2.188+2.2637+2.2928+ 2.196) 
    & 1.18
    % (1.17+1.2549+1.165+ 1.1576\mathop{+}1.162)
    & & & 2.24& 1.18    \\
   ~\cite[Example 3.8]{plump1995ontermination}
 & 2.95& 1.90 & 2.94 &1.87  & 3.49  &1.87   \\
    %~\cite[Example 3]{plump2018modular}
    % & 3.45& 3.44 & 3.55 &2.38  & 3.96  &17.60 &   4.24 AT &  3.45 at& 4.07 t  \\
   ~\cite[Example 4]{plump2018modular} &4.26& 3.19&  4.24 & 3.13 &
    5.82
    % (5.77+5.83+5.775+5.84+5.818+5.89)
    & timeout  \\

   ~\cite[Example 5]{plump2018modular} & 5.54
    &5.55
    % (5.53 +5.660\mathop{+}5.5213+5.5369+5.484689\mathop{+}5.6404\mathop{+}5.4877\mathop{+}5.5026)
    & 5.53& 5.50& 9.11&  5.62  \\
   ~\cite[Example 6]{plump2018modular} &  &  & & &  &   \\
    % (* en haut j'ai enverse tropical et arctic *)
    %~\cite[Example 6]{plump2018modular} &  &6.93 & 6.38 & 6.30 & 7.87&  7.22  &  11.71 at & 7.71 AT& 13.60 ATNat \\
   ~\cite[Example 4]{bruggink2015proving} &
    2.44
    % (2.4415+ 2.4181+2.4466+2.4366+2.4624+2.4479)
    & 
    2.46
    % (2.4405+2.5180+2.4137+2.4539+2.4480+2.4750)
    & 
    2.47
    % (2.4660+2.4457+2.4559+2.4526+2.4588+2.4847+2.5731)/7
    &
    2.54
    % (2.4380+2.4557+2.5590+2.5631+2.7245+2.5041+2.5182)/7
    & 4.58 & 
    2.46
    % (2.3987+2.4521+2.4620+2.4558+2.5128+2.4507)/6
     \\
   ~\cite[Example 5]{bruggink2015proving} &  &  &&& 7.80& timeout  \\
   ~\cite[Example 6]{bruggink2015proving} &  &  &&& 9.75& timeout   \\
   ~\cite[Example 1]{bruggink2014termination} & 
     2.26
    %  (2.2887\mathop{+}2.2386 +2.2719\mathop{+}2.2735 +2.3025\mathop{+}2.1925)/6
     &1.18
    %  (1.1764+1.1837+1.1709+1.2222+1.1608+1.1645)/6
     & & &2.24
    %  (2.2365+2.2388+2.2124+2.2395+2.2550+2.2513)/6
     & 1.18  
    %  (1.2498+1.1710+1.1634+1.1655+1.1687+1.1733)/6
     \\
   ~\cite[Example 4]{bruggink2014termination} &  2.25
    % (2.2545+2.2512+2.2529+2.3498+2.2137+ 2.2257\mathop{+}2.2167\mathop{+}2.2413)/8
    & 1.22 & 2.24
    % (2.2981+2.2026\mathop{+}2.2229+2.2524+2.2421+2.2415)/6
    &1.18
    % (1.1261+1.1850+1.1848+1.1706+1.2052+1.1920)/6
    &2.25
    % (2.2347+2.3106+2.2130+2.2391+2.2466+2.2493)/6 
    & 1.19 

    \\
   ~\cite[Example 5]{bruggink2014termination} &  4.23 & 3.23  & 4.25 &3.28  & 5.82 & timeout \\
    %~\cite[Example 6]{bruggink2014termination} &  2.58& 1.63 & 2.61&1.65&2.67 & 1.50 \\
    %~\cite[Routing]{bruggink2014termination} &  & & & & &    
    % \\
    %~\autoref{nonwf} with $n=3$ & - & 1.90 &-& 1.95 &- & 2.00 & - & 2.47 a & 3.56 nT  \\
    \hline
    \end{tabular}
    % \begin{table}[htbp] 
    %     \renewcommand{\arraystretch}{1.2}
    %     \centering
    % \begin{tabular}{|c|c|c|c|c|c|c|c|c|c|}
    %     \hline
    %      &\;\;A\;\;&\;\;a\;\;&\;\;T\;\;&\;\;t\;\;&\;\; N\;\;&\;\;n\;\; &\;\;ATN\;\;& \;\;atn\;\;&\;\;ATNatn\;\;\\
    %     \hline
    %    ~\cite[Example 6.2]{endrullis2024generalized_arxiv_v2} & & & & & 2.80 &1.29 &   3.30 N&  2.38 n & 2.50 n \\
    %     %~\cite[Example 6.4]{endrullis2024generalized_arxiv_v2}&  && & &  &   &  &  & our only Graph\\
    %    ~\cite[Example D.3]{endrullis2024generalized_arxiv_v2} & 2.30 & 1.63 & & & 2.98& 1.33  &    2.83 N &  2.09 a& 2.19 a  \\
    %    ~\cite[Example 3.8]{plump1995ontermination}
    %  & 3.45& 3.44 & 3.55 &2.38  & 3.96  &17.60 &   4.24 AT &  3.45 at& 4.07 t  \\
    %    ~\cite[Example 3]{plump2018modular}
    %     & 3.45& 3.44 & 3.55 &2.38  & 3.96  &17.60 &   4.24 AT &  3.45 at& 4.07 t  \\
    %    ~\cite[Example 4]{plump2018modular} &  5.17 & 5.31  & 5.31 &4.18  & 6.95 & 21.68 & 7.10 ATN & 5.55 atn &7.15 atnA \\
    %    ~\cite[Example 5]{plump2018modular} & 6.27&6.93 & 6.38 & 6.30 & 7.87&  7.22  &  11.71 AT & 7.71 at& 13.60 atnAT \\
    %    ~\cite[Example 6]{plump2018modular} &  &  & & &  &  & & & \\
    %     % (* en haut j'ai enverse tropical et arctic *)
    %     %~\cite[Example 6]{plump2018modular} &  &6.93 & 6.38 & 6.30 & 7.87&  7.22  &  11.71 at & 7.71 AT& 13.60 ATNat \\
    %    ~\cite[Example 4]{bruggink2015proving} &2.76 & 2.91& 2.75 &2.79  & 5.05 & 2.69&  4.92 AT & 3.28 at &5.86 anT\\
    %    ~\cite[Example 5]{bruggink2015proving} &  &  &&& 10.17& timeout&  10.96 ATN& timeout& 21.07 atN  \\
    %    ~\cite[Example 6]{bruggink2015proving} &  &  &&& 21.50& timeout  &  25.39 ATN & timeout &78.28 atnAN\\
    %    ~\cite[Example 1]{bruggink2014termination} &  2.42&1.64& & &3.41& 1.33  &   3.08 A& 3.32 a &3.53 n \\
    %    ~\cite[Example 4]{bruggink2014termination} &  2.58& 1.63 & 2.61&1.65&2.67 & 1.50& 2.79 N& 1.66 a &2.57 t\\
    %    ~\cite[Example 5]{bruggink2014termination} &  5.17 & 5.31  & 5.31 &4.18  & 6.95 & 21.68 & 7.10 ATN & 5.55 atn &7.15 atnA \\
    %    ~\cite[Example 6]{bruggink2014termination} &  2.58& 1.63 & 2.61&1.65&2.67 & 1.50& 2.79 N& 1.66 a &2.57 t\\
    %    ~\cite[Routing]{bruggink2014termination} &  & & & & & & 11.38 ATN & 14.52 atn & 16.44 at  
    %     \\
    %     %~\autoref{nonwf} with $n=3$ & - & 1.90 &-& 1.95 &- & 2.00 & - & 2.47 a & 3.56 nT  \\
    %     \hline
    %     \end{tabular}%
    
    \medskip
    \caption{
    Runtime performance in seconds (200s timeout).
    Columns denote semirings:
    ``A'', ``T'', ``N'' denote the arctic, tropical, and arithmetic semirings over the (extended) natural numbers;
    ``a'', ``t'', ``n'' denote the arctic, tropical, and arithmetic semirings over the (extended) real numbers.
    % ``ATN", ``atn", ``ATNatn" denote combinations of different semirings. For these combined configurations, the specific semirings involved are also indicated. 
    Empty cells indicate termination not proven.
     }
    \label{tabular:benchmarks}
\end{table}

% \begin{table}[htbp]  
%     \renewcommand{\arraystretch}{1.2}
%     \centering
% \begin{tabular}{|c|c|c|c|c|c|c |}
%     \hline
%      &\;\;A\;\;&\;\;a\;\;&\;\;T\;\;&\;\;t\;\;&\;\; N\;\;&\;\;n\;\; \\
%     \hline
%    ~\cite[Example 6.2]{endrullis2024generalized_arxiv_v2} & & & & & 2.80 &1.29   \\
%    ~\cite[Example 6.3]{endrullis2024generalized_arxiv_v2} & & & & & 2.74 &1.16   \\
%     %~\cite[Example 6.4]{endrullis2024generalized_arxiv_v2}&  && & &  &   &  &  & our only Graph\\
%    ~\cite[Example D.3]{endrullis2024generalized_arxiv_v2} & 2.30 & 1.63 & & & 2.98& 1.33    \\
%    ~\cite[Example 3.8]{plump1995ontermination}
%  & 3.45& 3.44 & 3.55 &2.38  & 3.96  &17.60   \\
%     %~\cite[Example 3]{plump2018modular}
%     % & 3.45& 3.44 & 3.55 &2.38  & 3.96  &17.60 &   4.24 AT &  3.45 at& 4.07 t  \\
%    ~\cite[Example 4]{plump2018modular} &  5.17 & 5.31  & 5.31 &4.18  & 6.95 & 21.68   \\
%    ~\cite[Example 5]{plump2018modular} & 6.27&6.93 & 6.38 & 6.30 & 7.87&  7.22    \\
%    ~\cite[Example 6]{plump2018modular} &  &  & & &  &   \\
%     % (* en haut j'ai enverse tropical et arctic *)
%     %~\cite[Example 6]{plump2018modular} &  &6.93 & 6.38 & 6.30 & 7.87&  7.22  &  11.71 at & 7.71 AT& 13.60 ATNat \\
%    ~\cite[Example 4]{bruggink2015proving} &2.76 & 2.91& 2.75 &2.79  & 5.05 & 2.69 \\
%    ~\cite[Example 5]{bruggink2015proving} &  &  &&& 7.87& timeout  \\
%    ~\cite[Example 6]{bruggink2015proving} &  &  &&& 9.77& timeout   \\
%    ~\cite[Example 1]{bruggink2014termination} &  2.42&1.64& & &3.41& 1.33   \\
%    ~\cite[Example 4]{bruggink2014termination} &  2.58& 1.63 & 2.61&1.65&2.67 & 1.50 \\
%    ~\cite[Example 5]{bruggink2014termination} &  5.17 & 5.31  & 5.31 &4.18  & 6.95 & 21.68  \\
%    ~\cite[Example 6]{bruggink2014termination} &  2.58& 1.63 & 2.61&1.65&2.67 & 1.50 \\
%     %~\cite[Routing]{bruggink2014termination} &  & & & & &    
%     % \\
%     %~\autoref{nonwf} with $n=3$ & - & 1.90 &-& 1.95 &- & 2.00 & - & 2.47 a & 3.56 nT  \\
%     \hline
%     \end{tabular}
%     % \begin{table}[htbp] 
%     %     \renewcommand{\arraystretch}{1.2}
%     %     \centering
%     % \begin{tabular}{|c|c|c|c|c|c|c|c|c|c|}
%     %     \hline
%     %      &\;\;A\;\;&\;\;a\;\;&\;\;T\;\;&\;\;t\;\;&\;\; N\;\;&\;\;n\;\; &\;\;ATN\;\;& \;\;atn\;\;&\;\;ATNatn\;\;\\
%     %     \hline
%     %    ~\cite[Example 6.2]{endrullis2024generalized_arxiv_v2} & & & & & 2.80 &1.29 &   3.30 N&  2.38 n & 2.50 n \\
%     %     %~\cite[Example 6.4]{endrullis2024generalized_arxiv_v2}&  && & &  &   &  &  & our only Graph\\
%     %    ~\cite[Example D.3]{endrullis2024generalized_arxiv_v2} & 2.30 & 1.63 & & & 2.98& 1.33  &    2.83 N &  2.09 a& 2.19 a  \\
%     %    ~\cite[Example 3.8]{plump1995ontermination}
%     %  & 3.45& 3.44 & 3.55 &2.38  & 3.96  &17.60 &   4.24 AT &  3.45 at& 4.07 t  \\
%     %    ~\cite[Example 3]{plump2018modular}
%     %     & 3.45& 3.44 & 3.55 &2.38  & 3.96  &17.60 &   4.24 AT &  3.45 at& 4.07 t  \\
%     %    ~\cite[Example 4]{plump2018modular} &  5.17 & 5.31  & 5.31 &4.18  & 6.95 & 21.68 & 7.10 ATN & 5.55 atn &7.15 atnA \\
%     %    ~\cite[Example 5]{plump2018modular} & 6.27&6.93 & 6.38 & 6.30 & 7.87&  7.22  &  11.71 AT & 7.71 at& 13.60 atnAT \\
%     %    ~\cite[Example 6]{plump2018modular} &  &  & & &  &  & & & \\
%     %     % (* en haut j'ai enverse tropical et arctic *)
%     %     %~\cite[Example 6]{plump2018modular} &  &6.93 & 6.38 & 6.30 & 7.87&  7.22  &  11.71 at & 7.71 AT& 13.60 ATNat \\
%     %    ~\cite[Example 4]{bruggink2015proving} &2.76 & 2.91& 2.75 &2.79  & 5.05 & 2.69&  4.92 AT & 3.28 at &5.86 anT\\
%     %    ~\cite[Example 5]{bruggink2015proving} &  &  &&& 10.17& timeout&  10.96 ATN& timeout& 21.07 atN  \\
%     %    ~\cite[Example 6]{bruggink2015proving} &  &  &&& 21.50& timeout  &  25.39 ATN & timeout &78.28 atnAN\\
%     %    ~\cite[Example 1]{bruggink2014termination} &  2.42&1.64& & &3.41& 1.33  &   3.08 A& 3.32 a &3.53 n \\
%     %    ~\cite[Example 4]{bruggink2014termination} &  2.58& 1.63 & 2.61&1.65&2.67 & 1.50& 2.79 N& 1.66 a &2.57 t\\
%     %    ~\cite[Example 5]{bruggink2014termination} &  5.17 & 5.31  & 5.31 &4.18  & 6.95 & 21.68 & 7.10 ATN & 5.55 atn &7.15 atnA \\
%     %    ~\cite[Example 6]{bruggink2014termination} &  2.58& 1.63 & 2.61&1.65&2.67 & 1.50& 2.79 N& 1.66 a &2.57 t\\
%     %    ~\cite[Routing]{bruggink2014termination} &  & & & & & & 11.38 ATN & 14.52 atn & 16.44 at  
%     %     \\
%     %     %~\autoref{nonwf} with $n=3$ & - & 1.90 &-& 1.95 &- & 2.00 & - & 2.47 a & 3.56 nT  \\
%     %     \hline
%     %     \end{tabular}%
    
%     \medskip
%     \caption{
%     Runtime performance in seconds using the automated strategy (200s timeout).
%     Columns denote semirings:
%     ``A'', ``T'', ``N'' denote the arctic, tropical, and arithmetic semirings over (extended) natural numbers;
%     ``a'', ``t'', ``n'' denote the arctic, tropical, and arithmetic semirings over (extended) real numbers.
%     % ``ATN", ``atn", ``ATNatn" denote combinations of different semirings. For these combined configurations, the specific semirings involved are also indicated. 
%     Empty cells indicate termination not proven.
%      }
%     \label{tabular:benchmarks}
%     \end{table}
     

 

% \section{Discussion}
% \label{sec:type_graph:related_work}
% \paragraph{Acceleration with the real tropical~/~arctic semiring.} 
Consider the examples in Table~\ref{tabular:benchmarks} whose termination is provable using the tropical or arctic semirings. For all such examples except \cite[Example 5]{plump2018modular} and \cite[Example 4]{bruggink2015proving}, the real-valued semiring approach achieves better runtime performance than the integer-valued semiring approach. The acceleration rate varies across examples due to Z3's internal heuristics. This result is notable because termination proofs for these systems only require weighted type graphs with small edge weights (e.g., values less than 2 for most examples and at most 3 for all), which favors the implementation of the integer-valued semiring approach.

\paragraph{Timeout with the real arithmetic semiring.}
The systems whose termination is not proven by our tool are \cite[Example 6]{plump2018modular}, \cite[Examples 5 and 6]{bruggink2015proving}, \cite[Example 4]{plump2018modular}, and \cite[Example 5]{bruggink2014termination}.
The first case (\cite[Example 6]{plump2018modular}) is due to a limitation of the type graph method in general, as the system has a rule whose right-hand side graph can be embedded into its left-hand side graph, which is problematic for the type graph method, as pointed out in \cite[Example D.4]{endrullis2024generalized_arxiv_v2}. For the remaining cases,
 the failure stems from the double-exponential time complexity~\cite{collins1974quantifier,z3realarithmetic} in the number of variables \( (x_{u,v,l})_{u,v \mathop{\in} \{1,...,k\}, l \mathop{\in} \Sigma} \) and \( (y_{u,v,l})_{u,v \mathop{\in} \{1,...,k\}, l \mathop{\in} \Sigma} \), which is also influenced by the length of the conditions of Theorem~\ref{nwf:thm:termination_grs} encoded in Z3.
\paragraph{Acceleration with the real arithmetic semiring.} For some examples, the approach with semirings over real numbers can achieve a better runtime performance than the approach with semirings over natural numbers. These are (1) examples which require weighted type graphs with 1 or 2 nodes, and the constraints can be expressed in Z3 with a short formula, or (2) examples which can first be simplified by eliminating some rules using weighted type graphs with 1 node, and then the remaining rules can be 
easily shown to terminate
as in the first case.
 \paragraph{Interest of the approach with semiring over natural numbers.}
For~\cite[Example 4]{plump2018modular},~\cite[Example 5 and 6]{bruggink2015proving} and~\cite[Example 5]{bruggink2014termination}, the semiring over natural numbers is more efficient, as it allows one to restrict the search space to an extremely small size by bounding the maximum weight of edges, which is not possible for the semiring over real numbers. However, this advantage can be achieved only if termination can be proven with a weighted type graph whose weights lie in an extremely small set.
\paragraph{Limitations of our experiments.} 
While our search strategy is very close to the real-world scenario in which a non-expert user would use to prove termination of DPO rewriting systems, 
it is difficult to estimate the runtime performance gain of our approach over the one proposed by Endrullis and Overbeek~\cite{endrullis2024generalized_arxiv_v2}.  
This is because the reported runtimes include (1) translation of constraints into Z3's input format, (2) the time spent in Z3 to solve the constraint system, and (3) the parallelism overhead. 
Another problem is that the examples in our experiments are limited to those whose termination can be proven with a weighted type graph with small edge weights (at most 3). These examples favor the approach with semirings over natural numbers, as the search space is extremely small when the maximum edge weight is bounded by 3.
\paragraph{Comparison with \texttt{GraphTT-wtg}.}
A comparison of our implementation of the approach with semirings over real numbers with the one proposed in \texttt{GraphTT-wtg} by Endrullis and Overbeek~\cite{endrullis2024generalized_arxiv_v3} is desired. However, to the best of our knowledge, at the time of writing, this tool is not publicly available,
and their article does not provide sufficient implementation details to enable a detailed comparison.


% \section{Conclusion}
% \label{sec:type_graph:conclusion}
% The type graph method is a technique for proving termination of DPO rewriting systems. 
To apply this method, one needs to construct a weighted type graph over a semiring that witnesses termination of the rewriting system, and three concrete well-founded semirings over the natural numbers have been proposed in prior work: the tropical semiring, the arctic semiring, and the arithmetic semiring.
However, constructing suitable weighted type graphs over these concrete semirings for DPO rewriting systems on edge-labeled directed multigraphs is difficult. 
To address this challenge, we investigate weighted type graphs over non-well-founded semirings, and propose three corresponding concrete semirings over the real numbers: the tropical semiring, the arctic semiring, and the arithmetic semiring. Construction of weighted type graphs over these concrete semirings is computationally easier.
    
We implemented both our approach and the one proposed by Endrullis and Overbeek~\cite{endrullis2024generalized_arxiv_v2} for DPO rewriting systems on edge-labeled directed multigraphs into a unified tool. Experiments show that for the tropical and arctic semirings, the weighted type graphs over semirings over the real numbers can be constructed more easily, but for the arithmetic semiring, the weighted type graphs over semirings over the real numbers are more difficult to construct than those over the natural numbers.

  
 
% % \begin{credits}
% %     \ifanonymous   
% %     \else
% %       \subsubsection{\ackname} This study was partially funded by the project SAPPORO, ANR, 2019-CE25-0005.  
% %       \fi
% %   \end{credits} 
 
% % \bibliographystyle{eptcs}[2]
% % \bibliographystyle{eptcsalpha}


% % \iflongversion
% % \newpage 
% % \section{Appendix}
% % 
The following proposition ensures that every morphism to a type graph $(T,S,\mathbb{E}, w)$ has a weight different from $0_S$. Additionally, if every T-valued element has a weight greater than $1_S$, then the weight of a morphism to a type graph is greater than $1_S$ too. This lemma will be used in~\autoref{lem_4d13}.
\begin{proposition}  
    \label{prop_endrullis_2d7}
    Let $(S, \mathop{\oplus}, \mathop{\odot}, 0_S, 1_S, \prec, \mu)$ be a strongly monotonic measurable semiring. We have for all $x,y\in S$:
    \begin{align*}
        0_S \mathop{\neq} x \mathop{\land} 0_S \mathop{\neq} y 
        &\mathop{\Rightarrow} 0_S \mathop{\neq} x \mathop{\odot} y 
        \tag{S10} \label{eq:prop_neq0_mul_neq0}  
        \\
        1_S \mathop{\preceq} x \mathop{\land} 1_S \mathop{\preceq} y \mathop{\land} 0_S \mathop{\neq} x \mathop{\land} 0_S \mathop{\neq} y  
        &\mathop{\Rightarrow}
         1_S \mathop{\preceq} x \mathop{\odot} y 
         \tag{S11} \label{eq:prop_neg0_ge1_mul_ge1}  
         \\
         0_S \mathop{\neq} x \mathop{\land} 0_S \mathop{\neq} y   
         &\mathop{\Rightarrow} 0_S \mathop{\neq} x \mathop{\oplus} y
         \tag{S12} \label{eq:prop_neq0_plus_neq0}  
    \end{align*}
\end{proposition}

\begin{proof}
    \label{proof_prop_endrullis_2d7}
    Let $x,y \mathop{\in} S$ such that $x, y \mathop{\neq} 0_S$. By definition, $\prec$ is not empty, therefore there exist $a, b \mathop{\in} S$ such that $a \mathop{\prec} b$.
    \begin{itemize}
        \item $\prec$ is irreflexive, because $\mu: (S, \prec) \mathop{\to} (\overline{\mathbb{R}}, <)$ is a homomorphism.
        \item \ref*{eq:prop_neq0_mul_neq0}:  
        Suppose $(x \mathop{\odot} y)=0_S$. 
        We have 
        \begin{flalign*}
             0_S &= a \mathop{\odot} 0_S & \text{$0_S$ is annihilator for $\mathop{\odot}$}\\
                 &= a \mathop{\odot} (x \mathop{\odot} y) &\text{by assumption on $x \mathop{\odot} y$}\\ 
                 &= a \mathop{\odot} x \mathop{\odot} y &\text{by associativity} \\
                 &\mathop{\prec} b \mathop{\odot} x \mathop{\odot} y &\text{by \eqref{ax:s4} and $x,y\mathop{\neq} 0_S$}\\
                 &= b \mathop{\odot} (x \mathop{\odot} y)  &\text{by associativity}  \\
                 &= b \mathop{\odot} 0_S &\text{by assumption on $x \mathop{\odot} y$} \\
                 &= 0_S
        \end{flalign*}
         which contradicts the irreflexivity of $\prec$. 
        \item \ref*{eq:prop_neg0_ge1_mul_ge1}:
        Suppose
          $1_S \mathop{\preceq} x$ and $1_S \mathop{\preceq} y$. We have either $1_S \mathop{=} x$ or $1_S \mathop{\prec} x$. If $1_S \mathop{=} x$ then 
          \begin{flalign*}
            x \mathop{\odot} y &= 1_S \mathop{\odot} y & \\
                      &= y  & \\
                      & \mathop{\succeq} 1_S &\text{by assumption}
          \end{flalign*}
          If $1_S \mathop{\prec} x$ then $
        %   1_S \mathop{\preceq} y \mathop{=} 
          1_S \mathop{\odot} y \mathop{\prec} x \mathop{\odot} y$ by \eqref{ax:s4} since $y \mathop{\neq} 0$ by assumption.
        \item \ref*{eq:prop_neq0_plus_neq0}:  
        By \eqref{ax:s4}, we have $a \mathop{\odot} x \mathop{\prec} b \mathop{\odot} x$ and $a\mathop{\odot} y \mathop{\prec} b \mathop{\odot} y $. 
        \begin{flalign*}
            a \mathop{\odot} \left(x  \mathop{\oplus} y \right) &= \left( a \mathop{\odot} x \right)  \mathop{\oplus} \big(a \mathop{\odot} y \big)  & \text{by distributivity}\\
            & \mathop{\prec} \left(b \mathop{\odot} x \right)   \mathop{\oplus} \left( b \mathop{\odot} y \right)  & \text{by \eqref{ax:s2}\newline}\\
            & \mathop{=} b \mathop{\odot} \left(x  \mathop{\oplus} y \right) & \text{by distributivity}
        \end{flalign*} 
        if $x  \mathop{\oplus} y \mathop{=} 0_S$ then we have $0_S \mathop{=} a \mathop{\odot} 0_S \mathop{=} a \mathop{\odot} \left(x  \mathop{\oplus} y\right) \mathop{\prec} b \mathop{\odot} \left(x  \mathop{\oplus} y\right) \mathop{=} b \mathop{\odot} 0_S \mathop{=} 0_S$ which contradicts the irreflexivity of $\prec$. 
    \end{itemize}
\qed
\end{proof} 

The following lemma guaranteeing that, under certain constraints: exact weights of host graphs can be computed, and upper bounds for result-graph weights can be derived. 
\begin{lemma}[\cite{endrullis2024generalized_arxiv_v2}]
    \label{lem_4d13}
\ \newline
\begin{minipage}{0.7\textwidth}
    Let $\mathcal{T} \mathop{=} (T,\mathbb{E}, (S, \mathop{\oplus}, \mathop{\odot}, 0_S, 1_S, \prec, \mu), w)$ be a finitary weighted type graph. Consider the pushout square $\delta$ illustrated on the right. We define
\end{minipage}
\begin{minipage}{0.3\textwidth}
    \begin{center}{\normalfont
        \begin{tikzpicture}[node distance=12mm]
            \node (A) {$A$};
            \node (B) [right of=A] {$B$};
            \node (C) [below of=A] {$C$};
            \node (D) [right of=C] {$D$};
            
            \draw [->] (A) to node [above, label] {$\alpha$} (B);
            \draw [->] (A) to node [left, label] {$\beta$} (C);
            \draw [->] (B) to node [right, label] {$\beta'$} (D);
            \draw [->] (C) to node [below, label] {$\alpha'$} (D);
            
            \node [at=($(A)!.5!(D)$)] {$\delta$};
        \end{tikzpicture}
    }\end{center}
\end{minipage}
     \[k \mathop{=} \underset{t_A:A \mathop{\rightarrow} T}{\mathop{\bigoplus}}
            \left ( 
                \underset{\substack{t_C:C \mathop{\rightarrow} T\\
                                            t_A \mathop{=} \beta \mathop{\star} t_C }}{\mathop{\bigoplus}}
                        w_\mathcal{T}(t_C - \beta)     
                 \right ) 
            \mathop{\odot} 
                w_\mathcal{T}(\set{\alpha \mathop{\star} - \mathop{=} t_A})
    \]
    The following conditions hold
    \begin{enumerate}[label=(\Alph*)]
        \item  $w_\mathcal{T}(D)=k$ if $\delta$ is weighable with $\mathcal{T}$.
        \item  $w_\mathcal{T}(D)\mathop{\preceq} k$ if $\delta$ is bounded above by $\mathcal{T}$  and \(w(e) \mathop{\succeq} 1_S\) for all $e \mathop{\in} \mathbb{E}$.
    \end{enumerate}
\end{lemma}

The proof of the following lemma follows the structure of the proof of \cite[Theorem C.3]{endrullis2024generalized_arxiv_v2} to make the comparaison with the original proof easier.

\noindent\begin*{\textbf{\autoref{lem:decreasing_step}}}
\newline
\begin{minipage}{0.7\textwidth}
    Let $\mathcal{T} \mathop{=} (T,\mathbb{E}, (S, \mathop{\oplus}, \mathop{\odot}, 0_S, 1_S, \prec, \mu), w)$ be a finitary weighted type graph, $\rho$ a rewriting rule and $\Delta \mathop{\in} \mathfrak{F}(\rho)$ a DPO diagram
    (shown on the right)   such that the following conditions hold:
\end{minipage}  
\begin{minipage}{0.3\textwidth}
    \begin{center}
        \begin{tikzpicture}[node distance=11mm]
          \node (I) {$K$};
          \node (L) [left of= I] {$L$};
          \node (R) [right of=I] {$R$}; 
          \node (G) [below of=L] {$G$};
          \node (C) [below of=I] {$C$};
          \node (H) [below of=R] {$H$};
        %   \node (T) [left=of $(L)!0.5!(G)$] {$T$};
        %   \draw [->] (L) to  node [label, above] {$c$}  (T);
        %   \draw [->] (G) to  node [label, below] {$\alpha$} (T);
          \draw [->] (I) to node [label, below] {$l$} (L);
          \draw [->] (I) to node [label, below] {$r$} (R);
          \draw [->] (L) to  (G);
          \draw [->] (I) to (C);
          \draw [->] (R) to (H);
          \draw [->] (C) to (G);
          \draw [->] (C) to (H);
        \end{tikzpicture}
      \end{center}
\end{minipage}
   \begin{itemize}
       \item $\operatorname{left}(\Delta)$ is weighable with \(\mathcal{T}\),
       \item $\operatorname{right}(\Delta)$ is bounded above by \(\mathcal{T}\), 
       \item $w(e) \mathop{\succeq} 1_S$ for all $e \mathop{\in} \mathbb{E}$.
   \end{itemize}

   \noindent
  We have:
   \begin{itemize}
       \item $\mu(w_\mathcal{T}(G)) \mathop{\succeq} \mu(w_\mathcal{T}(H))$ if $\rho$ is weakly decreasing,
       \item $\mu(w_\mathcal{T}(G)) \mathop{>} \mu(w_\mathcal{T}(H))\mathop{+}\delta$ if $\rho$ is $\delta$-uniformly or $\delta$-closure decreasing for some $\delta >0$ and $w(e) \mathop{\succeq} 1_S$ for all $e \mathop{\in} \mathbb{E}$.
   \end{itemize}
\end*{}

\begin{proof}
    \label{proof:decreasing_step}
    \noindent For every \( t_K: K \mathop{\rightarrow} T \), we define
$
        S_{t_K} \overset{\operatorname{def}}{=}   
        \underset{\substack{t_C:C \mathop{\rightarrow} T \\
        t_K \mathop{=} h_{KC} \mathop{\star} t_C }}{\mathop{\bigoplus}} 
        w_\mathcal{T}(t_C - h_{KC})  
$.
    
    \noindent For all $t_K: K \mathop{\to} T$ and $X,Y \mathop{\in} S$, the following claims hold:
    \begin{enumerate}[label=(\alph*)] 
        \item \label{s_nz} $S_{t_K} \ne 0_S$ if there is $t_C$ with $ t_K \mathop{=} h_{KC} \mathop{\star} t_C$.  
        \begin{proof}
            By definition of weighted type graph, for all $e \mathop{\in} \mathbb{E}$, we have 
            \begin{flalign}
                w(e) \mathop{\neq} 0_S \label{eq_we_neq_0s1111}
            \end{flalign}
            For every $t_C:C \mathop{\to} T$, we have 
            \begin{flalign*}
                &w_\mathcal{T}(t_C - h_{KC}) \\
               =&\mathop{\bigodot}_{e\in \mathbb{E}} w_e(t_C - h_{KC}) & \text{by Definition~\ref{def:weight_excluding}}\\
               =&\mathop{\bigodot}_{e\in \mathbb{E}} 
                 \mathop{\bigodot}_{\substack{\alpha \mathop{\in} \{- * t_C \mathop{=} e\}\\
                    \alpha \notin \left\{ \iota \mathop{\in} \operatorname{Hom}(X, C)~\middle|~\exists \zeta:X \mathop{\to} K,~\zeta \mathop{\star} h_{KC} \mathop{=} \iota \right\}
                 }
                 } w(e)  & \text{by Definition~\ref{def:weight_excluding_pre}} \\
               \mathop{\neq}&0_S & \text{by \eqref{eq_we_neq_0s1111}, \eqref{eq:prop_neq0_mul_neq0} and Definition~\ref{def:bigodot}}  
            \end{flalign*}

            Therefore, $S_{t_K} \overset{\operatorname{def}}{=}   
            \underset{\substack{t_C:C \mathop{\rightarrow} T \\
            t_K \mathop{=} h_{KC} \mathop{\star} t_C }}{\mathop{\bigoplus}} 
            w_\mathcal{T}(t_C - h_{KC}) \mathop{\neq} 0_S$ if there exists be a morphism such that $t_K \mathop{=} h_{KC} \mathop{\star} t_C$ by \eqref{eq:prop_neq0_plus_neq0}.
            % For every $t_C:C \mathop{\to} T$ such that $t_K \mathop{=} h_{KC} \mathop{\star} t_C$, we have $w_\mathcal{T}(t_C - h_{KC}) \mathop{\neq} 0_S$ by \eqref{eq:prop_neq0_mul_neq0}. 
        \end{proof}
        
        \item \label{s_ge1} $S_{t_K} \mathop{\succeq} 1_S$ if there is $t_C$ with $ t_K \mathop{=} h_{KC} \mathop{\star} t_C$ and $w_\mathcal{T}(e) \mathop{\succeq} 1_S$ for all $e \mathop{\in} \mathbb{E}$,
        \begin{proof}
            By the definition of weighted type graph, for all $e \mathop{\in} \mathbb{E}$, we have $w(e) \mathop{\neq} 0_S$.  
            By assumption, we have $w_\mathcal{T}(e) \mathop{\succeq} 1_S$ for all $e \mathop{\in} \mathbb{E}$. Thus, we have 
            \begin{flalign}
                1_S \mathop{\preceq} w(e) \mathop{\neq} 0_S \label{eq_we_neq_0s_geq1_0}
            \end{flalign}
            By \eqref{eq:prop_neg0_ge1_mul_ge1}, we have
            \begin{flalign}
                1_S \mathop{\preceq} w_\mathcal{T}(t_C - h_{KC}) \label{eq_we_neq_0s_geq1}
            \end{flalign}

            Therefore, $S_{t_K} \overset{\operatorname{def}}{=}   
            \underset{\substack{t_C:C \mathop{\rightarrow} T \\
            t_K \mathop{=} h_{KC} \mathop{\star} t_C }}{\mathop{\bigoplus}} 
            w_\mathcal{T}(t_C - h_{KC}) \mathop{\succeq} 1_S$ if there exists be a morphism such that $t_K \mathop{=} h_{KC} \mathop{\star} t_C$ by \eqref{ax:s1}.
        \end{proof}
        
        % \item \label{claim:le} $Y \mathop{\succeq} X \implies  Y \mathop{\odot} S_{t_K} \mathop{\succeq} X \mathop{\odot} S_{t_K}$
        % \\ by Axiom \eqref{ax:s3}. \todo{to delete: inutile}
         
        \item \label{claim:st} if there exists $t_C$ with $t_K \mathop{=} h_{KC} \mathop{\star} t_C$ then
        $$ \mu(Y) \mathop{>} \mu(X)\mathop{+}\delta  \implies \mu(Y \mathop{\odot} S_{t_K}) \mathop{>} \mu(X \mathop{\odot} S_{t_K})$$
                % $$\left (\exists \alpha \mathop{\geq} \delta.~\mu(Y) \mathop{>} \mu(X)\mathop{+}\alpha \right ) \implies (\mu(Y \mathop{\odot} S_{t_K}) \mathop{>} \mu(X \mathop{\odot} S_{t_K}))$$
        \begin{proof}
           Suppose that there is $t_C$ with $t_K \mathop{=} h_{KC} \mathop{\star} t_C$. We have $S_{t_K} \mathop{\neq} 0_S$ by \ref{s_nz}, and we conclude by \eqref{ax:s4''}.
        \end{proof}
    
        \item \label{claim:sh_{DT}elta} 
        if there exists $t_C$ with $t_K \mathop{=} h_{KC} \mathop{\star} t_C$, and  $w_\mathcal{T}(e) \mathop{\succeq} 1_S$ for all $e \mathop{\in} \mathbb{E}$ then
        % $$\left (\exists \alpha \mathop{\geq} \delta.~\mu(Y) \mathop{>} \mu(X)\mathop{+} \alpha \right ) \implies (\exists \beta \mathop{\geq} \delta. \mu(Y \mathop{\odot} S_{t_K}) \mathop{>} \mu(X \mathop{\odot} S_{t_K}) \mathop{+}\beta)$$
        $$\mu(Y) \mathop{>} \mu(X)\mathop{+} \delta \implies \mu(Y \mathop{\odot} S_{t_K}) \mathop{>} \mu(X \mathop{\odot} S_{t_K}) \mathop{+}\delta $$
        \begin{proof}
            Suppose that there is $t_C$ with $t_K \mathop{=} h_{KC} \mathop{\star} t_C$. We have $1_S \mathop{\preceq} S_{t_K} \mathop{\neq} 0_S$ by \ref{s_nz} and \ref{s_ge1}, and we conclude by \eqref{ax:s4'}. 
        \end{proof}

        \item \label{claim:0} 
        if there is no $t_C$ with $t_K \mathop{=} h_{KC} \mathop{\star} t_C$ then  $S_{t_K} \mathop{=} 0_S$, thus
        $$Y \mathop{\odot} S_{t_K} \mathop{=} 0_S \mathop{=} X \mathop{\odot} S_{t_K} $$
    
        \item \label{claim:exist_st} 
        If there is a context closure $t_L$ for $\rho$ and $T$ in $\mathfrak{F}$ , then, let $t_K \mathop{=} l \mathop{\star} t_L$, we have
        $$ \mu(Y) \mathop{>} \mu(X)\mathop{+}\delta \implies \mu(Y \mathop{\odot} S_{t_K}) \mathop{>} \mu(X \mathop{\odot} S_{t_K})$$
        \begin{proof}
            
       By Definition~\ref{def:context_closure} of context closure, there exists $t_G : G \mathop{\rightarrow} T$ such that 
        \begin{flalign*}
             t_L \mathop{=} h_{LG} \mathop{\star} t_G \tag{1} \label{eq_tl_hlg_tg}
        \end{flalign*}
      i.e. we have the following commutative diagram
     
    \begin{center}
        \begin{tikzpicture}[node distance=11mm]
          \node (I) {$K$};
          \node (L) [left of= I] {$L$};
          \node (R) [right of=I] {$R$};
          \node (G) [below of=L] {$G$};
          \node (C) [below of=I] {$C$};
          \node (H) [below of=R] {$H$};
          \node (T) [left=of $(L)!0.5!(G)$] {$T$};
          \draw [->] (L) to  node [label, above] {$t_L$}  (T);
          \draw [->] (G) to  node [label, below] {$t_G$} (T);
          \draw [->] (I) to node [label, above] {$l$} (L);
          \draw [->] (I) to node [label,above] {$r$} (R);
        %   \draw [->] (L) to node [label, right] {$m$} (G);
        \draw [->] (L) to node [label, right] {} (G);
          \draw [->] (I) to (C);
          \draw [->] (R) to (H);
          \draw [->] (C) to (G);
          \draw [->] (C) to (H);
        \end{tikzpicture}
      \end{center}
    
        Let $t_C \overset{\operatorname{def}}{=} h_{CG} \mathop{\star} t_G$ and $t_K \overset{\operatorname{def}}{=} l \mathop{\star} t_L$. We have\\
        \begin{flalign*}
              t_K  &=  l \mathop{\star} t_L &\text{by definition of $t_K$}
            \\ &=   l \mathop{\star} (h_{LG}  \mathop{\star} t_G) & \text{by~\autoref{eq_tl_hlg_tg}}
            \\ &= (l \mathop{\star} h_{LG}) \mathop{\star} t_G &\text{by associativity }
            \\ &= (h_{KC} \mathop{\star} h_{CG}) \mathop{\star} t_G & \text{by commutative of $\square KLGC$}
            \\ &= h_{KC} \mathop{\star} (h_{CG}  \mathop{\star} t_G) & \text{by associativity}
            \\ & \mathop{=} h_{KC} \mathop{\star} t_C &\text{by definition of $t_C$}
        \end{flalign*}
        and the claim follows from \ref{claim:st}, sinc $t_C$ is a morphism such that $t_K \mathop{=} h_{KC} \mathop{\star} t_C$.
    \end{proof}

        \item \label{claim:exist_sh_{DT}elta} 
        If there is a context closure $t_L$ for $\rho$ and $T$ in $\mathfrak{F}$, and $w_\mathcal{T}(e) \mathop{\succeq} 1_S$ for all $e \mathop{\in} \mathbb{E}$ then, let $t_K \mathop{=} l \mathop{\star} t_L$, we have 
            % $$\left (\exists \alpha \mathop{\geq} \delta. Y \mathop{\succ} X\mathop{+} \alpha \right ) \implies (\exists \beta \mathop{\geq} \delta. Y \mathop{\odot} S_{t_K} \mathop{\succ} X \mathop{\odot} S_{t_K} \mathop{+}\beta)$$
        $$Y \mathop{\succ} X\mathop{+}\delta \implies Y \mathop{\odot} S_{t_K} \mathop{\succ} X \mathop{\odot} S_{t_K} \mathop{+}\delta$$ 
        \begin{proof}
            The proof is analogous to the proof of \ref{claim:exist_st} but with \ref{claim:sh_{DT}elta} instead of \ref{claim:st} in the end.
        \end{proof} 
    \end{enumerate}
    
    \noindent For every \( t_K: K \mathop{\rightarrow} T \), let
    \begin{flalign*}
        \Lambda_{t_K} &\overset{\operatorname{def}}{=}  w_\mathcal{T}(\{l \mathop{\star} - \mathop{=} t_K\})
        \\
        \Omega_{t_K} &\overset{\operatorname{def}}{=}  w_\mathcal{T}(\{r \mathop{\star} - \mathop{=} t_K\})
    \end{flalign*}
  By~\autoref{lem_4d13}, we have 
        \begin{flalign*} 
            w_\mathcal{T}(G) &=
                \underset{\substack{t_K: K \mathop{\rightarrow} T}}{\mathop{\bigoplus}}     \ \
            (S_{t_K} \mathop{\odot} \Lambda_{t_K})
              \\
            w_\mathcal{T}(H) &\mathop{\preceq}
            \underset{\substack{t_K: K \mathop{\rightarrow} T}}{\mathop{\bigoplus}}     \ \
                (S_{t_K} \mathop{\odot} \Omega_{t_K})
        \end{flalign*}

    \noindent We complete the proof with a analysis by cases:
    \begin{enumerate}
        \item  If $\rho$ is weakly decreasing, then, by Definition~\ref{def:decreasing_rule} of weakly decreasing rule, we have $\Lambda_{t_K} \mathop{\geq} \Omega_{t_K}$
        for every $t_K: K \mathop{\rightarrow} T$. 
        By \eqref{ax:s3}, for every  $ t_K : K \mathop{\rightarrow} T$, we have 
                \begin{flalign*} 
                    S_{t_K} \mathop{\odot} \Lambda_{t_K} \mathop{\succeq} S_{t_K} \mathop{\odot} \Omega_{t_K} \tag{NE} \label{steps:weightC:ge} 
                \end{flalign*}
        Thus, we have $w_\mathcal{T}(G) \mathop{\succeq} w_\mathcal{T}(H)$, from \eqref {ax:s1}.
        % \item
        %     If $\rho$ is $\delta$-uniformly decreasing for some $\delta \mathop{\in} \mathbb{R}_{>0}$, then, by Definition~\ref{def:decreasing_rule} of $\delta$-uniformly decreasing rule,
        %     for all $t_K : K \mathop{\to} T$, we have 
        %                     \begin{itemize}                                
        %                         \item $\exists \alpha \mathop{\geq} \delta.~\mu(\Lambda_{t_K}) \mathop{>} \mu(\Omega_{t_K})\mathop{+}\alpha$, or
        %                         \item $\{l \mathop{\star} - \mathop{=} t_K\} \mathop{=} \emptyset \mathop{=} \{r \mathop{\star} - \mathop{=} t_K\}$
        %                     \end{itemize}
        %     From \ref{claim:st} and \ref{claim:0}, for every \( t_K: K \mathop{\rightarrow} T \), we have
        %     \begin{enumerate}[label=(\roman*)]
        %         \item $S_{t_K} \mathop{\odot} \Lambda_{t_K} \mathop{=} 0_S \mathop{=}  S_{t_K} \mathop{\odot} \Omega_{t_K}$, or        
        %         \item  \label{it:strict} $ \mu(\Lambda_{t_K} \mathop{\odot} S_{t_K}) \mathop{>}  \mu(S_{t_K} \mathop{\odot} \Omega_{t_K})$
        %     \end{enumerate}
        %     To establish $ \mu(w_\mathcal{T}(G)) \mathop{>} \mu(w_\mathcal{T}(H))$, using \eqref{ax:s2}, 
        %     it suffices to show that we have case~\ref{it:strict} for some $t_K : K \mathop{\to} T$.
        %     This follows from \ref{claim:exist_st} since we have a context closure for $\rho$ and $\mathcal{T}$, by assumption.
        \item  
            Suppose that $\rho$ is $\delta$-uniformly decreasing for $\delta \mathop{\in} \mathbb{R}_{>0}$, and $w_\mathcal{T}(e) \mathop{\succeq} 1$ for all $e \mathop{\in} \mathbb{E}$. By Definition~\ref{def:decreasing_rule} of $\delta$-uniformly decreasing rule,
            for all $t_K : K \mathop{\to} T$, we have  
                            \begin{itemize}                                
                                \item $\mu(\Lambda_{t_K}) \mathop{>} \mu(\Omega_{t_K})\mathop{+}\delta$,
                                % $\exists \alpha \mathop{\geq} \delta  .~\mu(\Lambda_{t_K}) \mathop{>} \mu(\Omega_{t_K})\mathop{+}\alpha$, 
                                 or
                                \item $\{l \mathop{\star} - \mathop{=} t_K\} \mathop{=} \emptyset \mathop{=} \{r \mathop{\star} - \mathop{=} t_K\}$
                            \end{itemize}
            From \ref{claim:sh_{DT}elta} and \ref{claim:0}, for every \( t_K: K \mathop{\rightarrow} T \), we obtain
            \begin{enumerate}[label=(\roman*)]
                \item $S_{t_K} \mathop{\odot} \Lambda_{t_K} \mathop{=} 0_S \mathop{=}  S_{t_K} \mathop{\odot} \Omega_{t_K}$, or
                \item  \label{it:strich_{DT}elta}  $\mu(\Lambda_{t_K} \mathop{\odot} S_{t_K}) \mathop{>} \mu(\Omega_{t_K} \mathop{\odot} S_{t_K})\mathop{+}\delta$
                % $\exists \beta \mathop{\geq} \delta  .\  \mu(\Lambda_{t_K} \mathop{\odot} S_{t_K}) \mathop{>} \mu(\Omega_{t_K} \mathop{\odot} S_{t_K})\mathop{+}\beta$
            \end{enumerate}
            To establish $ \mu(w_\mathcal{T}(G)) \mathop{>} \mu(w_\mathcal{T}(H))\mathop{+}\delta$, using \eqref{ax:s2'}, 
            it suffices to show that we have case~\ref{it:strich_{DT}elta} for some $t_K : K \mathop{\to} T$.
            This follows from \ref{claim:exist_sh_{DT}elta} since we have a context closure for $\rho$ and $\mathcal{T}$ by assumption.
            % \item
            % If $\rho$ is $\delta$-closure decreasing, 
            % then it is also weakly decreasing and we obtain \eqref{steps:weightC:ge} for every $t_K : K \mathop{\to} T$.
            % Since the semiring is strictly ordered museurable, it suffices to show that there exists some $t_K : K \mathop{\to} T$ such that
            % \begin{align}
            %    \mu( S_{t_K} \mathop{\odot} \Lambda_{t_K})  \mathop{\succ} \mu(S_{t_K} \mathop{\odot} \Omega_{t_K})
            %   \tag{$\star$} \label{steps:weightC:gt}
            % \end{align}
            % in order to conclude $\mu(w_\mathcal{T}(G)) \mathop{>} \mu(w_\mathcal{T}(H))$ by Equation~\eqref{ax:s5}.
            % By Definition~\ref{def:decreasing_rule}, there is a context closure $t_L$ for $\rho$ and $T$, and 
            % % $\exists \delta' \mathop{\geq} \delta  .\  \Lambda_{t_K} \mathop{>} \Omega_{t_K}\mathop{+}\delta'$
            % $\mu(\Lambda_{t_K}) \mathop{>} \mu(\Omega_{t_K})\mathop{+}\delta$
            % for $t_K \mathop{=} l \mathop{\star} t_L$. Thus, we obtain  Equation~\eqref{steps:weightC:gt} by~\ref{claim:exist_st}.
        \item
            If $\rho$ is $\delta$-closure decreasing, and $w_\mathcal{T}(e) \mathop{\succeq} 1$ for all $e \mathop{\in} \mathbb{E}$ then it is also weakly decreasing and we obtain \eqref{steps:weightC:ge} for every $t_K : K \mathop{\to} T$.
            Since the semiring is a strictly monotonic museurable semiring,  by~\autoref{ax:s2'} and \eqref{ax:s5'}, it suffices to show that there exists some $t_K : K \mathop{\to} T$ such that 
            \begin{align}
                % \exists \alpha \mathop{\geq} \delta.~
                \mu(S_{t_K} \mathop{\odot} \Lambda_{t_K}) \mathop{>} \mu(S_{t_K} \mathop{\odot} \Omega_{t_K})\mathop{+}
                \delta
                % \alpha
              \tag{$\star\star$}\label{steps:weightC:gh_{DT}elta}
            \end{align}
            in order to conclude $ \mu(w_\mathcal{T}(G)) \mathop{>} \mu(w_\mathcal{T}(H))\mathop{+}\delta$.
            There is a context closure $t_L$ for $\rho$ and $T$, and
            $\mu(\Lambda_{t_K}) \mathop{>} \mu(\Omega_{t_K})\mathop{+}\delta$
            % $\exists \delta' \mathop{\geq} \delta  .\  \Lambda_{t_K} \mathop{>} \Omega_{t_K}\mathop{+}\delta'$
            for $t_K \mathop{=} l \mathop{\star} t_L$. Thus, we obtain Equation~\eqref{steps:weightC:gh_{DT}elta} by~\ref{claim:exist_sh_{DT}elta}.
    \end{enumerate}
    \end{proof} 
    
    

\noindent\begin*{\textbf{\autoref{thm:termination_grs}}}
Let $\mathcal{A}$ and $\mathcal{B}$ be sets of DPO rewriting rules, $\mathcal{T} \mathop{=} (T,\mathbb{E}, (S, \mathop{\oplus}, \mathop{\odot}, 0_S, 1_S, \prec, \mu), w)$ a finitary weighted type graph and $\mathfrak{F}$ a DPO rewriting framework such that

\begin{enumerate}[label=\roman*)]
   \item\label{thm1:hyp3} $w(e) \mathop{\succeq} 1_S$ for all $e \mathop{\in} \mathbb{E}$,
   % \item\label{thm1:hyp4} $\{s \mathop{\in} S\mid 1_S \leq s \mathop{\neq} 0_S\} \mathop{\subseteq} \mathbb{R}_{>0}$ 
   % \item\label{thm1:hyp4} for all $x \mathop{\in} S$, if $ 1_S \mathop{\preceq} x \mathop{\neq} 0_S$ then $\mu(x) \mathop{\geq} \mu(1_S)$ and $\mu(x) \mathop{\in} \mathbb{R}$,
   \item\label{thm1:hyp4} for all $x \mathop{\in} S$, if $ 1_S \mathop{\preceq} x \mathop{\neq} 0_S$ then $\mu(x) \mathop{\geq} \mu(1_S)$ and $\mu(x) \mathop{\in} \mathbb{R}$,
   \item for every rule $\rho \mathop{\in} (\mathcal{A }\mathop{\cup} \mathcal{B })$ and every double pushout diagram  
   $\Delta \mathop{\in} \mathfrak{F}(\rho)$ 
   \begin{itemize}
       \item \(\operatorname{left}(\Delta)\) is weighable with \(\mathcal{T}\),
       \item \(\operatorname{right}(\Delta)\) is bounded above by \(\mathcal{T}\). 
   \end{itemize}
\end{enumerate}       

\noindent If the following conditions hold:
\begin{enumerate}
   \item there exists $\delta >0$ such that either every $\rho \mathop{\in} \mathcal{A}$ is $\delta$-uniformly, or every $\rho \mathop{\in} \mathcal{A}$ is $\delta$-closure decreasing,
   \item every rule $\rho \mathop{\in} \mathcal{B}$ is weakly decreasing,
\end{enumerate}
then $\mathop{\Rightarrow}_{\mathcal{A},\mathfrak{F}}$ is \textbf{terminating} relative to $\mathop{\Rightarrow}_{\mathcal{B},\mathfrak{F}}$.
\end*{}


\begin{proof} 
    \label{proof_termination_grs}
    From the definition of weighted type graph, we have 
    $$\text{for all}~e\in\mathbb{E}, w(e) \mathop{\neq} 0_S$$ 
    From ssumption \eqref{thm1:hyp3}, we have 
    $$\text{for all}~e\in\mathbb{E},1_S \mathop{\preceq} w(e)$$
    Therefore, we have 
    \begin{flalign}
        \text{for all}~e\in\mathbb{E},1_S \mathop{\preceq} w(e)\mathop{\neq} 0_S \label{thm_eq_we_neq0_geq1}
    \end{flalign} 
    Let $G$ be a graph admitting a match of a DPO rewriting rule. We have 
    \begin{flalign*}
        w_\mathcal{T}(G) &\overset{\operatorname{def}}{=} 
            \underset{h \mathop{\in} \operatorname{Hom}(G,T)}{\mathop{\bigoplus}}  w_\mathcal{T}(h) \\
        & \overset{\operatorname{def}}{=} 
        \underset{h \mathop{\in} \operatorname{Hom}(G,T)}{\mathop{\bigoplus}} 
            \left ( \underset{e \mathop{\in} \mathbb{E}}{\mathop{\bigodot}} 
            \left(  
                \underset{\alpha \mathop{\in} \{- \mathop{\star} h \mathop{=} e\}}{\mathop{\bigodot}}w(e) 
            \right)
            \right )\\
    \end{flalign*} 
    By~\autoref{thm_eq_we_neq0_geq1},~\autoref{prop_endrullis_2d7}, {def:bigodot} and $1_S \mathop{\neq} 0_S$, for every $h \mathop{\in} \operatorname{Hom}(G,T)$, we have
    \begin{flalign}
        1_S \mathop{\preceq} 
        \underset{e \mathop{\in} \mathbb{E}}{\mathop{\bigodot}} 
                \left(  
                    \underset{\alpha \mathop{\in} \{- \mathop{\star} h \mathop{=} e\}}{\mathop{\bigodot}}w(e) 
                \right) 
        \mathop{\neq} 0_S
    \end{flalign}
    Since $G$ be a graph admitting a match of a DPO rewriting rule, by~\autoref{prop_endrullis_2d7} and \eqref{ax:s0}, we have $$1_S \mathop{\preceq} w_\mathcal{T}(G) \mathop{\neq} 0_S$$
    % By~\autoref{prop_endrullis_2d7} and \eqref{ax:s0}, we have 
    % $$\forall G\in\mathcal{C}_0, (\exists H\in \mathcal{C}_0. G \mathop{\Rightarrow}_\mathcal{R} H) \mathop{\rightarrow} (1_S \mathop{\preceq} w_\mathcal{T}(G) \mathop{\neq} 0_S)$$
    By Assumption \eqref{thm1:hyp4}, we have 
    % $$\forall G\in\mathcal{C}_0, (\exists H\in \mathcal{C}_0. G \mathop{\Rightarrow}_\mathcal{R} H) \mathop{\rightarrow} (\mu(w_\mathcal{T}(G)) \mathop{\geq} \mu(1_S))$$
      $$\mu(w_\mathcal{T}(G)) \mathop{\geq} \mu(1_S)$$
 
    % Let $G \mathop{\in} \mathcal{C}_0$ be an object. By Assumption \ref{thm1:hyp4}, we have $\mu(w_\mathcal{T}(G)) \mathop{\in} \mathbb{R}$ and $\mu(w_\mathcal{T}(G)) \mathop{\geq} \mu(1_S)$.

    By~\autoref{lem:decreasing_step}, every rewriting step with rules in $\mathcal{A}$ strictly decreases the weight by at least $\delta$ and no rewriting step with rules in $\mathcal{B}$ increases the weight.
    Consequently, there is no infinite rewriting sequence with an infinite rewriting steps with rules in $\mathcal{A}$ from $G$.
\end{proof}

% \noindent\begin*{\textbf{\autoref{thm:termination_gls}}}
%     Let $\mathfrak{F}$ be a DPO rewriting framework, $\mathcal{A}$ and $\mathcal{B}$ sets of graph relabelling rules and $\mathcal{T} \mathop{=} (T,\mathbb{E}, S, w)$ a finitary weighted type graph such that 
%     \begin{itemize}
%             \item for all $x \mathop{\in} S$, if $x \mathop{\neq} 0_S$ then $\mu(x) \mathop{\in} \mathbb{R}$.
%     \end{itemize}

% \noindent If the following conditions hold
%     \begin{itemize}
%     \item every rule $\rho \mathop{\in} \mathcal{B}$ is weakly decreasing,
%     \item for every $\rho \mathop{\in} \mathcal{A}$, $\rho$ is $0$-uniform or $0$-closure decreasing,
%     \end{itemize}
%     then $\mathcal{A }$ is \textbf{terminating} relative to $\mathcal{B }$.
% \end*{}

% \begin{proof} 
%     By the definition, we have 
%     $$\forall e \mathop{\in} \mathbb{E}. w(e) \mathop{\neq} 0_S$$

%     By~\autoref{prop_endrullis_2d7}, we have 
%         $$\forall G\in\mathcal{C}_0, (\exists H\in \mathcal{C}_0. G \mathop{\Rightarrow}_\mathcal{R} H) \mathop{\rightarrow}  w_\mathcal{T}(G) \mathop{\neq} 0_S$$

%     By assumption, we have 
%      $$\forall G\in\mathcal{C}_0, (\exists H\in \mathcal{C}_0. G \mathop{\Rightarrow}_\mathcal{R} H) \mathop{\rightarrow}  \mu(w_\mathcal{T}(G)) \mathop{\in} \mathcal{R}$$

%     Let $G \mathop{\in} \mathcal{C}_0$ be an object. We have $w_\mathcal{T}(G) \mathop{\in} \mathbb{R}$.

%     By~\autoref{lem:decreasing_step}, every rewriting step with rules in $\mathcal{A}$ strictly decreases the weight by at least $\delta$ and no rewriting step with rules in $\mathcal{R}$ increases the weight.

%     Consequently, there is no infinite rewriting sequence with an infinite rewriting steps with rules in $\mathcal{A}$ from some $G \mathop{\in} \mathcal{C}$, because otherwise there would be an infinite number of accessible objects from $G$ which contradicts the fact that with a given finite unlabeled graph, there are only a finite number of possible labeled graphs for the set of label is finite.
% \end{proof}
    
% % \subsection{Graphs}
% % \label{preliminaries:graphs} 
% % \begin{definition}[Unlabeled graph~\cite{barr1990category}]
    \label{def:graph:unlabeled}
    An \textbf{unlabeled graph} \( G \) consists of a collection of \textbf{nodes} (also called \textbf{objects}) and a collection of \textbf{edges} equipped with a \textbf{source} (or \textbf{domain}) node and a \textbf{target} (or \textbf{codomain}) node. 
    
    For an unlabeled graph \( G \), we denote by \( G_0 \) its collection of nodes, \( G_1 \) its collection of edges, \( \operatorname{dom}:G_1{\to}G_0 \) the domain function, and \( \operatorname{cod}:G_1{\to}G_0 \) the codomain function. An unlabeled graph is \textbf{finite} if \( G_0 \) and \( G_1 \) are both finite sets.
    We write \( a: s \mathop{\to} t \) to indicate that \( a \) is a directed edge from \( s \) to \( t \).
\end{definition}   
A homomorphism of unlabeled graphs is a mapping between the nodes and edges of two graphs that preserves the graph structure.
\begin{definition}[Homomorphism of unlabeled graphs]
    \label{def:unlabeled_graph:homomorphism}
    Let \( G \) and \( H \) be unlabeled graphs. A \textbf{homomorphism of unlabeled graphs} \( h: G \mathop{\to} H \) is a pair of functions \( h_0: G_0 \mathop{\to} H_0 \) and \( h_1: G_1 \mathop{\to} H_1 \) such that for every edge \( a: s \mathop{\to} t \) in \( G \), we have \( h_1(a) : h_0(s) \mathop{\to} h_0(t) \) in \( H \).
\end{definition}
\begin{definition}[Labeled graph \cite{konig2018tutorial}]
    \label{def:graph}
    Let \(\Sigma\) be a finite set of labels. A labeled graph is an ordered pair \((G,\lambda)\) where \( G \) is an unlabeled graph and \( \lambda : G_1 \mathop{\rightarrow} \Sigma\) is an edge-labeling function. 
    It is called \textbf{finite} if its underlying unlabeled graph is finite.  
\end{definition}
By $a : s\overset{l}{\rightarrow} t$, we denote the arrow $a$ labeled by $l$ from $s$ to $t$. Unless otherwise specified, the term \enquote{graph} will refer to finite labeled graphs. Note that unlabeled graphs can be regarded as labeled graphs. A homomorphism of labeled graphs is a homomorphism of unlabeled graphs that preserves the labels assigned to the edges.
\begin{definition}[Homomorphism of labeled graphs~\cite{konig2018tutorial}]
    \label{def:graph:homomorphism}
    Let \( (G,\lambda) \) and \( (H,\lambda') \) be labeled graphs. A \textbf{homomorphism of labeled graphs} \( h:(G,\lambda) \mathop{\rightarrow} (H,\lambda') \) is a homomorphism of unlabeled graphs such that for each edge \( a \) in \( G \), we have \( \lambda \mathop{=} \lambda' \circ h_1 \).
\end{definition}

% % \subsection{Pushout}
% % \label{preliminaries:pushout}
% % \begin{definition}[Category \cite{pierce1991basic, barr1990category}]
    \label{def:cat}
    A \textbf{category} is an unlabeled graph \( C \) together with a total function \( u : C_0 \to C_1 \) and a partial function \( \star: C_1 \times C_1 \to C_1 \) such that 
        (i) for all edges \( f:X \to Y \) and \( g:Y \to Z \), the edge \( f \star g :X \to Z \) is defined; 
        (ii) for every node \( X \), \( u(X) \) is an edge from \( X \) to \( X \);
        (iii) for every \( f:X \to Y \), we have \(u(X) \star f = f = f \star u(Y)\);
        (iv) for all edges \( f \), \( g \) and \(h\), we have \( (f \star g) \star h = f \star (g \star h) \) whenever either side is defined.
    Edges are called \textbf{morphisms}. The function $\star$ is called \textbf{composition}. For all \( X \in C_0 \), the edge \( u(X) \) is denoted \( \operatorname{id}_X \) and is called the \textbf{identity} of the object \( X \).
    % \( C \) is called the \textbf{underlying graph} of the category \( \mathcal{C} \).
\end{definition} 
% % \begin{notation}
    The composition of morphisms \( f : X \mathop{\to} Y \) and \( g : Y \mathop{\to} Z \) is written in diagrammatic order as \( f \mathop{\star} g \), rather than in functional order \( g \circ f \). 
    % The advantage is that, when reading from left to right, the morphisms appear in the same order as in the corresponding diagram, making the notation more intuitive for visual reasoning.
\end{notation}  
% % \begin{definition}[Monomorphism~\cite{pierce1991basic,barr1990category}]
    \label{def:cat:homo}
    A morphism \( f : X \to Y \) is \textbf{monic} (or a \textbf{monomorphism}) if for all morphisms \( g \) and \( h \), if \( g \star f = h \star f \), then \( g = h \). A monomorphism is denoted by \( f : X \rightarrowtail Y \).
\end{definition} 
% % \begin{definition}[Span~\cite{lowe2010graph}]
    An ordered pair \( (\alpha : A \mathop{\to} B,~\beta : A \mathop{\to} C) \) of morphisms with a common domain is called a \textbf{span}, denoted by \( B \overset{\alpha}{\leftarrow} A \overset{\beta}{\rightarrow} C \).
\end{definition}
% % \begin{definition}[Cospan]
    An ordered pair \( (\beta' : B \to D,~\alpha' : C \to D) \) of morphisms with a common codomain is called a \textbf{cospan}, denoted by \( B \overset{\beta'}{\rightarrow} D \overset{\alpha'}{\leftarrow} C \). 
\end{definition} 
% % 
\begin{definition}[Diagram \cite{barr1990category}]
    \label{def:cat:diagram}
    Let \( G \) be an unlabeled graph. A \textbf{diagram} (in \( \mathcal{C} \) of shape \( G \)) is a homomorphism of unlabeled graphs \( h : G \to C \) where \( C \) is the underlying unlabeled graph of the category \( \mathcal{C} \). A diagram is \textbf{commutative} if, for all nodes \( u \), \( v \), and any two paths from \( u \) to \( v \) in the unlabeled graph \( G \):

    \begin{center}
    \resizebox{12cm}{!}{
        \begin{tikzpicture}
        \node (u) at (0,0) {\( u \)};
        \node (k1) at (2,0.5) {\( k_1 \)};
        \node (k2) at (4,0.5) {\( k_2 \)};
        \node (ketc) at (6,0.5) {\( \dots \)};
        \node (knm2) at (8,0.5) {\( k_{n-2} \)};
        \node (knm1) at (10,0.5) {\( k_{n-1} \)};
        \node (v) at (12,0) {\( v \)};
        \node (l1) at (2,-0.5) {\( l_1 \)};
        \node (l2) at (4,-0.5) {\( l_2 \)};
        \node (letc) at (6,-0.5) {\( \dots \)};
        \node (lnm2) at (8,-0.5) {\( l_{m-2} \)};
        \node (lnm1) at (10,-0.5) {\( l_{m-1} \)};
        \draw[->] (u) -- (k1) node [midway,above] {\( s_1 \)};
        \draw[->] (k1) -- (k2) node [midway,above] {\( s_2 \)};
        \draw[->] (k2) -- (ketc);
        \draw[->] (ketc) -- (knm2); 
        \draw[->] (knm2) -- (knm1) node[midway,above] {\( s_{n-2} \)}; 
        \draw[->] (knm1) -- (v) node[midway,above] {\( s_{n-1} \)}; 
        \draw[->] (u) -- (l1) node[midway,below] {\( t_1 \)};
        \draw[->] (l1) -- (l2) node[midway,below] {\( t_2 \)};
        \draw[->] (l2) -- (letc);
        \draw[->] (letc) -- (lnm2); 
        \draw[->] (lnm2) -- (lnm1) node[midway,below] {\( t_{m-2} \)}; 
        \draw[->] (lnm1) -- (v) node[midway,below] {\( t_{m-1} \)}; 
        \end{tikzpicture}
    }
    \end{center}
    \noindent
    the equality \( h(s_1) \star h(s_2) \star \dots  \star h(s_{n-1}) = h(t_1) \star h(t_2) \star \dots  \star h(t_{m-1}) \) holds.
\end{definition}
% % Pushouts play a central role in the double-pushout approach to graph rewriting considered in this thesis. The concepts and notation in this section follow the treatments of Pierce~\cite{pierce1991basic} and Barr and Wells~\cite{barr1990category}.
\begin{definition}
    \label{def:cat}
    A \textbf{category}\index{Category} is an unlabeled graph \( C \) together with a total function \( u : V(C)  \mathop{\to} E(C) \) and a partial function \( \star: E(C) \mathop{\times} E(C)  \mathop{\to} E(C) \) such that 
        \begin{itemize}
            \item for all edges \( f:X  \mathop{\to} Y \) and \( g:Y  \mathop{\to} Z \), the edge \( f \mathop{\star} g :X  \mathop{\to} Z \) is defined; 
            \item  for every node \( X \), \( u(X) \) is an edge from \( X \) to \( X \); 
            \item for every \( f:X  \mathop{\to} Y \), we have \(u(X) \mathop{\star} f \mathop{=} f \mathop{=} f \mathop{\star} u(Y)\);
            \item for all edges \( f \), \( g \) and \(h\), we have \( (f \mathop{\star} g) \mathop{\star} h \mathop{=} f \mathop{\star} (g \mathop{\star} h) \) whenever either side is defined.
        \end{itemize}
    Edges are called \textbf{morphisms}\index{Morphism}. The function $\star$ is called \textbf{composition}\index{Composition}. For all \( X \mathop{\in} V(C) \), the edge \( u(X) \) is denoted by \( \operatorname{id}_X \) and is called the \textbf{identity}~\index{Indentity} of the object \( X \).
    % \( C \) is called the \textbf{underlying graph} of the category \( \mathcal{C} \).
\end{definition}    
\begin{definition}
    A category \(\mathcal{C}\) is said to be \textbf{locally small}\index{Category!locally small} if for all objects \(X,Y\) in \(\mathcal{C}\), the collection $\opn{Hom}(X,Y)$\index{hom(@$\opn{Hom}(X,Y)$} of morphisms from \(X\) to \(Y\) is a set (called a \textbf{hom-set})~\index{Hom-set}. For a locally small category, $\opn{Mono}(X,Y)$\index{mono(@$\opn{Mono}(X,Y)$} denotes the set of all monomorphisms from $X$ to $Y$.
\end{definition}
\textbf{Throughout this section, fix a locally small category \( \mathcal{C} \).}
\begin{example}
    Consider the unlabeled graph shown below.
    %  in Figure~\ref{fig:preliminaries:category}. 
     It can be considered as a category where the objects are the nodes and the morphisms are the paths between nodes; composition is path concatenation. The identity of a node is the self-loop of the node. There are at least three morphisms from the left node to itself: the identity morphism (the self-loop), the path that traverses the self-loop twice, and the path that goes to the right node and back. 
        % \begin{figure}[H]
        % \centering
        \begin{center}
            \resizebox{0.4\textwidth}{!}{
        \begin{tikzpicture}
            \graphbox{}{0mm}{0mm}{32mm}{16mm}{-10mm}{-9mm}{
                \node[draw,circle] (1) at (0,0) {};
                \node[draw,circle] (2) at (2,0) {};
                \draw[->] (1) edge[loop above] node[midway, above] { } (1) ;
                \draw[->] (2) edge[loop above] node[midway, above] { } (2) ;
                \draw[->] (1) edge[bend left] node[midway, above] {}  (2)  ;
                \draw[->] (2) edge[bend left] node[midway, below] {} (1)   ;
            }
        \end{tikzpicture}
            }
    %     \caption{}
    %     \label{fig:preliminaries:category}
    % \end{figure}
        \end{center}
\end{example}

% cat notation * 
\begin{notation}
    The composition of morphisms \( f : X  \mathop{\to} Y \) and \( g : Y  \mathop{\to} Z \) is written in diagrammatic order as \( f \mathop{\star} g \), rather than in functional order \( g \circ f \) as is common in litterature. The advantage is that, when reading from left to right, the morphisms appear in the same order as in the corresponding diagram, making the reasoning accompanying diagrams more intuitive. 
\end{notation}  

\begin{definition} 
    \label{def:cat:homo}
    A morphism \( f : X  \mathop{\to} Y \) is said to be a \textbf{monomorphism}\index{Monomorphism} (is \textbf{monic}\index{Monic}) if given any morphisms \( g,h: Z  \mathop{\to} X  \), \( g \mathop{\star} f \mathop{=} h \mathop{\star} f \) implies \( g \mathop{=} h \). 
    In this case, we write $f : X \rightarrowtail Y$\index{a@$\rightarrowtail$} to indicate that $f$ is a monomorphism.
\end{definition} 

When visualizing a monomorphism, we often use $\rightarrowtail$ instead of $\to$ to emphasize that it is monic. For example, a monomorphism of labeled graphs can be represented as follows:
\begin{center}
        \resizebox{0.7\textwidth}{!}{
        \begin{tikzpicture}
            \graphbox{$K$}{40mm}{0mm}{24mm}{15mm}{2mm}{-5mm}{
                \coordinate (o) at (5mm,-3mm); 
                \node[draw,circle] (l1) at ($(o)+(-10mm,0mm)$) {1};
                \node[draw,circle] (l2) at ($(l1)+(1,0)$) {2};
            }    
            \graphbox{$R$}{70mm}{0mm}{45mm}{15mm}{2mm}{-5mm}{
                \coordinate (o) at (-5mm,-3mm); 
                \node[draw,circle] (l1) at ($(o)+(-10mm,0mm)$) {1};
                \node[draw,circle] (l2) at ($(l1)+(3,0)$) {2};
                \node[draw,circle] (l3) at ($(l1)+(1,0)$) {4};
                \node[draw,circle] (l4) at ($(l1)+(2,0)$) {5};
                \draw[->] (l1) -- (l3) node[midway,above] {$a$};
                \draw[->] (l3) -- (l4) node[midway,above] {$b$};
                \draw[->] (l4) -- (l2) node[midway,above] {$a$};
            }    
            \node () at (67mm,-8mm) {$\rightarrowtail$};
        \end{tikzpicture}
        }
    \end{center}

\begin{example}
    \index{set@\(\mathbf{Set}\)}\index{Category!set}
The category \(\mathbf{Set}\) has sets as objects and total functions between them as morphisms. For \(f\mathop{\colon} A \mathop{\to} B\) and \(g\mathop{\colon} B \mathop{\to} C\), composition is given by \(g\circ f\), and the identity morphism on a set \(A\) is the identity function \(\mathrm{id}_A\).
\end{example}

\begin{example} 
    Finite labeled graphs and their homomorphisms form a category, hereafter denoted by \textbf{Graph}\index{graph@\textbf{Graph}}\index{Category!graph}. Its objects are labeled graphs, its morphisms are graph homomorphisms, and the monomorphisms are homomorphisms. 
    $\textbf{Graph}$ is locally small. 
\end{example}

% \begin{definition}[Span \cite{lowe2010graph}]
%     A pair \( (\alpha : A  \mathop{\to} B,~\beta : A  \mathop{\to} C) \) of morphisms with a common domain is called a \textbf{span}, denoted by \( B \overset{\alpha}{\leftarrow} A \overset{\beta}{\rightarrow} C \).
% \end{definition}
 
% \begin{definition}[Cospan]
%     A pair \( (\beta' : B  \mathop{\to} D,~\alpha' : C  \mathop{\to} D) \) of morphisms with a common codomain is called a \textbf{cospan}, denoted by \( B \overset{\beta'}{\rightarrow} D \overset{\alpha'}{\leftarrow} C \). 
% \end{definition} 
A span (resp. cospan) is a couple of morphisms with a common domain (resp. codomain).
\begin{definition}
An ordered pair \((\alpha : A  \mathop{\to} B,\, \beta : A  \mathop{\to} C)\) of morphisms with a common domain is called a \textbf{span}\index{Span} \cite{lowe2010graph}, denoted by
\(
B \overset{\alpha}{\leftarrow} A \overset{\beta}{\rightarrow} C
\). 
% An example of a span $(\alpha, \beta)$ is shown below.
Likewise, an ordered pair \((\beta' : B  \mathop{\to} D,\, \alpha' : C  \mathop{\to} D)\) of morphisms with a common codomain is called a \textbf{cospan}\index{Cospan}, denoted by
\(
B \overset{\beta'}{\rightarrow} D \overset{\alpha'}{\leftarrow} C
\). 
\end{definition}
\begin{example}
Consider the diagram below in the category \textbf{Graph}, where the numbers inside nodes and the subgraphs in different colors illustrate how the morphisms map nodes and edges. $(\alpha, \beta)$ is a span, and $(\beta', \alpha')$ is a cospan.


\begin{center}
        \resizebox{0.8\textwidth}{!}{
        \begin{tikzpicture} 
            \graphbox{\( L \)}{40mm}{20mm}{34mm}{12mm}{2mm}{2mm}{
                \coordinate (o) at (0mm,-8mm); 
                \node[draw,circle] (l1) at ($(o)+(-10mm,0mm)$) {1};
                \node[draw,circle] (l2) at ($(l1)+(2,0)$) {2};
                \node[draw,circle,red] (l3) at ($(l1)+(1,0)$) {3};
                \draw[->,red] (l1) -- (l3) node[midway,above] {$a$};
                \draw[->,red] (l3) -- (l2) node[midway,above] {$a$};
            } 
    
            \graphbox{\( K \)}{0mm}{0mm}{34mm}{12mm}{2mm}{2mm}{
                \coordinate (o) at (0mm,-8mm); 
                \node[draw,circle] (l1) at ($(o)+(-10mm,0mm)$) {1};
                \node[draw,circle] (l2) at ($(l1)+(2,0)$) {2};
            }  
            \graphbox{\(G\)}{90mm}{5mm}{34mm}{24mm}{2mm}{-3mm}{
                \coordinate (o) at (0mm,-5mm); 
                \node[draw,circle] (l1) at ($(o)+(-10mm,0mm)$) {1};
                \node[draw,circle] (l2) at ($(l1)+(2,0)$) {2};
                \node[draw,circle,red] (l3) at ($(l1)+(1,0)$) {3};
                \node[draw,circle,blue] (l4) at ($(l2)+(0,-1)$) {6};
                \draw[->,red] (l1) -- (l3) node[midway,above] {$a$};
                \draw[->,red] (l3) -- (l2) node[midway,above] {$a$};
                \draw[->,blue] (l2) -- (l4) node[midway,right] {$a$};
                \node[draw,circle,blue] (l6) at ($(l1)+(0,-1)$) {7};
                \draw[<-,blue] (l1) -- (l6) node[midway,left] {$a$};
                \draw[->,blue] (l2) edge[out=-135,in=-45]node[midway,below] {$a$} (l1) ;
            }   
     
            \graphbox{\( C \)}{40mm}{-20mm}{34mm}{24mm}{2mm}{-3mm}{
                \coordinate (o) at (0mm,-5mm); 
                \node[draw,circle] (l1) at ($(o)+(-10mm,0mm)$) {1};
                \node[draw,circle] (l2) at ($(l1)+(2,0)$) {2};
                \node[draw,circle,blue] (l4) at ($(l2)+(0,-1)$) {6};
                \draw[->,blue] (l2) -- (l4) node[midway,right] {$a$};
                \draw[->,blue] (l2) edge[out=-135,in=-45]node[midway,below] {$a$} (l1) ;
                \node[ draw,circle,blue] (l6) at ($(l1)+(0,-1)$) {7};
                \draw[<-,blue] (l1) -- (l6) node[midway,left] {$a$};
            }      
            % K to L
            \draw[->] (17mm,5mm) -- node[above] {$\alpha$} (37mm,15mm);
            % C to G
            \draw[->] (76mm,-28mm)-- node[below] {$\alpha'$} (104mm,-21mm) ;
            % K to C
            \draw[->] (17mm,-17mm) -- node[below] {$\beta$} (37mm,-28mm);
            % L to G
            \draw[->] (76mm,16mm) -- node[above] {$\beta'$} (104mm,7mm);
            % \node () at (57mm,-6mm) {$PO$};
        \end{tikzpicture}
        }
    \end{center}
\end{example}

\begin{definition}[\cite{barr1990category}]
    \label{def:cat:diagram}
    Let $\mathcal{C}$ be a category, and \( G \) an unlabeled graph. A \textbf{diagram}\index{Diagram} (of shape \( G \)) is a homomorphism of unlabeled graphs \( h : G  \mathop{\to} \mathcal{C} \) where \( \mathcal{C} \) is considered as an unlabeled graph. A diagram is said to be \textbf{commutative}\index{Diagram!commutative} if, for all nodes \( u \), \( v \), and any two paths from \( u \) to \( v \) in the unlabeled graph \( G \):

    \begin{center}
    \resizebox{12cm}{!}{
        \begin{tikzpicture}
        \node (u) at (0,0) {\( u \)};
        \node (k1) at (2,0.5) {\( k_1 \)};
        \node (k2) at (4,0.5) {\( k_2 \)};
        \node (ketc) at (6,0.5) {\( \dots \)};
        \node (knm2) at (8,0.5) {\( k_{n-2} \)};
        \node (knm1) at (10,0.5) {\( k_{n-1} \)};
        \node (v) at (12,0) {\( v \)};
        \node (l1) at (2,-0.5) {\( l_1 \)};
        \node (l2) at (4,-0.5) {\( l_2 \)};
        \node (letc) at (6,-0.5) {\( \dots \)};
        \node (lnm2) at (8,-0.5) {\( l_{m-2} \)};
        \node (lnm1) at (10,-0.5) {\( l_{m-1} \)};
        \draw[->] (u) -- (k1) node [midway,above] {\( s_1 \)};
        \draw[->] (k1) -- (k2) node [midway,above] {\( s_2 \)};
        \draw[->] (k2) -- (ketc);
        \draw[->] (ketc) -- (knm2); 
        \draw[->] (knm2) -- (knm1) node[midway,above] {\( s_{n-2} \)}; 
        \draw[->] (knm1) -- (v) node[midway,above] {\( s_{n-1} \)}; 
        \draw[->] (u) -- (l1) node[midway,below] {\( t_1 \)};
        \draw[->] (l1) -- (l2) node[midway,below] {\( t_2 \)};
        \draw[->] (l2) -- (letc);
        \draw[->] (letc) -- (lnm2); 
        \draw[->] (lnm2) -- (lnm1) node[midway,below] {\( t_{m-2} \)}; 
        \draw[->] (lnm1) -- (v) node[midway,below] {\( t_{m-1} \)}; 
        \end{tikzpicture}
    }
    \end{center}
    \noindent
    the equality \( h(s_1) \mathop{\star} h(s_2) \mathop{\star} \dots  \mathop{\star} h(s_{n-1}) \mathop{=} h(t_1) \mathop{\star} h(t_2) \mathop{\star} \dots  \mathop{\star} h(t_{m-1}) \) holds.
\end{definition}

\begin{example}
    A commutative diagram in the category \textbf{Graph} of finite, directed, edge-labeled multigraphs is illustrated below. The numbers inside nodes and the subgraphs in different colors illustrate how the morphisms map nodes and edges. 
     The symbol $\mathop{=}$ in the center of the diagram
    is used to indicate that the diagram is commutative,
    i.e. the composition of morphisms along every path from node 1 to node 2 is the same.
    \begin{center}
        \resizebox{0.8\textwidth}{!}{
        \begin{tikzpicture} 
            \graphbox{\( L \)}{40mm}{20mm}{34mm}{12mm}{2mm}{2mm}{
                \coordinate (o) at (0mm,-8mm); 
                \node[draw,circle] (l1) at ($(o)+(-10mm,0mm)$) {1};
                \node[draw,circle] (l2) at ($(l1)+(2,0)$) {2};
                \node[draw,circle,red] (l3) at ($(l1)+(1,0)$) {3};
                \draw[->,red] (l1) -- (l3) node[midway,above] {$a$};
                \draw[->,red] (l3) -- (l2) node[midway,above] {$a$};
            } 
    
            \graphbox{\( K \)}{0mm}{0mm}{34mm}{12mm}{2mm}{2mm}{
                \coordinate (o) at (0mm,-8mm); 
                \node[draw,circle] (l1) at ($(o)+(-10mm,0mm)$) {1};
                \node[draw,circle] (l2) at ($(l1)+(2,0)$) {2};
            }  
            \graphbox{\(G\)}{90mm}{5mm}{34mm}{24mm}{2mm}{-3mm}{
                \coordinate (o) at (0mm,-5mm); 
                \node[draw,circle] (l1) at ($(o)+(-10mm,0mm)$) {1};
                \node[draw,circle] (l2) at ($(l1)+(2,0)$) {2};
                \node[draw,circle,red] (l3) at ($(l1)+(1,0)$) {3};
                \node[draw,circle,blue] (l4) at ($(l2)+(0,-1)$) {6};
                \draw[->,red] (l1) -- (l3) node[midway,above] {$a$};
                \draw[->,red] (l3) -- (l2) node[midway,above] {$a$};
                \draw[->,blue] (l2) -- (l4) node[midway,right] {$a$};
                \node[draw,circle,blue] (l6) at ($(l1)+(0,-1)$) {7};
                \draw[<-,blue] (l1) -- (l6) node[midway,left] {$a$};
                \draw[->,blue] (l2) edge[out=-135,in=-45]node[midway,below] {$a$} (l1) ;
            }   
     
            \graphbox{\( C \)}{40mm}{-20mm}{34mm}{24mm}{2mm}{-3mm}{
                \coordinate (o) at (0mm,-5mm); 
                \node[draw,circle] (l1) at ($(o)+(-10mm,0mm)$) {1};
                \node[draw,circle] (l2) at ($(l1)+(2,0)$) {2};
                \node[draw,circle,blue] (l4) at ($(l2)+(0,-1)$) {6};
                \draw[->,blue] (l2) -- (l4) node[midway,right] {$a$};
                \draw[->,blue] (l2) edge[out=-135,in=-45]node[midway,below] {$a$} (l1) ;
                \node[ draw,circle,blue] (l6) at ($(l1)+(0,-1)$) {7};
                \draw[<-,blue] (l1) -- (l6) node[midway,left] {$a$};
            }      
            % K to L
            \draw[->] (17mm,5mm) -- node[above] {$\alpha$} (37mm,15mm);
            % C to G
            \draw[->] (76mm,-28mm)-- node[below] {$\alpha'$} (104mm,-21mm) ;
            % K to C
            \draw[->] (17mm,-17mm) -- node[below] {$\beta$} (37mm,-28mm);
            % L to G
            \draw[->] (76mm,16mm) -- node[above] {$\beta'$} (104mm,7mm);
            \node () at (57mm,-6mm) {$\mathop{=}$};
        \end{tikzpicture}
        }
    \end{center}
\end{example}

\begin{notation}   
    When the context makes it clear, \( h_{AB} \) denotes a morphism \( h : A  \mathop{\to} B \), and we refer to diagrams by listing their nodes, as is standard in geometry. 
    % For example, the diagram shown in Definition~\ref{def:cat:po} is denoted by \( ACDB \) or \( ABDC \).
\end{notation}   

The pushout is a construction in category theory that can often be thought of as the construction of a new structure from two given structures by gluing them along a common interface structure.
\begin{definition}
    \label{def:cat:po} 
    A \textbf{pushout}\index{Pushout} of a span \( B \overset{\alpha}{\leftarrow} A \overset{\beta}{\rightarrow} C \), shown in the following diagram,
    %  in Figure~\ref{fig:preliminaries:pushout_sdfkjasdlgjfl}
    is defined as a cospan \( B \overset{\beta'}{\rightarrow} D \overset{\alpha'}{\leftarrow} C \) such that the following conditions hold:
    \begin{itemize}
        \item \( \alpha \mathop{\star} \beta' \mathop{=} \beta \mathop{\star} \alpha' \),
        \item for every cospan \( B \overset{\gamma'}{\rightarrow} E \overset{\gamma}{\leftarrow} C \), if \( \alpha \mathop{\star} \gamma' \mathop{=} \beta \mathop{\star} \gamma \) holds, then there is a unique morphism \(\delta : D  \mathop{\to} E\) such that \( \gamma' \mathop{=} \beta' \mathop{\star} \delta \) and \( \gamma \mathop{=} \alpha' \mathop{\star} \delta \).
    \end{itemize} 
    \begin{center}
        \resizebox{0.45\textwidth}{!}{
            \begin{tikzpicture}
                    \node (i) at (0,0) {A};
                    \node (r) at (1,1) {B};
                    \node (c) at (1,-1) {C};
                    \node (h) at (2,0) {D};
                    % \node () at (1,-1) {\( \Delta \)};
                    \draw[->]  (i) -- (r) node [midway,left] {$ \alpha $};
                    \draw[->] (c) -- (h) node [midway,left] {$ \alpha' $};
                    \draw[->] (r) -- (h) node[midway, left] {$ \beta' $};
                    \draw[->] (i) -- (c) node[midway, left] {$ \beta $};
                    \node (d') at (4,0) {E};
                    \draw[->] (c) -- (d') node [midway,below]{$ \gamma $};
                    \draw[->] (r) -- (d') node [midway,above]{$ \gamma' $};
                    \draw[->,dashed] (h) -- (d') node [midway]{$ \delta $};
                \end{tikzpicture}
        }
            \end{center}
The diagram involving \( (\alpha, \beta, \alpha', \beta') \) is called a \textbf{pushout square}\index{Pushout!square}, or simply a \textbf{pushout}, with \(D\) as the \textbf{pushout object}\index{Pushout!object}. The existence of a unique morphism is known as the \textbf{universal mapping property of the pushout}\index{Pushout!universal mapping property}.
\end{definition} 

\begin{example}
    \label{ex:cat:posfjsdlkgja}
     Pushouts of a span always exist in \(\mathbf{Set}\), and (up to isomorphism) can be described as follows. Let
    \( B \overset{\alpha}{\leftarrow} A \overset{\beta}{\rightarrow} C \) be a span. Its pushout is the cospan \( B \overset{\beta'}{\rightarrow} D \overset{\alpha'}{\leftarrow} C \) where the pushout object of \((\alpha,\beta)\) is the quotient set
    \[
    D \;=\; (B\mathop{+}C)/{\sim}
    \]
    where \(B\mathop{+}C\) denotes the disjoint union of $B$ and $C$ and \(\sim\) is the smallest equivalence relation that includes \(\set{(\alpha(a),\beta(b))\mid a \mathop{\in} A }\). The maps
    \(\beta' \mathop{\colon} B \mathop{\to} D\) and \(\alpha' \mathop{\colon} C \mathop{\to} D\) send each element to its equivalence class.

    For example, consider the functions \(\alpha\) and \(\beta\) in the category \(\mathbf{Set}\) in the diagram, illustrated in Figure~\ref{fig:preliminaries:a_rewriting_step_dfjalsdkjflg}.
    In this diagram, each set is drawn as a box and its elements are represented by circles. The numbers inside circles indicate how the functions map those elements.
    \begin{figure}[H]
      \centering 
      \resizebox{0.6\textwidth}{!}{
      \begin{tikzpicture}
          \graphbox{\( A\)}{40mm}{-3mm}{34mm}{12mm}{2mm}{2mm}{
              \coordinate (o) at (0mm,-8mm); 
              \node[draw,circle] (l1) at ($(o)+(-10mm,0mm)$) {1};
              \node[draw,circle] (l2) at ($(l1)+(2,0)$) {2};
          }  
          \graphbox{\( B \)}{80mm}{-3mm}{45mm}{12mm}{2mm}{2mm}{
              \coordinate (o) at (-5mm,-8mm); 
              \node[draw,circle] (l1) at ($(o)+(-10mm,0mm)$) {1};
              \node[draw,circle] (l2) at ($(l1)+(3,0)$) {2};
              \node[draw,circle] (l3) at ($(l1)+(1,0)$) {4};
          }     
          \graphbox{\( C  \)}{40mm}{-22mm}{34mm}{22mm}{2mm}{-3mm}{
              \coordinate (o) at (0mm,-3mm); 
              \node[draw,circle] (l1) at ($(o)+(-10mm,0mm)$) {1};
              \node[draw,circle] (l2) at ($(l1)+(2,0)$) {2};
              \node[ draw,circle] (l6) at ($(l1)+(0,-1)$) {3};
          }    
          \graphbox{\( D \)}{80mm}{-22mm}{45mm}{22mm}{2mm}{-3mm}{
              \coordinate (o) at (-5mm,-3mm); 
              \node[draw,circle] (l1) at ($(o)+(-10mm,0mm)$) {1};
              \node[draw,circle] (l2) at ($(l1)+(3,0)$) {2};
              \node[draw,circle] (l3) at ($(l1)+(1,0)$) {4};
              \node[ draw,circle] (l6) at ($(l1)+(0,-1)$) {3};
          }    
          \node () at (77mm,-8mm) {\( \overset{\alpha}{\rightarrow} \)}; % K -> R
          \node () at (58mm,-18mm) {\( \beta\downarrow \)};
          \node () at (102mm,-18mm) {\( \beta'\downarrow \)};
          \node () at (77mm,-33mm) {\( \overset{\alpha'}{\rightarrow} \)}; % C -> H
      \end{tikzpicture}
      }
      \caption{}
      \label{fig:preliminaries:a_rewriting_step_dfjalsdkjflg}
  \end{figure}
    Throughout this example, an element labeled \(n\) in a set \(X\) is denoted by \(n_X\) to avoid ambiguity.        
    The binary relation \(\sim\) is the reflexive, symmetric and transitive closure of the binary relation $\{(1_B,1_C),(2_B,2_C)\}$.
 
    The disjoint union of \(B\) and \(C\) is
    \[ 
    D' \mathop{=} \{1_B,2_B,4_B,1_C,2_C,3_C\},
    \]
    and the quotient set is
    \[
    D'/\sim \mathop{=} \{[1_B],[2_B],[3_C],[4_B]\},
    \]
    where $[x]$ denotes the equivalence class of the element \(x\).
    We have \([1_C]=[1_B]\) and \([2_C]=[2_B]\), and the maps
    \(\beta'' \mathop{\colon} B \mathop{\to} D'\) and \(\alpha'' \mathop{\colon} C \mathop{\to} D'\) send each element to its equivalence class. Note that \(\{[1_B],[2_B],[3_C],[4_B]\}\) is isomorphic to \(D\), shown in Figure~\ref{fig:preliminaries:a_rewriting_step_dfjalsdkjflg}, which is expected because the pushout of a span is unique up to isomorphism.
\end{example}

\begin{proposition}{\cite[p.188]{corradini1997algebraic}}
    \label{prop:pushout_graph_always_exists}
    In category \textbf{Graph}, the pushout of two arrows always exists: It can be computed componentwise (as a pushout in \textbf{Set}) for the nodes and for the edges, and the source, target, and labeling mappings are uniquely determined.
\end{proposition}

\begin{example}
    The diagram in the category \textbf{Graph} shown below
    %  in Figure~\ref{fig:preliminaries:pushout_injective} 
     is a pushout square. The numbers inside nodes and the subgraphs in different colors illustrate how the morphisms map nodes and edges. In this example, both $\alpha$ and $\beta$ are injective morphisms. Therefore, the pushout object $G$ can be constructed easily by taking the interface graph $K$ and adding elements from $L$ and $C$ which are not present in $K$.
    % \begin{figure}[H]
    %     \centering
    \begin{center}
        \resizebox{0.8\textwidth}{!}{
        \begin{tikzpicture} 
            \graphbox{\( L \)}{40mm}{15mm}{34mm}{12mm}{2mm}{2mm}{
                \coordinate (o) at (0mm,-8mm); 
                \node[draw,circle] (l1) at ($(o)+(-10mm,0mm)$) {1};
                \node[draw,circle] (l2) at ($(l1)+(2,0)$) {2};
                \node[draw,circle,red] (l3) at ($(l1)+(1,0)$) {3};
                \draw[->,red] (l1) -- (l3) node[midway,above] {$a$};
                \draw[->,red] (l3) -- (l2) node[midway,above] {$a$};
            } 
    
            \graphbox{\( K \)}{0mm}{0mm}{34mm}{12mm}{2mm}{2mm}{
                \coordinate (o) at (0mm,-8mm); 
                \node[draw,circle] (l1) at ($(o)+(-10mm,0mm)$) {1};
                \node[draw,circle] (l2) at ($(l1)+(2,0)$) {2};
            }  
            \graphbox{\(G  \)}{90mm}{5mm}{34mm}{24mm}{2mm}{-3mm}{
                \coordinate (o) at (0mm,-5mm); 
                \node[draw,circle] (l1) at ($(o)+(-10mm,0mm)$) {1};
                \node[draw,circle] (l2) at ($(l1)+(2,0)$) {2};
                \node[draw,circle,red] (l3) at ($(l1)+(1,0)$) {3};
                \node[draw,circle,blue] (l4) at ($(l2)+(0,-1)$) {6};
                \draw[->,red] (l1) -- (l3) node[midway,above] {$a$};
                \draw[->,red] (l3) -- (l2) node[midway,above] {$a$};
                \draw[->,blue] (l2) -- (l4) node[midway,right] {$a$};
                \node[draw,circle,blue] (l6) at ($(l1)+(0,-1)$) {7};
                \draw[<-,blue] (l1) -- (l6) node[midway,left] {$a$};
                \draw[->,blue] (l2) edge[out=-135,in=-45]node[midway,below] {$a$} (l1) ;
            }   
     
            \graphbox{\( C \)}{40mm}{-15mm}{34mm}{24mm}{2mm}{-3mm}{
                \coordinate (o) at (0mm,-5mm); 
                \node[draw,circle] (l1) at ($(o)+(-10mm,0mm)$) {1};
                \node[draw,circle] (l2) at ($(l1)+(2,0)$) {2};
                \node[draw,circle,blue] (l4) at ($(l2)+(0,-1)$) {6};
                \draw[->,blue] (l2) -- (l4) node[midway,right] {$a$};
                \draw[->,blue] (l2) edge[out=-135,in=-45]node[midway,below] {$a$} (l1) ;
                \node[ draw,circle,blue] (l6) at ($(l1)+(0,-1)$) {7};
                \draw[<-,blue] (l1) -- (l6) node[midway,left] {$a$};
            }      
            % K to L
            \draw[>->] (17mm,5mm) -- node[above] {$\alpha$} (37mm,10mm);
            % C to G
            \draw[>->] (76mm,-28mm)-- node[below] {$\alpha'$} (104mm,-21mm) ;
            % K to C
            \draw[>->] (17mm,-17mm) -- node[below] {$\beta$} (37mm,-28mm);
            % L to G
            \draw[>->] (76mm,10mm) -- node[above] {$\beta'$} (104mm,7mm);
            \node () at (57mm,-6mm) {$PO$};
        \end{tikzpicture}
        }
    \end{center}
    %     \caption{Pushout square with injective morphisms.}
    %     \label{fig:preliminaries:pushout_injective}
    % \end{figure}
\end{example}

\begin{example}
    \label{ex:cat:pushout_non_injective_ssss}
    Consider the diagram in the category \textbf{Graph} shown below, where the numbers inside nodes and the subgraphs in different colors illustrate how the morphisms map nodes and edges. 
    % The diagram 
    % in Figure~\ref{fig:preliminaries:pushout_non_injective} is a pushout square in the category \textbf{Graph}. 
    In this example, $\beta$ is not injective. Therefore, some elements are merged in the pushout object $G$.

    % \begin{figure}[H]
    %     \centering 
    \begin{center}
        \resizebox{0.8\textwidth}{!}{
        \begin{tikzpicture} 
            \graphbox{\( L \)}{40mm}{20mm}{34mm}{20mm}{2mm}{2mm}{
                \coordinate (o) at (0mm,-11mm); 
                \node[draw,circle] (l1) at ($(o)+(-10mm,0mm)$) {1};
                \node[draw,circle] (l2) at ($(l1)+(2,0)$) {2};
                \draw[->,red] (l2) edge[out=-135,in=-45]node[midway,below] {$a$} (l1) ;
                \node[draw,circle,red] (l3) at ($(l1)+(1,0)$) {3};
                \draw[->,red] (l1) -- (l3) node[midway,above] {$a$};
                \draw[->,red] (l3) -- (l2) node[midway,above] {$a$};
            } 
    
            \graphbox{\( K \)}{0mm}{0mm}{34mm}{12mm}{2mm}{2mm}{
                \coordinate (o) at (0mm,-8mm); 
                \node[draw,circle] (l1) at ($(o)+(-10mm,0mm)$) {1};
                \node[draw,circle] (l2) at ($(l1)+(2,0)$) {2};
            }  
            \graphbox{\(G  \)}{90mm}{10mm}{34mm}{40mm}{2mm}{-3mm}{
                \coordinate (o) at (0mm,-20mm); 
                 \node[draw,circle] (l1) at ($(o)+(0,0)$) {1\ 2};
                % \node[draw,circle] (l1) at ($(o)+(-10mm,0mm)$) {1};
                % \node[draw,circle] (l2) at ($(l1)+(2,0)$) {2};
                \draw[->,red] (l1) edge[loop below] node[midway, below] {$a$} (l1) ;
                \node[draw,circle,red] (l3) at ($(l1)+(0,1.4)$) {3};
                \node[draw,circle,blue] (l4) at ($(l1)+(1,-1)$) {6};
                \draw[->,red] (l1) edge[bend left] node[midway,left] {$a$} (l3);
                \draw[->,red] (l3) edge[bend left] node[midway,right] {$a$} (l1);
                \draw[->,blue] (l1) edge  node[midway,right] {$a$} (l4);
                \node[draw,circle,blue] (l6) at ($(l1)+(-1,-1)$) {7};
                \draw[<-,blue] (l1) edge node[midway,left] {$a$} (l6) ;
                % \draw[->,blue] (l1) edge[out=-135,in=-45]node[midway,below] {$a$} (l1) ;
            }   
     
            \graphbox{\( C \)}{40mm}{-13mm}{34mm}{25mm}{2mm}{-3mm}{
                \coordinate (o) at (-2mm,-6mm); 
                \node[draw,circle] (l1) at ($(o)+(0,0)$) {1\ 2};
                % \node[draw,circle] (l2) at ($(l1)+(2,0)$) {2};
                \node[draw,circle,blue] (l4) at ($(l1)+(1,-1)$) {6};
                \draw[->,blue] (l1) -- (l4) node[midway,right] {$a$};
                % \draw[->,blue] (l1) edge[loop above] node[midway, above] {$a$} (l1) ;
                \node[ draw,circle,blue] (l6) at ($(l1)+(-1,-1)$) {7};
                \draw[<-,blue] (l1) -- (l6) node[midway,left] {$a$};
            }
            % K to L  
            \draw[>->] (17mm,5mm) -- node[above] {$\alpha$} (37mm,15mm);
            % C to G
            \draw[>->] (76mm,-28mm)-- node[below] {$\alpha'$} (88mm,-24mm) ;
            % K to C
            \draw[->] (17mm,-17mm) -- node[below] {$\beta$} (37mm,-28mm);
            % L to G
            \draw[->] (76mm,16mm) -- node[above] {$\beta'$} (88mm,7mm);
            \node () at (57mm,-6mm) {$PO$};
        \end{tikzpicture}
        } 
    \end{center}
\end{example}
The \emph{pullback} is the dual construction of the pushout, and it can be thought of as construction of the interface structure along which two structures are glued together. 
\begin{definition} 
    \label{def:cat:pb}
   A \textbf{pullback}\index{Pullback} of a cospan \(B \overset{\beta'}{\rightarrow} D \overset{\alpha'}{\leftarrow} C \) 
%    , shown below,
%    in Figure~\ref{fig:preliminaries:pullback_ssdsfd},
is a span \( B \overset{\alpha}{\leftarrow} A \overset{\beta}{\rightarrow} C \) such that the following conditions hold:
\begin{itemize}
    \item  \( \alpha \mathop{\star} \beta' \mathop{=} \beta \mathop{\star} \alpha' \),
    \item for every span \( B \overset{\gamma'}{\leftarrow} E \overset{\gamma}{\rightarrow} C \) if \(\gamma' \mathop{\star} \beta' \mathop{=} \gamma \mathop{\star} \alpha'\) holds, then there is a unique morphism \(\delta: E  \mathop{\to} A\) such that $\gamma' \mathop{=} \delta \mathop{\star} \alpha$ and $\gamma \mathop{=} \delta \mathop{\star} \beta$.
\end{itemize}  
    % \begin{figure}[H]
    %     \centering
    \begin{center}
        \resizebox{0.4\textwidth}{!}{
                    \begin{tikzpicture}
                        \node (i) at (0,0) {A};
                        \node (r) at (1,1) {B};
                        \node (c) at (1,-1) {C};
                        \node (h) at (2,0) {D}; 
                        % \node () at (1,-1) {\( \Delta \)};
                        \draw[->]  (i) -- (r) node [midway,left] {$\alpha$};
                        \draw[->] (c) -- (h) node [midway,left] {$\alpha'$};
                        \draw[->] (r) -- (h) node[midway, left] {$\beta'$};
                        \draw[->] (i) -- (c) node[midway, left] {$\beta$};
                        \node (d') at (-2,0) {E};
                        \draw[<-] (c) -- (d') node [midway,below]{$\gamma$};
                        \draw[<-] (r) -- (d') node [midway,above]{$\gamma'$};
                        \draw[->, dashed] (d') -- (i) node [midway]{$\delta$};
                    \end{tikzpicture}
        }
    %     \caption{}
    %     \label{fig:preliminaries:pullback_ssdsfd}
    % \end{figure}
                \end{center}
The diagram involving \( (\alpha, \beta, \alpha', \beta') \) is called a \textbf{pullback square}, or simply a \textbf{pullback}\index{Pullback!square}, with \(A\) as the \textbf{pullback object}\index{Pullback!object}. The existence of a unique morphism is known as the \textbf{universal mapping property of the pullback}\index{Pullback!universal mapping property}.
\end{definition} 

\begin{example} 
    \label{ex:cat:pbfsdljkgjasssss}
    In the category \textbf{Set},
    the pullback 
    of a cospan \(B \overset{\beta'}{\rightarrow} D \overset{\alpha'}{\leftarrow} C \) is the span \( B \overset{\alpha}{\leftarrow} A \overset{\beta}{\rightarrow} C \) where
    the pullback object is the set $A \mathop{=} \set{(b,c)\in B \mathop{\times} C \mathop{\mid} \beta (c) \mathop{=} \alpha (b)}$; $\alpha$ and $\beta$ are defined as the corresponding projections, e.g., $\alpha((b, c)) \mathop{=} b$ and $\beta((b, c)) \mathop{=} c$.
    Consider the following diagram in the category \textbf{Set}, where 
    sets are drawn as boxes,
    circles represent elements of sets, and numbers inside circles indicate how the functions map those elements.
    \begin{center}
      \resizebox{0.7\textwidth}{!}{
      \begin{tikzpicture}
          \graphbox{\( A\)}{40mm}{-3mm}{34mm}{12mm}{2mm}{2mm}{
              \coordinate (o) at (0mm,-8mm); 
              \node[draw,circle] (l1) at ($(o)+(-10mm,0mm)$) {1};
              \node[draw,circle] (l2) at ($(l1)+(2,0)$) {2};
          }  
          \graphbox{\( B \)}{80mm}{-3mm}{45mm}{12mm}{2mm}{2mm}{
              \coordinate (o) at (-5mm,-8mm); 
              \node[draw,circle] (l1) at ($(o)+(-10mm,0mm)$) {1};
              \node[draw,circle] (l2) at ($(l1)+(3,0)$) {2};
              \node[draw,circle] (l3) at ($(l1)+(1,0)$) {4};
            %   \node[draw,circle] (l4) at ($(l1)+(2,0)$) {5};
            %   \draw[ ] (l1) -- (l3) node[midway,above] {$a$};
            %   \draw[ ] (l3) -- (l4) node[midway,above] {$b$};
            %   \draw[ ] (l4) -- (l2) node[midway,above] {$a$};
          }     
          \graphbox{\( C  \)}{40mm}{-22mm}{34mm}{22mm}{2mm}{-3mm}{
              \coordinate (o) at (0mm,-3mm); 
              \node[draw,circle] (l1) at ($(o)+(-10mm,0mm)$) {1};
              \node[draw,circle] (l2) at ($(l1)+(2,0)$) {2};
            %   \node[draw,circle] (l4) at ($(l2)+(0,-1)$) {6};
              \node[ draw,circle] (l6) at ($(l1)+(0,-1)$) {3};
            %   \draw[ ] (l1) -- (l6) node[midway,left] {$a$};
            %   \draw[ ] (l2) -- (l4) node[midway,right] {$a$};
          }    
          \graphbox{\( D \)}{80mm}{-22mm}{45mm}{22mm}{2mm}{-3mm}{
              \coordinate (o) at (-5mm,-3mm); 
              \node[draw,circle] (l1) at ($(o)+(-10mm,0mm)$) {1};
              \node[draw,circle] (l2) at ($(l1)+(3,0)$) {2};
              \node[draw,circle] (l3) at ($(l1)+(1,0)$) {4};
            %   \node[draw,circle] (l4) at ($(l1)+(2,0)$) {5};
            %   \node[ draw,circle] (l5) at ($(l2)+(0,-1)$) {6};
              \node[ draw,circle] (l6) at ($(l1)+(0,-1)$) {3};
            %   \draw[ ] (l1) -- (l6) node[midway,left] {$a$};
            %   \draw[] (l1) -- (l3) node[midway,above] {$a$};
            %   \draw[] (l3) -- (l4) node[midway,above] {$b$};
            %   \draw[ ] (l4) -- (l2) node[midway,above] {$a$};
            %   \draw[ ] (l2) -- (l5) node[midway,right] {$a$};
          }    
          \node () at (77mm,-8mm) {\( \overset{\alpha}{\rightarrow} \)};
          \node () at (58mm,-18mm) {\( \beta\downarrow \)};
          \node () at (102mm,-18mm) {\( \beta'\downarrow \)};
          \node () at (77mm,-33mm) {\( \overset{\alpha'}{\rightarrow} \)};
      \end{tikzpicture}
      }
    \end{center} 
    The span \( B \overset{\alpha}{\leftarrow} A \overset{\beta}{\rightarrow} C \)
    % shown in Figure~\ref{fig:preliminaries:a_rewriting_step_dfjalsdkdfsdfjflg} 
    is the pullback of the cospan \(B \overset{\beta'}{\rightarrow} D \overset{\alpha'}{\leftarrow} C \).
    Indeed, the pullback object can be taken as the set $A' \mathop{=} \set{(1_B,1_C),(2_B,2_C)} \mathop{\subseteq} B \mathop{\times} C$, and the morphisms $\alpha'' \mathop{\colon} A'  \mathop{\to} B$ and $\beta'' \mathop{\colon} A'  \mathop{\to} C$ are defined as the corresponding projections, e.g., $\alpha''((x,y)) \mathop{=} x$ and $\beta''((x,y)) \mathop{=} y$. Note that $A$ is isomorphic to $A'$, which is expected because the pullback of a cospan is unique up to isomorphism.
\end{example}
In the category \textbf{Graph} pullbacks always exist and are computed pointwise: take the pullback in \textbf{Set} of the node sets and of the edge sets, and equip the resulting graph with source, target and labeling maps induced componentwise; the projection graph homomorphisms are the corresponding componentwise projections and they satisfy the universal property.
\begin{example}
    \label{ex:cat:pbfsdljkgjasssss2222ssss}
    Consider the diagram in the category \textbf{Graph}, shown below. The numbers inside nodes and the subgraphs in different colors illustrate how the morphisms map nodes and edges. 
    The span $L \overset{\alpha}{\leftarrow} K \overset{\beta}{\rightarrow} C$ is a pullback of the cospan $L \overset{\beta'}{\rightarrow} G \overset{\alpha'}{\leftarrow} C$.
    % \begin{figure}[H]
    %     \centering
    \begin{center}
        \resizebox{0.8\textwidth}{!}{
        \begin{tikzpicture} 
            \graphbox{\( L \)}{40mm}{20mm}{34mm}{20mm}{2mm}{2mm}{
                \coordinate (o) at (0mm,-10mm); 
                \node[draw,circle] (l1) at ($(o)+(-10mm,0mm)$) {1};
                \node[draw,circle] (l2) at ($(l1)+(2,0)$) {2};
                \draw[->,red] (l2) edge[out=-135,in=-45]node[midway,below] {$a$} (l1) ;
                \node[draw,circle,red] (l3) at ($(l1)+(1,0)$) {3};
                \draw[->,red] (l1) -- (l3) node[midway,above] {$a$};
                \draw[->,red] (l3) -- (l2) node[midway,above] {$a$};
            } 
    
            \graphbox{\( K \)}{0mm}{0mm}{34mm}{12mm}{2mm}{2mm}{
                \coordinate (o) at (0mm,-8mm); 
                \node[draw,circle] (l1) at ($(o)+(-10mm,0mm)$) {1};
                \node[draw,circle] (l2) at ($(l1)+(2,0)$) {2};
            }  
            \graphbox{\(G  \)}{90mm}{10mm}{34mm}{40mm}{2mm}{-3mm}{
                \coordinate (o) at (0mm,-20mm); 
                 \node[draw,circle] (l1) at ($(o)+(0,0)$) {1\ 2};
                % \node[draw,circle] (l1) at ($(o)+(-10mm,0mm)$) {1};
                % \node[draw,circle] (l2) at ($(l1)+(2,0)$) {2};
                \draw[->,red] (l1) edge[loop below] node[midway, below] {$a$} (l1) ;
                \node[draw,circle,red] (l3) at ($(l1)+(0,1.4)$) {3};
                \node[draw,circle,blue] (l4) at ($(l1)+(1,-1)$) {6};
                \draw[->,red] (l1) edge[bend left] node[midway,left] {$a$} (l3);
                \draw[->,red] (l3) edge[bend left] node[midway,right] {$a$} (l1);
                \draw[->,blue] (l1) edge  node[midway,right] {$a$} (l4);
                \node[draw,circle,blue] (l6) at ($(l1)+(-1,-1)$) {7};
                \draw[<-,blue] (l1) edge node[midway,left] {$a$} (l6) ;
                % \draw[->,blue] (l1) edge[out=-135,in=-45]node[midway,below] {$a$} (l1) ;
            }   
     
            \graphbox{\( C \)}{40mm}{-12mm}{34mm}{24mm}{2mm}{-3mm}{
                \coordinate (o) at (0mm,-5mm); 
                \node[draw,circle] (l1) at ($(o)+(0,0)$) {1\ 2};
                % \node[draw,circle] (l2) at ($(l1)+(2,0)$) {2};
                \node[draw,circle,blue] (l4) at ($(l1)+(1,-1)$) {6};
                \draw[->,blue] (l1) -- (l4) node[midway,right] {$a$};
                % \draw[->,blue] (l1) edge[loop above] node[midway, above] {$a$} (l1) ;
                \node[ draw,circle,blue] (l6) at ($(l1)+(-1,-1)$) {7};
                \draw[<-,blue] (l1) -- (l6) node[midway,left] {$a$};
            }
            % K to L  
            \draw[>->] (17mm,5mm) -- node[above] {$\alpha$} (37mm,15mm);
            % C to G
            \draw[>->] (76mm,-28mm)-- node[below] {$\alpha'$} (88mm,-26mm) ;
            % K to C
            \draw[->] (17mm,-17mm) -- node[below] {$\beta$} (37mm,-28mm);
            % L to G 
            \draw[->] (76mm,16mm) -- node[above] {$\beta'$} (88mm,10mm);
            \node () at (57mm,-6mm) {$PB$};
        \end{tikzpicture}
        }
    \end{center}
\end{example} 
Consider the following diagram in the category \textbf{Graph}:\begin{center}
        \resizebox{0.8\textwidth}{!}{
        \begin{tikzpicture} 
            \graphbox{\( L \)}{40mm}{15mm}{34mm}{12mm}{2mm}{2mm}{
                \coordinate (o) at (0mm,-8mm); 
                \node[draw,circle] (l1) at ($(o)+(-10mm,0mm)$) {1};
                \node[draw,circle] (l2) at ($(l1)+(2,0)$) {2};
                \node[draw,circle,red] (l3) at ($(l1)+(1,0)$) {3};
                \draw[->,red] (l1) -- (l3) node[midway,above] {$a$};
                \draw[->,red] (l3) -- (l2) node[midway,above] {$a$};
            } 
    
            \graphbox{\( K \)}{0mm}{0mm}{34mm}{12mm}{2mm}{2mm}{
                \coordinate (o) at (0mm,-8mm); 
                \node[draw,circle] (l1) at ($(o)+(-10mm,0mm)$) {1};
                \node[draw,circle] (l2) at ($(l1)+(2,0)$) {2};
            }  
            \graphbox{\(G  \)}{90mm}{5mm}{34mm}{24mm}{2mm}{-3mm}{
                \coordinate (o) at (0mm,-5mm); 
                \node[draw,circle] (l1) at ($(o)+(-10mm,0mm)$) {1};
                \node[draw,circle] (l2) at ($(l1)+(2,0)$) {2};
                \node[draw,circle,red] (l3) at ($(l1)+(1,0)$) {3};
                \node[draw,circle,blue] (l4) at ($(l2)+(0,-1)$) {6};
                \draw[->,red] (l1) -- (l3) node[midway,above] {$a$};
                \draw[->,red] (l3) -- (l2) node[midway,above] {$a$};
                \draw[->,blue] (l2) -- (l4) node[midway,right] {$a$};
                \node[draw,circle,blue] (l6) at ($(l1)+(0,-1)$) {7};
                \draw[<-,blue] (l1) -- (l6) node[midway,left] {$a$};
                \draw[->,blue] (l2) edge[out=-135,in=-45]node[midway,below] {$a$} (l1) ;
            }   
     
            \graphbox{\( C \)}{40mm}{-15mm}{34mm}{24mm}{2mm}{-3mm}{
                \coordinate (o) at (0mm,-5mm); 
                \node[draw,circle] (l1) at ($(o)+(-10mm,0mm)$) {1};
                \node[draw,circle] (l2) at ($(l1)+(2,0)$) {2};
                \node[draw,circle,blue] (l4) at ($(l2)+(0,-1)$) {6};
                \draw[->,blue] (l2) -- (l4) node[midway,right] {$a$};
                \draw[->,blue] (l2) edge[out=-135,in=-45]node[midway,below] {$a$} (l1) ;
                \node[ draw,circle,blue] (l6) at ($(l1)+(0,-1)$) {7};
                \draw[<-,blue] (l1) -- (l6) node[midway,left] {$a$};
            }      
            % K to L
            \draw[>->] (17mm,5mm) -- node[above] {$\alpha$} (37mm,10mm);
            % C to G
            \draw[>->] (76mm,-28mm)-- node[below] {$\alpha'$} (104mm,-21mm) ;
            % K to C
            \draw[>->] (17mm,-17mm) -- node[below] {$\beta$} (37mm,-28mm);
            % L to G
            \draw[>->] (76mm,10mm) -- node[above] {$\beta'$} (104mm,7mm);
            \node () at (57mm,-6mm) {$PO$};
        \end{tikzpicture}
        }
    \end{center}
We observe that it is a pushout square as well as a pullback square. This is not a coincidence, as stated in the following proposition.
\begin{proposition}[\text{\cite[Lemma 13]{lack2004adhesive}}]
    \label{prop:pb_eq_po}
    In the category \textbf{Graph}, pushouts along monomorphisms are also pullbacks. 
\end{proposition}
This proposition is illustrated in Example~\ref{ex:cat:pushout_non_injective_ssss} and Example~\ref{ex:cat:pbfsdljkgjasssss2222ssss}.

% % \begin{notation}
    When the context makes it clear, a morphism \( h : A \to B \) will be denoted by \( h_{AB} \), and diagrams will be referred to by their nodes, as is standard in geometry. For example, the diagram involving the morphisms \( \alpha, \beta, \alpha', \beta' \) in~\autoref{def:cat:po} will be denoted by \( ACDB \) or \( ABDC \).
\end{notation}   
% % \else
% % \fi  







% \chapter{Termination of Injective DPO Graph Rewriting
% Systems using Subgraph Counting}

% \section{abstract}
% We present a machine-checkable sufficient condition for relative termination of double-pushout graph rewriting systems with injective rules on edge-labeled multigraphs. 
% Our method defines a graph's weight as the sum of weights of occurrences of a set of graphs within it. By ensuring 
% (1) every rewriting step using rules in a set A strictly decreases the host graph's weight, and 
% (2) every rewriting step using rules in a set B never increases it, we guarantee that rules in the set A can be applied only finitely many times in any rewriting sequence with rules in the union of A and B.  
% Our method resolves termination cases that prior interpretation-based methods cannot. 
% We also propose an implementation of our technique.

% \section{Introduction}
% \label{sec:intro}
% We present a new automated method for relative termination of double-pushout (DPO) graph rewriting systems with injective rules~\cite{corradini1997algebraic,habel2001double}.~\footnote{This work resulted in a publication at the 18th International Conference on Graph Transformation (ICGT 2025)~\cite{qiu2025termination_icgt}. Reproduced with permission from Springer Nature.} To prove the relative termination of a set of rules \(\mathcal{A}\) with respect to another set of rules \(\mathcal{B}\), the method requires identifying patterns whose counts decrease with each application of a rule from \(\mathcal{A}\) and do not increase with each application of a rule from \(\mathcal{B}\). 
Left-hand sides of the rules are typical examples of such patterns, and we leave it to future work to explore whether other patterns are useful.
 
Let \( \mathbb{X} \) denote a fixed set of patterns. We assign weights (natural numbers) to monomorphisms from graphs in \( \mathbb{X} \), and define the weight of a graph $G$ as the sum of the weights of all monomorphisms from graphs in \( \mathbb{X} \) to $G$. 
  
Let $\mathcal{A}$ and $\mathcal{B}$ be sets of DPO rewriting rules. Let $X \mathop{\in} \mathbb{X}$ and $G \mathop{\Rightarrow} H$ be a rewriting step defined by the double-pushouts diagram shown below. 
\begin{center}
     \begin{tikzpicture}
            \node (I) at (0,0) {$K$};
            \node (L)  at (-2,0) {$L$};
            \node (R)  at (2,0) {$R$};
            \node (G)  at (-2,-2) {$G$};
            \node (C)  at (0,-2) {$C$};
            \node (H)  at (2,-2) {$H$};
            \draw [>->] (I) to  node [midway,below] {$l$} (L);
            \draw [>->] (I) to  node [midway,below] {$r$} (R);
            \draw [>->] (L) to node [midway,right] {$m$} (G);
            \draw [>->] (I) to  node [midway,right] 
            {} (C);
            \draw [>->] (R) to  node [midway,right] 
            {$m'$}
            (H);
            \draw [>->] (C) to node [midway,above] {$l'$} (G);
            \draw [>->] (C) to node [midway,above] 
            {$r'$} 
            (H);
        \end{tikzpicture}
    \end{center}
The image of a monomorphism from $X$ to $G$ falls into exactly one of the following three mutually exclusive cases:
\begin{enumerate}
    \item the image is fully included in $m(L)$;
    \item the image is fully included in $l'(C)$; 
    \item the image is neither fully included in $m(L)$ nor fully included in $l'(C)$.
\end{enumerate} 
 Similary, the image of a monomorphism from $X$ to $H$ falls into exactly one of the following three mutually exclusive cases:
 \begin{enumerate}
    \item the image is fully included in $m'(R)$;
    \item the image is fully included in $r'(C)$;
    \item the image is neither fully included in $m'(R)$ nor fully included in $r'(C)$.
 \end{enumerate}
  The number of monomorphisms $h: X \to G$ whose image is fully included in $m(L)$ can be computed exactly; likewise, the number of monomorphisms $x: X \to H$ whose image is fully included in $m'(R)$ can be computed exactly. Moreover, the number of monomorphisms $x: X \to G$ whose image is fully included in $l'(C)$ equals to the number of monomorphisms $x: X \to H$ whose image is fully included in $r'(C)$. Therefore, to estimate the change in weight when applying a rewriting rule, it suffices to compare the numbers of monomorphisms in the third cases above.
  
  To enable this comparison, we suppose that for every $X \in \mathbb{X}$, 
  the following conditions hold: 
\begin{itemize}
    % \item For every rule in \( \mathcal{A} \), the left-hand side graph's weight is strictly greater than the right-hand side graph's weight; 
    \item For every rule in \( \mathcal{A} \), the number of monomorphisms $x: X \to G$ whose image is fully included in $m(L)$ is strictly greater than the number of monomorphisms $x: X \to H$ whose image is fully included in $m'(R)$;
    % \item For every rule in \( \mathcal{B} \), the left-hand side graph's weight is greater than or equal to the right-hand side graph's weight.
    \item For every rule in \( \mathcal{B} \), the number of monomorphisms $x: X \to G$ whose image is fully included in $m(L)$ is greater than or equal to the number of monomorphisms $x: X \to H$ whose image is fully included in $m'(R)$;
\end{itemize}
    We suppose additionaly that there is an injective mapping from the set of monomorphisms $x : X \to H$ whose the image is neither fully included in $m'(R)$ nor fully included in $r'(C)$ to the set of monomorphisms from $x : X \to G$ whose image is neither fully included in $m(L)$ nor fully included in $l'(C)$. 


Under these conditions, rewriting steps using rules in \( \mathcal{A} \) strictly decrease the weights of the host graphs, while rewriting steps using rules in \( \mathcal{B} \) do not increase them.
Consequently, the rewriting rules in \( \mathcal{A} \) can only be applied a finite number of times.
% , since the weight of a finite graph is a natural number and strictly decreases with each rewriting step using a rule in \( \mathcal{A} \), the process must terminate after a finite number of iterations.  

   
This chapter is organized as follows.
We first define the weight of a graph in~\textsection~\ref{subgraph_counting:sec:interpretation}. 
Using this definition, we outline the general approach and identify the key challenge in~\textsection~\ref{subgraph_counting:sec:general_idea} and introduce the concept of non-increasing rules in \textsection~\ref{subgraph_counting:sec:non-increasing}. 
A solution to the challenge is then proposed in~\textsection~\ref{subgraph_counting:sec:solution_to_the_key_challenge}, culminating in our termination criterion in~\textsection~\ref{subgraph_counting:sec:termination}.
\textsection~\ref{subgraph_counting:sec:examples} illustrates the method with some examples.
\textsection~\ref{subgraph_counting:sec:related_work} compares our approach with some existing methods.
Finally,~\textsection~\ref{subgraph_counting:sec:conclusion} concludes with remarks and future research directions. Proofs of some propositions, lemmas, and theorems of this chapter
 have been moved to Appendix~\ref{subgraph_counting:sec:appendix} to improve readability.
% \section{Preliminaries:to do //modification needed<=content deleted} 
% \label{sec:pre} 
% \begin{definition}[Rewriting framework \cite{endrullis2024generalized_arxiv_v2}]
    A \textbf{DPO rewriting framework} $\mathfrak{F}$ is a mapping of DPO rewriting rules to classes of DPO diagrams such that, for every rule $\rho$, $\mathfrak{F}(\rho)$ is a class of DPO diagrams with top-span $\rho$.
    The \textbf{DPO rewriting relation $\Rightarrow_{\rho,\mathfrak{F}}$ induced by a DPO rewriting rule $\rho$ in $\mathfrak{F}$} is defined as follows: $G \Rightarrow_{\rho,\mathfrak{F}} H$ iff $G \Rightarrow_\rho^\delta H$ for some $\delta \in \mathfrak{F}(\rho)$. 
    % for some $\delta \in \mathfrak{F}(\rho)$
     The \textbf{DPO rewriting relation $\Rightarrow_{\mathcal{R},\mathfrak{F}}$ induced by a set $\mathcal{R}$ of DPO rewriting rules in $\mathfrak{F}$} is given by: $G \Rightarrow_{\mathcal{R},\mathfrak{F}} H$ iff $G \Rightarrow_{\rho,\mathfrak{F}} H$ for some $\rho \in \mathcal{R}$. When $\mathfrak{F}$ is clear from the context, we 
    suppress $\mathfrak{F}$ and 
    write $\Rightarrow_{\rho}$ and $\Rightarrow_{\mathcal{R}}$.
  \end{definition}
Let \(\mathfrak{F}\) denote the DPO rewriting framework that associates any rule with the class of all DPO diagrams with the rule as the top span, and let \(\mathfrak{M}\) denote the DPO rewriting framework that associates any rule with the class of all DPO diagrams having monic matches and the rule as the top span.

% \subsection{DPO rewriting} 
% \begin{definition}[Rewriting rule and match~\cite{corradini1997algebraic}]
  \label{def:grs:dpo_rule}
A \textbf{DPO rewriting rule} $\rho$ is a span \( L \overset{l}{\leftarrow} K \overset{r}{\rightarrow} R \), where \( K \) is the \textbf{interface}, \( L \) is the \textbf{left-hand-side graph}, denoted \( \operatorname{lhs}(\rho) \), and \( R \) is the \textbf{right-hand-side graph}, denoted \( \operatorname{rhs}(\rho) \). The rule is \textbf{monic} if $l$ and $r$ are both monic.
A match of the rule in an graph \( G \) is a morphism \( m: L \rightarrow G \).   
\end{definition}
   In this paper, we use examples from the category \textbf{Graph} of edge-labeled directed multigraphs (see~\cite{konig2018atutorial}) to illustrate the discussed concepts. To facilitate this, we introduce the following notation for visualizing graph homomorphisms.
\begin{notation}[\cite{qiu2025termination}]
    We use the notation from~\cite[Notation 1]{overbeek2023apbpotutorial} to visualize edge-labeled graph homomorphisms. Labeled graphs are enclosed in boxes with their names displayed in the top-left corner. Nodes and edges are assigned subsets of \(\mathbb{N}\) as identifiers, and these identifiers are chosen such that: (i) Each node or edge \( y \) in the codomain graph is assigned the union of the identifiers of all nodes or edges in the domain graph that are mapped to \( y \); (ii) The graph homomorphism is uniquely determined by this assignment. To further improve readability, we represent sets by listing their elements. Additionally, we omit identifiers when doing so does not cause confusion. This is illustrated in the following representation of a homomorphism \( h: G \to H \).
    
   \begin{center}
        \resizebox{0.5\textwidth}{!}{
        \begin{tikzpicture}
            \graphbox{\( G \)}{00mm}{-20mm}{45mm}{25mm}{2mm}{-10mm}{
                \coordinate (o) at (-5mm,-8mm); 
                \node[draw,circle] (l1) at ($(o)+(-10mm,0mm)$) {1};
                \node[draw,circle] (l2) at ($(l1)+(3,0)$) {2};
                \node[draw,circle] (l3) at ($(l1)+(1,0)$) {3};
                \node[draw,circle] (l4) at ($(l1)+(2,0)$) {4};
                \draw[->] (l1) -- (l3) node[midway,above] {a};
                \draw[->] (l3) -- (l4) node[midway,above] {b};
                \draw[->] (l4) -- (l2) node[midway,above] {a};
            }  
            \graphbox{\( H \)}{52mm}{-20mm}{50mm}{25mm}{2mm}{-10mm}{
                \coordinate (o) at (-5mm,-8mm); 
                \node[draw,circle] (l1) at ($(o)+(-1,0mm)$) {1};
                \node[draw,circle] (l2) at ($(l1)+(3,0)$) {2};
                \node[draw,circle] (l3) at ($(l1)+(1.5,0)$) {3\ 4};
                \draw[->] (l1) edge node[midway,above] {a} (l3);
                \draw[->] (l3) edge [loop above] node[midway,above] {b} (l3) ;
                \draw[->] (l3) -- (l2) node[midway,above] {a};
            }      
            \node () at (48mm,-30mm) {$\rightarrow$};
        \end{tikzpicture}
    }
    \end{center}  
    In this example, the sets \(\{1\}\), \(\{2\}\), \(\{3\}\), \(\{4\}\), and \(\{3,4\}\) are represented as \(1\), \(2\), \(3\), \(4\), and \(3\ 4\), respectively. Edge identifiers are omitted.
\end{notation}
\begin{example}
  \label{ex:grsaa}
  The injective DPO rule from \cite[Example 6]{bruggink2014termination} will be used to illustrate the concepts discussed throughout this paper.
  The rule can be visualized as follows:
  \begin{center} 
      \resizebox{0.7\textwidth}{!}{
      \begin{tikzpicture}
          \graphbox{$L$}{0mm}{0mm}{34mm}{15mm}{2mm}{-5mm}{
              \coordinate (o) at (0mm,-3mm); 
              \node[draw,circle] (l1) at ($(o)+(-10mm,0mm)$) {1};
              \node[draw,circle] (l2) at ($(l1)+(2,0)$) {2};
              \node[draw,circle] (l3) at ($(l1) + (1,0)$) {3};
              \draw[->] (l1) -- (l3) node[midway,above] {a};
              \draw[->] (l3) -- (l2) node[midway,above] {a};
          }     
          \graphbox{$K$}{40mm}{0mm}{24mm}{15mm}{2mm}{-5mm}{
              \coordinate (o) at (5mm,-3mm); 
              \node[draw,circle] (l1) at ($(o)+(-10mm,0mm)$) {1};
              \node[draw,circle] (l2) at ($(l1)+(1,0)$) {2};
              % \node[draw,circle] (l3) at ($(l1) + (1,0)$) {$\ $};
              % \draw[->] (l1) -- (l3) node[midway,above] {a};
              % \draw[->] (l3) -- (l2) node[midway,above] {a};
          }    
          \graphbox{$R$}{70mm}{0mm}{45mm}{15mm}{2mm}{-5mm}{
              \coordinate (o) at (-5mm,-3mm); 
              \node[draw,circle] (l1) at ($(o)+(-10mm,0mm)$) {1};
              \node[draw,circle] (l2) at ($(l1)+(3,0)$) {2};
              \node[draw,circle] (l3) at ($(l1) + (1,0)$) {4};
              \node[draw,circle] (l4) at ($(l1) + (2,0)$) {5};
              \draw[->] (l1) -- (l3) node[midway,above] {a};
              \draw[->] (l3) -- (l4) node[midway,above] {b};
              \draw[->] (l4) -- (l2) node[midway,above] {a};
          }    
          \node () at (37mm,-8mm) {$\overset{l}{\leftarrowtail}$};
          \node () at (67mm,-8mm) {$\overset{r}{\rightarrowtail}$};
          % \draw[>->] (51mm,2mm) -- (52mm,3mm);
      \end{tikzpicture}
      }
  \end{center}
\end{example}
\begin{definition}[DPO Rewriting step \cite{endrullis2024generalized_arxiv_v2}]
  \label{def:rewriting_step}
    \ \newline
    \noindent
    \begin{minipage}{0.72\textwidth}
      A DPO diagram $\delta$ is a diagram as shown on the right.
      This diagram $\delta$ is a witness for the \textbf{rewriting step} from \( G \) to \( H \) using the rule \( \rho \) and \textbf{match} \( m \), denoted \( G \mathop{\Rightarrow}_\rho^m H \) or \( G \mathop{\Rightarrow}_\rho^\delta H \). We denote $\operatorname{left}(\delta)$ and $\operatorname{right}(\delta)$ the pushout squares $KLGC$ and $KRHC$, respectively.
    \end{minipage}
    \hfill
    \begin{minipage}{0.28\textwidth}
          % \begin{center}
          \hfill
          \resizebox{0.85\textwidth}{!}{
          \begin{tikzpicture}
            % [node distance=11mm]
            \node (I) at (0,0) {$K$};
            \node (L) at (-2,0) {$L$};
            \node (R) at (2,0) {$R$};
            \node (G) at (-2,-2) {$G$};
            \node (C) at (0,-2) {$C$};
            \node (H) at (2,-2) {$H$};
            \draw [->] (I) to  node [midway,below] {$l$} (L);
            \draw [->] (I) to  node [midway,below] {$r$} (R);
            \draw [->] (L) to node [midway,right] {$m$} (G);
            \draw [->] (I) to node [midway,right] {$u$} (C);
            \draw [->] (R) to node [midway,left] {$m'$} (H);
            \draw [->] (C) to node [midway,above] {$l'$} (G);
            \draw [->] (C) to node [midway,above] {$r'$} (H);
            \node [at=($(I)!.5!(G)$)] {\normalfont PO};
            \node [at=($(I)!.5!(H)$)] {\normalfont PO};
          \end{tikzpicture}
        % \end{center}
        }
        \end{minipage}
  \end{definition}
\begin{example}
  \label{ex:rewriting_step_grs_aa}
  The DPO diagram below defines a rewriting step using the rule from Example~\ref{ex:grsaa}.
  \begin{center} 
      \resizebox{0.7\textwidth}{!}{
      \begin{tikzpicture}
          \graphbox{\( L \)}{0mm}{-3mm}{34mm}{12mm}{2mm}{2mm}{
              \coordinate (o) at (0mm,-8mm); 
              \node[draw,circle] (l1) at ($(o)+(-10mm,0mm)$) {1};
              \node[draw,circle] (l2) at ($(l1)+(2,0)$) {2};
              \node[draw,circle] (l3) at ($(l1)+(1,0)$) {3};
              \draw[] (l1) -- (l3) node[midway,above] {$a$};
              \draw[] (l3) -- (l2) node[midway,above] {$a$};
          } 
          \graphbox{\( K \)}{40mm}{-3mm}{34mm}{12mm}{2mm}{2mm}{
              \coordinate (o) at (0mm,-8mm); 
              \node[draw,circle] (l1) at ($(o)+(-10mm,0mm)$) {1};
              \node[draw,circle] (l2) at ($(l1)+(2,0)$) {2};
          }  
          \graphbox{\( R \)}{80mm}{-3mm}{45mm}{12mm}{2mm}{2mm}{
              \coordinate (o) at (-5mm,-8mm); 
              \node[draw,circle] (l1) at ($(o)+(-10mm,0mm)$) {1};
              \node[draw,circle] (l2) at ($(l1)+(3,0)$) {2};
              \node[draw,circle] (l3) at ($(l1)+(1,0)$) {4};
              \node[draw,circle] (l4) at ($(l1)+(2,0)$) {5};
              \draw[ ] (l1) -- (l3) node[midway,above] {$a$};
              \draw[ ] (l3) -- (l4) node[midway,above] {$b$};
              \draw[ ] (l4) -- (l2) node[midway,above] {$a$};
          }    
          \graphbox{\( G \)}{0mm}{-22mm}{34mm}{22mm}{2mm}{-3mm}{
              \coordinate (o) at (0mm,-3mm); 
              \node[draw,circle] (l1) at ($(o)+(-10mm,0mm)$) {1};
              \node[draw,circle] (l2) at ($(l1)+(2,0)$) {2};
              \node[draw,circle] (l3) at ($(l1)+(1,0)$) {3};
              \node[draw,circle] (l4) at ($(l2)+(0,-1)$) {6};
              \draw[] (l1) -- (l3) node[midway,above] {$a$};
              \draw[] (l3) -- (l2) node[midway,above] {$a$};
              \draw[ ] (l2) -- (l4) node[midway,right] {$a$};
              \node[draw,circle] (l6) at ($(l1)+(0,-1)$) {7};
              \draw[] (l1) -- (l6) node[midway,left] {$a$};
          }    
          \graphbox{\( C  \)}{40mm}{-22mm}{34mm}{22mm}{2mm}{-3mm}{
              \coordinate (o) at (0mm,-3mm); 
              \node[draw,circle] (l1) at ($(o)+(-10mm,0mm)$) {1};
              \node[draw,circle] (l2) at ($(l1)+(2,0)$) {2};
              \node[draw,circle] (l4) at ($(l2)+(0,-1)$) {6};
              \draw[ ] (l2) -- (l4) node[midway,right] {$a$};
              \node[ draw,circle] (l6) at ($(l1)+(0,-1)$) {7};
              \draw[ ] (l1) -- (l6) node[midway,left] {$a$};
          }    
          \graphbox{\( H \)}{80mm}{-22mm}{45mm}{22mm}{2mm}{-3mm}{
              \coordinate (o) at (-5mm,-3mm); 
              \node[draw,circle] (l1) at ($(o)+(-10mm,0mm)$) {1};
              \node[draw,circle] (l2) at ($(l1)+(3,0)$) {2};
              \node[draw,circle] (l3) at ($(l1)+(1,0)$) {4};
              \node[draw,circle] (l4) at ($(l1)+(2,0)$) {5};
              \node[ draw,circle] (l5) at ($(l2)+(0,-1)$) {6};
              \node[ draw,circle] (l6) at ($(l1)+(0,-1)$) {7};
              \draw[ ] (l1) -- (l6) node[midway,left] {$a$};
              \draw[] (l1) -- (l3) node[midway,above] {$a$};
              \draw[] (l3) -- (l4) node[midway,above] {$b$};
              \draw[ ] (l4) -- (l2) node[midway,above] {$a$};
              \draw[ ] (l2) -- (l5) node[midway,right] {$a$};
          }    
          \node () at (37mm,-8mm) {\( \leftarrowtail \)}; % K -> L
          \node () at (77mm,-8mm) {\( \rightarrowtail \)}; % K -> R
          \node () at (15mm,-18mm) {\( m\ \downarrowtail \)};
          \node () at (37mm,-33mm) {\( \leftarrowtail \)};
          \node () at (58mm,-18mm) {\( u\downarrowtail \)};
          \node () at (102mm,-18mm) {\( \downarrowtail \)};
          \node () at (77mm,-33mm) {\( \rightarrowtail \)}; % C -> H
      \end{tikzpicture}
      }
  \end{center}
\end{example} 

% \subsection{Relative termination}
% \begin{definition}[Rewriting sequence]
    Let \(\mathcal{R}\) be a set of DPO rewriting rules and $\mathfrak{F}$ a DPO rewriting framework in \textbf{Graph}.
    A \textbf{$(\mathcal{R},\mathfrak{F})$-rewriting sequence} is a finite sequence \(s_0,s_1,\hdots, s_m\) of objects such that for each \( 0 \leq n \leq m-1\), we have \(s_n \mathop{\Rightarrow}_{\mathcal{R},\mathfrak{F}} s_{n+1}\), or an infinite sequence \(s_0,s_1,\hdots\) of objects such that \(s_n \mathop{\Rightarrow}_{\mathcal{R},\mathfrak{F}} s_{n+1}\) for each \(n \mathop{\in} \mathbb{N}\).
\end{definition}
We adapt the concept of relative termination presented in \cite{klop1987term,geser1990relative}.
\begin{definition}[Relative termination]
    \label{termination:def:relative_termination}
     Let $\mathcal{A}$ and $\mathcal{B}$ be sets of rewriting rules, and $\mathfrak{F}$ a DPO rewriting framework in \textbf{Graph}. We say that $\mathop{\Rightarrow}_{\mathcal{A},\mathfrak{F}}$ is \textbf{terminating relative to} $\mathop{\Rightarrow}_{\mathcal{B}, \mathfrak{F}}$ if any infinite $(\mathcal{A} \mathop{\cup} \mathcal{B},\mathfrak{F})$-rewriting sequence has only a finite number of rewriting steps with rules in $\mathcal{A}$.
\end{definition}
   
% \section{A Termination Criterion} 
% \label{sec:termination_criterion}
% % In this section, we present a method for termination analysis of DPO graph rewriting systems with injective rules and monic matches. 
% Since it is an interpretation method~\cite{nipkow1998term}, we first define the weight of a graph in Definition~\ref{def:graph_weight}. 
% With this definition, we explain the general idea of the method and identify the key challenge in~\autoref{sec:general_idea}. 
% In~\autoref{sec:non-increasing}, we introduce the concept of non-increasing rules. 
% Using this concept, a solution to this challenge is then proposed in~\autoref{sec:solution_to_the_key_challenge}. 
% Finally, in~\autoref{sec:termination}, we state our termination criterion.

\section{Measuring graphs by counting morphisms from specific graphs}
\label{subgraph_counting:sec:interpretation}
Let $\mathcal{R} \mathop{=} \mathcal{A} \mathop{\cup} \mathcal{B}$ be a set of injective rules.
Define $\opn{Sub}(\opn{left}(\mathcal{R}))$ as the set of all subgraphs of left-hand sides of rules in $\mathcal{R}$.
We call the elements of $\opn{Sub}(\opn{left}(\mathcal{R}))$~\textbf{ruler-graphs}. Ruler-graphs in a set $\mathbb{X}$ provide a measure for a graph
G, and the total weight of G is determined by a weighted linear
combination of these measures.

\begin{definition} 
    \label{subgraph_counting:def:measurement}
    For a ruler-graph \( X \) and graph \( G \), the \textbf{measurement} \( m_X(G) \) is defined as the number of $X$-occurrences in $G$, i.e. the cardinality of the set of monomorphisms from $ X $ to $ G $:
    \(
        m_X(G) \mathop{=} \card{\operatorname{Mono}(X, G)}
    \).
\end{definition} 
Counting monomorphisms from ruler-graphs before and after a rewriting step provides a termination criterion: 
    If there is a ruler-graph $X$ such that every rewriting step using a rule in $\mathcal{A}$ strictly decreases the number monomorphism from $X$, and every rewriting step using a rule in $\mathcal{B}$ never increases it, then the rewriting relation induced by $\mathcal{A}$ terminates relative to the rewriting relation induced by $\mathcal{B}$.
 
For some ruler-graphs, the change in the number of monomorphism when applying a rewriting step using an injective rule is precisely computable by simply counting the numbers of monomorphisms from these ruler-graphs in the left- and right-hand side graphs. 

Examples include: 
\begin{itemize}
    \item  the ruler-graph \tikz[baseline=-0.5ex]{
        \node[draw,circle] (x) at (0,0) {};
    } consisting of a single node;
    \item the ruler-graph \tikz[baseline=-0.5ex]{
        \node[draw,circle] (x) at (0,0) {};
        \draw[->] (x) edge[loop right] node[midway, right] {$x$} (x) ;
    } consisting of a single node and a single loop labeled by $x\in \Sigma$; 
    \item the ruler-graph \tikz[baseline=-0.5ex]{
        \node[draw,circle] (x) at (0,0) { };
        \node[draw,circle] (y) at (1,0) { };
        \draw[->] (x) -- (y) node[midway, above] {$x$};
    } consisting of two nodes and a single labeled edge between them.  
\end{itemize}
    
However, systems like Example~\ref{subgraph_counting:ex:grsaa_rx} require more complex ruler-graphs such as \tikz[baseline=-0.5ex]{
        \node[draw,circle] (x) at (0,0) {};
        \node[draw,circle] (y) at (1,0) {};
        \node[draw,circle] (z) at (2,0) {};
        \draw[->] (x) -- (y) node[midway, above] {$a$};
        \draw[->] (y) -- (z) node[midway, above] {$a$};
    }, as we will demonstrate in Example~\ref{subgraph_counting:ex:termination:grsaa}.
    For such ruler-graph, the change in the number of monomorphisms when applying a rewriting rule is unpredictable in general, because the number of monomorphisms 
    % from such ruler-graph 
    depends on the context graph\textemdash a challenge that we address in later sections.
    
To combine measurements from multiple ruler-graphs (analogous to combining measurements of physical objects from multiple rulers), we introduce a weight function that assigns a weight to each ruler-graph.
\begin{definition} 
    \label{subgraph_counting:def:weight_function}
    A \textbf{weight function} for a set of ruler-graphs \( \mathbb{X} \) is a map \( s_{\mathbb{X}} \mathop{\colon} \mathbb{X} \mathop{\to} \mathbb{N} \) assigning a weight in $\mathbb{N}$ to each ruler-graph. The weight of a monomorphism from a ruler-graph $X$ is defined to be $s_\mathbb{X}(X)$. 
\end{definition}
Assigning different weights to measurements from distinct ruler-graphs enhences the technique, as we will demonstrate in Example~\ref{ex:overbeek_5d6}.
\begin{definition} 
    \label{subgraph_counting:def:graph_weight}  
    Let \( \mathbb{X} \) be a set of ruler-graphs, 
    \( s_{\mathbb{X}} \) a weight function, and \( G \) a graph. The \textbf{weight of graph} \( G \) relative to \( s_{\mathbb{X}} \), denoted by \( w_{s_{\mathbb{X}}}(G) \), is defined as: 
    \[
        w_{s_{\mathbb{X}}}(G) \mathop{=} \sum_{X \mathop{\in} \mathbb{X}} s_{\mathbb{X}}(X) \cdot m_X(G).  
    \]  
\end{definition} 

\section{General idea and key challenge}
\label{subgraph_counting:sec:general_idea}
For the remainder of this section, we assume: a set \( \mathcal{R} \mathop{=} \mathcal{A} \mathop{\cup} \mathcal{B} \) of injective DPO rewriting rules, a set \( \mathbb{X} \) of ruler-graphs, and a weight function \( s_{\mathbb{X}} \).  Let \( \rho \mathop{=} (L \overset{l}{\leftarrowtail} K \overset{r}{\rightarrowtail} R) \) denote a rule in \( \mathcal{R} \) and \( X \mathop{\in} \mathbb{X} \). 
Recall that \(\mathfrak{M}\), defined in Definition~\ref{def:rewriting_framework}, denotes the DPO rewriting framework that associates each rule with the collection of all DPO diagrams witnessing rewriting steps using the rule, where the matches are monic.

The rewriting relation \( \mathop{\Rightarrow}_{\mathcal{A},\mathfrak{M}} \) terminates relative to $\mathop{\Rightarrow}_{\mathcal{B},\mathfrak{M}}$ if the following hold: (i) for all \(G \mathop{\Rightarrow}_{\mathcal{A},\mathfrak{M}} H\), \( w_{s_\mathbb{X}}(G) \mathop{>} w_{s_\mathbb{X}}(H) \), and (ii) for all \(G \mathop{\Rightarrow}_{\mathcal{B},\mathfrak{M}} H\), \( w_{s_\mathbb{X}}(G) \mathop{\geq} w_{s_\mathbb{X}}(H) \). However, directly verifying all rewriting steps is infeasible due to their potential infiniteness. 
Thus, we aim to provide a rule-based sufficient condition for termination.
% \begin{remark}
%     To simplify our presentation, we treat the following three notions interchangeably when the monomorphism \( h: A \mathop{\to} B \) is understood: the domain graph \( A \), the monomorphism \( h: A \mathop{\to} B \), the image \( h(A) \mathop{\subseteq} B \) (an occurrence of \( A \) in \( B \)).
%     Consequently, in a rewriting step defined by the diagram in Definition~\ref{def:rewriting_step}, we streamline descriptions involving subgraph images. Rather than stating that there is an occurrence of a ruler-graph \( X \) formed by the image of subgraphs \( R' \mathop{\subseteq} R \) and \( C' \mathop{\subseteq} C \) around the image of \( K' \mathop{\subseteq} K \), we simply say \( X \) is formed by \( R' \) and \( C' \) around \( K' \). Rather than stating that $b \notin h(A)$, we simply say that $b \notin A$.
% \end{remark}
\begin{notation}
    \label{subgraph_counting:notation:mono_sets}
    % The disjoint union of two sets \( S \) and \( S' \) will be denoted by \( S \uplus S' \). Let \( X, A, B, G \) be graphs, and let \( \alpha \mathop{\colon} A \mathop{\to} G \) and \( \beta \mathop{\colon} B \mathop{\to} G \) be morphisms. We define the following sets of monomorphisms of $X$ in $G$: 
 The disjoint union of two sets \( S \) and \( S' \) will be denoted by \( S \uplus S' \). Let \( X, A, B, G \) be graphs, and let \( \alpha \mathop{\colon} A \mathop{\to} G \) and \( \beta \mathop{\colon} B \mathop{\to} G \) be morphisms shown below:
\begin{center}
            \begin{tikzpicture}
            \node (X) at (-1.5,-1) {$X$};
            \node (B) at (2,0) {$A$}; 
            \node (C) at (0,2) {$B$}; 
            \node (D) at (0,0) {$G$}; 
            \draw [>->] (B) to node [above,label,pos=0.45] {$\alpha$} (D); 
            \draw [>->] (C) to node [right,label,pos=0.45] {$\beta$} (D);
            \draw [>->,dashed] (X) to node [above,label,pos=0.45] {$\iota$} (D);
            \draw [>->,dashed] (X) to node [below,label,pos=0.45] {$\zeta$} (B);
            \draw [>->,dashed] (X) to node [above,label,pos=0.45] {$\eta$} (C);
        \end{tikzpicture}
\end{center}
 We define the following sets of monomorphisms from $X$ in $G$ based on their interaction with $\alpha$ and $\beta$:    
    \begin{align*}
        \operatorname{Mono}(X,G,\alpha) &= \left\{ \iota \mathop{\colon} X \rightarrowtail G \;\middle|\; \exists \zeta \mathop{\colon} X \rightarrowtail A.\, \iota \mathop{=} \zeta \mathop{\star} \alpha \right\},
        \\
        \operatorname{Mono}(X,G,\lnot \alpha) &= \left\{ \iota \mathop{\colon} X \rightarrowtail G \;\middle|\; \nexists \zeta \mathop{\colon} X \rightarrowtail A.\, \iota \mathop{=} \zeta \mathop{\star} \alpha \right\},
        \\
        \operatorname{Mono}(X,G,\lnot \alpha, \beta) &= \left\{ 
            \iota \mathop{\colon} X \rightarrowtail G \;\middle|\; 
                \begin{aligned}  
                    &(\nexists \zeta \mathop{\colon} X \rightarrowtail A.\, \iota \mathop{=} \zeta \mathop{\star} \alpha) \\ 
                    &\land (\exists \eta \mathop{\colon} X \rightarrowtail B.\, \iota \mathop{=} \eta \mathop{\star} \beta)
                \end{aligned}
        \right\},
        \\
        \operatorname{Mono}(X,G,\lnot \alpha, \lnot \beta) &= \left\{ 
            \iota \mathop{\colon} X \rightarrowtail G \;\middle|\; 
                \begin{aligned}
                    &(\nexists \zeta \mathop{\colon} X \rightarrowtail A.\, \iota \mathop{=} \zeta \mathop{\star} \alpha) \\
                    &\land (\nexists \eta \mathop{\colon} X \rightarrowtail B.\, \iota \mathop{=} \eta \mathop{\star} \beta)
                \end{aligned}
        \right\}.
    \end{align*}
\end{notation}
Consider a rewriting step \( G \mathop{\Rightarrow}_{\rho,\mathfrak{M}} H \) defined by the DPO diagram illustrated below:
\begin{center}
     \begin{tikzpicture}
            \node (I) at (0,0) {$K$};
            \node (L)  at (-2,0) {$L$};
            \node (R)  at (2,0) {$R$};
            \node (G)  at (-2,-2) {$G$};
            \node (C)  at (0,-2) {$C$};
            \node (H)  at (2,-2) {$H$};
            \draw [>->] (I) to  node [midway,below] {$l$} (L);
            \draw [>->] (I) to  node [midway,below] {$r$} (R);
            \draw [>->] (L) to node [midway,right] {$m$} (G);
            \draw [>->] (I) to  node [midway,right] 
            {} (C);
            \draw [>->] (R) to  node [midway,right] 
            {$m'$}
            (H);
            \draw [>->] (C) to node [midway,above] {$l'$} (G);
            \draw [>->] (C) to node [midway,above] 
            {$r'$} 
            (H);
            % \node [at=($(I)!.5!(G)$)] {\normalfont PO};
            % \node [at=($(I)!.5!(H)$)] {\normalfont PO};
        \end{tikzpicture}
\end{center}
 The sets of monomorphisms \( \operatorname{Mono}(X,G) \) and \( \operatorname{Mono}(X,H) \) can be decomposed into the following disjoint unions:
\begin{flalign*}
    \operatorname{Mono}(X,G) &= 
    \operatorname{Mono}(X,G,l')
    \uplus
    \operatorname{Mono}(X,G,\lnot l',m) 
    \uplus
    \operatorname{Mono}(X,G,\lnot l',\lnot m),
    \\
    \operatorname{Mono}(X,H) &= 
    \operatorname{Mono}(X,H,r')
    \uplus
    \operatorname{Mono}(X,H,\lnot r',m') 
    \uplus
    \operatorname{Mono}(X,H,\lnot r',\lnot m').
\end{flalign*}
Thus, the following equalities hold:
\begin{flalign*}
    \card{\operatorname{Mono}(X,G)} ={} &
    \card{\operatorname{Mono}(X,G,l')}
    +
    \card{\operatorname{Mono}(X,G,\lnot l',m)}
     \\
    &+
    \card{\operatorname{Mono}(X,G,\lnot l',\lnot m)},
    \\
    \card{\operatorname{Mono}(X,H)} ={} &
    \card{\operatorname{Mono}(X,H,r')} 
    +
    \card{\operatorname{Mono}(X,H,\lnot r',m')} \\
    &+\card{\operatorname{Mono}(X,H,\lnot r',\lnot m')}.
\end{flalign*}
\noindent By the following Lemma~\ref{subgraph_counting:lem:decomp_w_u}, the following equalities hold:\todo{to do todo: xu : we cannot use a lemma before stating it. Move the lemma up?}
\begin{flalign*}
&\card{\operatorname{Mono}(X,G)} =
\card{\operatorname{Mono}(X,C)}
+
\card{\operatorname{Mono}(X,L, \lnot l)}
+
\card{\operatorname{Mono}(X,G, \lnot l',\lnot m)},
 \\
 &\card{\operatorname{Mono}(X,H)} =
 \card{\operatorname{Mono}(X,C)}
 +
 \card{ \operatorname{Mono}(X,R, \lnot r) }
 +
 \card{ \operatorname{Mono}(X,H, \lnot r',\lnot m')}.  
\end{flalign*}
\begin{lemma}
    \label{subgraph_counting:lem:decomp_w_u}
        Let $X$ be a ruler-graph. For a pushout square as shown below:
        \begin{center}
            \begin{tikzpicture}
            \node (A) at (0,0) {$A$};
            \node (B) at (0,-2) {$B$}; 
            \node (C) at (-2,0) {$C$}; 
            \node (D) at (-2,-2) {$D$}; 
            \draw [>->] (A) to node [right,label,pos=0.5] {$\alpha$} (B);
            \draw [>->] (A) to node [above,label,pos=0.5] {$\beta$} (C);
            \draw [>->] (B) to node [below,label,pos=0.45] {$\beta'$} (D); 
            \draw [>->] (C) to node [left,label,pos=0.45] {$\alpha'$} (D);
        \end{tikzpicture}
        \end{center}
         the following equalities hold:
        % $\card{\operatorname{Mono}(X, B)} \mathop{=} \card{\operatorname{Mono}(X, D, \beta')}$ and $\card{\operatorname{Mono}(X, C, \lnot \beta)} \mathop{=} \card{\operatorname{Mono}(X, D, \lnot \beta', \alpha')}$.
        \begin{flalign*}
            \card{\operatorname{Mono}(X, B)} &= \card{\operatorname{Mono}(X, D, \beta')},
            \\
            \card{\operatorname{Mono}(X, C, \lnot \beta)} &= \card{\operatorname{Mono}(X, D, \lnot \beta', \alpha')}.
        \end{flalign*}
\end{lemma}
Therefore, we obtain:
\begin{flalign*}
      \card{\operatorname{Mono}(X,G)} -  \card{\operatorname{Mono}(X,H)}
     =& \card{\operatorname{Mono}(X,L, \lnot l)}- \card{\operatorname{Mono}(X,R, \lnot r)} \mathop{+}\\ 
     &\card{\operatorname{Mono}(X,G, \lnot l',\lnot m)} - 
     \card{\operatorname{Mono}(X,H, \lnot r',\lnot m') } 
\end{flalign*}
By the following Lemma~\ref{subgraph_counting:lem:xlnlmxrnr}, it can be simplified to:
\begin{flalign*}
     \card{\operatorname{Mono}(X,G)} -  \card{\operatorname{Mono}(X,H)}
    \mathop{=} &\card{\operatorname{Mono}(X,L)}- \card{\operatorname{Mono}(X,R)} \mathop{+}\\ 
      &\card{\operatorname{Mono}(X,G, \lnot l',\lnot m)} - 
    \card{\operatorname{Mono}(X,H, \lnot r',\lnot m') } 
\end{flalign*}
\begin{lemma}
    \label{subgraph_counting:lem:xlnlmxrnr}
    Let $X$ be a graph. Let $L \overset{l}{\leftarrowtail} K \overset{r}{\rightarrowtail} R$ be an injective DPO graph rewriting rule. We have 
    \[
       \card{\operatorname{Mono}(X, L, \lnot l)}  - \card{\operatorname{Mono}(X, R, \lnot r)} 
       \mathop{=} 
       \card{\operatorname{Mono}(X, L)}  - \card{\operatorname{Mono}(X, R)} 
        \]
\end{lemma}
Consequently, the weight difference satisfies:
\begin{flalign*}
     w_{s_\mathbb{X}}(G) - w_{s_\mathbb{X}}(H)
    \overset{\operatorname{def}}{=}&\sum_{X \mathop{\in} \mathbb{X}}^{}s_\mathbb{X}(X) * m_X(G) - \sum_{X \mathop{\in} \mathbb{X}}^{}s_\mathbb{X}(X) * m_X(H)\\
    \overset{\operatorname{def}}{=}&\sum_{X \mathop{\in} \mathbb{X}}^{}s_\mathbb{X}(X) * |\operatorname{Mono}(X,G)| - \sum_{X \mathop{\in} \mathbb{X}}^{}s_\mathbb{X}(X) * |\operatorname{Mono}(X,H)|\\
    =&\sum_{X \mathop{\in} \mathbb{X}}^{}s_\mathbb{X}(X) * \left( \card{\operatorname{Mono}(X,G)} -  \card{\operatorname{Mono}(X,H)} \right)\\
    =&\sum_{X \mathop{\in} \mathbb{X}}^{}s_\mathbb{X}(X) * \left(  \card{\operatorname{Mono}(X,L)}- \card{\operatorname{Mono}(X,R)} \right)+\\
       & \sum_{X \mathop{\in} \mathbb{X}}^{}s_\mathbb{X}(X) * \left( 
     \card{\operatorname{Mono}(X,G, \lnot l',\lnot m)} - 
    \card{\operatorname{Mono}(X,H, \lnot r',\lnot m') } \right).
\end{flalign*}
Since $\sum_{X \mathop{\in} \mathbb{X}}^{}s(X) * \left(  \card{\operatorname{Mono}(X,L)}- \card{\operatorname{Mono}(X,R)} \right)$ can be precisely computed, the key challenge is to establish a lower bound for
\begin{flalign*}
    % \label{ineq:key_challenge}
    \sum_{X \mathop{\in} \mathbb{X}}^{} s_\mathbb{X}(X) * 
         \left ( \card{\operatorname{Mono}(X,G, \lnot l',\lnot m)} - 
     \card{\operatorname{Mono}(X,H, \lnot r',\lnot m') } \right ).
    % \label{ineq:ge0_sss}
\end{flalign*} 
In the next section, we introduce the concept of non-increasing rules, which helps to address this challenge.
\section{Non-increasing rules}
\label{subgraph_counting:sec:non-increasing}
% \noindent
% \begin{minipage}{0.7\textwidth}
%     % \setlength{\parindent}{1em}
%     Let \( \rho \mathop{=} (L \overset{l}{\leftarrowtail} K \overset{r}{\rightarrowtail} R) \) be an injective DPO rewriting rule. 
%     A DPO diagram as illustrated on the right defines a rewriting step using $\rho$ with injective match \( m \).
% \end{minipage}%
% \hfill 
% \begin{minipage}{0.3\textwidth}
%     \hfill
%     % \begin{figure}[htbp] 
%     %     \center
%         \begin{tikzpicture}
%             % [node distance=15mm]
%             \node (I) {$K$};
%             \node (L) [left of=I] {$L$};
%             \node (R) [right of=I] {$R$};
%             \node (G) [below of=L] {$G$};
%             \node (C) [below of=I] {$C$};
%             \node (H) [below of=R] {$H$};
%             \draw [>->] (I) to  node [midway,below] {$l$} (L);
%             \draw [>->] (I) to  node [midway,below] {$r$} (R);
%             \draw [>->] (L) to node [midway,right] {$m$} (G);
%             \draw [>->] (I) to  node [midway,right] 
%             % {$u$}
%             {} (C);
%             \draw [>->] (R) to  node [midway,right] 
%             % {$m'$}
%             {} (H);
%             \draw [>->] (C) to node [midway,above] {$l'$} (G);
%             \draw [>->] (C) to node [midway,above] 
%             % {$r'$} 
%             {}
%             (H);
%             % \node [at=($(I)!.5!(G)$)] {\normalfont PO};
%             % \node [at=($(I)!.5!(H)$)] {\normalfont PO};
%         \end{tikzpicture}
%             % \caption{}
%     %         \label{diag:set_dpo_intuition_1}
%     % \end{figure}
% \end{minipage}

% \noindent
% \begin{minipage}{0.5\textwidth}
%     % \setlength{\parindent}{1em}
%      It follows that the DPO diagram depicted on the right, where all morphisms are inclusion functions and $H'$ is a graph isomorphic to $H$, holds.
% \end{minipage}%
% \hfill 
% \begin{minipage}{0.5\textwidth}  
%     \hfill
%     \begin{tikzpicture}
%         [node distance=20mm]
%         \node (I) at (0,0) {$\operatorname{Im}(l \mathop{\star} m)$};
%         \node (L) at (-2,0) {$\operatorname{Im}(m)$};
%         \node (R) at (2,0) {$\operatorname{Im}(m')$};
%         \node (G) at (-2,-1) {$G$};
%         \node (C) at (0,-1) {$\operatorname{Im}(l')$};
%         \node (H) at (2,-1) {$H'$};
%         \draw [>->] (I) to  node [midway,above] {} (L);
%         \draw [>->] (I) to  node [midway,above] {} (R);
%         \draw [>->] (L) to node [midway,right] {} (G);
%         \draw [>->] (I) to  node [midway,right] {} (C);
%         \draw [>->] (R) to  node [midway,right] {} (H);
%         \draw [>->] (C) to node [midway,above] {} (G);
%         \draw [>->] (C) to node [midway,above] {} (H);
%         % \node [at=($(I)!.5!(G)$)] {\normalfont PO};
%         % \node [at=($(I)!.5!(H)$)] {\normalfont PO};
%       \end{tikzpicture}
% \end{minipage}


%  \noindent
%     \begin{minipage}{0.6\textwidth}
%         By viewing a graph as a set o f nodes and edges and graph monomorphisms as set injections, the above DPO diagram can be considered as a DPO diagram in the category of sets, illustrated on the right.
%     \end{minipage}
%     \hfill
%     \begin{minipage}{0.3\textwidth}
%           \hfill
%           \resizebox{\textwidth}{!}{
%             \begin{tikzpicture}
%               \coordinate (k) at (0, 0);
%               \draw[fill=white] ($(k)+(0,0)$) rectangle ($(k)+(0.5,0.5)$);
%               \node () at ($(k)+(0.25,0.25)$) {\( \mathrm{K} \)};
          
%               \coordinate (c) at (0, -2.2);
%               \draw[fill=blue!20]
%               ($(c)+(0,-0.5)$)
%               -- ($(c)+(0,0.5)$) 
%               -- ($(c)+(1,0.5)$) 
%               arc[start angle=0, end angle=-90, radius=1]
%               -- cycle;
%               \node () at ($(c)+(0.75,0.25)$) {\( \mathrm{C'} \)};
%               \draw[fill=white] ($(c)+(0,0)$) rectangle ($(c)+(0.5,0.5)$);
%               \node () at ($(c)+(0.25,0.25)$) {\( \mathrm{K} \)};
          
%               \coordinate (l) at (-3, 0);
%               \draw[fill=orange!20] ($(l)+(-0.5,0)$) rectangle ($(l)+(0.5,1)$);
%               \node () at ($(l)+(-0.23,0.25)$) {\( \mathrm{L'} \)};
%               \draw[fill=white] ($(l)+(0,0)$) rectangle ($(l)+(0.5,0.5)$);
%               \node () at ($(l)+(0.25,0.25)$) {\( \mathrm{K} \)};
          
%               \coordinate (g) at (-3, -2.2);
%               \draw[fill=blue!20]
%               ($(g)+(0,-0.5)$)
%               -- ($(g)+(0,0.5)$)
%               -- ($(g)+(1,0.5)$) 
%               arc[start angle=0, end angle=-90, radius=1]
%               -- cycle;
%               \draw[fill=orange!20] ($(g)+(-0.5,0)$) rectangle ($(g)+(0.5,1)$);
%               \node () at ($(g)+(0.75,0.25)$) {\( \mathrm{C'} \)};
%               \node () at ($(g)+(-0.23,0.25)$) {\( \mathrm{L'} \)};
%               \draw[fill=white] ($(g)+(0,0)$) rectangle ($(g)+(0.5,0.5)$);
%               \node () at ($(g)+(0.25,0.25)$) {\( \mathrm{K} \)};
          
%               \coordinate (r) at (3,0);
%               \draw[fill=red!20] ($(r)+(-0.5,0)$)
%                 -- ($(r)+(-0.5,0.5)$)
%                 -- ($(r)+(0,1)$)
%                 --  ($(r)+(0.5,1)$)
%                 -- ($(r)+(0.5,0)$)
%                 -- cycle;
%               \node () at ($(r)+(-0.23,0.25)$) {\( \mathrm{R'} \)};
%               \draw[fill=white] ($(r)+(0,0)$) rectangle ($(r)+(0.5,0.5)$);
%               \node () at ($(r)+(0.25,0.25)$) {\( \mathrm{K} \)};
          
%               \coordinate (h) at (3, -2.2);
%               \draw[fill=blue!20]
%               ($(h)+(0,-0.5)$)
%               -- ($(h)+(0,0.5)$)
%               -- ($(h)+(1,0.5)$) 
%               arc[start angle=0, end angle=-90, radius=1]
%               -- cycle;
%               \draw[fill=red!20] ($(h)+(-0.5,0)$)
%               -- ($(h)+(-0.5,0.5)$)
%               -- ($(h)+(0,1)$)
%               --  ($(h)+(0.5,1)$)
%               -- ($(h)+(0.5,0)$)
%               -- cycle;
%              \node () at ($(h)+(0.75,0.25)$) {\( \mathrm{C'} \)};
%              \draw[fill=white] ($(h)+(0,0)$) rectangle ($(h)+(0.5,0.5)$);
%              \node () at ($(h)+(0.25,0.25)$) {\( \mathrm{K} \)};
%              \node () at ($(h)+(-0.23,0.25)$) {\( \mathrm{R'} \)};
          
%               \node[ font=\huge] (kl) at ($(k)!0.5!(l)+(0.25,0.25)$)
%              %    {\( \overset{l}{\leftarrowtail} \)}
%                 {\( \leftarrowtail \)}
%                ; 
%               \node[ font=\huge] (kr) at ($(k)!0.5!(r)+(0.25,0.25)$)
%                {\( \rightarrowtail \)}
%                ;  
%               \node[ font=\huge] (cg) at ($(c)!0.5!(g)+(0.25,0.25)$) 
%               {\( \leftarrowtail \)}
%              ;  
%               \node[ font=\huge] (ch) at ($(c)!0.5!(h)+(0.25,0.25)$)
%                {\( \rightarrowtail \)}
%              ; 
%               \node[ font=\huge] (kc) at ($(k)!0.5!(c)+(0.2,0.4)$) {\( \downarrowtail \)}; 
%             %   \node[ font=\LARGE] () at ($(l)!0.5!(g)+(0.5,0.4)$) {$m$}; 
%               \node[ font=\huge] (lg) at ($(l)!0.5!(g)+(0.1,0.4)$) {\( \downarrowtail \)}; 
%               \node[ font=\huge] (rh) at ($(r)!0.5!(h)+(0.1,0.4)$) {\( \downarrowtail \)}; 
%             %   \node[ font=\LARGE] () at ($(r)!0.5!(h)+(0.55,0.4)$) {$m'$}; 
%             %   \node[ font=\LARGE] () at ($(k)!0.5!(c)+(0.5,0.4)$) {$u$}; 
%             \end{tikzpicture}
%         }
%     \end{minipage}
% \begin{minipage}{0.59\textwidth}
%     % \setlength{\parindent}{1em}
%     Let \( \rho \mathop{=} (L \overset{l}{\leftarrowtail} K \overset{r}{\rightarrowtail} R) \) be an injective DPO rewriting rule. 
%     A rewriting step using $\rho$ with injective match \( m \) is defined by a DPO diagram as shown on the upper right.
%      % \setlength{\parindent}{1em}
%      It follows that both squares in the diagram depicted on the middle right, where all morphisms are inclusion functions and $H'$ is a graph isomorphic to $H$, are pushouts.
%      By viewing a graph as a set of nodes and edges and graph monomorphisms as set injections, the DPO diagram on the middle right can be considered as a DPO diagram in the category of sets, illustrated on the lower right. In
% \end{minipage}%
% \hfill 
% \begin{minipage}{0.39\textwidth}
%     % \hfill 
%     %     \resizebox{0.7\textwidth}{!}{
%     %         \begin{tikzpicture}
%     %             % [node distance=15mm]
%     %             \node (I) {$K$};
%     %             \node (L) [left of=I] {$L$};
%     %             \node (R) [right of=I] {$R$};
%     %             \node (G) [below of=L] {$G$};
%     %             \node (C) [below of=I] {$C$};
%     %             \node (H) [below of=R] {$H$};
%     %             \draw [>->] (I) to  node [midway,below] {$l$} (L);
%     %             \draw [>->] (I) to  node [midway,below] {$r$} (R);
%     %             \draw [>->] (L) to node [midway,right] {$m$} (G);
%     %             \draw [>->] (I) to  node [midway,right] 
%     %             % {$u$}
%     %             {} (C);
%     %             \draw [>->] (R) to  node [midway,right] 
%     %             % {$m'$}
%     %             {} (H);
%     %             \draw [>->] (C) to node [midway,above] {$l'$} (G);
%     %             \draw [>->] (C) to node [midway,above] 
%     %             % {$r'$} 
%     %             {}
%     %             (H);
%     %             % \node [at=($(I)!.5!(G)$)] {\normalfont PO};
%     %             % \node [at=($(I)!.5!(H)$)] {\normalfont PO};
%     %         \end{tikzpicture}
%     %     }
%     \centering
%     \resizebox{0.7\textwidth}{!}{
%         \begin{tikzpicture}
%             % [node distance=15mm]
%             \node (I) at (0,0) {$K$};
%             \node (L)  at (-2,0) {$L$};
%             \node (R)  at (2,0) {$R$};
%             \node (G)  at (-2,-2) {$G$};
%             \node (C)  at (0,-2) {$C$};
%             \node (H)  at (2,-2) {$H$};
%             \draw [>->] (I) to  node [midway,below] {$l$} (L);
%             \draw [>->] (I) to  node [midway,below] {$r$} (R);
%             \draw [>->] (L) to node [midway,right] {$m$} (G);
%             \draw [>->] (I) to  node [midway,right] 
%             % {$u$}
%             {} (C);
%             \draw [>->] (R) to  node [midway,right] 
%             {$m'$}
%             (H);
%             \draw [>->] (C) to node [midway,above] {$l'$} (G);
%             \draw [>->] (C) to node [midway,above] 
%             % {$r'$} 
%             {}
%             (H);
%             % \node [at=($(I)!.5!(G)$)] {\normalfont PO};
%             % \node [at=($(I)!.5!(H)$)] {\normalfont PO};
%         \end{tikzpicture}
%     }

%         \resizebox{0.7\textwidth}{!}{
%             \begin{tikzpicture}
%                 [node distance=20mm]
%                 \node (I) at (0,0) {$\operatorname{Im}(l \mathop{\star} m)$};
%                 \node (L) at (-2,0) {$\operatorname{Im}(m)$};
%                 \node (R) at (2,0) {$\operatorname{Im}(m')$};
%                 \node (G) at (-2,-2) {$G$};
%                 \node (C) at (0,-2) {$\operatorname{Im}(l')$};
%                 \node (H) at (2,-2) {$H'$};
%                 \draw [>->] (I) to  node [midway,above] {} (L);
%                 \draw [>->] (I) to  node [midway,above] {} (R);
%                 \draw [>->] (L) to node [midway,right] {} (G);
%                 \draw [>->] (I) to  node [midway,right] {} (C);
%                 \draw [>->] (R) to  node [midway,right] {} (H);
%                 \draw [>->] (C) to node [midway,above] {} (G);
%                 \draw [>->] (C) to node [midway,above] {} (H);
%                 % \node [at=($(I)!.5!(G)$)] {\normalfont PO};
%                 % \node [at=($(I)!.5!(H)$)] {\normalfont PO};
%             \end{tikzpicture}
%         }

%           \resizebox{0.7\textwidth}{!}{
%             \begin{tikzpicture}
%                 \coordinate (k) at (0, 0);
%                 \draw[fill=white] ($(k)+(0,0)$) rectangle ($(k)+(0.5,0.5)$);
%                 \node () at ($(k)+(0.25,0.25)$) {\( \mathrm{K} \)};
            
%                 \coordinate (c) at (0, -2.2);
%                 \draw[fill=blue!20]
%                 ($(c)+(0,-0.5)$)
%                 -- ($(c)+(0,0.5)$) 
%                 -- ($(c)+(1,0.5)$) 
%                 arc[start angle=0, end angle=-90, radius=1]
%                 -- cycle;
%                 \node () at ($(c)+(0.75,0.25)$) {\( \mathrm{C'} \)};
%                 \draw[fill=white] ($(c)+(0,0)$) rectangle ($(c)+(0.5,0.5)$);
%                 \node () at ($(c)+(0.25,0.25)$) {\( \mathrm{K} \)};
            
%                 \coordinate (l) at (-3, 0);
%                 \draw[fill=orange!20] ($(l)+(-0.5,0)$) rectangle ($(l)+(0.5,1)$);
%                 \node () at ($(l)+(-0.23,0.25)$) {\( \mathrm{L'} \)};
%                 \draw[fill=white] ($(l)+(0,0)$) rectangle ($(l)+(0.5,0.5)$);
%                 \node () at ($(l)+(0.25,0.25)$) {\( \mathrm{K} \)};
            
%                 \coordinate (g) at (-3, -2.2);
%                 \draw[fill=blue!20]
%                 ($(g)+(0,-0.5)$)
%                 -- ($(g)+(0,0.5)$)
%                 -- ($(g)+(1,0.5)$) 
%                 arc[start angle=0, end angle=-90, radius=1]
%                 -- cycle;
%                 \draw[fill=orange!20] ($(g)+(-0.5,0)$) rectangle ($(g)+(0.5,1)$);
%                 \node () at ($(g)+(0.75,0.25)$) {\( \mathrm{C'} \)};
%                 \node () at ($(g)+(-0.23,0.25)$) {\( \mathrm{L'} \)};
%                 \draw[fill=white] ($(g)+(0,0)$) rectangle ($(g)+(0.5,0.5)$);
%                 \node () at ($(g)+(0.25,0.25)$) {\( \mathrm{K} \)};
            
%                 \coordinate (r) at (3,0);
%                 \draw[fill=red!20] ($(r)+(-0.5,0)$)
%                 -- ($(r)+(-0.5,0.5)$)
%                 -- ($(r)+(0,1)$)
%                 --  ($(r)+(0.5,1)$)
%                 -- ($(r)+(0.5,0)$)
%                 -- cycle;
%                 \node () at ($(r)+(-0.23,0.25)$) {\( \mathrm{R'} \)};
%                 \draw[fill=white] ($(r)+(0,0)$) rectangle ($(r)+(0.5,0.5)$);
%                 \node () at ($(r)+(0.25,0.25)$) {\( \mathrm{K} \)};
            
%                 \coordinate (h) at (3, -2.2);
%                 \draw[fill=blue!20]
%                 ($(h)+(0,-0.5)$)
%                 -- ($(h)+(0,0.5)$)
%                 -- ($(h)+(1,0.5)$) 
%                 arc[start angle=0, end angle=-90, radius=1]
%                 -- cycle;
%                 \draw[fill=red!20] ($(h)+(-0.5,0)$)
%                 -- ($(h)+(-0.5,0.5)$)
%                 -- ($(h)+(0,1)$)
%                 --  ($(h)+(0.5,1)$)
%                 -- ($(h)+(0.5,0)$)
%                 -- cycle;
%             \node () at ($(h)+(0.75,0.25)$) {\( \mathrm{C'} \)};
%             \draw[fill=white] ($(h)+(0,0)$) rectangle ($(h)+(0.5,0.5)$);
%             \node () at ($(h)+(0.25,0.25)$) {\( \mathrm{K} \)};
%             \node () at ($(h)+(-0.23,0.25)$) {\( \mathrm{R'} \)};
            
%                 \node[ font=\huge] (kl) at ($(k)!0.5!(l)+(0.25,0.25)$)
%             %    {\( \overset{l}{\leftarrowtail} \)}
%                 {\( \leftarrowtail \)}
%                 ; 
%                 \node[ font=\huge] (kr) at ($(k)!0.5!(r)+(0.25,0.25)$)
%                 {\( \rightarrowtail \)}
%                 ;  
%                 \node[ font=\huge] (cg) at ($(c)!0.5!(g)+(0.25,0.25)$) 
%                 {\( \leftarrowtail \)}
%             ;  
%                 \node[ font=\huge] (ch) at ($(c)!0.5!(h)+(0.25,0.25)$)
%                 {\( \rightarrowtail \)}
%             ; 
%                 \node[ font=\huge] (kc) at ($(k)!0.5!(c)+(0.2,0.4)$) {\( \downarrowtail \)}; 
%             %   \node[ font=\LARGE] () at ($(l)!0.5!(g)+(0.5,0.4)$) {$m$}; 
%                 \node[ font=\huge] (lg) at ($(l)!0.5!(g)+(0.1,0.4)$) {\( \downarrowtail \)}; 
%                 \node[ font=\huge] (rh) at ($(r)!0.5!(h)+(0.1,0.4)$) {\( \downarrowtail \)}; 
%             %   \node[ font=\LARGE] () at ($(r)!0.5!(h)+(0.55,0.4)$) {$m'$}; 
%             %   \node[ font=\LARGE] () at ($(k)!0.5!(c)+(0.5,0.4)$) {$u$}; 
%             \end{tikzpicture}
%           }
% \end{minipage}
% the lower right diagram, $L', C', R', K$ are mutually disjoint sets such that $K$, 
%          $K \uplus L'$, 
%          $K \uplus R'$, 
%          $K \uplus C'$, 
%          $L' \uplus K \uplus C'$ and 
%          $R' \uplus K \uplus C'$ 
%             are corresponding sets of 
%          $\operatorname{Im}(l \mathop{\star} m)$, 
%          $\operatorname{Im}(m)$, 
%          $\operatorname{Im}(m')$,
%          $\operatorname{Im}(l')$,
%          $G$ and
%          $H'$ respectively,
%          and all morphisms as inclusion functions.
%     % Up to isomorphism, we can further consider all morphisms as inclusion functions, illustrated on the right. In this visualization, $L',R',C',K$ are mutually disjoint sets with $L' \uplus K \mathop{=} L$, $R' \uplus K= R$, $C' \uplus K \mathop{=} C$, $G \mathop{=} L \mathop{\cup} C$ and $H \mathop{=} R \mathop{\cup} C$.
%     Thus, any subgraph of $G$ (resp. $H$) can be considered as a union of subgraphs of $C$ and $L$ (resp. $R$), up to isomorphism.
%     % \color{red}\begin{definition}
% %         Consider a rewriting step defined by the DPO diagram above. Let $X$ be a ruler-graph and $G$ a graph. An \textbf{$X$-occurrence} in $G$ is a subgraph of $G$ isomorphic to $X$. An \textbf{$X$-occurrence $x$ implicitly created by the rewriting step} is an \( X \)-occurrence in $H$ which is included neither in \( R \) nor in $C$. 
% %         Let $R' \mathop{=} x \mathop{\cap} R$, $C' \mathop{=} x \mathop{\cap} C$ and $h_{R'L}:R' \rightarrowtail L$ a monomorphism preserving interface elements. The \textbf{corresponding $X$-occurrence relative to $h_{R'L}$} of $x$ is the $X$-occurrence in $G$ defined by $h_{R'L}(R') \mathop{\cup} C'$.
% %         An \textbf{$X$-occurrence $x$ implicitly destroyed by the rewriting step} is an \( X \)-occurrence in $G$ which is included neither in \( L \) nor in $C$. 
% %     \end{definition}
% % \color{black}

% With the above DPO diagrams in mind, we define:

% \begin{definition}
%     \label{def:corresponding_occurrence}
%      An \textbf{$X$-occurrence} in $G$ is a subgraph of $G$ isomorphic to $X$. An \textbf{$X$-occurrence $x$ implicitly created by the rewriting step} is an \( X \)-occurrence in $H$ which is included neither in \( R \) nor in $C$. 
%       The occurrence $x$ can be decomposed as the union of graphs $x \mathop{=} R_x \mathop{\cup} C_x$ where $R_x \mathop{=} x \mathop{\cap} R$ and $C_x \mathop{=} x \mathop{\cap} C$. For every graph monomorphism $h : R_x \rightarrowtail L$ that preserves interface elements (i.e., \( h(k) \mathop{=} k \) for all \( k \mathop{\in} K \)), we define the \textbf{corresponding $X$-occurrence of $x$ relative to $h$} as the $X$-occurrence in $G$ given by $h(R_x) \mathop{\cup} C_x$.
%     An \textbf{$X$-occurrence implicitly destroyed by the rewriting step} is an \( X \)-occurrence in $G$ which is included neither in \( L \) nor in $C$.
% \end{definition}
We define the set \(D(R,X)\) of all subgraphs of \( R \) which, when glued along some common interface elements with a subgraph $C'$ of a context graph \( C \), can form an implicit \( X \)-occurrence.

\begin{definition}[Distinguished subgraphs of the right-hand side graph]
    \label{subgraph_counting:def:rx}
    Let \(X\) be a graph and 
    \(
        L \overset{l}{\leftarrowtail} K \overset{r}{\rightarrowtail} R
    \) an injective DPO rewriting rule.
    %  such that $l$ is edge-injective and $r$ is node-injective.
    The set \(D(R,X)\) consists of all subgraphs \( R' \mathop{\subseteq} R \)
    satisfying the following conditions:
    \begin{itemize}
        \item $R'$ is not a subgraph of $r(K)$,
        \item $R'$ is not isomorphic to $X$,
        \item the following diagram where \(h_{K'K} \mathop{\colon} K' \rightarrowtail K \) and \(h_{R'R} \mathop{\colon} R' \rightarrowtail R \) are inclusion functions can be constructed:
        \begin{center}
                \begin{tikzpicture}[node distance=11mm]
                    \node (k) at (0, 0) {K};
                    \node (r) at (0, 2) {R};
                    \node (r') at (2,2) {$R'$};
                    \node (k') at (2,0) {K'};
                    \node (x) at (4,2) {X};
                    \node (c') at (4,0) {C'};
                    \node () [at=($(r)!0.5!(k')$)] {$PB$};
                    \node () [at=($(x)!0.5!(k')$)] {$PO$};
                    \draw[>->] (k) -- (r) node[pos= .5, left] {$r$}; 
                    \draw[>->] (r') -- (r) node[pos= .5, above] {$h_{R'R}$};
                    \draw[>->] (k') -- (k) node[pos= .5, below] {$h_{K'K}$};
                    \draw[>->] (k') -- (r');
                    \draw[>->] (k') -- (c');
                    \draw[>->] (c') -- (x);
                    \draw[>->] (r') -- (x);
                \end{tikzpicture}
        \end{center}
    \end{itemize}
\end{definition}  
The first and second conditions ensure that such $X$-occurrences are included neither in \( C \) nor in \( R \), thereby guaranteeing that they are implicit.
The third condition ensures that any $R' \mathop{\in} D(R,X)$ can be glued along some common interface elements with a subgraph $C'$ of a context graph \( C \) to form an \( X \)-occurrence. 

% In another word, the set $D(R,X)$ contains all subgraphs $R''$ of $R$ which is not a subgraph of $K$ such that there is a rewriting step defined by the diagram above such that $R'' \mathop{\cup} C''$ for a subgraph $C''$ of $C$ is an $X$-occurrence created by the rewriting step.

\begin{example}
    \label{subgraph_counting:ex:grsaa_rx}
    Consider the following injective DPO rule from~\cite[Example 6]{bruggink2014termination}:
    % in Figure~\ref{fig:subgraph_counting:grfksfdfgsdgsd}.  
    % \begin{figure}[H]
    %     \centering
    \begin{center}
        \begin{tikzpicture}
            \graphbox{$L$}{0mm}{0mm}{34mm}{15mm}{2mm}{-5mm}{
                \coordinate (o) at (0mm,-3mm); 
                \node[draw,circle] (l1) at ($(o)+(-10mm,0mm)$) {1};
                \node[draw,circle] (l2) at ($(l1)+(2,0)$) {2};
                \node[draw,circle] (l3) at ($(l1)+(1,0)$) {3};
                \draw[->] (l1) -- (l3) node[midway,above] {$a$};
                \draw[->] (l3) -- (l2) node[midway,above] {$a$};
            }     
            \graphbox{$K$}{40mm}{0mm}{24mm}{15mm}{2mm}{-5mm}{
                \coordinate (o) at (5mm,-3mm); 
                \node[draw,circle] (l1) at ($(o)+(-10mm,0mm)$) {1};
                \node[draw,circle] (l2) at ($(l1)+(1,0)$) {2};
            }    
            \graphbox{$R$}{70mm}{0mm}{45mm}{15mm}{2mm}{-5mm}{
                \coordinate (o) at (-5mm,-3mm); 
                \node[draw,circle] (l1) at ($(o)+(-10mm,0mm)$) {1};
                \node[draw,circle] (l2) at ($(l1)+(3,0)$) {2};
                \node[draw,circle] (l3) at ($(l1)+(1,0)$) {4};
                \node[draw,circle] (l4) at ($(l1)+(2,0)$) {5};
                \draw[->] (l1) -- (l3) node[midway,above] {$a$};
                \draw[->] (l3) -- (l4) node[midway,above] {$b$};
                \draw[->] (l4) -- (l2) node[midway,above] {$a$};
            }    

            \node () at (37mm,-8mm) {$\leftarrowtail$};
            \node () at (67mm,-8mm) {$\rightarrowtail$};

            % \draw[>->] (51mm,2mm) -- (52mm,3mm);
        \end{tikzpicture}
    %     \caption{}
    %     \label{fig:subgraph_counting:grfksfdfgsdgsd}
    % \end{figure}
    \end{center}
    Let $X$ be \tikz[baseline=-0.5ex]{
        \node[draw,circle] (x) at (0,0) { };
        \node[draw,circle] (y) at (1,0) { };
        \node[draw,circle] (z) at (2,0) { };
        \draw[->] (x) -- (y) node[midway, above] {$a$};
        \draw[->] (y) -- (z) node[midway, above] {$a$};
    } and $R' \mathop{\in} D(R,X)$. 
    $R'$ is neither the empty graph nor isomorphic to $X$, by Conditions 2 and 3 of Definition~\ref{subgraph_counting:def:rx}. 
    Therefore, $R'$ must contain an interface node (either node $1$ 
    % \tikz[baseline=-0.5ex]{ 
    %         \node[draw,circle] (y) at (1,0) {1};
    % }
     or node $2$
    % \tikz[baseline=-0.5ex]{ 
    %         \node[draw,circle] (y) at (1,0) {2};
    % }
    ), because if it contains no interface node, Condition 1 of Definition~\ref{subgraph_counting:def:rx} would be violated due to $X$'s connectivity.
    Additionally, $R'$ must contain node $4$
    % \tikz[baseline=-0.5ex]{ 
    %         \node[draw,circle] (y) at (1,0) {4};
    % }
     or node $5$
    % \tikz[baseline=-0.5ex]{ 
    %         \node[draw,circle] (y) at (1,0) {5};
    % } 
    to avoid being a subgraph of $r(K)$ (Condition 2). 
    Therefore, $R'$ must contain either 
    \tikz[baseline=-0.5ex]{ 
            \node[draw,circle] (x) at (0,0) {1}; 
            \node[draw,circle] (y) at (1,0) {4};
            \draw[->] (x) -- (y)  node[midway, above] {$a$};
    }
      or 
        \tikz[baseline=-0.5ex]{ 
            \node[draw,circle] (x) at (0,0) {5}; 
            \node[draw,circle] (y) at (1,0) {2};
            \draw[->] (x) -- (y) node[midway, above] {$a$};
    }
    , because if node $4$
    %     \tikz[baseline=-0.5ex]{ 
    %         \node[draw,circle] (y) at (1,0) {4};
    % }
     or node $5$
    %     \tikz[baseline=-0.5ex]{ 
    %         \node[draw,circle] (x) at (0,0) {5}; 
    % }
    is isolated, Condition 1 would be violated due to $X$'s connectivity.
    Also, $R'$ must not contain 
            \tikz[baseline=-0.5ex]{ 
            \node[draw,circle] (x) at (0,0) {4}; 
            \node[draw,circle] (y) at (1,0) {5};
            \draw[->] (x) -- (y) node[midway, above] {$b$};
    }
    by Condition 1.
    The following graph
    % in Figure~\ref{fig:subgraph_counting:rx_counter_exdfsdfadfs} 
    is not in $D(R,X)$ because it violates Condition 1.
    % \begin{figure}[H]
    %     \centering
    \begin{center}
            \begin{tikzpicture}
                \graphbox{}{70mm}{0mm}{45mm}{15mm}{2mm}{-5mm}{
                    \coordinate (o) at (-5mm,-3mm); 
                    \node[draw,circle] (l1) at ($(o)+(-10mm,0mm)$) {1};
                    \node[draw,circle] (l3) at ($(l1)+(1,0)$) {4};
                    \node[draw,circle] (l2) at ($(l1)+(3,0)$) {2};
                    \node[draw,circle] (l4) at ($(l1)+(2,0)$) {5};
                    \draw[->] (l4) -- (l2) node[midway,above] {$a$};
                    \draw[->] (l1) -- (l3) node[midway,above] {$a$};
                }    
            \end{tikzpicture}
        \end{center}
    Thus, $D(R,X)$ consists of the four graphs $R'_1$, $R'_2$, $R'_3$ and $R'_4$:
    \begin{center}
            \begin{tikzpicture}
                \graphbox{$R'_1$}{70mm}{0mm}{45mm}{15mm}{2mm}{-5mm}{
                    \coordinate (o) at (-5mm,-3mm); 
                    \node[circle] (l1) at ($(o)+(-10mm,0mm)$) {};
                    \node[draw,circle] (l2) at ($(l1)+(3,0)$) {2};
                    \node[draw,circle] (l4) at ($(l1)+(2,0)$) {5};
                    \draw[->] (l4) -- (l2) node[midway,above] {$a$};
                }    
            \end{tikzpicture}
            \begin{tikzpicture}
                \graphbox{$R'_2$}{70mm}{0mm}{45mm}{15mm}{2mm}{-5mm}{
                    \coordinate (o) at (-5mm,-3mm); 
                    \node[draw,circle] (l1) at ($(o)+(-10mm,0mm)$) {1};
                    \node[draw,circle] (l2) at ($(l1)+(3,0)$) {2};
                    \node[draw,circle] (l4) at ($(l1)+(2,0)$) {5};
                    \draw[->] (l4) -- (l2) node[midway,above] {$a$};
                }    
            \end{tikzpicture}
        \end{center}

        \begin{center}
            \begin{tikzpicture}
                \graphbox{$R'_3$}{70mm}{0mm}{45mm}{15mm}{2mm}{-5mm}{
                    \coordinate (o) at (-5mm,-3mm); 
                    \node[draw,circle] (l1) at ($(o)+(-10mm,0mm)$) {1};
                    \node[draw,circle] (l3) at ($(l1)+(1,0)$) {4};
                    \draw[->] (l1) -- (l3) node[midway,above] {$a$};
                }   
        \end{tikzpicture}
            \begin{tikzpicture}
                \graphbox{$R'_4$}{70mm}{0mm}{45mm}{15mm}{2mm}{-5mm}{
                    \coordinate (o) at (-5mm,-3mm); 
                    \node[draw,circle] (l1) at ($(o)+(-10mm,0mm)$) {1};
                    \node[draw,circle] (l2) at ($(l1)+(3,0)$) {2};
                    \node[draw,circle] (l3) at ($(l1)+(1,0)$) {4};
                    \draw[->] (l1) -- (l3) node[midway,above] {$a$};
                }    
            \end{tikzpicture}
    \end{center}
    For each of them, the construction of the pullback square in Condition 1 is demonstrated in Example~\ref{example:grs_aa:has_more_left} and the construction of the pushout square required by Condition 1 is straightforward. 
        
 
\end{example} 
 
\begin{example}
    \label{subgraph_counting:ex:rx_counter_ex}
    Let $X$ be the graph 
    \tikz[baseline=-0.5ex]{ 
            \node[draw,circle] (x) at (0,0) { }; 
            \node[draw,circle] (y) at (1,0) { };
            \node[draw,circle] (z) at (2,0) { };
            \draw[->] (x) -- (y)   {};
            \draw[->] (y) -- (z)   {};
    }. Consider the following rule:
    \begin{center}
            \begin{tikzpicture}
                \graphbox{$L$}{0mm}{0mm}{34mm}{15mm}{2mm}{-5mm}{
                    \coordinate (o) at (0mm,-3mm); 
                    \node[draw,circle] (l1) at ($(o)+(-10mm,0mm)$) {1};
                    % \node[draw,circle] (l2) at ($(l1)+(2,0)$) {2};
                    \node[draw,circle] (l3) at ($(l1)+(1,0)$) {2};
                    \draw[->] (l3) -- (l1) node[midway,above] {};
                    % \draw[->] (l3) -- (l2) node[midway,above] {$a$};
                }     
                \graphbox{$K$}{40mm}{0mm}{24mm}{15mm}{2mm}{-5mm}{
                    \coordinate (o) at (5mm,-3mm); 
                    \node[draw,circle] (l1) at ($(o)+(-10mm,0mm)$) {1};
                    \node[draw,circle] (l2) at ($(l1)+(1,0)$) {2};
                    % \node[draw,circle] (l3) at ($(l1)+(1,0)$) {$\ $};
                    \draw[->] (l2) -- (l1) node[midway,above] {};
                    % \draw[->] (l3) -- (l2) node[midway,above] {$a$};
                }    
                \graphbox{$R$}{70mm}{0mm}{40mm}{15mm}{2mm}{-5mm}{
                    \coordinate (o) at (-5mm,-3mm); 
                    \node[draw,circle] (l1) at ($(o)+(-10mm,0mm)$) {1};
                    % \node[draw,circle] (l2) at ($(l1)+(3,0)$) {2};
                    \node[draw,circle] (l3) at ($(l1)+(1,0)$) {2};
                    \node[draw,circle] (l4) at ($(l1)+(2,0)$) {3};
                    \draw[->] (l3) -- (l1) node[midway,above] {};
                    \draw[->] (l3) -- (l4) node[midway,above] {};
                    % \draw[->] (l4) -- (l2) node[midway,above] {$a$};
                }    
                \node () at (37mm,-8mm) {$\leftarrowtail$};
                \node () at (67mm,-8mm) {$\rightarrowtail$};
            \end{tikzpicture}
    \end{center}
    Let $R' \mathop{\in} D(R,X)$. $R'$ must contain node 3, otherwise it would be a subgraph of $r(K)$, violating Condition 2 of Definition~\ref{subgraph_counting:def:rx}. Node 3 must be connected to an interface node, otherwise Condition 1 of Definition~\ref{subgraph_counting:def:rx} would be violated. Therefore, $R'$ must include \raisebox{2pt}{\scalebox{0.6}{\tikz[baseline=-0.5ex]{
        \node [draw,circle] (x) at (0,0) {2};
        \node[draw,circle] (y) at (1,0) {3};
        \draw[->] (x) -- (y) {};
    }}}.
    The graph $R$ is not in $D(R,X)$ because it violates Condition 1. Thus, the set \( D(R,X) \) consists of the following graphs:
    \begin{itemize}
        \item $R'_1$:
    \raisebox{2pt}{\scalebox{0.6}{\tikz[baseline=-0.5ex]{
        \node [draw,circle] (x) at (0,0) {2};
        \node[draw,circle] (y) at (1,0) {3};
        \draw[->] (x) -- (y) {};
    }}},  
        \item $R'_2$:
    \raisebox{2pt}{\scalebox{0.6}{\tikz[baseline=-0.5ex]{
        \node [draw,circle] (node1) at (-1,0) {1};
        \node [draw,circle] (x) at (0,0) {2};
        \node[draw,circle] (y) at (1,0) {3};
        \draw[->] (x) -- (y) {};
    }}}.
    \end{itemize} 
\end{example}
We introduce the concept of non-increasing occurrences. Intuitively, if a rule $\rho$ is $X$-non-increasing, then for any rewriting step using $\rho$, there is an injective mapping from the set of \( X \)-occurrences implicitly created by the step to the set of \( X \)-occurrences implicitly destroyed by the step. This intuition is made precise in Lemma~\ref{subgraph_counting:lem:w_u_l_not_geq_r_not}.
\begin{definition}[$X$-non-increasing rule]
    \label{subgraph_counting:def:creates_more_x_on_the_left}
    % Let \(\rho \mathop{=} (L \overset{l}{\leftarrowtail} K \overset{r}{\rightarrowtail} R)\) be a rule and \(X\) a graph. 
    % Let \( \Psi \) be a function associating $R' \mathop{\in} D(R,X)$ to a homomorphism in $\operatorname{Mono}(R',L)$.
    % We say that \textbf{the number of occurrences of $X$ is non-increasing from left to right in any $\rho$-rewriting step under the mapping \(\Psi\)}
    % (or that $\rho$ is $X$-\textbf{non-increasing} under $\Psi$. When $\Psi$ is clear from context or irrelevant to the discussion, we may simply say that $\rho$ is $X$-\textbf{non-increasing}.)
    Let \(\rho \mathop{=} (L \overset{l}{\leftarrowtail} K \overset{r}{\rightarrowtail} R)\) be a rule and \(X\) a graph. 
    Let \( \Psi \) be a function associating $R' \mathop{\in} D(R,X)$ to a homomorphism in $\operatorname{Mono}(R',L)$.
    Rule $\rho$ is said to be $X$-\textbf{non-increasing} under $\Psi$ 
    if the following four conditions hold:
    \begin{enumerate}
        \item For all $R' \mathop{\in} D(R,X)$, we can construct the following diagram where all morphisms other than $\Psi(R')$, $l$ and $r$ are inclusion functions:
      \begin{center}
                \begin{tikzpicture} 
                    \node (k) at (0,-2) {$K$};
                    \node (k') at (0,0) {$K'$};
                    \node (l) at (-2,-2) {$L$};
                    \node (rb) at (-2,0) {$R'$};
                    \node (r) at (2,-2) {$R$};
                    \node (rb') at (2,0) {$R'$};

                    \draw[<-<]  (l) -- (k) node [midway,above] {$l$};
                    \draw[<-<]  (r) -- (k) node [midway,above] {$r$};
                    \draw[>->]  (k') -- (rb');
                    \draw[>->]  (rb') -- (r);
                    \draw[>->]  (k') -- (rb);
                    \draw[>->]  (k') -- (k);
                    \draw[>->]  (rb) -- (l) node [midway,left] {$\Psi(R')$};
                    \node () [at=($(k)!0.5!(rb)$)] {$PB$};
                    \node () [at=($(k)!0.5!(rb')$)] {$PB$}; 
                \end{tikzpicture}
        \end{center}
        % For all $R' \mathop{\in} D(R,X)$, in the diagram shown on the right, where $h_{R'R}$ is the inclusion function, both squares are pullbacks,
        % For all $R' \mathop{\in} D(R,X)$, there is a morphism \(h_{R'L}: R' \rightarrowtail L \) such that \trackedtext{both squares in the diagram shown on the right are pullbacks}, where all morphisms other than $h_{R'L}$, $l$ and $r$ are inclusion functions,
        \item \label{def:non_increasing:non_clapse} For all $R' \mathop{\in} D(R,X)$, for all nodes and edges $x$ in $R'$, if $ x \notin \operatorname{Im}(r)$ then 
        $\Psi(R')(x) \notin \operatorname{Im}(l)$,
        % $h_{R'L}(r') \notin \operatorname{Im}(l)$,
        \item \label{def:non_increasing_rule_img_edges_distinct} For all $R',R'' \mathop{\in} D(R,X)$, for all edges $x$ in $R'$ and $y$ in $R''$, if $x \mathop{\neq} y$, then $\Psi(R')(x) \mathop{\neq} \Psi(R'')(y)$,
        \item If $X$ has isolated nodes, then for all $R',R'' \mathop{\in} D(R,X)$ and nodes $x$ and $y$ in $R'$ and $R''$ respectively, if $x \mathop{\neq} y$, then $\Psi(R')(x) \mathop{\neq} \Psi(R'')(y)$.
    \end{enumerate} 
    When $\Psi$ is clear from context or irrelevant to the discussion, we may simply say that $\rho$ is $X$-\textbf{non-increasing}.
\end{definition}
% \begin{definition}[Non-increasing rule]
%     \label{def:creates_more_x_on_the_left}
%     Let \(\rho \mathop{=} (L \overset{l}{\leftarrowtail} K \overset{r}{\rightarrowtail} R)\) be a rule and \(X\) a ruler-graph. 
%     Rule \( \rho \) is \emph{\( X \)-non-increasing} if the following four conditions hold:
%     \newline
%     \vspace{1mm}
%     \noindent
%     \begin{minipage}{0.64\textwidth} 
%     \begin{enumerate}
%         \item For all $R' \mathop{\in} D(R,X)$, there is a morphism \(h_{R'L}: R' \rightarrowtail L \) such that both squares in the diagram shown on the right are pullbacks, where all morphisms other than $h_{R'L}$, $l$ and $r$ are inclusion functions, 
%         \item \label{def:non_increasing:non_clapse} For all $R' \mathop{\in} D(R,X)$, for all node or edge $r'$ in $R'$, if $ r' \notin \operatorname{Im}(r)$ then $h_{R'L}(r') \notin \operatorname{Im}(l)$,
%         \item \label{def:non_increasing_rule_img_edges_distinct} For all $R',R'' \mathop{\in} D(R,X)$, for all edges $x \mathop{\in} R'$ and $y \mathop{\in} R''$, if $x \mathop{\neq} y$, then $h_{R'L}(x) \mathop{\neq} h_{R''L}(y)$,
%         \item If $X$ has isolated nodes, then for all $R',R'' \mathop{\in} D(R,X)$ and nodes $x \mathop{\in} R', y \mathop{\in} R''$, if $x \mathop{\neq} y$, then $h_{R'L}(x) \mathop{\neq} h_{R''L}(y)$.
%     \end{enumerate} 
%     \end{minipage}
%     \hfill
%     \begin{minipage}{0.4\textwidth}
%         % \hfill
%         \begin{center}
%             % \resizebox{0.9\textwidth}{!}{ 
%                 % \begin{tikzpicture}[rotate=90]
%                 %     \node (k) {K}; 
%                 %     \node (k') [above=of k] {$K'$};
%                 %     \node (l) [left=of k] {$L$};
%                 %     \node (rb) [above=of l] {$R'$};
%                 %     \node (r) [right=of k] {$R$};
%                 %     \node (rb') [above=of r] {$R'$};
%                 %     \draw[<-<]  (l) -- (k) node [midway,below] {$l$};
%                 %     \draw[<-<]  (r) -- (k) node [midway,below] {$r$};
%                 %     \draw[>->]  (k') -- (rb');
%                 %     \draw[>->]  (rb') -- (r);
%                 %     \draw[>->]  (k') -- (rb);
%                 %     \draw[>->]  (k') -- (k);
%                 %     \draw[>->]  (rb) -- (l) node [midway,above,sloped] {$h_{R'L}$};
%                 %     \node () [at=($(k)!0.5!(rb)$)] {$PB$};
%                 %     \node () [at=($(k)!0.5!(rb')$)] {$PB$};
%                 % \end{tikzpicture}
%                 \begin{tikzpicture} 
%                     \node (k) {K}; 
%                     \node (k') [left=of k] {$K'$};
%                     \node (l) [above=of k] {$L$};
%                     \node (rb) [left=of l] {$R'$};
%                     \node (r) [below=of k] {$R$};
%                     \node (rb') [left=of r] {$R'$};
%                     \draw[<-<]  (l) -- (k) node [midway,right] {$l$};
%                     \draw[<-<]  (r) -- (k) node [midway,right] {$r$};
%                     \draw[>->]  (k') -- (rb');
%                     \draw[>->]  (rb') -- (r);
%                     \draw[>->]  (k') -- (rb);
%                     \draw[>->]  (k') -- (k);
%                     \draw[>->]  (rb) -- (l) node [midway,above] {$h_{R'L}$};
%                     \node () [at=($(k)!0.5!(rb)$)] {$PB$};
%                     \node () [at=($(k)!0.5!(rb')$)] {$PB$};
%                 \end{tikzpicture}
%             % }
%         \end{center}
%     \end{minipage}

%     %  \begin{enumerate}
%     %     % \item[(2)] \label{def:non_increasing:non_clapse} For all node or arrow $r'$ in $R'$, if $ r' \notin \operatorname{Im}(r)$ then $h_{R'L}(r') \notin \operatorname{Im}(l)$,
%     %     % \item there is a morphism $h_{R_XL}$ such that
%     %     %     \( h_{R'R_X} \mathop{\star} h_{R_XL} \mathop{=} h_{R'L} \) for all $R' \mathop{\in} D(R,X)$;
%     %     % \item[(3)] \label{def:non_increasing:edge_injective} For all $R',R'' \mathop{\in} D(R,X)$, for all edges $x \mathop{\in} R'$ and $y \mathop{\in} R''$, if $x \mathop{\neq} y$, then $h_{R'L}(x) \mathop{\neq} h_{R''L}(y)$,
%     %     %  \item $h$  is edge-injective 
%     %     % \item $\bigcup_{R' \mathop{\in} D(R,X)} h_{R'L}$ is an edge-injective morphism from $R_X$ to $L$;
%     %     % \item[(4)] \label{def:non_increasing:isolated_nodes} If $X$ has isolated nodes, then for all $R',R'' \mathop{\in} D(R,X)$ and nodes $x \mathop{\in} R', y \mathop{\in} R''$, if $x \mathop{\neq} y$, then $h_{R'L}(x) \mathop{\neq} h_{R''L}(y)$.
%     %     % \item $h$ is node-injective if $X$ has isolated nodes;
%     %  \end{enumerate}
%     %  \todo{est il util d'avoir "or arrow?" c'est une partie de edge injective, non ?}
% \end{definition} 
The first condition ensures that whenever $R' \mathop{\in} D(R,X)$ forms an $X$-occurrence in the result graph $H$, this occurrence has a corresponding $X$-occurrence relative to the morphism $h_{R'L}$ (as defined in Definition~\ref{def:x_occurrence}) in the host graph $G$.
The second condition guarantees that any $X$-occurrence implicitly created by the rewriting step has its corresponding $X$-occurrence not included in the context (Consequently, its corresponding $X$-occurrence is implicitly destroyed by the rewriting step). Example~\ref{subgraph_counting:ex:cond_2_necessaire} and Example~\ref{subgraph_counting:ex:cond_2_necessaire2} show the necessity of this condition.
The third and fourth conditions ensure that distinct $X$-occurrences created implicitly have distinct corresponding $X$-occurrences.
Example~\ref{subgraph_counting:ex:cond3_necessaire} and Example~\ref{subgraph_counting:ex:cond4_necessaire} show the necessity of these conditions.

The following example shows an $X$-non-increasing rule. 
\begin{example}
    \label{example:grs_aa:has_more_left}
    Consider the rule illustrated below and previously presented
    in Example~\ref{subgraph_counting:ex:grsaa_rx}.
    \begin{center}
        \begin{tikzpicture}
            \graphbox{$L$}{0mm}{0mm}{34mm}{15mm}{2mm}{-5mm}{
                \coordinate (o) at (0mm,-3mm); 
                \node[draw,circle] (l1) at ($(o)+(-10mm,0mm)$) {1};
                \node[draw,circle] (l2) at ($(l1)+(2,0)$) {2};
                \node[draw,circle] (l3) at ($(l1)+(1,0)$) {3};
                \draw[->] (l1) -- (l3) node[midway,above] {$a$};
                \draw[->] (l3) -- (l2) node[midway,above] {$a$};
            }     
            \graphbox{$K$}{40mm}{0mm}{24mm}{15mm}{2mm}{-5mm}{
                \coordinate (o) at (5mm,-3mm); 
                \node[draw,circle] (l1) at ($(o)+(-10mm,0mm)$) {1};
                \node[draw,circle] (l2) at ($(l1)+(1,0)$) {2};
            }    
            \graphbox{$R$}{70mm}{0mm}{45mm}{15mm}{2mm}{-5mm}{
                \coordinate (o) at (-5mm,-3mm); 
                \node[draw,circle] (l1) at ($(o)+(-10mm,0mm)$) {1};
                \node[draw,circle] (l2) at ($(l1)+(3,0)$) {2};
                \node[draw,circle] (l3) at ($(l1)+(1,0)$) {4};
                \node[draw,circle] (l4) at ($(l1)+(2,0)$) {5};
                \draw[->] (l1) -- (l3) node[midway,above] {$a$};
                \draw[->] (l3) -- (l4) node[midway,above] {$b$};
                \draw[->] (l4) -- (l2) node[midway,above] {$a$};
            }    

            \node () at (37mm,-8mm) {$\leftarrowtail$};
            \node () at (67mm,-8mm) {$\rightarrowtail$};

        \end{tikzpicture}
    \end{center}
    Let $X$ be the graph \tikz[baseline=-0.5ex]{
        \node[draw,circle] (x) at (0,0) { };
        \node[draw,circle] (y) at (1,0) { };
        \node[draw,circle] (z) at (2,0) { };
        \draw[->] (x) -- (y) node[midway, above] {$a$};
        \draw[<-] (z) -- (y) node[midway, above] {$a$};
    }.
    As explained in Example~\ref{subgraph_counting:ex:grsaa_rx}, the set \( D(R,X) \) consists of exactly four graphs: $R'_1$, $R'_2$, $R'_3$ and $R'_4$.
    The rule is $X$-non-increasing under the function $\Psi$ that maps each $R'_i$ to the inclusion morphism from $R'_i$ to $L$. specifically, it satisfies Condition 1 of Definition~\ref{subgraph_counting:def:creates_more_x_on_the_left}
     because the following diagrams
    %  in Figure~\ref{fig:subgraph_counting:decodkfjasddgsf} 
     can be constructed, and other conditions in Definition~\ref{subgraph_counting:def:creates_more_x_on_the_left} are straightforward to verify.
    %1
    \begin{center}
            \begin{tikzpicture}
                \graphbox{$R'_1$}{0mm}{0mm}{34mm}{15mm}{2mm}{-5mm}{
                    \coordinate (o) at (0mm,-3mm); 
                    \node (l1) at ($(o)+(-10mm,0mm)$) {};
                    \node[draw,circle] (l2) at ($(l1)+(2,0)$) {2};
                    \node[draw,circle] (l3) at ($(l1)+(1,0)$) {5};
                    % \draw[->] (l1) -- (l3) node[midway,above] {$a$};
                    \draw[->] (l3) -- (l2) node[midway,above] {$a$};
                }     
                \graphbox{$K'_1$}{40mm}{0mm}{24mm}{15mm}{2mm}{-5mm}{
                    \coordinate (o) at (5mm,-3mm); 
                    \node (l1) at ($(o)+(-10mm,0mm)$) {};
                    \node[draw,circle] (l2) at ($(l1)+(1,0)$) {2};
                    % \node[draw,circle] (l3) at ($(l1)+(1,0)$) {$\ $};
                    % \draw[->] (l1) -- (l3) node[midway,above] {$a$};
                    % \draw[->] (l3) -- (l2) node[midway,above] {$a$};
                }    
                \graphbox{$R'_1$}{70mm}{0mm}{45mm}{15mm}{2mm}{-5mm}{
                    \coordinate (o) at (-5mm,-3mm); 
                    \node (l1) at ($(o)+(-10mm,0mm)$) {};
                    \node[draw,circle] (l2) at ($(l1)+(3,0)$) {2};
                    % \node[draw,circle] (l3) at ($(l1)+(1,0)$) {4};
                    \node[draw,circle] (l4) at ($(l1)+(2,0)$) {5};
                    % \draw[->] (l1) -- (l3) node[midway,above] {$a$};
                    % \draw[->] (l3) -- (l4) node[midway,above] {$b$};
                    \draw[->] (l4) -- (l2) node[midway,above] {$a$};
                }    
                \graphbox{$L$}{0mm}{-20mm}{34mm}{15mm}{2mm}{-5mm}{
                    \coordinate (o) at (0mm,-3mm); 
                    \node[draw,circle] (l1) at ($(o)+(-10mm,0mm)$) {1};
                    \node[draw,circle] (l2) at ($(l1)+(2,0)$) {2};
                    \node[draw,circle] (l3) at ($(l1)+(1,0)$) {5};
                    \draw[->] (l1) -- (l3) node[midway,above] {$a$};
                    \draw[->] (l3) -- (l2) node[midway,above] {$a$};
                }     
                \graphbox{$K$}{40mm}{-20mm}{24mm}{15mm}{2mm}{-5mm}{
                    \coordinate (o) at (5mm,-3mm); 
                    \node[draw,circle] (l1) at ($(o)+(-10mm,0mm)$) {1};
                    \node[draw,circle] (l2) at ($(l1)+(1,0)$) {2};
                    % \node[draw,circle] (l3) at ($(l1)+(1,0)$) {$\ $};
                    % \draw[->] (l1) -- (l3) node[midway,above] {$a$};
                    % \draw[->] (l3) -- (l2) node[midway,above] {$a$};
                }    
                \graphbox{$R$}{70mm}{-20mm}{45mm}{15mm}{2mm}{-5mm}{
                    \coordinate (o) at (-5mm,-3mm); 
                    \node[draw,circle] (l1) at ($(o)+(-10mm,0mm)$) {1};
                    \node[draw,circle] (l2) at ($(l1)+(3,0)$) {2};
                    \node[draw,circle] (l3) at ($(l1)+(1,0)$) {4};
                    \node[draw,circle] (l4) at ($(l1)+(2,0)$) {5};
                    \draw[->] (l1) -- (l3) node[midway,above] {$a$};
                    \draw[->] (l3) -- (l4) node[midway,above] {$b$};
                    \draw[->] (l4) -- (l2) node[midway,above] {$a$};
                }    
                \node () at (37mm,-8mm) {$\leftarrowtail$};
                \node () at (17mm,-17mm) {$\Psi(R'_1)\downarrowtail$};
                \node () at (52mm,-17mm) {$\downarrowtail$};
                \node () at (92mm,-17mm) {$\downarrowtail$};
                \node () at (37mm,-18mm) {\text{PB}};
                \node () at (68mm,-18mm) {\text{PB}};
                \node () at (67mm,-8mm) {$\rightarrowtail$};
                \node () at (37mm,-28mm) {$\overset{l}{\leftarrowtail}$};
                \node () at (67mm,-28mm) {$\overset{r}{\rightarrowtail}$};
            \end{tikzpicture}
        \end{center}

    % 2
    \begin{center}
            \begin{tikzpicture}
                \graphbox{$R'_2$}{0mm}{0mm}{34mm}{15mm}{2mm}{-5mm}{
                    \coordinate (o) at (0mm,-3mm); 
                    \node[draw,circle] (l1) at ($(o)+(-10mm,0mm)$) {1};
                    \node[draw,circle] (l2) at ($(l1)+(2,0)$) {2};
                    \node[draw,circle] (l3) at ($(l1)+(1,0)$) {5};
                    % \draw[->] (l1) -- (l3) node[midway,above] {$a$};
                    \draw[->] (l3) -- (l2) node[midway,above] {$a$};
                }     
                \graphbox{$K'_2$}{40mm}{0mm}{24mm}{15mm}{2mm}{-5mm}{
                    \coordinate (o) at (5mm,-3mm); 
                    \node[draw,circle] (l1) at ($(o)+(-10mm,0mm)$) {1};
                    \node[draw,circle] (l2) at ($(l1)+(1,0)$) {2};
                    % \node[draw,circle] (l3) at ($(l1)+(1,0)$) {$\ $};
                    % \draw[->] (l1) -- (l3) node[midway,above] {$a$};
                    % \draw[->] (l3) -- (l2) node[midway,above] {$a$};
                }    
                \graphbox{$R'_2$}{70mm}{0mm}{45mm}{15mm}{2mm}{-5mm}{
                    \coordinate (o) at (-5mm,-3mm); 
                    \node[draw,circle] (l1) at ($(o)+(-10mm,0mm)$) {1};
                    \node[draw,circle] (l2) at ($(l1)+(3,0)$) {2};
                    % \node[draw,circle] (l3) at ($(l1)+(1,0)$) {4};
                    \node[draw,circle] (l4) at ($(l1)+(2,0)$) {5};
                    % \draw[->] (l1) -- (l3) node[midway,above] {$a$};
                    % \draw[->] (l3) -- (l4) node[midway,above] {$b$};
                    \draw[->] (l4) -- (l2) node[midway,above] {$a$};
                }    
                \graphbox{$L$}{0mm}{-20mm}{34mm}{15mm}{2mm}{-5mm}{
                    \coordinate (o) at (0mm,-3mm); 
                    \node[draw,circle] (l1) at ($(o)+(-10mm,0mm)$) {1};
                    \node[draw,circle] (l2) at ($(l1)+(2,0)$) {2};
                    \node[draw,circle] (l3) at ($(l1)+(1,0)$) {5};
                    \draw[->] (l1) -- (l3) node[midway,above] {$a$};
                    \draw[->] (l3) -- (l2) node[midway,above] {$a$};
                }     
                \graphbox{$K$}{40mm}{-20mm}{24mm}{15mm}{2mm}{-5mm}{
                    \coordinate (o) at (5mm,-3mm); 
                    \node[draw,circle] (l1) at ($(o)+(-10mm,0mm)$) {1};
                    \node[draw,circle] (l2) at ($(l1)+(1,0)$) {2};
                    % \node[draw,circle] (l3) at ($(l1)+(1,0)$) {$\ $};
                    % \draw[->] (l1) -- (l3) node[midway,above] {$a$};
                    % \draw[->] (l3) -- (l2) node[midway,above] {$a$};
                }    
                \graphbox{$R$}{70mm}{-20mm}{45mm}{15mm}{2mm}{-5mm}{
                    \coordinate (o) at (-5mm,-3mm); 
                    \node[draw,circle] (l1) at ($(o)+(-10mm,0mm)$) {1};
                    \node[draw,circle] (l2) at ($(l1)+(3,0)$) {2};
                    \node[draw,circle] (l3) at ($(l1)+(1,0)$) {4};
                    \node[draw,circle] (l4) at ($(l1)+(2,0)$) {5};
                    \draw[->] (l1) -- (l3) node[midway,above] {$a$};
                    \draw[->] (l3) -- (l4) node[midway,above] {$b$};
                    \draw[->] (l4) -- (l2) node[midway,above] {$a$};
                }    
                \node () at (37mm,-8mm) {$\leftarrowtail$};
                \node () at (17mm,-17mm) {$\Psi(R'_2)\downarrowtail$};
                \node () at (52mm,-17mm) {$\downarrowtail$};
                \node () at (92mm,-17mm) {$\downarrowtail$};
                \node () at (37mm,-18mm) {\text{PB}};
                \node () at (68mm,-18mm) {\text{PB}};
                \node () at (67mm,-8mm) {$\rightarrowtail$};
                \node () at (37mm,-28mm) {$\overset{l}{\leftarrowtail}$};
                \node () at (67mm,-28mm) {$\overset{r}{\rightarrowtail}$};
            \end{tikzpicture}
        \end{center}    
    
    %3
    \begin{center}
            \begin{tikzpicture}
                \graphbox{$R'_3$}{0mm}{0mm}{34mm}{15mm}{2mm}{-5mm}{
                    \coordinate (o) at (0mm,-3mm); 
                    \node[draw,circle] (l1) at ($(o)+(-10mm,0mm)$) {1};
                    % \node[draw,circle] (l2) at ($(l1)+(2,0)$) {2};
                    \node[draw,circle] (l3) at ($(l1)+(1,0)$) {4};
                    \draw[->] (l1) -- (l3) node[midway,above] {$a$};
                    % \draw[->] (l3) -- (l2) node[midway,above] {$a$};
                }     
                \graphbox{$K'_3$}{40mm}{0mm}{24mm}{15mm}{2mm}{-5mm}{
                    \coordinate (o) at (5mm,-3mm); 
                    \node[draw,circle] (l1) at ($(o)+(-10mm,0mm)$) {1};
                    % \node[draw,circle] (l2) at ($(l1)+(1,0)$) {2};
                    % \node[draw,circle] (l3) at ($(l1)+(1,0)$) {$\ $};
                    % \draw[->] (l1) -- (l3) node[midway,above] {$a$};
                    % \draw[->] (l3) -- (l2) node[midway,above] {$a$};
                }     
                \graphbox{$R'_3$}{70mm}{0mm}{45mm}{15mm}{2mm}{-5mm}{
                    \coordinate (o) at (-5mm,-3mm); 
                    \node[draw,circle] (l1) at ($(o)+(-10mm,0mm)$) {1};
                    % \node[draw,circle] (l2) at ($(l1)+(3,0)$) {4};
                    \node[draw,circle] (l3) at ($(l1)+(1,0)$) {4};
                    % \node[draw,circle] (l4) at ($(l1)+(2,0)$) {5};
                    \draw[->] (l1) -- (l3) node[midway,above] {$a$};
                    % \draw[->] (l3) -- (l4) node[midway,above] {$b$};
                    % \draw[->] (l4) -- (l2) node[midway,above] {$a$};
                }    
                \graphbox{$L$}{0mm}{-20mm}{34mm}{15mm}{2mm}{-5mm}{
                    \coordinate (o) at (0mm,-3mm); 
                    \node[draw,circle] (l1) at ($(o)+(-10mm,0mm)$) {1};
                    \node[draw,circle] (l2) at ($(l1)+(2,0)$) {2};
                    \node[draw,circle] (l3) at ($(l1)+(1,0)$) {4};
                    \draw[->] (l1) -- (l3) node[midway,above] {$a$};
                    \draw[->] (l3) -- (l2) node[midway,above] {$a$};
                }     
                \graphbox{$K$}{40mm}{-20mm}{24mm}{15mm}{2mm}{-5mm}{
                    \coordinate (o) at (5mm,-3mm); 
                    \node[draw,circle] (l1) at ($(o)+(-10mm,0mm)$) {1};
                    \node[draw,circle] (l2) at ($(l1)+(1,0)$) {2};
                    % \node[draw,circle] (l3) at ($(l1)+(1,0)$) {$\ $};
                    % \draw[->] (l1) -- (l3) node[midway,above] {$a$};
                    % \draw[->] (l3) -- (l2) node[midway,above] {$a$};
                }    
                \graphbox{$R$}{70mm}{-20mm}{45mm}{15mm}{2mm}{-5mm}{
                    \coordinate (o) at (-5mm,-3mm); 
                    \node[draw,circle] (l1) at ($(o)+(-10mm,0mm)$) {1};
                    \node[draw,circle] (l2) at ($(l1)+(3,0)$) {2};
                    \node[draw,circle] (l3) at ($(l1)+(1,0)$) {4};
                    \node[draw,circle] (l4) at ($(l1)+(2,0)$) {5};
                    \draw[->] (l1) -- (l3) node[midway,above] {$a$};
                    \draw[->] (l3) -- (l4) node[midway,above] {$b$};
                    \draw[->] (l4) -- (l2) node[midway,above] {$a$};
                }    
                \node () at (37mm,-8mm) {$\leftarrowtail$};
                \node () at (17mm,-17mm) {$\Psi(R'_3)\downarrowtail$};
                \node () at (52mm,-17mm) {$\downarrowtail$};
                \node () at (92mm,-17mm) {$\downarrowtail$};
                \node () at (37mm,-18mm) {\text{PB}};
                \node () at (68mm,-18mm) {\text{PB}};
                \node () at (67mm,-8mm) {$\rightarrowtail$};
                \node () at (37mm,-28mm) {$\overset{l}{\leftarrowtail}$};
                \node () at (67mm,-28mm) {$\overset{r}{\rightarrowtail}$};
            \end{tikzpicture}
    \end{center}


    %4
    \begin{center}
            \begin{tikzpicture}
                \graphbox{$R'_4$}{0mm}{0mm}{34mm}{15mm}{2mm}{-5mm}{
                    \coordinate (o) at (0mm,-3mm); 
                    \node[draw,circle] (l1) at ($(o)+(-10mm,0mm)$) {1};
                    \node[draw,circle] (l2) at ($(l1)+(2,0)$) {2};
                    \node[draw,circle] (l3) at ($(l1)+(1,0)$) {4};
                    \draw[->] (l1) -- (l3) node[midway,above] {$a$};
                    % \draw[->] (l3) -- (l2) node[midway,above] {$a$};
                }     
                \graphbox{$K'_4$}{40mm}{0mm}{24mm}{15mm}{2mm}{-5mm}{
                    \coordinate (o) at (5mm,-3mm); 
                    \node[draw,circle] (l1) at ($(o)+(-10mm,0mm)$) {1};
                    \node[draw,circle] (l2) at ($(l1)+(1,0)$) {2};
                    % \node[draw,circle] (l3) at ($(l1)+(1,0)$) {$\ $};
                    % \draw[->] (l1) -- (l3) node[midway,above] {$a$};
                    % \draw[->] (l3) -- (l2) node[midway,above] {$a$};
                }     
                \graphbox{$R'_4$}{70mm}{0mm}{45mm}{15mm}{2mm}{-5mm}{
                    \coordinate (o) at (-5mm,-3mm); 
                    \node[draw,circle] (l1) at ($(o)+(-10mm,0mm)$) {1};
                    \node[draw,circle] (l2) at ($(l1)+(3,0)$) {2};
                    \node[draw,circle] (l3) at ($(l1)+(1,0)$) {4};
                    % \node[draw,circle] (l4) at ($(l1)+(2,0)$) {5};
                    \draw[->] (l1) -- (l3) node[midway,above] {$a$};
                    % \draw[->] (l3) -- (l4) node[midway,above] {$b$};
                    % \draw[->] (l4) -- (l2) node[midway,above] {$a$};
                }    
                \graphbox{$L$}{0mm}{-20mm}{34mm}{15mm}{2mm}{-5mm}{
                    \coordinate (o) at (0mm,-3mm); 
                    \node[draw,circle] (l1) at ($(o)+(-10mm,0mm)$) {1};
                    \node[draw,circle] (l2) at ($(l1)+(2,0)$) {2};
                    \node[draw,circle] (l3) at ($(l1)+(1,0)$) {4};
                    \draw[->] (l1) -- (l3) node[midway,above] {$a$};
                    \draw[->] (l3) -- (l2) node[midway,above] {$a$};
                }     
                \graphbox{$K$}{40mm}{-20mm}{24mm}{15mm}{2mm}{-5mm}{
                    \coordinate (o) at (5mm,-3mm); 
                    \node[draw,circle] (l1) at ($(o)+(-10mm,0mm)$) {1};
                    \node[draw,circle] (l2) at ($(l1)+(1,0)$) {2};
                    % \node[draw,circle] (l3) at ($(l1)+(1,0)$) {$\ $};
                    % \draw[->] (l1) -- (l3) node[midway,above] {$a$};
                    % \draw[->] (l3) -- (l2) node[midway,above] {$a$};
                }    
                \graphbox{$R$}{70mm}{-20mm}{45mm}{15mm}{2mm}{-5mm}{
                    \coordinate (o) at (-5mm,-3mm); 
                    \node[draw,circle] (l1) at ($(o)+(-10mm,0mm)$) {1};
                    \node[draw,circle] (l2) at ($(l1)+(3,0)$) {2};
                    \node[draw,circle] (l3) at ($(l1)+(1,0)$) {4};
                    \node[draw,circle] (l4) at ($(l1)+(2,0)$) {5};
                    \draw[->] (l1) -- (l3) node[midway,above] {$a$};
                    \draw[->] (l3) -- (l4) node[midway,above] {$b$};
                    \draw[->] (l4) -- (l2) node[midway,above] {$a$};
                }    
                \node () at (37mm,-8mm) {$\leftarrowtail$};
                \node () at (17mm,-17mm) {$\Psi(R'_4)\downarrowtail$};
                \node () at (52mm,-17mm) {$\downarrowtail$};
                \node () at (92mm,-17mm) {$\downarrowtail$};
                \node () at (37mm,-18mm) {\text{PB}};
                \node () at (68mm,-18mm) {\text{PB}};
                \node () at (67mm,-8mm) {$\rightarrowtail$};
                \node () at (37mm,-28mm) {$\overset{l}{\leftarrowtail}$};
                \node () at (67mm,-28mm) {$\overset{r}{\rightarrowtail}$};
            \end{tikzpicture}
    \end{center}

    To illustrate the $X$-non-increasing property captured by Definition~\ref{subgraph_counting:def:creates_more_x_on_the_left},
    consider the rewriting step defined by the DPO diagram below.
    \begin{center}
        \resizebox{0.8\textwidth}{!}{
        \begin{tikzpicture}
            \graphbox{\( L \)}{0mm}{5mm}{34mm}{20mm}{2mm}{-5mm}{
                \coordinate (o) at (0mm,-8mm); 
                \node[draw,circle] (l1) at ($(o)+(-10mm,0mm)$) {1};
                \node[draw,circle] (l2) at ($(l1)+(2,0)$) {2};
                \node[draw,circle] (l3) at ($(l1)+(1,0)$) {3};
                \draw[] (l1) -- (l3) node[midway,above] {$a$};
                \draw[] (l3) -- (l2) node[midway,above] {$a$};
            } 

            \graphbox{\( K \)}{40mm}{5mm}{34mm}{20mm}{2mm}{-5mm}{
                \coordinate (o) at (0mm,-8mm); 
                \node[draw,circle] (l1) at ($(o)+(-10mm,0mm)$) {1};
                \node[draw,circle] (l2) at ($(l1)+(2,0)$) {2};
            }  

            \graphbox{\( R \)}{80mm}{5mm}{45mm}{20mm}{2mm}{-5mm}{
                \coordinate (o) at (-5mm,-8mm); 
                \node[draw,circle] (l1) at ($(o)+(-10mm,0mm)$) {1};
                \node[draw,circle] (l2) at ($(l1)+(3,0)$) {2};
                \node[draw,circle] (l3) at ($(l1)+(1,0)$) {4};
                \node[draw,circle] (l4) at ($(l1)+(2,0)$) {5};
                \draw[ ] (l1) -- (l3) node[midway,above] {$a$};
                \draw[ ] (l3) -- (l4) node[midway,above] {$b$};
                \draw[ ] (l4) -- (l2) node[midway,above] {$a$};
            }    

            \graphbox{\( G \)}{0mm}{-22mm}{34mm}{30mm}{2mm}{-10mm}{
                \coordinate (o) at (0mm,-3mm); 
                \node[draw,circle] (l1) at ($(o)+(-10mm,0mm)$) {1};
                \node[draw,circle] (l2) at ($(l1)+(2,0)$) {2};
                \node[draw,circle] (l3) at ($(l1)+(1,0)$) {3};
                \node[draw,circle] (l4) at ($(l2)+(0,-1)$) {6};
                \draw[] (l1) -- (l3) node[midway,above] {$a$};
                \draw[] (l3) -- (l2) node[midway,above] {$a$};
                \draw[ ] (l2) -- (l4) node[midway,right] {$a$};
                \node[draw,circle] (l6) at ($(l1)+(0,-1)$) {7};
                \draw[] (l1) -- (l6) node[midway,left] {$a$};
            }    

            \graphbox{\( C  \)}{40mm}{-22mm}{34mm}{30mm}{2mm}{-10mm}{
                \coordinate (o) at (0mm,-3mm); 
                \node[draw,circle] (l1) at ($(o)+(-10mm,0mm)$) {1};
                \node[draw,circle] (l2) at ($(l1)+(2,0)$) {2};
                \node[draw,circle] (l4) at ($(l2)+(0,-1)$) {6};
                \draw[ ] (l2) -- (l4) node[midway,right] {$a$};
                \node[ draw,circle] (l6) at ($(l1)+(0,-1)$) {7};
                \draw[ ] (l1) -- (l6) node[midway,left] {$a$};
            }    

            \graphbox{\( H  \)}{80mm}{-22mm}{45mm}{30mm}{2mm}{-10mm}{
                \coordinate (o) at (-5mm,-3mm); 
                \node[draw,circle] (l1) at ($(o)+(-10mm,0mm)$) {1};
                \node[draw,circle] (l2) at ($(l1)+(3,0)$) {2};
                \node[draw,circle] (l3) at ($(l1)+(1,0)$) {4};
                \node[draw,circle] (l4) at ($(l1)+(2,0)$) {5};
                \node[ draw,circle] (l5) at ($(l2)+(0,-1)$) {6};
                \node[ draw,circle] (l6) at ($(l1)+(0,-1)$) {7};
                \draw[ ] (l1) -- (l6) node[midway,left] {$a$};
                \draw[] (l1) -- (l3) node[midway,above] {$a$};
                \draw[] (l3) -- (l4) node[midway,above] {$b$};
                \draw[ ] (l4) -- (l2) node[midway,above] {$a$};
                \draw[ ] (l2) -- (l5) node[midway,right] {$a$};
            }    
            \node () at (37mm,-18mm) {\text{PO}};
            \node () at (78mm,-18mm) {\text{PO}};
            \node () at (37mm,-8mm) {\( \leftarrowtail \)}; % K -> L
            \node () at (77mm,-8mm) {\( \rightarrowtail \)}; % K -> R
            \node () at (17mm,-18mm) {\( m\ \downarrowtail \)};
            \node () at (37mm,-33mm) {\( \leftarrowtail \)};
            \node () at (52mm,-18mm) {\( \downarrowtail \)};
            \node () at (92mm,-18mm) {\( \downarrowtail \)};
            \node () at (77mm,-33mm) {\( \rightarrowtail \)}; % C -> H
        \end{tikzpicture}
        }
    \end{center}
    Whenever \( R'_i \) (for \( 1 \leq i \leq 4 \)) forms an \( X \)-occurrence $x$ in $H$ that is included neither in \( R \) nor in $C$ with a subgraph $C' \mathop{\subseteq} C \mathop{\cap} x \mathop{\subseteq} C$, the graph
    \( \operatorname{\Psi(R')} \mathop{\subseteq} L\) forms a \( X \)-occurrence in $G$ with $C'$ that is included neither in \( L \) nor in $C$.
    Furthermore, distinct $X$-occurrences in $H$ that are included neither in \( R \) nor in $C$ have distinct corresponding $X$-occurrences in $G$ that are included neither in \( L \) nor in $C$.
\end{example}

% cond 2 necessaire  1
\begin{example} 
    \label{subgraph_counting:ex:cond_2_necessaire}
    Let $X$ be the graph 
    \tikz[baseline=-0.5ex]{ 
            \node[draw,circle] (x) at (0,0) {}; 
            \node[draw,circle] (y) at (1,0) {};
            \node[draw,circle] (z) at (2,0) {};
            \draw[->] (x) -- (y)   {};
            \draw[->] (y) -- (z)   {};
    }. The rewriting rule shown below
    % in Figure~\ref{fig:subgraph_counting:rx_counter_ex} 
    is not necessarily $X$-non-increasing.
    % \begin{figure}[H]
    %     \centering
    \begin{center}
            \begin{tikzpicture}
                \graphbox{$L$}{0mm}{0mm}{34mm}{15mm}{2mm}{-5mm}{
                    \coordinate (o) at (0mm,-3mm); 
                    \node[draw,circle] (l1) at ($(o)+(-10mm,0mm)$) {1};
                    % \node[draw,circle] (l2) at ($(l1)+(2,0)$) {2};
                    \node[draw,circle] (l3) at ($(l1)+(1,0)$) {2};
                    \draw[->] (l3) -- (l1) node[midway,above] {};
                    % \draw[->] (l3) -- (l2) node[midway,above] {$a$};
                }     
                \graphbox{$K$}{40mm}{0mm}{24mm}{15mm}{2mm}{-5mm}{
                    \coordinate (o) at (5mm,-3mm); 
                    \node[draw,circle] (l1) at ($(o)+(-10mm,0mm)$) {1};
                    \node[draw,circle] (l2) at ($(l1)+(1,0)$) {2};
                    % \node[draw,circle] (l3) at ($(l1)+(1,0)$) {$\ $};
                    \draw[->] (l2) -- (l1) node[midway,above] {};
                    % \draw[->] (l3) -- (l2) node[midway,above] {$a$};
                }    
                \graphbox{$R$}{70mm}{0mm}{45mm}{15mm}{2mm}{-5mm}{
                    \coordinate (o) at (-5mm,-3mm); 
                    \node[draw,circle] (l1) at ($(o)+(-10mm,0mm)$) {1};
                    % \node[draw,circle] (l2) at ($(l1)+(3,0)$) {2};
                    \node[draw,circle] (l3) at ($(l1)+(1,0)$) {2};
                    \node[draw,circle] (l4) at ($(l1)+(2,0)$) {3};
                    \draw[->] (l3) -- (l1) node[midway,above] {};
                    \draw[->] (l3) -- (l4) node[midway,above] {};
                    % \draw[->] (l4) -- (l2) node[midway,above] {$a$};
                }    
                \node () at (37mm,-8mm) {$\leftarrowtail$};
                \node () at (67mm,-8mm) {$\rightarrowtail$};
            \end{tikzpicture}
      \end{center}

    As shown in Example~\ref{subgraph_counting:ex:rx_counter_ex}, \( D(R,X) \) contains the graph $R'$:
    \raisebox{2pt}{
        \scalebox{0.6}{\tikz[baseline=-0.5ex]{
        \node [draw,circle] (x) at (0,0) {2};
        \node[draw,circle] (y) at (1,0) {3};
        \draw[->] (x) -- (y) {};
    }}}. The unique monomorphism $h_{R'L}:R' \rightarrowtail L$ fails the second condition of Definition~\ref{subgraph_counting:def:creates_more_x_on_the_left}. 
    %  as illustrated below 
    % \begin{center}
    %     \resizebox{0.6\textwidth}{!}{
    %     \begin{tikzpicture}
    %             \graphbox{$R'$}{40mm}{-20mm}{24mm}{15mm}{2mm}{-5mm}{
    %                 \coordinate (o) at (5mm,-3mm); 
    %                 \node[draw,circle] (l1) at ($(o)+(-10mm,0mm)$) {2};
    %                 \node[draw,circle] (l2) at ($(l1)+(1,0)$) {3};
    %                 % \node[draw,circle] (l3) at ($(l1)+(1,0)$) {$\ $};
    %                 \draw[->] (l1) -- (l2) node[midway,above] {};
    %                 % \draw[->] (l3) -- (l2) node[midway,above] {$a$};
    %             }    
    %             \graphbox{$L$}{70mm}{-20mm}{45mm}{15mm}{2mm}{-5mm}{
    %                 \coordinate (o) at (-5mm,-3mm); 
    %                 \node[draw,circle] (l1) at ($(o)+(-10mm,0mm)$) {3};
    %                 % \node[draw,circle] (l2) at ($(l1)+(3,0)$) {2};
    %                 \node[draw,circle] (l2) at ($(l1)+(1,0)$) {2};
    %                 \draw[->] (l2) -- (l1) node[midway,above] {};
    %                 % \draw[->] (l4) -- (l2) node[midway,above] {$a$};
    %             }    
    %             \node () at (67mm,-28mm) {$\rightarrowtail$};
    %         \end{tikzpicture}
    %     }
    % \end{center}
    For any rewriting step using this rule, the corresponding $X$-occurrence relative to $h_{R'L}$ of any implicitly created $X$-occurrence is included in the context.
\end{example}
% cond 2 necessaire  2
\begin{example}
    \label{subgraph_counting:ex:cond_2_necessaire2}
    Let $X$ be the graph 
    \tikz[baseline=-0.5ex]{ 
            \node[draw,circle] (x) at (0,0) {}; 
            \node[draw,circle] (y) at (1,0) {};
            \node[draw,circle] (z) at (2,0) {};
            \draw[->] (x) -- (y)   {};
            % \draw[->] (y) -- (z)   {};
    }. The rewriting rule shown below
    % in Figure~\ref{fig:subgraph_counting:degkdjfqaaacodkfjasddgsf} 
    is not necessarily $X$-non-increasing.
    % \begin{figure}[H]
    %     \centering
    \begin{center}
            \begin{tikzpicture}
                \graphbox{$L$}{0mm}{0mm}{34mm}{15mm}{2mm}{-5mm}{
                    \coordinate (o) at (0mm,-3mm); 
                    \node[draw,circle] (l1) at ($(o)+(-10mm,0mm)$) {1};
                    \node[draw,circle] (l3) at ($(l1)+(1,0)$) {2};
                }     
                \graphbox{$K$}{40mm}{0mm}{24mm}{15mm}{2mm}{-5mm}{
                    \coordinate (o) at (5mm,-3mm); 
                    \node[draw,circle] (l1) at ($(o)+(-10mm,0mm)$) {1};
                    \node[draw,circle] (l2) at ($(l1)+(1,0)$) {2};
                }    
                \graphbox{$R$}{70mm}{0mm}{45mm}{15mm}{2mm}{-5mm}{
                    \coordinate (o) at (-5mm,-3mm); 
                    \node[draw,circle] (l1) at ($(o)+(-10mm,0mm)$) {1};
                    \node[draw,circle] (l3) at ($(l1)+(1,0)$) {2};
                    \node[draw,circle] (l4) at ($(l1)+(2,0)$) {3};
                }    
                \node () at (37mm,-8mm) {$\leftarrowtail$};
                \node () at (67mm,-8mm) {$\rightarrowtail$};
            \end{tikzpicture}
    \end{center}
    The set \( D(R,X) \) contains $R'$:
    \raisebox{2pt}{ \scalebox{0.6}{\tikz[baseline=-0.5ex]{
        \node [draw,circle] (x) at (0,0) {1};
        \node[draw,circle] (y) at (1,0) {3};
    }}}.  
    Despite the existence of a unique monomorphism $h_{R'L}: R' \rightarrowtail L$ preserving interface elements, this rule fails the second condition of Definition~\ref{subgraph_counting:def:creates_more_x_on_the_left}.
    For any rewriting step using this rule, the corresponding $X$-occurrence relative to $h_{R'L}$ of any implicitly created $X$-occurrence is included in the context.
\end{example}
\begin{example}
    \label{subgraph_counting:ex:cond3_necessaire}
    Let $X$ be the graph 
    \tikz[baseline=-0.5ex]{ 
            \node[draw,circle] (x) at (0,0) { }; 
            \node[draw,circle] (y) at (1,0) { };
            \node[draw,circle] (z) at (2,0) { };
            \draw[->] (x) -- (y)   {};
            \draw[->] (y) -- (z) {};
    }. The rewriting rule shown below
    % in Figure~\ref{fig:subgraph_counting:decodkfjasdddfsfdsagsf} 
    is not necessarily $X$-non-increasing.
    % \begin{figure}[H]
    %     \centering
    \begin{center}
            \begin{tikzpicture}
                \graphbox{$L$}{0mm}{0mm}{30mm}{20mm}{0}{0}{
                    \node[draw,circle]  (x) at (-6mm,-12mm) {x};
                    \node[draw,circle] (y) at (6mm,-12mm) {};
                    % \node[draw,circle]  (z) at (6mm,0mm) {};
                    \draw[->]  (x) to (y);
                    % \draw[->] (y) to[bend right=20] (z);
                    % \draw[->]  (z) to[bend right=20] (y);
                }
                \graphbox{$K$}{40mm}{0mm}{30mm}{20mm}{0}{0}{
                    \node[draw,circle]  (x) at (-6mm,-12mm) {x};
                    % \node[draw,circle]  (y) at (6mm,-12mm) {y};
                }
                \graphbox{$R$}{80mm}{0mm}{30mm}{20mm}{0}{0}{
                    \node[draw,circle]  (x) at (-6mm,-12mm) {x};
                        \node[draw,circle]  (y) at (6mm,-8mm) {y};
                        \node[draw,circle]  (z) at (6mm,-16mm) {z};
                        \draw[->]  (x) to (y);
                        \draw[->]  (x) to (z);
                }
                \node () at (35mm,-12mm) {$\leftarrowtail$};
                \node () at (75mm,-12mm) {$\rightarrowtail$};
            \end{tikzpicture}
    \end{center}
    The set \( D(R,X) \) contains exactly two elements $R'$:
    \raisebox{2pt}{ 
        \scalebox{0.6}{\tikz[baseline=-0.5ex]{
        \node [draw,circle] (x) at (0,0) {x};
        \node[draw,circle] (y) at (1,0) {y};
        \draw[->] (x) -- (y) {};
    }}} and $R''$:\raisebox{2pt}{ 
        \scalebox{0.6}{\tikz[baseline=-0.5ex]{
        \node [draw,circle] (x) at (0,0) {x};
        \node[draw,circle] (y) at (1,0) {z};
        \draw[->] (x) -- (y) {};
    }}}. 
    Despite unique monomorphisms \( h_{R'L}: R' \rightarrowtail L \) and \( h_{R''L}: R'' \rightarrowtail L \) preserving interface elements, they fail the third condition of Definition~\ref{subgraph_counting:def:creates_more_x_on_the_left}. 
    For any rewriting step using this rule, implicitly created $X$-occurrences with the same subgraph of the context have the same corresponding $X$-occurrence. 
\end{example}
 
\begin{example}
    \label{subgraph_counting:ex:cond4_necessaire} 
    Let $X$ be the graph 
    \tikz[baseline=-0.5ex]{ 
            \node[draw,circle] (x) at (0,0) { }; 
            \node[draw,circle] (y) at (1,0) { };
            \node[draw,circle] (z) at (2,0) { };
            \draw[->] (x) -- (y)   {};
    }, which has an isolated node. The rewriting rule shown below
    % in Figure~\ref{fig:subgraph_counting:decodkfjasddgdfsgassf} 
    is not necessarily $X$-non-increasing.
    % \begin{figure}[H]
    %     \centering
    \begin{center}
            \begin{tikzpicture}
                \graphbox{$L$}{0mm}{0mm}{30mm}{20mm}{0}{0}{
                    \node[draw,circle]  (x) at (-6mm,-10mm) {x};
                    \node[draw,circle] (y) at (6mm,-10mm) {y};
                }
                \graphbox{$K$}{40mm}{0mm}{30mm}{20mm}{0}{0}{
                    \node[draw,circle]  (x) at (-6mm,-10mm) {x};
                }
                \graphbox{$R$}{80mm}{0mm}{30mm}{20mm}{0}{0}{
                    \node[draw,circle]  (x) at (-6mm,-10mm) {x};
                        \node[draw,circle]  (y) at (6mm,-6mm) {z};
                        \node[draw,circle]  (z) at (6mm,-14mm) {w};
                }
                \node () at (35mm,-10mm) {$\leftarrowtail$};
                \node () at (75mm,-10mm) {$\rightarrowtail$};
            \end{tikzpicture}
    \end{center}
    The set \( D(R,X) \) contains exactly two elements $R'$:
    \raisebox{2pt}{\scalebox{0.6}{\tikz[baseline=-0.5ex]{
        \node [draw,circle] (x) at (0,0) {x};
        \node[draw,circle] (y) at (1,0) {z};
        % \draw[->] (x) -- (y) {};
    }}} and $R''$:
    \raisebox{2pt}{\scalebox{0.6}{\tikz[baseline=-0.5ex]{
        \node [draw,circle] (x) at (0,0) {x};
        \node[draw,circle] (y) at (1,0) {w};
        % \draw[->] (x) -- (y) {};
    }}}. 
    There are unique monomorphisms $h_{R'L}:R' \rightarrowtail L$ and $h_{R''L}:R'' \rightarrowtail L$ preserving interface elements, but they fail the fourth condition of Definition~\ref{subgraph_counting:def:creates_more_x_on_the_left}.
    For any rewriting step using this rule, implicitly created $X$-occurrences with the same subgraph of the context have the same corresponding $X$-occurrence.
\end{example}


\section{Solution to the key challenge}
\label{subgraph_counting:sec:solution_to_the_key_challenge}
% the cardinality of the set of \( X \)-occurrences implicitly destroyed by the rewriting step is at least as large as the cardinality of the set of \( X \)-occurrences implicitly created by the rewriting step.
% There are more implicitly destroyed X-occurrences than implicitly created X-occurrences.
Using the notion of X-non-increasing, we state the following lemma.
 It formalizes the intuition that if $\rho$ is $X$-non-increasing, then, for any rewriting step using $\rho$, 
more X-occurrences are implicitly destroyed than implicitly created by the rewriting step.
\begin{lemma} 
    \label{subgraph_counting:lem:w_u_l_not_geq_r_not}
        Let $X$ be a ruler-graph and $\rho \mathop{=} (L \overset{l}{\leftarrowtail} K \overset{r}{\rightarrowtail} R)$ an injective DPO rewriting rule.
        Suppose that $\rho$ is $X$-non-increasing. For every rewriting step induced by the following DPO diagram:
        \begin{center}
            \begin{tikzpicture}
                \node (k) at (0,1) {K};
                \node (l) at (-2,1) {L};
                \node (r) at (2,1) {R};
                \node (c) at (0,-1) {C};
                \node (g) at (-2,-1) {G};
                \node (h) at (2,-1) {H};
                \draw[<-<]  (l) -- (k) node [midway,below] {$l$};
                \draw[>->]  (k) -- (r) node [midway,below] {$r$};
                \draw[>->] (c) -- (g) node [midway, above] {$l'$};
                \draw[>->] (c) -- (h) node [midway,above] {$r'$};
                \draw[>->] (l) -- (g) node[midway, right] {$m$};
                \draw[>->] (r) -- (h) node[midway, left] {$m'$};
                \draw[>->] (k) -- (c) node[midway, left] {};
                \node () [at=($(l)!0.5!(c)$)] {$PO$};
                \node () [at=($(r)!0.5!(c)$)] {$PO$};
            \end{tikzpicture}
        \end{center}
       The following inequality holds:
        \[
            |\operatorname{Mono}(X, G, \lnot m, \lnot l')| \mathop{\geq} |\operatorname{Mono}(X, H, \lnot m', \lnot r')|
        \]
\end{lemma}
 
 
% \subsection{Termination criterion}
% \label{sec:termination} 
% Finally, we present a lemma that allows us to overapproximate the change in weight of a graph when a rule is applied, followed by our termination criterion. 

\begin{lemma}[Decreasing step]
    \label{subgraph_counting:lem:w_g_geq_w_h_leq}
    Let $\rho \mathop{=} (L \overset{l}{\leftarrowtail} K \overset{r}{\rightarrowtail} R)$ be an injective DPO rewriting rule,
    \( \mathbb{X} \) a set of ruler-graphs,
    \( s_{\mathbb{X}} \mathop{\colon} \mathbb{X} \mathop{\to} \mathbb{N} \) a weight function,
    and \( G \mathop{\Rightarrow}_{\rho,\mathfrak{M}} H \) a rewriting step. 
    If $\rho$ is \( X \)-non-increasing for every ruler-graph \( X \mathop{\in} \mathbb{X} \), then 
    $$
        w_{s_\mathbb{X}}(G) - w_{s_\mathbb{X}}(H) 
        \mathop{\geq} 
        w_{s_\mathbb{X}}(L) - w_{s_\mathbb{X}}(R).
    $$
\end{lemma}
\begin{proof}
    \iflongversion
        See~\textsection~\ref{subgraph_counting:proof:lem:w_g_geq_w_h_leq}.
    \else
        See \cite[Lemma 41]{qiu2025termination}.
    \fi 
\end{proof} 
\begin{theorem}[Termination] 
    \label{subgraph_counting:thm:termination_grs}
    Let \(\mathcal{A}\) and \(\mathcal{B}\) be sets of injective DPO rewriting rules, $\mathbb{X}$ a set of ruler-graphs, and $s_\mathbb{X}$ a weight function. If the following conditions hold:
    \begin{enumerate}
        \item  for every $\rho \mathop{\in} \mathcal{A} \mathop{\cup} \mathcal{B}$ and for every $X \mathop{\in} \mathbb{X}$, the rule $\rho$ is $X$-non-increasing,
        \item for every \(\rho \mathop{\in} \mathcal{A}\), we have \( w_{s_\mathbb{X}}(lhs(\rho)) \mathop{>} w_{s_\mathbb{X}}(rhs(\rho)) \),
        \item for every \(\rho \mathop{\in} \mathcal{B}\), we have \( w_{s_\mathbb{X}}(lhs(\rho)) \mathop{\geq} w_{s_\mathbb{X}}(rhs(\rho)) \).
    \end{enumerate}
    Then \(\mathop{\Rightarrow}_{\mathcal{A},\mathcal{M}}\) terminates relative to \(\mathop{\Rightarrow}_{\mathcal{B},\mathcal{M}}\).
\end{theorem}
\begin{proof}
    \iflongversion
        See~\textsection~\ref{subgraph_counting:proof:lem:w_g_geq_w_h_leq}.
    \else
        See \cite[Lemma 41]{qiu2025termination}.
    \fi 
\end{proof} 
% \noindent Our technique is described for DPO rewriting with injective matches due to their expressiveness, but it also extends to non-injective matches~\cite{habel2001double}.

% \begin{definition}[\cite{habel2001double}] 
%     \ \newline 
%     \noindent
%     \begin{minipage}{0.7\textwidth}
%         Given a rule $\rho \mathop{=} L \mathop{\leftarrow} K \mathop{\rightarrow} R$, a rule $\rho' \mathop{=} (L' \mathop{\leftarrow} K' \mathop{\rightarrow} R')$ is called a \textbf{quotient rule} of $\rho$ if there exists a DPO diagram (shown on the right)
%     where all vertical graph morphisms are surjective. The set of quotient rules of $\rho$ is denoted by $Q(\rho)$, and 
%     for a rule set $\mathcal{A}$, we write $Q(\mathcal{A}) \mathop{=} \bigcup_{\rho\in\mathcal{A}} Q(\rho)$.
%     \end{minipage}
%     \hfill
%     \begin{minipage}{0.29\textwidth}
%         \hfill
%         \begin{tikzpicture}[scale=0.7]
%             \node (k) at (0,0) {K};
%             \node (l) at (-2,0) {L};
%             \node (r) at (2,0) {R};
%             \node (k') at (0,-2) {K'};
%             \node (l') at (-2,-2) {L'};
%             \node (r') at (2,-2)  {R'};
%             \draw[->] (k) -> (l);
%             \draw[->] (k) -> (r); 
%             \draw[->] (k') -> (l'); 
%             \draw[->] (k') -> (r'); 
%             \draw[->] (k) -> (k'); 
%             \draw[->] (l) -> (l'); 
%             \draw[->] (r) -> (r'); 
%             \node () [at=($(l)!0.5!(k')$)] {$PO$};
%             \node () [at=($(r)!0.5!(k')$)] {$PO$};
%         \end{tikzpicture}
%     \end{minipage}
% \end{definition}

% \begin{lemma}[\cite{habel2001double}]
%     Let $\rho$ be a DPO rewriting rule. For any graphs $G$ and $H$,
%     $G \mathop{\Rightarrow}_{\rho,\mathfrak{F}} H$ if and only if $G \mathop{\Rightarrow}_{Q(\rho),\mathfrak{M}} H$.
% \end{lemma}
% \begin{corollary}
%     \label{cor:termination}
%     Let \(\mathcal{A}\) and \(\mathcal{B}\) be sets of injective DPO rewriting rules. 
%     The rewriting relation $\mathop{\Rightarrow}_{\mathcal{A},\mathfrak{F}}$ terminates relative to $\mathop{\Rightarrow}_{\mathcal{B},\mathfrak{F}}$ 
%     if 
%     $\mathop{\Rightarrow}_{Q(\mathcal{A}),\mathfrak{M}}$ terminates relative to $\mathop{\Rightarrow}_{Q(\mathcal{B}),\mathfrak{M}}$.
% \end{corollary} 

% \section{Examples}
% \label{sec:examples}
% While the type graph method~\cite{bruggink2014termination, bruggink2015proving,endrullis2024generalized_arxiv_v2} can prove termination for this example, it requires constructing an appropriate weighted type graph, a task that is difficult in general~\cite[\textsection 6]{bruggink2015proving}. 
In contrast, our criterion only requires checking simpler conditions.
\begin{example}
    \label{ex:termination:grsaa}
    Consider the rewriting rule $\rho$ from \autoref{ex:grsaa_rx}. Let $X$ be the ruler-graph \tikz[baseline=-0.5ex]{
        \node (x) at (0,0) {$\bullet$};
        \node (y) at (1,0) {$\bullet$ }; 
        \node (z) at (2,0) { $\bullet$};
        \draw[->] (x) -- (y) node[midway, above] {$a$};
        \draw[<-] (z) -- (y) node[midway, above] {$a$};
    }, $\mathbb{X} = \{X\}$ and $s_\mathbb{X}(X) = 1$. The rule is $X$-non-increasing as shown in \autoref{example:grs_aa:has_more_left}. 
    Since \(w_{s_\mathbb{X}}(\operatorname{lhs}(\rho)) = 1 > 0 = w_{s_\mathbb{X}}(\operatorname{rhs}(\rho)\)),
    it terminates by \autoref{thm:termination_grs}.
\end{example}
We consider an example for which the techniques from \cite{bruggink2014termination,bruggink2015proving,endrullis2024generalized_arxiv_v2,plump2018modular,overbeek2024termination_lmcs} fail.
% \begin{example}
%     \label{ex_contrib_variant}
%     The rewriting rule illustrated below is a variant of a rewriting rule presented in \cite[Example 6]{plump2018modular}, obtained by removing all edges in the interface.
    
%     % \begin{figure}[hbt] 
%     %     \center
%     \begin{center}
%         \resizebox{0.6\textwidth}{!}{
%             \begin{tikzpicture}
%                 \graphbox{$L$}{0mm}{0mm}{35mm}{35mm}{2mm}{-5mm}{
%                     \coordinate (delta) at (0,-18mm);
%                     \node[draw,circle] (l1) at ($(delta) + (-1,1.5)$) {1};
%                     \node[draw,circle] (l2) at ($(delta) + (1,1.5)$) {2};
%                     \node[draw,circle] (l3) at ($(delta) + (0,0)$) {3};
%                     \draw[->] (l1) -- (l3) node[midway,left] {s};
%                     \draw[->] (l2) -- (l3) node[midway,right] {s};
%                     \draw[->] (l3) edge [loop below] node {0} (l3);
%                 }
%                 \graphbox{$K$}{40mm}{0mm}{35mm}{35mm}{2mm}{-5mm}{
%                     \coordinate (delta) at (0,-18mm);
%                     \coordinate (interfaceorigin) at ($(delta) +(5,0)$);
%                     \node[draw,circle] (r1) at ($(delta) +(-1,1.5)$) {1};
%                     \node[draw,circle] (r2) at ($(delta) +(0.5,1.5)$) {2};
%                     \node[draw,circle] (r3) at ($(delta) + (0,0)$) {3};
%                     % \draw[->] (r1) -- (r3) node[midway,left] {s};
%                     % \draw[->] (r3) edge [loop below] node {0} (r3);
%                 }
%                 \graphbox{$R$}{80mm}{0mm}{50mm}{35mm}{2mm}{-5mm}{
%                     \coordinate (delta) at (-10mm,-18mm);
%                     \node[draw,circle] (r1) at ($(delta) + (-1,1.5)$) {1};
%                     \node[draw,circle] (r2) at ($(delta) + (0.5,1.5)$) {2};
%                     \node[draw,circle] (r3) at ($(delta) + (0,0)$) {3};
%                     \node[draw,circle] (r4) at ($(delta) + (1,0)$) {4};
%                     \draw[->] (r1) -- (r3) node[midway,left] {s};
%                     \draw[->] (r2) -- (r4) node[midway,right] {s};
%                     \draw[->] (r4) edge [loop below] node {0} (r4);
%                     \draw[->] (r3) edge [loop below] node {0} (r3);
%                     \node[draw,circle] (r5) at ($(r2) + (1.5,0)$) {};
%                     \draw[->] (r5) edge [loop below] node {0} (r5);
%                     \draw[->] (r5) edge [loop right] node {0} (r5);
%                     \draw[->] (r5) edge [loop left] node {0} (r5);
%                 }
%                 % \graphbox{$R_x$}{40mm}{40mm}{35mm}{35mm}{2mm}{-5mm}{
%                 %     \coordinate (delta) at (0,-18mm);
%                 %     \coordinate (rxorigin) at ($(interfaceorigin)+(0,6)$);
%                 %     \node[draw,circle] (r1) at ($(delta) + (-1,1.5)$) {1};
%                 %     \node[draw,circle] (r2) at ($(delta) +  (0.5,1.5)$) {2};
%                 %     \node[draw,circle] (r3) at ($(delta) +  (0,0)$) {3};
%                 %     \draw[->] (r1) -- (r3) node[midway,left] {s};
%                 %     % \draw[->] (r3) edge [loop below] node {0} (r3);
%                 % }
%                 \node () at (38mm,-18mm) {$\leftarrowtail$};
%                 \node () at (77mm,-18mm) {$\rightarrowtail$};
%                 % \node () at (57mm,2mm) {$\uparrowtail$};
%                 % \node () at (38mm,2mm) {$\swarrowtail$};
%                 % \node () at (79mm,2mm) {$\searrowtail$};
%             \end{tikzpicture}
%             }
%     \end{center}
%         %     \caption{Diagram of \autoref{ex_contrib_variant}}
%         %     \label{fig:contrib_variant}
%         % \end{figure}
%         Let $X$ be the graph 
%         \tikz[baseline=-0.5ex]{ 
%                 \node (x) at (0,0) {$\bullet$}; 
%                 \node (y) at (1,0) {$\bullet$};
%                 \node (z) at (2,0) {$\bullet$};
%                 \draw[->] (x) -- (y) node[midway, above] {$s$};
%                 \draw[->] (z) -- (y) node[midway, above] {$s$};
%         } with weight $1$ and $\mathbb{X} = \{X\}$.
%         The set \( D(R,X) \) has a unique element $R'$:
%         \raisebox{2pt}{
%             \scalebox{0.7}{\tikz[baseline=-0.5ex]{
%             \node [draw,circle] (x) at (0,0) {1};
%             \node[draw,circle] (y) at (1,0) {3};
%             \draw[->] (x) -- (y) node[midway, above] {$s$};
%         }}}. The rule is $X$-non-increasing with the unique monomorphism $h_{R'L}$ which preserves the interface elements.
%         % morphism illustrated below:
%         % \begin{center}
%         %     \resizebox{0.5\textwidth}{!}{
%         % \begin{tikzpicture}
%         %     \graphbox{$L$}{60mm}{0mm}{35mm}{35mm}{2mm}{-5mm}{
%         %         \coordinate (delta) at (0,-18mm);
%         %         \node[draw,circle] (l1) at ($(delta) + (-1,1.5)$) {1};
%         %         \node[draw,circle] (l2) at ($(delta) + (1,1.5)$) {2};
%         %         \node[draw,circle] (l3) at ($(delta) + (0,0)$) {3};
%         %         \draw[->] (l1) -- (l3) node[midway,left] {s};
%         %         \draw[->] (l2) -- (l3) node[midway,right] {s};
%         %         \draw[->] (l3) edge [loop below] node {0} (l3);
%         %     }
%         %     % \graphbox{$K$}{40mm}{0mm}{35mm}{35mm}{2mm}{-5mm}{
%         %     %     \coordinate (delta) at (0,-18mm);
%         %     %     \coordinate (interfaceorigin) at ($(delta) +(5,0)$);
%         %     %     \node[draw,circle] (r1) at ($(delta) +(-1,1.5)$) {1};
%         %     %     \node[draw,circle] (r2) at ($(delta) +(0.5,1.5)$) {2};
%         %     %     \node[draw,circle] (r3) at ($(delta) + (0,0)$) {3};
%         %     %     % \draw[->] (r1) -- (r3) node[midway,left] {s};
%         %     %     % \draw[->] (r3) edge [loop below] node {0} (r3);
%         %     % }
%         %     % \graphbox{$R$}{80mm}{0mm}{35mm}{35mm}{2mm}{-5mm}{
%         %     %     \coordinate (delta) at (0,-18mm);
%         %     %     \node[draw,circle] (r1) at ($(delta) + (-1,1.5)$) {1};
%         %     %     \node[draw,circle] (r2) at ($(delta) + (0.5,1.5)$) {2};
%         %     %     \node[draw,circle] (r3) at ($(delta) + (0,0)$) {3};
%         %     %     \node[draw,circle] (r4) at ($(delta) + (1,0)$) {};
%         %     %     \draw[->] (r1) -- (r3) node[midway,left] {s};
%         %     %     \draw[->] (r2) -- (r4) node[midway,right] {s};
%         %     %     \draw[->] (r4) edge [loop below] node {0} (r4);
%         %     %     \draw[->] (r3) edge [loop below] node {0} (r3);
%         %     % }
%         %     \graphbox{}{0}{0}{35mm}{35mm}{2mm}{-5mm}{
%         %         \coordinate (delta) at (0,-18mm);
%         %         \coordinate (rxorigin) at ($(interfaceorigin)+(0,6)$);
%         %         \node[draw,circle] (r1) at ($(delta) + (-1,1.5)$) {1};
%         %         % \node[draw,circle] (r2) at ($(delta) +  (0.5,1.5)$) {2};
%         %         \node[draw,circle] (r3) at ($(delta) +  (0,0)$) {3};
%         %         \draw[->] (r1) -- (r3) node[midway,left] {s};
%         %         % \draw[->] (r3) edge [loop below] node {0} (r3);
%         %     }
%         %     % \node () at (37mm,-18mm) {$\leftarrowtail$};
%         %     \node () at (48mm,-18mm) {$\rightarrowtail$};
%         %     % \node () at (57mm,2mm) {$\uparrowtail$};
%         %     % \node () at (38mm,2mm) {$\swarrowtail$};
%         %     % % \node () at (79mm,2mm) {$\searrowtail$};
%         % \end{tikzpicture}
%         % }
%         % \end{center}
%         % Let $s_\mathbb{X}$ be the weight function which associates the weight of $1$ to $X$.
%         % Since \(w_{s_\mathbb{X}}(L) = 1 > 0 = w_{s_\mathbb{X}}(R)\), 
%         Since \(w(L) = 1 > 0 = w(R)\), 
%         the rule terminates by \autoref{thm:termination_grs}.
% \end{example}
\begin{example}
    \label{ex_contrib_variant}
    The rewriting rule below is a variant of a rule presented in \cite[Example 6]{plump2018modular}, obtained by removing all edges in the interface:
    
    % \begin{figure}[hbt] 
    %     \center
    \begin{center}
        \resizebox{0.6\textwidth}{!}{
            \begin{tikzpicture}
                \graphbox{$L$}{0mm}{0mm}{35mm}{35mm}{2mm}{-5mm}{
                    \coordinate (delta) at (0,-18mm);
                    \node[draw,circle] (l1) at ($(delta) + (-1,1.5)$) {1};
                    \node[draw,circle] (l2) at ($(delta) + (1,1.5)$) {2};
                    \node[draw,circle] (l3) at ($(delta) + (0,0)$) {3};
                    \draw[->] (l1) -- (l3) node[midway,left] {s};
                    \draw[->] (l2) -- (l3) node[midway,right] {s};
                    \draw[->] (l3) edge [loop below] node {0} (l3);
                }
                \graphbox{$K$}{40mm}{0mm}{35mm}{35mm}{2mm}{-5mm}{
                    \coordinate (delta) at (0,-18mm);
                    \coordinate (interfaceorigin) at ($(delta) +(5,0)$);
                    \node[draw,circle] (r1) at ($(delta) +(-1,1.5)$) {1};
                    \node[draw,circle] (r2) at ($(delta) +(0.5,1.5)$) {2};
                    \node[draw,circle] (r3) at ($(delta) + (0,0)$) {3};
                    % \draw[->] (r1) -- (r3) node[midway,left] {s};
                    % \draw[->] (r3) edge [loop below] node {0} (r3);
                }
                \graphbox{$R$}{80mm}{0mm}{50mm}{35mm}{2mm}{-5mm}{
                    \coordinate (delta) at (-10mm,-18mm);
                    \node[draw,circle] (r1) at ($(delta) + (-1,1.5)$) {1};
                    \node[draw,circle] (r2) at ($(delta) + (0.5,1.5)$) {2};
                    \node[draw,circle] (r3) at ($(delta) + (0,0)$) {3};
                    \node[draw,circle] (r4) at ($(delta) + (1,0)$) {4};
                    \draw[->] (r1) -- (r3) node[midway,left] {s};
                    \draw[->] (r2) -- (r4) node[midway,right] {s};
                    \draw[->] (r4) edge [loop below] node {0} (r4);
                    \draw[->] (r3) edge [loop below] node {0} (r3);
                    \node[draw,circle] (r5) at ($(r2) + (1.5,0)$) {};
                    \draw[->] (r5) edge [loop below] node {0} (r5);
                    \draw[->] (r5) edge [loop right] node {0} (r5);
                    \draw[->] (r5) edge [loop left] node {0} (r5);
                }
                % \graphbox{$R_x$}{40mm}{40mm}{35mm}{35mm}{2mm}{-5mm}{
                %     \coordinate (delta) at (0,-18mm);
                %     \coordinate (rxorigin) at ($(interfaceorigin)+(0,6)$);
                %     \node[draw,circle] (r1) at ($(delta) + (-1,1.5)$) {1};
                %     \node[draw,circle] (r2) at ($(delta) +  (0.5,1.5)$) {2};
                %     \node[draw,circle] (r3) at ($(delta) +  (0,0)$) {3};
                %     \draw[->] (r1) -- (r3) node[midway,left] {s};
                %     % \draw[->] (r3) edge [loop below] node {0} (r3);
                % }
                \node () at (38mm,-18mm) {$\leftarrowtail$};
                \node () at (77mm,-18mm) {$\rightarrowtail$};
                % \node () at (57mm,2mm) {$\uparrowtail$};
                % \node () at (38mm,2mm) {$\swarrowtail$};
                % \node () at (79mm,2mm) {$\searrowtail$};
            \end{tikzpicture}
            }
    \end{center}
        %     \caption{Diagram of \autoref{ex_contrib_variant}}
        %     \label{fig:contrib_variant}
        % \end{figure}
        Let $X$ be the ruler-graph 
        \tikz[baseline=-0.5ex]{ 
                \node (x) at (0,0) {$\bullet$}; 
                \node (y) at (1,0) {$\bullet$};
                \node (z) at (2,0) {$\bullet$};
                \draw[->] (x) -- (y) node[midway, above] {$s$};
                \draw[->] (z) -- (y) node[midway, above] {$s$};
        }, $\mathbb{X} = \{X\}$ and $s_\mathbb{X}(X)=1$.
        The set \( D(R,X) \) consists of two elements $R'_1$:
        \raisebox{2pt}{
            \scalebox{0.7}{\tikz[baseline=-0.5ex]{
            \node [draw,circle] (x) at (0,0) {1};
            \node[draw,circle] (y) at (1,0) {3};
            \draw[->] (x) -- (y) node[midway, above] {$s$};
        }}} and $R'_2$:
        \raisebox{2pt}{
            \scalebox{0.7}{\tikz[baseline=-0.5ex]{
            \node [draw,circle] (z) at (-1,0) {2};
            \node [draw,circle] (x) at (0,0) {1};
            \node[draw,circle] (y) at (1,0) {3};
            \draw[->] (x) -- (y) node[midway, above] {$s$};
        }}}. 
        Each \( R'_i \) admits a unique monomorphism \( h_{R'_i L} \colon R'_i \rightarrowtail L \) preserving interface elements. 
        The rule is $X$-non-increasing with $h_{R'_1L}$ and $h_{R'_2L}$ because all conditions in \autoref{def:creates_more_x_on_the_left} are satisfied.
        Since \(w_{s_\mathbb{X}}(L) = 1 > 0 = w_{s_\mathbb{X}}(R)\), 
        it follows that the system terminates by \autoref{thm:termination_grs}.
\end{example}
The following example demonstrates the usefulness of a weight function that assigns distinct weight to measurements from different ruler-graphs.
\begin{example} 
    \label{ex:overbeek_5d6}
    Consider the rewriting rules presented in \cite[Example 5.6]{overbeek2024termination_lmcs}:
    \begin{center} 
      $\rho = $\scalebox{0.7}{{
\begin{tikzpicture}[baseline=-15mm,scale=1.2]
    \graphbox{$L$}{0mm}{0mm}{31mm}{20mm}{2mm}{-13mm}{
      \node [draw, circle] (x) at (-7mm,0mm) {1};
      \node [draw, circle] (y) at (0mm,0mm) {2};
      \draw[->] (x) edge[loop above] node  {$a$} (x);
      \draw[->] (y) edge [loop above] node {$a$} (y);
    }
    \graphbox{$K$}{35mm}{-0mm}{18mm}{20mm}{0mm}{-10mm}{
    }
    \begin{scope}  
    \graphbox{$R$}{57mm}{-0mm}{38mm}{20mm}{2mm}{-13mm}{
      \node [draw, circle] (x) at (-7mm,0mm) {3};
      \node [draw, circle] (y) at (0mm,0mm) {4};
      \node [draw, circle] (z) at (7mm,0mm) {5};
      \draw[->] (x) edge[loop above] node  {$b$} (x);
      \draw[->] (y) edge[loop above] node  {$b$} (y);
      \draw[->] (z) edge[loop above] node  {$b$} (z);
    }
    \end{scope}
    \node () at (33mm,-12mm) {$\leftarrowtail$};
    \node () at (55mm,-12mm) {$\rightarrowtail$};
\end{tikzpicture}
}}
    \end{center}
    \begin{center}
    $\tau = $\scalebox{0.7}{{
  \begin{tikzpicture}[baseline=-15mm,scale=1.5,scale=0.9]
    \graphbox{L}{0mm}{0}{31mm}{20mm}{2mm}{-13mm}{
        \node[draw,circle] (x) at (-7mm,0mm) {1};
        \node[draw,circle] (y) at (0mm,0mm) {2};
        \draw[->] (x) edge[loop above] node {$b$} (x);
        \draw[->] (y) edge [loop above] node {$b$} (y);
      }
      \graphbox{K}{35mm}{0}{18mm}{20mm}{2mm}{-13mm}{
      }
      \begin{scope} 
      \graphbox{R}{57mm}{0}{25mm}{20mm}{1mm}{-13mm}{
        \node[draw,circle] (x) at (-3.5mm,0mm) {3};
        \draw[->] (x) edge[loop above] node {$a$} (x);
      }
      \end{scope}
      \node () at (33mm,-12mm) {$\leftarrowtail$};
     \node () at (55mm,-12mm) {$\rightarrowtail$};
  \end{tikzpicture}}
    }
    \end{center}
     Let $X$ be the ruler-graph
    \tikz[baseline=-0.5ex]{
        \node (x) at (0,0) {$\bullet$};
        \draw[->] (x) edge [loop right] node {a} (x);
    }, $Y$ the ruler-graph
    \tikz[baseline=-0.5ex]{
        \node (x) at (0,0) {$\bullet$};
        \draw[->] (x) edge [loop right] node {b} (x);
    } and $\mathbb{X} = \{X, Y\}$.
    The sets $D(R_\rho, X)$, $D(R_\rho, Y)$, $D(R_\tau, X)$ and $D(R_\tau, Y)$ are all empty. Therefore, both rules $\rho$ and $\tau$ are $X$- and $Y$-non-increasing because all conditions in \autoref{def:creates_more_x_on_the_left} are trivially satisfied.
    
    For $s_\mathbb{X}(X) = 5$ and $s_\mathbb{X}(Y) = 3$, we have $
    w_{s_\mathbb{X}}(L_\rho) = 10 > 9 = w_{s_\mathbb{X}}(R_\rho)
    $ and $
    w_{s_\mathbb{X}}(L_\tau) = 6 > 5 = w_{s_\mathbb{X}}(R_\tau)$.
    Thus, the system terminates by \autoref{thm:termination_grs}.

  However, for $s_\mathbb{X}'(X) = 1$ and $s_\mathbb{X}'(Y) = 1$, we have
    \(
        w_{s_\mathbb{X}'}(L_\rho) = 2 \not > 3 = w_{s_\mathbb{X}'}(R_\rho)
    \) which prevents us from applying \autoref{thm:termination_grs} to prove termination of the rewriting system.
    
\end{example}

% \begin{example}[Limitation]
%     \label{ex:plump95_4d1}
%    Consider a rule presented in \cite[Example 4.1]{Plump1995}:
 
%     \begin{center}
%         \resizebox{0.7\textwidth}{!}{
%       $\rho = $  \begin{tikzpicture}[baseline=-20mm]
%         \graphbox{$L$}{0mm}{0}{32mm}{40mm}{0}{0}{
%             \node[draw,circle]  (n1) at (0,-6mm) {1};
%             \node[draw,circle]   (n2) at (0mm,-26mm) {2};
%             \node[draw,circle]   (n3) at (-10mm,-26mm) {3};
%             \node[draw,circle]   (n4) at (10mm,-26mm) {4};
%             \draw[->]  (n2) edge [loop below] node  {a} (n2);
%             \draw[->]  (n3) edge [loop below] node  {b} (n3);
%             \draw[->]  (n1) to node [right] {f} (n2) ;
%             \draw[->]  (n1) to node [right] {f}  (n3);
%             \draw[->]  (n1) to node [right] {f}  (n4);
%           }
%           \graphbox{$K$}{33mm}{0}{32mm}{40mm}{0}{-0}{
%             \node[draw,circle]  (n1) at (0,-6mm) {1};
%             \node[draw,circle]   (n2) at (0mm,-26mm) {2};
%             \node[draw,circle]   (n3) at (-10mm,-26mm) {3};
%             \node[draw,circle]   (n4) at (10mm,-26mm) {4};
%             \draw[->]  (n2) edge [loop below] node  {a} (n2);
%             \draw[->]  (n3) edge [loop below] node  {b} (n3);
%           }
%           \graphbox{$R$}{66mm}{0}{32mm}{40mm}{0}{-0}{
%             \node[draw,circle]  (n1) at (0,-6mm) {1};
%             \node[draw,circle]   (n2) at (0mm,-26mm) {2};
%             \node[draw,circle]   (n3) at (-10mm,-26mm) {3};
%             \node[draw,circle]   (n4) at (10mm,-26mm) {4};
%             \draw[->]  (n2) edge [loop below] node  {a} (n2);
%             \draw[->]  (n3) edge [loop below] node  {b} (n3);
%             \draw[->]  (n1) edge [bend left] node [right] {f} (n4) ;
%             \draw[->]  (n1) edge [bend right] node [left] {f}  (n4);
%             \draw[->]  (n1) to node [right] {f}  (n4);
%           }   
%     \end{tikzpicture}
%         }
%   \end{center}
%   \noindent
%   \begin{minipage}{0.7\textwidth}
%     To prove its termination with our method, a ruler-graph $X$ containing an edge labeled by $f$ must be used, because the number of occurrences of every graph containing only edges labeled by $a$ and $b$ does not change by any rewriting step using $\rho$. But in this case, condition \eqref{def:non_increasing:edge_injective} of \autoref{def:creates_more_x_on_the_left} cannot be satisfied.
%   \end{minipage}
%   \begin{minipage}{0.3\textwidth}
%     \hfill
%   \begin{center}
%     \begin{tikzpicture} 
%         \graphbox{}{0}{0}{30mm}{20mm}{-10mm}{-10mm}{
%             \node[draw,circle]  (n1) at (0,0mm) {1};
%             % \node[draw,circle]   (n2) at (0mm,-26mm) {2};
%             % \node[draw,circle]   (n3) at (-10mm,-26mm) {3};
%             \node[draw,circle]   (n4) at (20mm,0mm) {4};
%             % \draw[->]  (n2) edge [loop below] node  {a} (n2);
%             % \draw[->]  (n3) edge [loop below] node  {b} (n3);
%             \draw[->]  (n1) edge [bend left] node [above] {f} (n4) ;
%             \draw[->]  (n1) edge [bend right] node [below] {f}  (n4);
%             \draw[->]  (n1) to node {f}  (n4);
%           }  
%     \end{tikzpicture}
%   \end{center}
% \end{minipage}
  
%   \end{example}
% Finally, we present a limitation of our approach.
\begin{example}[Limitation]
    \label{rem:d3_limitation}  
    Our method fails to prove termination for the rewriting rule in \cite[Example D.3]{endrullis2024generalized_arxiv_v2}.
    % because our approach cannot take into account antipatterns(see~\cite[Remark 6.2]{overbeek2024termination_lmcs}).
    This failure occurs because, for any system $\mathcal{R}$, rewriting step $G \Rightarrow_{\mathcal{R},\mathfrak{M}} H$, set of ruler-graphs $\mathbb{X}$, and weight function $s_\mathbb{X}$, we have $w_{s_\mathbb{X}}(G) \geq w_{s_\mathbb{X}}(H)$ due to the existence of an injective graph homomorphism from $G$ to $H$.
    
    \begin{center}
        \resizebox{0.7\textwidth}{!}{
    \begin{tikzpicture}  
       \graphbox{$L$}{0mm}{-11mm}{32mm}{15mm}{2mm}{-8mm}{  
           \node[draw,circle]  (x) at (-6mm,0mm) {1};  
           \node[draw,circle]  (y) at (6mm,0mm) {};  
         }  
         \graphbox{$K$}{33mm}{-11mm}{32mm}{15mm}{2mm}{-8mm}{  
           \node[draw,circle]  (x) at (-6mm,0mm) {1};  
         }  
         \graphbox{$R$}{66mm}{-11mm}{32mm}{15mm}{1mm}{-8mm}{  
          \node[draw,circle]  (x) at (-6mm,0mm) {1};  
          \node[draw,circle]  (y) at (6mm,0mm) {};  
          \draw[->]  (x) to (y);  
         }    
   \end{tikzpicture}
        }  
    \end{center}
  \end{example} 

  
% \section{Related work}
% \label{sec:related_work} 
% A comparative analysis of termination techniques for DPO graph rewriting systems, drawn from prior work~\cite{plump1995ontermination,plump2018modular,bruggink2014termination,bruggink2015proving,endrullis2024generalized_arxiv_v2,overbeek2024termination_lmcs}, is summarized in Table~\ref{tab:comparison:subgraph_counting}. Our approach successfully proves termination for 14 of these systems. In the remainder of this section, we compare our method with some existing methods in more detail.


{% local group
  \setlength{\tabcolsep}{3pt}
  \renewcommand{\arraystretch}{1}
\begin{table}[H]
   \centering
  \small % Reduce font size
  \caption{Applicability of termination techniques to DPO rewriting examples.
  The symbol \ding{51} indicates termination can be proved by the technique,
  \ding{55} indicates it cannot be proved, and 
  $-$ denotes irrelevance or out-of-scope cases.
        }
 \label{tab:comparison:subgraph_counting}
% \setlength{\tabcolsep}{4pt} % Reduce horizontal padding
   \begin{NiceTabular}{ccccccccc}[vlines] % <-- 9 columns now (was 7)
    \Hline
                % \diagbox{\enskip \textbf{Examples}}{\textbf{Techniques}} 
    \Block{1-2}{\diagbox{\enskip \textbf{Examples}}{\textbf{Techniques}}} & 
    &
    \RowStyle{\rotate}
     \makecell{Forward closure \cite{plump1995ontermination}} % NEW column #1
    & \RowStyle{\rotate}
     \makecell{Modular criterion \cite{plump2018modular}} % NEW column #2
    & \RowStyle{\rotate}
     \makecell{Type graph \cite{bruggink2014termination}}  
    & \RowStyle{\rotate}
     \makecell{Type graph \cite{bruggink2015proving}} 
    & \RowStyle{\rotate}
     \makecell{Type graph \cite{endrullis2024generalized_arxiv_v2}} 
    & \RowStyle{\rotate}
     \makecell{Subgraph 
            counting \cite{overbeek2024termination_lmcs}} 
    & \RowStyle{\rotate}
     \makecell{Morphism
                Counting 
                \\(Chapter~\ref{chap:subgraph_counting})} \\
    \Hline
    \Hline
% ok examples
Chapter~\ref{chap:subgraph_counting} 
&Example~\ref{subgraph_counting:ex_contrib_variant}
  & \ding{51} & \ding{55} & \ding{55} & \ding{55} & \ding{55} & \ding{55} & \ding{51} \\ \Hline
  
\cite{plump1995ontermination} &
\hyperref[ex:overbeek_5d8_plump1995_3d8_plump2018_3_overbeek_5d8]{Example 3.8}
             & \ding{51} & -- & -- & -- & -- &
          --
             & \ding{51}\\ 
\hline

\Block{2-1}{\cite{plump2018modular}} & \hyperref[ex:overbeek_5d8_plump1995_3d8_plump2018_3_overbeek_5d8]{Example 3} 
          & -- & \ding{51} &  -- & -- & -- & 
          --
          & \ding{51}\\ 

\Hline

%~\cite{plump2018modular} 
& Example 5 &  -- &  \ding{51} &   -- & -- & -- &  
            --
          & \ding{51}\\  
\Hline

\cite{bruggink2014termination} & \hyperref[ex:termination:grsaa]{Example 4 and 6}  
  & -- & -- & \ding{51} & -- & -- & 
            --
          & \ding{51} \\ \Hline

\Block{2-1}{\cite{bruggink2015proving}} & \hyperref[ex:termination:grsaa]{Example 2}  
  & -- & -- & -- & \ding{51} & -- & 
  -- &  \ding{51}\\ \Hline
  
%~\cite{bruggink2015proving}
 & Example 4 
  & -- & -- & -- & \ding{51} & -- & 
  --& \ding{51} \\ \Hline


% ----- from endrullis2024generalized_arxiv_v2 -----
\Block{3-1}{\cite{endrullis2024generalized_arxiv_v2}} & \hyperref[ex:endrullis2024_6d2]{Example 6.2}  
  & -- & -- & -- & -- & \ding{51} & -- & \ding{51}\\ \Hline

%~\cite{endrullis2024generalized_arxiv_v2}
 & \hyperref[ex_endrullis_6d3_endrullis_5d8]{Example 6.3}
  & -- & -- & -- & -- & \ding{51} &% \ding{55} 
  \ding{55} & \ding{51}\\ \Hline

%~\cite{endrullis2024generalized_arxiv_v2} 
& \hyperref[ex:overbeek_5d8_plump1995_3d8_plump2018_3_overbeek_5d8]{Example D.1}
  & -- & -- & -- & -- & \ding{51} & -- & \ding{51}\\ \Hline

  \Block{4-1}{\cite{overbeek2024termination_lmcs}} & \hyperref[ex:overbeek_5d3]{Example 5.3}
  & -- & -- & -- & -- & -- & \ding{51} & \ding{51}\\ \Hline

%~\cite{overbeek2024termination_lmcs} \hyperref[ex:overbeek_5d3]{Example 5.3 monic matches}
%     & -- & -- & -- & -- & -- & \ding{51} & \ding{51}\\ \Hline
%~\cite{overbeek2024termination_lmcs} 
& \hyperref[ex:overbeek_5d5]{Example 5.5} 
  & -- & -- & -- & -- & -- & \ding{51} & \ding{51}\\ \Hline

%~\cite{overbeek2024termination_lmcs} 
& \hyperref[ex:overbeek_5d6]{Example 5.6}
  & -- & -- & -- & -- & -- & \ding{51} & \ding{51} \\ \Hline

%~\cite{overbeek2024termination_lmcs} \hyperref[ex:overbeek_5d6]{Example 5.6 bis}
%     & -- & -- & -- & -- & -- & \ding{51} & -- \\ \Hline


%~\cite{overbeek2024termination_lmcs}  
& \hyperref[ex:overbeek_5d8_plump1995_3d8_plump2018_3_overbeek_5d8]{Example 5.8}
  & -- & -- & -- & -- & -- & \ding{51} & \ding{51}\\ \Hline

      % not supported examples  
     ~\cite{plump2018modular} &  \hyperref[ex:plump2018_ex6_endrullis_d4]{Example 6} &  -- & \ding{51} & -- & -- & -- & 
      --
          & -- \\
      \Hline

\Block{3-1}{\cite{endrullis2024generalized_arxiv_v2}}
 & Example 6.4  
      & -- & -- & -- & -- & \ding{51} & -- & -- \\ \Hline

%  ~\cite{endrullis2024generalized_arxiv_v2}
  &  Example 6.5  
      & -- & -- & -- & -- &  \ding{51} & -- & -- \\ \Hline

    %  ~\cite{endrullis2024generalized_arxiv_v2}
       &\hyperref[ex:plump2018_ex6_endrullis_d4]{Example D.4} 
      & -- & -- & -- & -- & \ding{51} & -- & --\\ \Hline

   % ----- from overbeek2024termination_lmcs -----
   \Block{3-1}{\cite{overbeek2024termination_lmcs}} & Example 5.2
      & -- & -- & -- & -- & -- & \ding{51} & -- \\ \Hline

    %  ~\cite{overbeek2024termination_lmcs} 
      & Example 5.7 
      & -- & -- & -- & -- & -- & \ding{51} & -- \\ \Hline
      
%  ~\cite{overbeek2024termination_lmcs} 
  & Example 5.9 
      & -- & -- & -- & -- & -- & \ding{51} & --\\ \Hline
 

    % not ok examples
   ~\cite{plump1995ontermination} & \hyperref[ex:plump95_4d1]{Example 4.1} & \ding{51} & -- & -- & -- & -- & 
              \ding{51}
              
              & \ding{55}\\ 
   \Hline
  ~\cite{plump2018modular} & \hyperref[ex:plump_ex4]{Example 4} &  -- &  \ding{51} &  -- & -- & -- & 
               --
               & \ding{55}\\ 
   \Hline

   \Block{3-1}{\cite{bruggink2014termination}} & Example 1 
   & -- & -- & \ding{51} & -- & -- & 
                 --
               &  \ding{55}\\ 
   \Hline

%   ~\cite{bruggink2014termination} 
   & Routing Protocol
       & -- & -- & \ding{51} & -- & -- & 
           --
           &  \ding{55}\\  
           \Hline
% \Block{2-1}{\cite{bruggink2014termination}}
 & \hyperref[ex:plump_ex4]{Example 5}
   & -- & -- & \ding{51} & -- & -- & -- &  \ding{55}\\ 
\Hline

\Block{2-1}{\cite{bruggink2015proving}} & \hyperref[ex:bruggink2015_ex5]{Example 5}
   & -- & -- & -- & \ding{51} & -- &  
   -- &  \ding{55}\\
   \Hline

%   ~\cite{bruggink2015proving} 
   & \hyperref[ex:bruggink2015_ex6_endrullis2024_d2]{Example 6} 
   & -- & -- & -- & \ding{51} & -- &  
   --&  \ding{55}\\  
   \Hline

   \Block{2-1}{\cite{endrullis2024generalized_arxiv_v2}} & \hyperref[ex:bruggink2015_ex6_endrullis2024_d2]{Example D.2} 
   & -- & -- & -- & -- & \ding{51} & -- & \ding{55}\\ 
   \Hline

%   ~\cite{endrullis2024generalized_arxiv_v2}
   & \hyperref[rem:d3_limitation]{Example D.3}
   & -- & -- & -- & -- & \ding{51} & \ding{55} & \ding{55}\\ \Hline

  \end{NiceTabular}
  % \vspace{2mm}
  % \caption{Applicability of termination techniques to DPO rewriting examples.
  %  The symbol \ding{51} indicates termination can be proved by the technique,
  %  \ding{55} indicates it cannot be proved, and 
  %  $-$ denotes irrelevance or out-of-scope cases.
  %        }
  % \label{tab:comparison}
  \end{table}
} 


The subgraph-counting method by Overbeek and Endrullis~\cite{overbeek2024termination_lmcs} is designed for the PBPO+~\cite{overbeek2023graph, overbeek2023apbpotutorial}—a rewriting formalism capable of simulating left-injective DPO rewriting. It can prove termination for systems like~\cite[Examples 5.2, 5.7, 5.9]{overbeek2024termination_lmcs} and~\cite[Example 6]{plump2018modular}, which lie beyond the scope of our technique. Additionally, this method can be applied for many categories. 

It is very closely related to our work in the setting where they are both defined. Both methods weigh objects by summing weighted morphisms targeting them, and, for both methods, the key challenge lies in estimating weights for morphisms whose images partially overlap with the rewriting context, because morphisms fully embedded in the context are shared between host and resulting graphs, while those entirely within left- or right-hand-side graphs are trivial to quantify.  

The two methods employ distinct strategies to overcome this challenge. The subgraph-counting method relies on type-morphisms (see~\cite[page 9, remark 4.11, Lemma 4.23]{overbeek2024termination_lmcs}), whereas our approach requires injective mappings from (i) subgraph occurrences partially overlapping the rewriting context in the result graph to (ii) those in the host graph.

Both approaches have limitations and strengths. Endrullis and Overbeek's approach suffers from discrete interface\textemdash interface graph with no edges, as mentioned in~\cite[Example 5.5]{overbeek2024termination_lmcs}. For instance, it proves termination for~\cite[Example 5.5]{overbeek2024termination_lmcs} but fails for~\cite[Example 6.3]{endrullis2024generalized_arxiv_v2}, where the rules differ only by a discrete interface. 
Similarly, it fails to prove termination forExample~\ref{subgraph_counting:ex_contrib_variant}, but if an edge labeled by ``s'' from node $1$ to node $3$ is added, then it succeeds. Our approach, however, can handle all the aforementioned cases. Nevertheless, their method can address~\cite[Example 4.1]{plump1995ontermination}, which remains beyond the scope of our technique. Finally, both approaches cannot handle~\cite[Example D.3]{endrullis2024generalized_arxiv_v2} (see~\cite[Remark 6.2]{overbeek2024termination_lmcs} and Example~\ref{ex:endrullis:d3:limitation}).

% The type graph method, which weighs an object by summing the weights of morphisms from the object to a type graph, was initially introduced by Zantema, K{\"o}nig and Bruggink~\cite{zantema2014termination} for cycle-rewriting systems. 
% This method has since been generalized to edge-labeled multigraphs by Bruggink et al.~\cite{bruggink2014termination} for DPO rewriting with monic matches and injective rules, later extended to DPO rewriting in general by Bruggink et al.~\cite{bruggink2015proving}, and further adapted to more categories and different DPO variants by Endrullis et al.~\cite{endrullis2024generalized_arxiv_v2}. 
% These methods are not directly comparable with our technique in general.

On the one hand, the termination of Example~\ref{subgraph_counting:ex_contrib_variant} can be proved by our method but not by the type graph methods due to the exsitence of a surjection from the output graph to the input graph as explained in~\cite[Example D.4]{endrullis2024generalized_arxiv_v2}. On the other hand, our method cannot prove the termination of~\cite[Example 1, 5 and Ad-hoc Routing Protocol]{bruggink2014termination}, nor~\cite[Example 5, 6]{bruggink2015proving}, nor~\cite[Examples D2 and D3]{endrullis2024generalized_arxiv_v2}.

Plump~\cite{plump1995ontermination} introduced a necessary and sufficient termination condition for left-injective DPO hypergraph rewriting via forward closure, though verifying this condition is undecidable because the condition satisfies if and only if the rewriting system is terminating.
While this method proves termination of Example~\ref{subgraph_counting:ex_contrib_variant}, our approach succeeds on~\cite[Example 3.8]{plump1995ontermination} but fails in proving termination of~\cite[Example 4.1]{plump1995ontermination}. 

Plump~\cite{plump2018modular} later proposed a modular critical pair-based strategy for left-injective DPO hypergraph rewriting with monic matches. 
Our method complements this: while modularity reduces global complexity, each subsystem requires individual termination proofs. For example, the measure based on the indegree proposed in~\cite{plump2018modular} cannot prove the termination of Example~\ref{subgraph_counting:ex_contrib_variant} due to the additional loops. Specifically, if the context graph is identical to the interface graph, then the weight does not decrease because we have $3^k < 2^k+2^k+3^k$ for all $k \mathop{\in} \mathbb{N}$. In contrast, our method succeeds.
Additionally, our method proves termination of~\cite[Examples 1 and 5]{plump2018modular} but not~\cite[Example 4]{plump2018modular}, and non-injective rules (e.g.,~\cite[Example 6]{plump2018modular}) fall outside our scope.

In~\cite{levendovszky2007termination}, a termination criterion for DPO rewriting with monic matches, injective rules and negative application conditions on finite typed attributed graphs is proposed. It is based on the fact that if the application of every infinite sequence of rules requires the initial graph to be infinite, then the system terminates. This technique is theoretically very interesting, but it is hard to automatically check the termination condition.

Bottoni et al.~\cite{bottoni2005termination} presents a termination criterion for DPO and SPO rewriting on high-level replacement units. The high-level replacement units that they consider are rewriting systems with very restrictive external control mechanisms. \todo{todo to do: xu : jamais introduit} The method relies on a mesearing function satisfying a very strong constrainte, and the instance of such a mesearing function proposed is node and edge counting, which is subsumed by our method in the setting of DPO rewriting with injective rules.
% Bottoni et al.~\cite{bottoni2005termination} propose a termination criterion for DPO/SPO rewriting on high-level replacement units. Their method imposes a strongly constrained measuring function and the only concretes measuring function proposed are node-counting and edge-counting.

Bottoni et al.~\cite{bottoni2010atermination} presents a criterion for termination of DPO rewriting with monic matches, injective rules and negative application conditions, based on the construction of a labeled transition system.
whose states represent overlaps between the negative application condition and the right hand side that can give rise to cycles.
% Bottoni et al.~\cite{bottoni2010atermination} presents a criterion for termination of DPO rewriting with monic matches, injective rules and negative application conditions, based on the construction of a labeled transition system. 

Levendovszky et al.~\cite{levendovszky2007termination} propose a termination criterion for DPO rewriting (monic matches, injective rules, negative application condition), though automated verification is hard as explained in~\cite[\textsection 6]{levendovszky2007termination}. 







% \section{Conclusion and future work}
% \label{sec:conclusion} 
% We have presented a machine-checkable sufficient condition for relative termination of DPO systems with injective rules on edge-labeled multigraphs.  
This method resolves cases where prior interpretation-based techniques \cite{zantema2014termination,bruggink2014termination,bruggink2015proving,endrullis2024generalized_arxiv_v2,
overbeek2024termination_lmcs} fail (e.g., \autoref{subgraph_counting:ex_contrib_variant}) and, unlike the subgraph-counting approach in \cite{overbeek2024termination_lmcs}, avoids limitations arising from interfaces with no edges.
The method has been implemented in an OCaml tool. 

An interesting direction for future work is to extend our method to DPO systems with negative application conditions and to address the limitation highlighted in \autoref{ex:endrullis:d3:limitation}.

% % \chapter{Conclusion and Futur Works}
% % \section{Conclusion}
% % \begin{itemize}
    \item[bruggink2014] the paper "high-level replacement units and their termination properties 2005" considers high-level replacement units (HLRU), which are transformation systems with externam control expressions. The paper introduces a general framework for proving termination of such HLRUs, but the only concrete termination criteria considered are node and edge counting, which are subsumed by the weighted type graph method.
    \item[bruggink2014] in "termination criteria for model transformation 2005" layered graph transformation systems are considered, which are graph transformation systems where interleaving creation and deletion of edges with the same label is prohibited and creation of nodes is bounded. The paper shows such graph transformation systems are terminating.
    \item[bruggink2014] The paper "Termination analysis of model transformations by petri nets. 2006" simulates a graph transformation system by a Petri-net. Thus, the presence of edges with certain labels and the causal relationships between them are modeled, but no other structural properties of the graph. The paper uses typed graph transformation systems; thus, a type graph is used but, unlike in our weighted type graph method, it is fixed by the graph transformation system. Finally, [3] was one of the inspirations for this paper. 
    \item[bruggink2014] "Towards a systematic method for proving termination of graph transformation systems" is one of the inspirations for the paper of bruggink2014. its termination argument is subsumed by the weighted type graph technique.
\end{itemize}

% % \section{Future Works} 
% % 
\begin{itemize}
    \item[bruggink2014]Although all theorems have been stated and proved for (binary) multigraphs, a generalization to hypergraphs would be trivial. On the other hand, transferring the results to other graph transformation formalisms is harder. For example, in the single pushout approach, the graph transformation system corresponding to the one of Ex. 1 is non-terminating, so the result of Ex. 2 (in which it is proved that this system is terminating) shows that the weighted type graph technique cannot be transferred one-to-one to single pushout graph transformation. It is left as future research to find similar arguments for the single pushout approach and other formalisms.
    
    Another direction for further research is to allow for graph transformation systems with negative application conditions or more general application conditions. Note, however, that the implicit negative application condition of double pushout graph transformation, the dangling edge condition, can in some cases already be handled (see Ex. 2). 
    
    Finally, for interesting real-world applications, it would be interesting to generalize the technique to more expressive methods of specifying the initial graph languages, so that we can, for example, restrict to trees or rings of arbitrary size (both graph languages cannot be expressed by a type graph). 
\end{itemize}

\begin{idea}
    example majority A or B\\
    implementation of the algorithms proposed by plump\\
    propose a reinforced version of the algorithms\\
\end{idea}


% \chapter{Appendix for Subgraph Counting}
% \section*{Pullback, Adhesive Category and VK-square}
% \input{shared_paragrahs/adhesive}
% \newpage
% \section*{Auxiliary lemmas}
% 


\begin{lemma}
    \label{lem:b_c_same_img_exist_a}
    Consider the pushout square in $\mathbf{Graph}$ of form
    \begin{center}{\normalfont
        \begin{tikzpicture}[node distance=12mm]
          \node (A) {$A$};
          \node [right of=A] (B) {$B$}; 
          \node [below of=A] (C) {$C$}; 
          \node [below of=B] (D) {$D$}; 
          \node () [at=($(A)!0.5!(D)$)] {\normalfont PO};
          \draw [>->] (A) to (B);
          \draw [>->] (A) to (C);
          \draw [>->] (B) to (D); 
          \draw [>->] (C) to (D);
        \end{tikzpicture}
    }\end{center} 
    The following hold 
    \begin{enumerate}
        \item  If $ b\in B $ and $ c \mathop{\in} C $ such that $ h_{BD}(b) \mathop{=} h_{CD}(c) $ then there is $ a \mathop{\in} A $ such that $ h_{AB}(a) \mathop{=} b $ and $ h_{AC}(a)= c $.
        \item  For all \( b \mathop{\in} B \mathop{\setminus} h_{AB}(A) \) and \( c \mathop{\in} C \), we have \( \beta'(b) \mathop{\neq} \alpha'(c) \).
    \end{enumerate}
\end{lemma}
\begin{proof}
     Let \( b \mathop{\in} B \) and \( c \mathop{\in} C \) such that \( h_{BD}(b) \mathop{=} h_{CD}(c) \). 
     
     There are two cases: if $b$ and $c$ are two nodes then let \( S \) be the singleton graph with a unique node \( u \), otherwise\textemdash$b$ and $c$ must be two edges with the same label $l$\textemdash let \( S \) be the graph with two distinct nodes and a unique edge \( u \) labeled by $l$.
     
     Define morphisms \( h_{SB}: S \mathop{\to} B \) and \( h_{SC}: S \mathop{\to} C \) such that \( h_{SB}(u) \mathop{=} b \) and \( h_{SC}(u) \mathop{=} c \). We have \( h_{SB} \mathop{\star} h_{BD} \mathop{=} h_{SC} \mathop{\star} h_{CD} \). The pushout square \( ACDB \) along monomorphisms is also a pullback square by Proposition~\ref{prop:pb_eq_po}, and the universal property of the pullback guarantees the existence of a unique morphism \( h_{SA}: S \mathop{\to} A \) such that \( h_{SA} \mathop{\star} h_{AB} \mathop{=} h_{SB} \) and \( h_{SA} \mathop{\star} h_{AC} \mathop{=} h_{SC} \). Let \( a \mathop{=} h_{SA}(u) \). We have \( a \mathop{\in} A \), and by composition: 
     \begin{itemize}
        \item \(h_{AB}(a) \mathop{=} (h_{SA} \mathop{\star} h_{AB})(u) \mathop{=} h_{SB}(u) \mathop{=} b \)
        \item \(h_{AC}(a) \mathop{=} (h_{SA} \mathop{\star} h_{AC})(u) \mathop{=} h_{SC}(u) \mathop{=} c\)
     \end{itemize} 

    The second assertion is a direct consequentce of the first. Suppose, for contradiction, there exist \( b \mathop{\in} B \mathop{\setminus} h_{AB}(A) \) and \( c \mathop{\in} C \) such that \( h_{BD}(b) \mathop{=} h_{CD}(c) \). Since the first assertion holds, there must exist \( a \mathop{\in} A \) with \( h_{AB}(a) \mathop{=} b \) and \( h_{AC}(a) \mathop{=} c \). However, this contradicts the assumption that \( b \notin h_{AB}(A) \).
    \qed
\end{proof}


\begin{lemma}
    \label{kpcpck_pullback}
    Consider the following commutative diagram in \textbf{Graph}.
    \begin{center}
        \resizebox{6cm}{!}{
            \begin{tikzpicture}
                \node (k) at (0,0) {K};
                \node (r) at (4,0) {R};
                \node (c) at (0,-3) {C};
                \node (h) at (4,-3) {H};
                \draw[<-<]  (r) -- (k) node [midway,above] {};
                \draw[>->] (c) -- (h) node [midway, below] {};
                \draw[>->] (r) -- (h) node[midway, left] {};
                \draw[>->] (k) -- (c) node[midway, left] {};
                \node (k') at (1,-1) {K'};
                \node (r') at (2,-1) {R'};
                \node (c') at (1,-2) {C'};
                \node () at (1.5,-1.5) {$PB$};
                \node () at (3,-1.5) {$PB$};
                \node () at (1.5,-2.5) {$PB$};
                \node (x) at (2,-2) {X};
                \draw [>->] (c') -- (x);
                \draw [>->] (r') -- (x);
                \draw [>->] (k') -- (r');
                \draw [>->] (k') -- (c');
                \draw [>->] (c') -- (c);
                \draw[>->] (r') -- (r);
                \draw[>->] (x) -- (h);
                \draw[->] (k') -- (k);
            \end{tikzpicture}
        }
        \end{center}  
        The squares $K'KCC'$ and $K'KRR'$ are pulllback square.
\end{lemma}
\begin{proof}
    Let $K \mathop{\leftarrow} A \mathop{\rightarrow} C'$ be a span such that 
    \begin{flalign}
        h_{AK} \mathop{\star} h_{KC} \mathop{=} h_{AC'} \mathop{\star} h_{C'C} \label{hakhkchacp}
    \end{flalign}
    We have the following commutative diagram 

    \begin{center}
        \resizebox{0.3\textwidth}{!}{
            \begin{tikzpicture}
                \node (k) at (0,0) {K};
                \node (r) at (4,0) {R};
                \node (c) at (0,-3) {C};
                \node (h) at (4,-3) {H};
                \node (h) at (4,-3) {H};
                \node (a) at (-1,0) {A};
                \draw[<-<]  (r) -- (k) node [midway,above] {};
                \draw[>->] (c) -- (h) node [midway, below] {};
                \draw[>->] (r) -- (h) node[midway, left] {};
                \draw[>->] (k) -- (c) node[midway, left] {};
                \node (k') at (1,-1) {K'};
                \node (r') at (2,-1) {R'};
                \node (c') at (1,-2) {C'};
                \node () at (1.5,-1.5) {$PB$};
                \node () at (3,-1.5) {$PB$};
                \node () at (1.5,-2.5) {$PB$};
                \node (x) at (2,-2) {X};
                \draw [>->] (c') -- (x);
                \draw [>->] (r') -- (x);
                \draw [>->] (k') -- (r');
                \draw [>->] (k') -- (c');
                \draw [>->] (c') -- (c);
                \draw[>->] (r') -- (r);
                \draw[>->] (x) -- (h);
                \draw[->] (k') -- (k);
                \draw[->,red] (a) -- (c');
                \draw[->,red] (a) -- (k);
            \end{tikzpicture}
        }
        \end{center} 
    We show first the existence of a morphism $h_{AK'} \mathop{\colon} A \mathop{\to} K'$ such that 
    \begin{flalign*}
        h_{AK'} \mathop{\star} h_{K'C'} \mathop{=} h_{AC'}  
        \\
        h_{AK'} \mathop{\star} h_{K'K} \mathop{=} h_{AK} 
    \end{flalign*}
    From Equation~\eqref{hakhkchacp}, we obtain 
    \begin{flalign}
        h_{AK} \mathop{\star} h_{KC} \mathop{\star} h_{CH}= h_{AC'} \mathop{\star} h_{C'C} \mathop{\star} h_{CH} \label{hakhkchchhacp}
    \end{flalign}
    From Equation~\eqref{hakhkchchhacp} and commutativity of $C'CHX$ and $KCHR$, we have $h_{KC} \mathop{\star} h_{CH} \mathop{=} h_{KR} \mathop{\star} h_{RH}$ and $h_{C'C} \mathop{\star} h_{CH} \mathop{=} h_{C'X} \mathop{\star} h_{XH}$ and deduce  
     \begin{flalign}
        h_{AK} \mathop{\star} h_{KR} \mathop{\star} h_{RH}= h_{AC'} \mathop{\star} h_{C'X} \mathop{\star} h_{XH} \label{hakhkrhrhhacp}
     \end{flalign}
    Therefore, the following commutative diagram holds 

    \begin{center}
        \resizebox{0.3\textwidth}{!}{
            \begin{tikzpicture}
                \node (k) at (0,0) {K};
                \node (r) at (4,0) {R};
                \node (h) at (4,-3) {H};
                \node (h) at (4,-3) {H};
                \node (a) at (-1,0) {A};
                \draw[<-<]  (r) -- (k) node [midway,above] {};
                \draw[>->] (r) -- (h) node[midway, left] {};
                \node (r') at (2,-1) {R'};
                \node (c') at (1,-2) {C'};
                \node () at (3,-1.5) {$PB$};
                \node (x) at (2,-2) {X};
                \draw [>->] (c') -- (x);
                \draw [>->] (r') -- (x);
                \draw[>->] (r') -- (r);
                \draw[>->] (x) -- (h);
                \draw[->] (a) -- (c');
                \draw[->] (a) -- (k);
            \end{tikzpicture}
        }
        \end{center} 
    Since $R'XHR$ is a pullback square, by the universal property Definition~\ref{def:cat:pb}, there exists a unique morphism $h_{AR'} \mathop{\colon} A \mathop{\to} R'$ such that 
    \begin{flalign}
        h_{AR'} \mathop{\star} h_{R'R} \mathop{=} h_{AK} \mathop{\star} h_{KR}  \label{harprprakkr}
        \\
        h_{AR'} \mathop{\star} h_{R'X} \mathop{=} h_{AC'} \mathop{\star} h_{C'X} \label{harprpxacpcpx}
    \end{flalign}
    We have thus the following commutative diagram 

        \begin{center}
            \resizebox{0.2\textwidth}{!}{
                \begin{tikzpicture}
                    \node (a) at (-1,0) {A};
                    \node (k') at (1,-1) {K'};
                    \node (r') at (2,-1) {R'};
                    \node (c') at (1,-2) {C'};
                    \node () at (1.5,-1.5) {$PB$};
                    \node (x) at (2,-2) {X};
                    \draw [>->] (c') -- (x);
                    \draw [>->] (r') -- (x);
                    \draw [>->] (k') -- (r');
                    \draw [>->] (k') -- (c');
                    \draw[->] (a) -- (c');
                    \draw[->] (a) -- (r');
                \end{tikzpicture}
            }
        \end{center} 
    Since $K'C'XR'$ is a pullback square, there exists a unique morphism $h_{AK'} \mathop{\colon} A \mathop{\to} K'$ such that 
    \begin{flalign}
        h_{AK'} \mathop{\star} h_{K'C'} \mathop{=} h_{AC'}  \label{hakphkpcphacp}
        \\
        h_{AK'} \mathop{\star} h_{K'R'} \mathop{=} h_{AR'} \label{hakpkprparp}
    \end{flalign}
    We have $h_{AK'} \mathop{\star} h_{K'C'} \mathop{=} h_{AC'}$ by Equation~\eqref{hakphkpcphacp}, and $h_{AK'} \mathop{\star} h_{K'K} \mathop{=} h_{AK}$ because $h_{KR}$ is injective and the following holds:
    \begin{flalign*}
        (h_{AK'} \mathop{\star} h_{K'K}) \mathop{\star} h_{KR} &= h_{AK} \mathop{\star} h_{K'R'} \mathop{\star} h_{R'R} \hspace{0.5cm} &\text{by commutativity of $K'KRR'$}\\
        &= h_{AR'} \mathop{\star} h_{R'R} &\text{by Equation~\eqref{hakpkprparp}} \\
        &= h_{AK} \mathop{\star} h_{KR} &\text{by Equation~\eqref{harprprakkr}}
    \end{flalign*}
    Now, we show the unicity of such a morphism from $A$ to $K'$. 
    Let $h_{AK'}' \mathop{\colon} A \mathop{\to} K'$ be a morphism  such that 
    \begin{flalign*}
        h_{AK'}' \mathop{\star} h_{K'C'} \mathop{=} h_{AC'}  
        \\
        h_{AK'}' \mathop{\star} h_{K'K} \mathop{=} h_{AK} 
    \end{flalign*}
    From $h_{AK'} \mathop{\star} h_{K'C'} \mathop{=} h_{AC'}$, $h_{AK'}' \mathop{\star} h_{K'C'} \mathop{=} h_{AC'}$ and injectivity of $h_{K'C'}$, we deduce $h_{AK'} \mathop{=} h_{AK'}$.

    \qed
\end{proof}


\begin{lemma}
    \label{kpcpxrp_po}
    Consider the following commutative diagram in \textbf{Graph} where $KCHR$ is a pushout.
    \begin{center}
        \resizebox{6cm}{!}{
            \begin{tikzpicture}
                \node (k) at (0,0) {K};
                \node (r) at (4,0) {R};
                \node (c) at (0,-3) {C};
                \node (h) at (4,-3) {H};
                % \node (rb) at ($\scl*(1.5,-0.5)$) {$R_X$};
                % \node (h') at ($\scl*(1.5,-1.5)$) {$H'$};
                % \draw[>->]  (rb) -- (h') node [midway,above] {};
                % \draw[>->]  (c) -- (h') node [midway,above] {};
                \draw[<-<]  (r) -- (k) node [midway,above] {};
                \draw[>->] (c) -- (h) node [midway, below] {};
                \draw[>->] (r) -- (h) node[midway, left] {};
                \draw[>->] (k) -- (c) node[midway, left] {};
                % \draw[->] (rb) to (l);
                % \draw[<-<] (rb) to (k);
                \node (k') at (1,-1) {K'};
                \node (r') at (2,-1) {R'};
                \node (c') at (1,-2) {C'};
                % \node () at (1.5,-1.5) {$PB$};
                \node () at (3,-1.5) {$PB$};
                \node () at (1.5,-2.5) {$PB$};
                \node () at (1.5,-0.5) {$PB$};
                \node () at (0.5,-1.5) {$PB$};
                \node (x) at (2,-2) {X};
                % \draw [->] (x) -- (h') node[midway] {!};
                \draw [>->] (c') -- (x);
                \draw [>->] (r') -- (x);
                \draw [>->] (k') -- (r');
                \draw [>->] (k') -- (c');
                \draw [>->] (c') -- (c);
                % \draw [->] (r') -- (rb);
                % \draw [->] (r') -- (rb);
                % \draw [->] (h') -- (h);
                \draw[>->] (r') -- (r);
                \draw[>->] (x) -- (h) node[midway,right] {};
                % \node (rb) at ($\scl*(\sclx*1.5,-0.5)$) {$R_X$};
                % \node (h') at ($\scl*(\sclx*1.5,-1.2)$) {$H'$};
                % \draw[>->]  (rb) -- (h') node [midway,above] {};
                % \draw[>->]  (c) -- (h') node [midway,above] {};
                % \draw[>->]  (rb) -- (h') node [midway,above] {};
                % \draw[>->] (rb) to (r);
                % \draw[<-<] (rb) to (k);
                \draw[>->] (k') -- (k) ;
            \end{tikzpicture}
        }
        \end{center} 
        The square $K'C'XR'$ is a pushout.
\end{lemma}
\begin{proof}
    \textbf{Graph} is adhesive by Proposition~\ref{prop:graph_adhesive}. By Definition~\ref{def:adhesive_cat}, pushouts along monomorphisms are VK squares in \textbf{Graph}. The square $KCHR$ is a pushout along monomorphisms in \textbf{Graph}. Hence $KCHR$ is VK square. By Definition~\ref{def:vk_square}, the square $K'C'XR'$ is pushout if the squares $C'CHX$ and $R'XHR$ are pullbacks. By assumption, the squares $C'CHX$ and $R'XHR$ are pullbacks. Therefore, $K'C'XR'$ is pushout.
\end{proof}

\begin{lemma}
    \label{lem:xinXcpinCrpinR}
        Consider the following pushout square in \textbf{Graph}.
    \begin{center}
        \begin{tikzpicture}
            \node (k') at (-1,-1) {K'};
            \node (r') at (-2,-1) {R'};
            \node (c') at (-1,-2) {C'};
            \node (x) at (-2,-2) {X};
            \node () at (-1.5,-1.5) {$PO$};
            \draw [>->] (c') -- (x);
            \draw [>->] (r') -- (x);
            \draw [>->] (k') -- (r');
            \draw [>->] (k') -- (c');
        \end{tikzpicture}
    \end{center}
     Let $x$ be a node (resp. an edge) in $X$. There is either a node (resp. an edge) $r'$ in $R'$ such that $h_{R'X}(r') \mathop{=} x$ or a node (resp. an edge) $c'$ in $C'$ such that $h_{C'X}(c') \mathop{=} x$.
\end{lemma}
\begin{proof}
    Suppose, by contradiction, that there exists neither a node (resp. an edge) $r'$ in $R'$ such that $h_{R'X}(r') \mathop{=} x$ nor a node (resp. an edge) $c'$ in $C'$ such that $h_{C'X}(c') \mathop{=} x$.

    Let $G$ be a graph with two distinct nodes (resp. edges) $y$ and $y'$.
    By the universal property of pushout square Definition~\ref{def:cat:po}, there exists a unique morphism $h_{XG}:X \rightarrowtail G$ such that $h_{C'G} \mathop{=} h_{C'X} \mathop{\star} h_{XG}$ and  $h_{R'G} \mathop{=} h_{R'X} \mathop{\star} h_{XG}$.
    \begin{center}
        \begin{tikzpicture}
            \node (g) at (-4,-4) {$G$};
            \node (k') at (-1,-1) {K'};
            \node (r') at (-2,-1) {R'};
            \node (c') at (-1,-2) {C'};
            \node (x) at (-2,-2) {X};
            \node () at (-1.5,-1.5) {$PO$};
            \draw [>->] (c') -- (x);
            \draw [>->] (r') -- (x);
            \draw [>->] (k') -- (r');
            \draw [>->] (k') -- (c');
            \draw [>->] (c') -- (g);
            \draw [>->] (r') -- (g);
            \draw [->] (x) -- (g) node[midway,above] {!};
        \end{tikzpicture}
    \end{center}
    However, if we define two morphisms $f$ and $g$ by 
    \begin{flalign*}
        f(z) &= h_{XG}(z) \text{if $z \mathop{\neq} x$}\\
        f(x) &= y \\
        g(z) &= h_{XG}(z) \text{if $z \mathop{\neq} x$}\\
        g(x) &= y' \\
    \end{flalign*}
    we have $f \mathop{\neq} g$ and 
    \begin{flalign*}
        h_{C'G} \mathop{=} h_{C'X} \mathop{\star} f\\
        h_{R'G} \mathop{=} h_{R'X} \mathop{\star} f\\
        h_{C'G} \mathop{=} h_{C'X} \mathop{\star} g\\
        h_{R'G} \mathop{=} h_{R'X} \mathop{\star} g
    \end{flalign*}
    which is a contradiction.
\end{proof}


\begin{lemma}
    \label{lem:g_monic}
    Consider the following commutative diagram in \textbf{Graph}.
    \begin{center}
        \begin{tikzpicture}
            \node (k) at (0,0) {K};
            \node (c) at (0,-4) {C};
            \node (l) at (-4,0) {$L$};
            \node (g) at (-4,-4) {$G$};
            \draw[>->] (l) -- (g) node [midway,above] {};
            \draw[>->] (c) -- (g) node [midway,above] {};
            \draw[>->] (k) -- (c) node[midway, left] {};
            \draw[<-<] (l) to (k);
            \node (k') at (-1,-1) {K'};
            \node (r') at (-2,-1) {R'};
            \node (c') at (-1,-2) {C'};
            \node (x) at (-2,-2) {X};
            \node () at (-1.5,-1.5) {$PO$};
            \node () at (-0.5,-1.5) {$PB$};
            \draw [>->] (c') -- (x);
            \draw [>->] (r') -- (x);
            \draw [>->] (k') -- (r');
            \draw [>->] (k') -- (c');
            \draw [>->] (c') -- (c);
            \draw [>->] (r') -- (l);
            \draw [>->] (k') -- (k);
        \end{tikzpicture}
    \end{center}
    There exists a unique monomorphism $h_{XG}:X \rightarrowtail G$ such that $h_{C'C} \mathop{\star} h_{CG} \mathop{=} h_{C'X} \mathop{\star} h_{XG}$ and  $h_{R'L} \mathop{\star} h_{LG} \mathop{=} h_{R'X} \mathop{\star} h_{XG}$.
\end{lemma}
\begin{proof}
    There exists a unique morphism $h_{XG}:X \mathop{\to} G$, by Definition~\ref{def:cat:po}, because $K'C'XR'$ is a pushout\todo{$K'C'XR'$ PO} and 
    $h_{K'C'} \mathop{\star} h_{C'C} \mathop{\star} h_{CG}
    =h_{K'R'} \mathop{\star} h_{R'L} \mathop{\star} h_{LG}$.
    % the following hold
    % \begin{flalign*}
    %     &h_{K'C'} \mathop{\star} h_{C'C} \mathop{\star} h_{CG} &\\ 
    %    =&h_{K'K} \mathop{\star} h_{KC} \mathop{\star} h_{CG} &\hspace{2cm}\text{by commutativity of $K'C'CK$}\\
    %    =&h_{K'K} \mathop{\star} h_{KL} \mathop{\star} h_{LG} &\text{by the commutativity of $KCGL$} \\
    %    =&h_{K'R'} \mathop{\star} h_{R'L} \mathop{\star} h_{LG} &\text{by the commutativity of $K'KLR'$}
    % \end{flalign*}
    Therefore, the following commutative diagram holds
    \begin{center}
        \begin{tikzpicture}
            \node (k) at (0,0) {K};
            \node (c) at (0,-4) {C};
            \node (l) at (-4,0) {$L$};
            \node (g) at (-4,-4) {$G$};
            \draw[>->] (l) -- (g) node [midway,above] {};
            \draw[>->] (c) -- (g) node [midway,above] {};
            \draw[>->] (k) -- (c) node[midway, left] {};
            \draw[<-<] (l) to (k);
            \node (k') at (-1,-1) {K'};
            \node (r') at (-2,-1) {R'};
            \node (c') at (-1,-2) {C'};
            \node (x) at (-2,-2) {X};
            \node () at (-1.5,-1.5) {$PO$};
            \node () at (-0.5,-1.5) {$PB$};
            \draw [>->] (c') -- (x);
            \draw [>->] (r') -- (x);
            \draw [>->] (k') -- (r');
            \draw [>->] (k') -- (c');
            \draw [>->] (c') -- (c);
            \draw [>->] (r') -- (l);
            \draw [>->] (k') -- (k);
            \draw [->] (x) -- (g) node[midway,above] {!};
        \end{tikzpicture}
    \end{center}
    It remains to show that $h_{XG}$ is injective.
    
    Let $x_1,x_2$ be two elements in the graph $X$ such that $x_1 \mathop{\neq} x_2$. Since $K'R'XC'$ is a pushout\todo{$K'R'XC'$ P.O.} square, by Lemma~\ref{lem:xinXcpinCrpinR}, there are 4 possible cases:
    \begin{enumerate}
        \item there are $c_1, c_2 \mathop{\in} C'$ such that $h_{C'X}(c_1) \mathop{=} x_1$ and $h_{C'X}(c_2) \mathop{=} x_2$;
        \item there are $r_1, r_2 \mathop{\in} R'$ such that $h_{R'X}(r_1) \mathop{=} x_1$ and $h_{R'X}(r_2) \mathop{=} x_2$;
        \item there are $r' \mathop{\in} R'$ and $c' \mathop{\in} C'$ such that $h_{C'X}(c') \mathop{=} x_1$ and $h_{R'X}(r') \mathop{=} x_2$;
        \item there are $r' \mathop{\in} R'$ and $c' \mathop{\in} C'$ such that $h_{C'X}(c') \mathop{=} x_2$ and $h_{R'X}(r') \mathop{=} x_1$;
    \end{enumerate}
    Since the second and fourth cases are symmetric to the first and third case, respectively, it suffices to show that in the first and third cases, we have $h_{XG}(x_1) \mathop{\neq} h_{XG}(x_2)$. 
    \begin{itemize}
        \item[Case (1)] Suppose that there are $c_1, c_2 \mathop{\in} C'$ such that $h_{C'X}(c_1) \mathop{=} x_1$ and $h_{C'X}(c_2) \mathop{=} x_2$. 
        
        We have $c_1 \mathop{\neq} c_2$ since $x_1 \mathop{\neq} x_2$ and $h_{C'X}$ is injective.
        
        Therefore we have 
        \begin{flalign*}
            h_{XG}(x_1)&= h_{XG}(h_{C'X}(c_1)) & \text{by definition of $c_1$}\\
                        &= (h_{C'X} \mathop{\star} h_{XG})(c_1)  \\
                        &= (h_{C'C} \mathop{\star} h_{CG})(c_1) & \text{by commutativity of $C'CGX$} \\
                        &\mathop{\neq} (h_{C'C} \mathop{\star} h_{CG})(c_2) & \text{by injectivity of $h_{C'C} \mathop{\star} h_{CG}$ and $c_1 \mathop{\neq} c_2$} \\
                        &= (h_{C'X} \mathop{\star} h_{XG})(c_2) & \text{by commutativity of $C'CGX$} \\
                        &= h_{XG}(h_{C'X}(c_2)) \\
                        &= h_{XG}(x_2) & \text{by definition of $c_2$}
        \end{flalign*}
        
        % \begin{center}
        %     \begin{tikzpicture}
        %         \node (k) at (0,0) {K};
        %         \node (c) at (0,-4) {C};
        %         \node (l) at (-4,0) {$L$};
        %         \node (g) at (-4,-4) {$G$};
        %         \draw[>->] (l) -- (g) node [midway,above] {};
        %         \draw[>->] (c) -- (g) node [midway,above] {};
        %         \draw[>->] (k) -- (c) node[midway, left] {};
        %         \draw[<-<] (l) to (k);
        %         \node (k') at (-1,-1) {K'};
        %         \node (r') at (-2,-1) {R'};
        %         \node (c') at (-1,-2) {C'};
        %         \node (x) at (-2,-2) {X};
        %         \node () at (-1.5,-1.5) {$PO$};
        %         \node () at (-0.5,-1.5) {$PB$};
        %         \draw [>->] (c') -- (x);
        %         \draw [>->] (r') -- (x);
        %         \draw [>->] (k') -- (r');
        %         \draw [>->] (k') -- (c');
        %         \draw [>->] (c') -- (c);
        %         \draw [>->] (r') -- (l);
        %         \draw [>->] (k') -- (k);
        %         \draw [->] (x) -- (g) node[midway,above] {!};
        %     \end{tikzpicture}
        % \end{center}
        % \item[(2)] Suppose that there are $r_1, r_2 \mathop{\in} R'$ such that $h_{R'X}(r_1) \mathop{=} x_1$ and $h_{R'X}(r_2) \mathop{=} x_2$. The proof of this case is analogous to the previous case.

        % We have $r_1 \mathop{\neq} r_2$ because $x_1 \mathop{\neq} x_2$ and $h_{R'X}$ is injective, and
        % \begin{flalign*}
        %     h_{XG}(x_1) &= h_{XG}(h_{R'X}(r_1)) & \text{by definition of $r_1$}\\
        %                  &= (h_{R'X} \mathop{\star} h_{XG})(r_1)  \\
        %                  &= (h_{R'L} \mathop{\star} h_{LG})(r_1) & \text{by commutativity of $R'XGL$} \\
        %                  &\mathop{\neq} (h_{R'L} \mathop{\star} h_{LG})(r_2) & \text{by injectivity of $h_{R'L} \mathop{\star} h_{LG}$ and $r_1 \mathop{\neq} r_2$} \\
        %                  &= (h_{R'X} \mathop{\star} h_{XG})(r_2) & \text{by commutativity of $R'XGL$} \\
        %                  &= h_{XG}(h_{R'X}(r_2)) \\
        %                  &= h_{XG}(x_2) & \text{by definition of $r_2$}
        % \end{flalign*}
        
        % \begin{center}
        %     \begin{tikzpicture}
        %         \node (k) at (0,0) {K};
        %         \node (c) at (0,-4) {C};
        %         \node (l) at (-4,0) {$L$};
        %         \node (g) at (-4,-4) {$G$};
        %         \draw[>->] (l) -- (g) node [midway,above] {};
        %         \draw[>->] (c) -- (g) node [midway,above] {};
        %         \draw[>->] (k) -- (c) node[midway, left] {};
        %         \draw[<-<] (l) to (k);
        %         \node (k') at (-1,-1) {K'};
        %         \node (r') at (-2,-1) {R'};
        %         \node (c') at (-1,-2) {C'};
        %         \node (x) at (-2,-2) {X};
        %         \node () at (-1.5,-1.5) {$PO$};
        %         \node () at (-0.5,-1.5) {$PB$};
        %         \draw [>->] (c') -- (x);
        %         \draw [>->] (r') -- (x);
        %         \draw [>->] (k') -- (r');
        %         \draw [>->] (k') -- (c');
        %         \draw [>->] (c') -- (c);
        %         \draw [>->] (r') -- (l);
        %         \draw [>->] (k') -- (k);
        %         \draw [->] (x) -- (g) node[midway,above] {!};
        %     \end{tikzpicture}
        % \end{center}
        \item[Case (3)] Suppose that there are $r' \mathop{\in} R'$ and $c' \mathop{\in} C'$ such that $h_{C'X}(c') \mathop{=} x_1$ and $h_{R'X}(r') \mathop{=} x_2$.
        % , or $h_{C'X}(c') \mathop{=} x_2$ and $h_{R'X}(r') \mathop{=} x_1$. By symmetry, we suppose that the first case holds. Let $c \mathop{=} h_{C'C}(c')$ and $l \mathop{=} h_{R'L}(r')$
        
        We have three cases:
            \begin{itemize}
                \item[(3.1)] Suppose that $h_{C'C}(c') \notin \operatorname{Im}(h_{KC})$. We have
                    \begin{flalign*}
                        h_{XG}(x_1) &= h_{XG}(h_{C'X}(c')) & \text{by definition of $c'$}\\
                                     &= (h_{C'X} \mathop{\star} h_{XG})(c') \\
                                     &= (h_{C'C} \mathop{\star} h_{CG})(c') & \text{by commutativity of $C'CGX$} \\
                                     &= h_{CG}(h_{C'C}(c')) & \\
                                    %  &= h_{CG}(c) & \\
                                     &\mathop{\neq} 
                                    %  h_{LG}(l) & \text{by Lemma~\ref{lem:b_c_same_img_exist_a} and $c\notin \operatorname{Im}(h_{KC})$}  
                                    h_{LG}(h_{R'L}(r')) & \text{by Lemma~\ref{lem:b_c_same_img_exist_a} and $h_{C'C}(c') \notin \operatorname{Im}(h_{KC})$}   
                                    \\
                                     &= (h_{R'L} \mathop{\star} h_{LG})(r') 
                                     \\
                                     &= (h_{R'X} \mathop{\star} h_{XG})(r') & \text{by commutativity of $R'XGL$} 
                                     \\
                                     &= h_{XG}(h_{R'X}(r'))
                                     \\
                                     &= h_{XG}(x_2) & \text{by definition of $r'$}
                    \end{flalign*}
                    
                    % \begin{center}
                    %     \begin{tikzpicture}
                    %         \node (k) at (0,0) {K};
                    %         \node (c) at (0,-4) {C};
                    %         \node (l) at (-4,0) {$L$};
                    %         \node (g) at (-4,-4) {$G$};
                    %         \draw[>->] (l) -- (g) node [midway,above] {};
                    %         \draw[>->] (c) -- (g) node [midway,above] {};
                    %         \draw[>->] (k) -- (c) node[midway, left] {};
                    %         \draw[<-<] (l) to (k);
                    %         \node (k') at (-1,-1) {K'};
                    %         \node (r') at (-2,-1) {R'};
                    %         \node (c') at (-1,-2) {C'};
                    %         \node (x) at (-2,-2) {X};
                    %         \node () at (-1.5,-1.5) {$PO$};
                    %         \node () at (-0.5,-1.5) {$PB$};
                    %         \draw [>->] (c') -- (x);
                    %         \draw [>->] (r') -- (x);
                    %         \draw [>->] (k') -- (r');
                    %         \draw [>->] (k') -- (c');
                    %         \draw [>->] (c') -- (c);
                    %         \draw [>->] (r') -- (l);
                    %         \draw [>->] (k') -- (k);
                    %         \draw [->] (x) -- (g) node[midway,above] {!};
                    %     \end{tikzpicture}
                    % \end{center}
                \item[(3.2)] Supppose $h_{R'L}(r') \notin \operatorname{Im}(h_{KL})$. This case is symmetric to the previous case.
                % where $c \notin \operatorname{Im}(h_{KC})$.
                \item[(3.3)] Suppose that there are $k_1, k_2 \mathop{\in} K$ such that $h_{KC}(k_1) \mathop{=} h_{C'C}(c')$ and $h_{KL}(k_2) \mathop{=} h_{R'L}(r')$. 
                
                We cannot have $k_1=k_2$. 
                
                Suppose, by contradiction, that we have 
                \begin{flalign}
                    k_1=k_2 \label{k1eqk2}
                \end{flalign} 
                
                % \begin{center}
                %     \begin{tikzpicture}
                %         \node (k) at (0,0) {K};
                %         \node (c) at (0,-4) {C};
                %         \node (l) at (-4,0) {$L$};
                %         \node (g) at (-4,-4) {$G$};
                %         \draw[>->] (l) -- (g) node [midway,above] {};
                %         \draw[>->] (c) -- (g) node [midway,above] {};
                %         \draw[>->] (k) -- (c) node[midway, left] {};
                %         \draw[<-<] (l) to (k);
                %         \node (k') at (-1,-1) {K'};
                %         \node (r') at (-2,-1) {R'};
                %         \node (c') at (-1,-2) {C'};
                %         \node (x) at (-2,-2) {X};
                %         \node () at (-1.5,-1.5) {$PO$};
                %         \node () at (-0.5,-1.5) {$PB$};
                %         \draw [>->] (c') -- (x);
                %         \draw [>->] (r') -- (x);
                %         \draw [>->] (k') -- (r');
                %         \draw [>->] (k') -- (c');
                %         \draw [>->] (c') -- (c);
                %         \draw [>->] (r') -- (l);
                %         \draw [>->] (k') -- (k);
                %         \draw [->] (x) -- (g) node[midway,above] {!};
                %     \end{tikzpicture}
                % \end{center}
                Since $K'C'CK$ is also a pullback square by Proposition~\ref{prop:pb_eq_po} and $h_{KC}(k_1) \mathop{=} h_{C'C}(c')$ and $h_{C'C}(c') \mathop{=}  h_{C'C}(c')$, there is $k' \mathop{\in} K'$ such that $h_{K'K}(k') \mathop{=} k_1$ and $h_{K'C'}(k') \mathop{=} c'$.
 
                We have $h_{K'R'}(k') \mathop{=} r'$ because $h_{R'L}$ is injective, $h_{R'L}(r') \mathop{=} h_{R'L}(r')$ and 
                \begin{flalign*}
                   &h_{R'L}(h_{K'R'}(k')) \hspace{2cm}&\\
                  =&(h_{K'R'} \mathop{\star} h_{R'L})(k') &\\
                  =&(h_{K'K} \mathop{\star} h_{KL})(k') & \text{by commutativity of $K'R'LK$}\\
                  =&(h_{KL})(h_{K'K}(k'))\\
                  =&(h_{KL})(k_1)\\
                  =&(h_{KL})(k_2) & \text{by assumption Equation~\eqref{k1eqk2}}
                %  \\
                %   =& l
                \end{flalign*}
 
             Therefore, we have the contradiction:
                     \begin{flalign*}
                         x_1 &= h_{C'X}(c')  & \text{by definition of $c'$}\\
                             &= h_{C'X}(h_{K'C'}(k')) & \text{by definition of $k'$}\\
                             &= (h_{K'C'} \mathop{\star} h_{C'X})(k') &  \\
                             &= (h_{K'R'} \mathop{\star} h_{R'X})(k') & \text{by the commutativity of $K'C'XR'$} \\
                             &= h_{R'X}(h_{K'R'}(k')) &  \\
                             &= h_{R'X}(r') &  \text{by definition of $r'$}\\
                             &= x_2& \text{by definition of $x_2$}
                     \end{flalign*}

                Thus, the following inequality holds 
                \begin{flalign}
                    k_1 \mathop{\neq} k_2 \label{k1neqk2}
                \end{flalign}

                \begin{flalign*}
                    h_{XG}(x_1) &= h_{XG}(h_{C'X}(c')) & \text{by definition of $c'$} \\
                            &= (h_{C'X} \mathop{\star} h_{XG})(c') &  \\
                            &= (h_{C'C} \mathop{\star} h_{CG})(c') & \text{by commutativity of $C'CGX$} \\
                            % &= (h_{CG})(c) & \text{by definition of $c$} \\
                            &= (h_{CG})(h_{KC}(k_1)) & \text{by definition of $k_1$} 
                            \\
                            &= (h_{KC} \mathop{\star} h_{CG})(k_1) & 
                            \\
                            &\mathop{\neq} (h_{KC} \mathop{\star} h_{CG})(k_2)  & \text{by injectivity of $h_{KC}\star h_{CG}$ and Equation~\eqref{k1neqk2}} 
                            \\
                            &= (h_{KL} \mathop{\star} h_{LG})(k_2) & \text{by commutativity of $KLGC$}
                            \\
                            &= h_{LG}(h_{KL}(k_2)) & 
                            \\
                            % &= h_{LG}(l) & \text{by the definition of $k_2$}
                            % \\
                            &= h_{LG}(h_{R'L}(r')) & \text{by the definition of $k_2$}
                            \\
                            &= (h_{R'L} \mathop{\star} h_{LG})(r') & 
                            \\
                            &= (h_{R'X} \mathop{\star} h_{XG})(r') & \text{by commutativity of $R'XLG$} \\
                            &= h_{XG}(h_{R'X}(r'))\\
                                &= h_{XG}(x_2) & \text{by definition of $r'$}
                \end{flalign*}
                
                % \begin{center}
                %     \begin{tikzpicture}
                %         \node (k) at (0,0) {K};
                %         \node (c) at (0,-4) {C};
                %         \node (l) at (-4,0) {$L$};
                %         \node (g) at (-4,-4) {$G$};
                %         \draw[>->] (l) -- (g) node [midway,above] {};
                %         \draw[>->] (c) -- (g) node [midway,above] {};
                %         \draw[>->] (k) -- (c) node[midway, left] {};
                %         \draw[<-<] (l) to (k);
                %         \node (k') at (-1,-1) {K'};
                %         \node (r') at (-2,-1) {R'};
                %         \node (c') at (-1,-2) {C'};
                %         \node (x) at (-2,-2) {X};
                %         \node () at (-1.5,-1.5) {$PO$};
                %         \node () at (-0.5,-1.5) {$PB$};
                %         \draw [>->] (c') -- (x);
                %         \draw [>->] (r') -- (x);
                %         \draw [>->] (k') -- (r');
                %         \draw [>->] (k') -- (c');
                %         \draw [>->] (c') -- (c);
                %         \draw [>->] (r') -- (l);
                %         \draw [>->] (k') -- (k);
                %         \draw [->] (x) -- (g) node[midway,above] {!};
                %     \end{tikzpicture}
                % \end{center}
            \end{itemize}            
    \end{itemize}
\end{proof}



\begin{lemma}
    \label{lem:h_hp_diff_g_gp_diff}
    
    Let $X$ be a graph. Let $\rho=(L \overset{l}{\leftarrowtail} K \overset{r}{\rightarrowtail} R)$ be an injective DPO rewriting rule which is \(X\)-non-increasing under $\Psi$. Let $G \mathop{\Rightarrow}_{\rho,\mathcal{M}} H$ be a rewriting step defined by the following DPO diagram:

    \begin{center}
        \begin{tikzpicture}
            \node (k) at (0,1) {K};
            \node (l) at (-2,1) {L};
            \node (r) at (2,1) {R};
            \node (c) at (0,-1) {C};
            \node (g) at (-2,-1) {G};
            \node (h) at (2,-1) {H};
            % \node (rb) at (1.5,-0.5) {$R_X$};
            % \node (h') at (1.5,-1.5) {$H'$};
            % \draw[>->]  (rb) -- (h') node [midway,above] {};
            % \draw[>->]  (c) -- (h') node [midway,above] {};
            \draw[<-<]  (l) -- (k) node [midway,below] {$l$};
            \draw[>->]  (k) -- (r) node [midway,below] {$r$};
            \draw[>->] (c) -- (g) node [midway, above] {$l'$};
            \draw[>->] (c) -- (h) node [midway,above] {$r'$};
            \draw[>->] (l) -- (g) node[midway, right] {$m$};
            \draw[>->] (r) -- (h) node[midway, left] {$m'$};
            \draw[>->] (k) -- (c) node[midway, left] {};
            \node () [at=($(l)!0.5!(c)$)] {$PO$};
            \node () [at=($(r)!0.5!(c)$)] {$PO$};
            % \draw[>->] (rb) to (r);
            % \draw[->] (rb) to (l);
            % \draw[<-<] (rb) to (k);
        \end{tikzpicture}
    \end{center}
   
    Let $h_{XH}',h_{XH}'':X \rightarrowtail H$ be monomorphisms.
    In the category \textbf{Graph}, we can construct the following commutative diagram
    \begin{center}
        \begin{tikzpicture} 
            \node (k) at (0,0) {K};
            \node (c) at (0,-4) {C};
            \node (l) at (-4,0) {$R$};
            \node (g) at (-4,-4) {$H$};
            \node (r) at (4,0) {$L$};
            \node (h) at (4,-4) {$G$};
            \draw [>->] (k) -- (r); 
            \draw [>->] (c) -- (h);
            \draw [>->] (r) -- (h);
            \draw[>->] (l) -- (g) node [midway,above] {};
            \draw[>->] (c) -- (g) node [midway,above] {};
            \draw[>->] (k) -- (c) node[midway, left] {};
            \draw[<-<] (l) to (k);
            \node (k') at (-2,-1) {K'};
            \node (r') at (-3.5,-1.5) {R'};
            \node (c') at (-1.5,-2.2) {C'};
            \node (x) at (-3,-3) {X};
            \draw [>->] (c') -- (x);
            \draw [>->] (r') -- (x); 
            \draw [>->] (k') -- (r');
            \draw [>->] (k') -- (c');
            \draw [>->] (c') -- (c);
            \draw [>->] (r') -- (l);
            \draw [>->] (k') -- (k);

            \node (k'') at (-0.5,-1.5) {K''};
            \node (r'') at (-2.5,-2) {R''};
            \node (c'') at (-0.5,-2.5) {C''};
            \draw [>->] (c'') -- (x);
            \draw [>->] (r'') -- (x);
            \draw [>->] (k'') -- (r'');
            \draw [>->] (k'') -- (c'');
            \draw [>->] (k'') -- (k);
            \draw [>->] (c'') -- (c);
            \draw [>->] (r'') -- (l);

            \draw [>->] (x) edge[bend left] node[pos=0.1,below]{$h_{XH}'$} (g);
            \draw [>->] (x) edge[bend right] node[midway,above]{$h_{XH}''$} (g);

        \end{tikzpicture}
    \end{center}
    where $K'R'XC'$, $K''R''XC''$, $K'KCC'$, $K''KCC''$, $R'RHX$ and $R''RHX$ are pullbacks.

    By Lemma~\ref{kpcpck_pullback}, $K'KLR'$ and $K''KCC''$ are pullbacks. 

    By Lemma~\ref{kpcpxrp_po}, $K'R'XC'$ and $K''R''XC''$ are pushout.

    Since $\rho$ is $X$-non-increasing under $\Psi$, there are $\Psi(R'): R' \rightarrowtail L$ and $\Psi(R''):R'' \rightarrowtail L$ such that $K'R'LK$ and $K''R''LK$ are pullbacks.

    By Lemma~\ref{lem:g_monic}, we have the following commutative diagram
    \begin{center}
        \begin{tikzpicture} 
            \node (k) at (0,0) {K};
            \node (c) at (0,-4) {C};
            \node (l) at (-4,0) {L};
            \node (g) at (-4,-4) {G};
            \node (r) at (4,0) {L};
            \node (h) at (4,-4) {G};
            \draw [>->] (k) -- (r); 
            \draw [>->] (c) -- (h);
            \draw [>->] (r) -- (h);
            \draw[>->] (l) -- (g) node [midway,above] {};
            \draw[>->] (c) -- (g) node [midway,above] {};
            \draw[>->] (k) -- (c) node[midway, left] {};
            \draw[<-<] (l) to (k);
            \node (k') at (-2,-1) {K'};
            \node (r') at (-3.5,-1.5) {R'};
            \node (c') at (-1.5,-2.2) {C'};
            \node (x) at (-3,-3) {X};
            \draw [>->] (c') -- (x);
            \draw [>->] (r') -- (x); 
            \draw [>->] (k') -- (r');
            \draw [>->] (k') -- (c');
            \draw [>->] (c') -- (c);
            \draw [>->] (r') -- (l);
            \draw [>->] (k') -- (k);

            \node (k'') at (-0.5,-1.5) {K''};
            \node (r'') at (-2.5,-2) {R''};
            \node (c'') at (-0.5,-2.5) {C''};
            \draw [>->] (c'') -- (x);
            \draw [>->] (r'') -- (x);
            \draw [>->] (k'') -- (r'');
            \draw [>->] (k'') -- (c'');
            \draw [>->] (k'') -- (k);
            \draw [>->] (c'') -- (c);
            \draw [>->] (r'') -- (l);

            \draw [>->] (x) edge[bend left] node[pos=0.1,below]{$h_{XG}'$} (g);
            \draw [>->] (x) edge[bend right] node[midway,above]{$h_{XG}''$} (g);

        \end{tikzpicture}
    \end{center}
    where  
    \begin{itemize}
        \item $K'R'XC'$ is pushout and pullback;
        \item $K''R''XC''$ is pushout and pullback;
        \item $K'KCC'$ and $K''KCC''$ are pullback;
        \item $h_{XG}':X \mathop{\to} G$ is the unique morphism such that $h_{C'C} \mathop{\star} h_{CG} \mathop{=} h_{C'X} \mathop{\star} h_{XG}'$ and  $h_{R'L} \mathop{\star} h_{LG} \mathop{=} h_{R'X} \mathop{\star} h_{XG}'$ where $h_{R'L} \mathop{=} \Psi(R')$;
        \item $h_{XG}'':X \mathop{\to} G$ is the unique morphism such that $h_{C''C} \mathop{\star} h_{CG} \mathop{=} h_{C''X} \mathop{\star} h_{XG}''$ and  $h_{R''L} \mathop{\star} h_{LG} \mathop{=} h_{R''X} \mathop{\star} h_{XG}''$ where $h_{R''L} \mathop{=} \Psi(R'')$.
    \end{itemize}
    If $h_{XH}' \mathop{\neq} h_{XH}''$ then $h_{XG}' \mathop{\neq} h_{XG}''$.
\end{lemma}
\begin{proof}
    Suppose that $h_{XH}' \mathop{\neq} h_{XH}''$. We are going to show $h_{XG}' \mathop{\neq} h_{XG}''$.

    From $h_{XH}' \mathop{\neq} h_{XH}''$, we deduce there exists an element $x \mathop{\in} X$ such that $h_{XH}(x) \mathop{\neq} h_{XH}''(x)$. We can suppose that \begin{flalign}
        \text{$x$ is either an isolated nodes or an edge.} \label{x_isolated_or_edge}
    \end{flalign}  \todo{$x$ is either an isolated node or an edge}
    because otherwise $x$ would be a non-isolated node and we can take an incident edge of $x$.

    \noindent
    Since $K'R'XC'$ and $K''R''XC''$ are pushouts, by Lemma~\ref{lem:xinXcpinCrpinR}, there are 4 possible cases:
    \begin{enumerate}
        \item there are $c' \mathop{\in} C', c'' \mathop{\in} C''$ such that $h_{C'X}(c') \mathop{=} h_{C'X}(c'') \mathop{=} x$;
        \item  there are $r' \mathop{\in} R', r'' \mathop{\in} R''$ such that $h_{R'X}(r') \mathop{=} h_{R'X}(r'') \mathop{=} x$
        \item there are arrows $r' \mathop{\in} R'$ and $c'' \mathop{\in} C''$ such that 
        \begin{flalign*}
            h_{C''X}(c'') \mathop{=} x
            \\
            h_{R'X}(r') \mathop{=} x
            \\
            \nexists c' \mathop{\in} C'. h_{C'X}(c') \mathop{=} x 
            \\
            \nexists r'' \mathop{\in} R''. h_{R''X}(r'') \mathop{=} x 
        \end{flalign*}
        \item there are arrows $r'' \mathop{\in} R''$ and $c' \mathop{\in} C'$ such that 
        \begin{flalign*}
            h_{C'X}(c') \mathop{=} x
            \\
            h_{R''X}(r'') \mathop{=} x
            \\
            \nexists c'' \mathop{\in} C''. h_{C''X}(c'') \mathop{=} x 
            \\
            \nexists r' \mathop{\in} R'. h_{R'X}(r') \mathop{=} x   
        \end{flalign*}
    \end{enumerate}

    We are going to show that in each case we have $h_{XG}'(x) \mathop{\neq} h_{XG}''(x)$. Since the fourth case is symmetric to the third case, it suffices to analyse the first three cases.

    \begin{itemize}
        \item[(1)] Suppose that there are $c' \mathop{\in} C', c'' \mathop{\in} C''$ such that $h_{C'X}(c') \mathop{=} h_{C'X}(c'') \mathop{=} x$. 
        Suppose, by contradiction, that the following equality holds
        \begin{flalign}
            h_{C'C}(c') \mathop{=} h_{C''C}(c'') \label{cpccp_eq_cppccpp}
        \end{flalign}
         we have the following contradiction:
        \begin{flalign*}
            &h_{XH}'(x) \\
           =&h_{XH}'(h_{C'X}(c')) \\
           =&(h_{C'X} \mathop{\star} h_{XH}')(c') \\
           =&(h_{C'C} \mathop{\star} h_{CH})(c') &\text{By commutativity of $C'XHC$}\\
           =&h_{CH}(h_{C'C}(c')) \\
           =&h_{CH}(h_{C''C}(c'')) & \text{ by Equation~\eqref{cpccp_eq_cppccpp}}\\
           =&(h_{C''C} \mathop{\star} h_{CH})(c'') \\
           =&(h_{C''X} \mathop{\star} h_{XH}'')(c'') &\text{By commutativity of $C''CHX$}\\
           =&h_{XH}''(h_{C''X}(c'')) \\
           \mathop{=} &h_{XH}''(x) \\
        \end{flalign*}
        
        % \begin{center}
        %     \begin{tikzpicture} 
        %         \node (k) at (0,0) {K};
        %         \node (c) at (0,-4) {C};
        %         \node (l) at (-4,0) {$L$};
        %         \node (g) at (-4,-4) {$G$};
        %         \draw[>->] (l) -- (g) node [midway,above] {};
        %         \draw[>->] (c) -- (g) node [midway,above] {};
        %         \draw[>->] (k) -- (c) node[midway, left] {};
        %         \draw[<-<] (l) to (k);
        %         \node (k') at (-2,-1) {K'};
        %         \node (r') at (-3.5,-1.5) {R'};
        %         \node (c') at (-1.5,-2.2) {C'};
        %         \node (x) at (-3,-3) {X};
        %         \draw [>->] (c') -- (x);
        %         \draw [>->] (r') -- (x); 
        %         \draw [>->] (k') -- (r');
        %         \draw [>->] (k') -- (c');
        %         \draw [>->] (c') -- (c);
        %         \draw [>->] (r') -- (l);
        %         \draw [>->] (k') -- (k);
    
        %         \node (k'') at (-0.5,-1.5) {K''};
        %         \node (r'') at (-2.5,-2) {R''};
        %         \node (c'') at (-0.5,-2.5) {C''};
        %         \draw [>->] (c'') -- (x);
        %         \draw [>->] (r'') -- (x);
        %         \draw [>->] (k'') -- (r'');
        %         \draw [>->] (k'') -- (c'');
        %         \draw [>->] (k'') -- (k);
        %         \draw [>->] (c'') -- (c);
        %         \draw [>->] (r'') -- (l);
    
        %         \draw [>->] (x) edge[bend left] node[pos=0.3,below]{!g''} (g);
        %         \draw [>->] (x) edge[bend right] node[midway,above]{!g'} (g);
    
        %     \end{tikzpicture}
        % \end{center}
        Therefore, we have 
        \begin{flalign}
            h_{C'C}(c') \mathop{\neq} h_{C''C}(c'') \label{cpccp_neq_cppccpp}
        \end{flalign} 
        \begin{flalign*}
            h_{XG}'(x)&= h_{XG}'(h_{C'X}(c')) & \text{by definition of $c'$}\\
                        &= (h_{C'X} \mathop{\star} h_{XG}')(c')  \\
                        &= (h_{C'C} \mathop{\star} h_{CG})(c') & \text{by commutativity of $C'CGX$} \\
                        &= h_{CG}(h_{C'C}(c')) &  \\
                        &\mathop{\neq} h_{CG}(h_{C''C}(c'')) & \text{by Equation~\eqref{cpccp_neq_cppccpp} and the injectivity of $h_{CG}$} \\
                        &= (h_{C''C} \mathop{\star} h_{CG})(c'') &  \\
                        &= (h_{C''X} \mathop{\star} h_{XG}'')(c'') & \text{by commutativity of $C''CGX$} \\
                        &= h_{XG}''(h_{C''X}(c'')) \\
                        &= h_{XG}''(x) & \text{by definition of $c''$}
        \end{flalign*}
 
        \item[(2)] Suppose that there are $r' \mathop{\in} R', r'' \mathop{\in} R''$ such that $h_{R'X}(r') \mathop{=} h_{R'X}(r'') \mathop{=} x$. We have 
        \begin{flalign}
            h_{R'R}(r') \mathop{\neq} h_{R''R}(r'') \label{rprrppr}
        \end{flalign}  
        by an argument analogous to the argument for $h_{C'C}(c') \mathop{\neq} h_{C''C}(c'')$ in the previous case. Since $\rho$ is $X$-non-increasing, Equation~\eqref{rprrppr} and Equation~\eqref{x_isolated_or_edge} hold, we have \todo{use : x is either an isolated node or an edge}       
        \begin{flalign}
            h_{R'L}(r') \mathop{\neq} h_{R''L}(r'') \label{rplrprpplrpp}
        \end{flalign}  

        % Therefore, we have  
        % \begin{flalign*}
        %     h_{XG}'(x)&= h_{XG}'(h_{R'X}(r')) & \text{by definition of $r'$}\\
        %         &= (h_{R'X} \mathop{\star} h_{XG}')(r')  \\
        %         &= (h_{R'L} \mathop{\star} h_{LG})(r') & \text{by commutativity of $R'LGX$} \\
        %         &= h_{LG}(h_{R'L}(r')) &  \\
        %         &\mathop{\neq} h_{LG}(h_{R''L}(r'')) & \text{by $h_{R'R}(r') \mathop{\neq} h_{R''R}(r'')$ and the injectivity of $h_{RG}$} \\
        %         &= (h_{R''L} \mathop{\star} h_{LG})(r'') &  \\
        %         &= (h_{R''X} \mathop{\star} h_{XG}'')(r'') & \text{by commutativity of $R''LGX$} \\
        %         &= h_{XG}''(h_{R''X}(r'')) \\
        %         &= h_{XG}''(x) & \text{by definition of $r''$}
        % \end{flalign*}
         
        % \todo{$h_{R_XL}$ edge-injective} Since $h_{R_XL}$ is edge-injective by assumption, because it is $X$-non-increasing by assumption. Therefore, we have $h_{R'L}(r') \mathop{\neq} h_{R''L}(r'')$ because \( h_{R'R}(r') \mathop{\neq} h_{R''R}(r'') \). Hence, from~\ref{rprneqrppr}, we have 
        % \begin{flalign}
        %     h_{R'L}(r') \mathop{\neq} h_{R''L}(r'') \label{rplrprpplrpp}
        % \end{flalign}
        Hence, the following inequality holds
        
        % \begin{center}
        %     \begin{tikzpicture} 
        %         \node (k) at (0,0) {K};
        %         \node (c) at (0,-4) {C};
        %         \node (l) at (-4,0) {$L$};
        %         \node (g) at (-4,-4) {$G$};
        %         \draw[>->] (l) -- (g) node [midway,above] {};
        %         \draw[>->] (c) -- (g) node [midway,above] {};
        %         \draw[>->] (k) -- (c) node[midway, left] {};
        %         \draw[<-<] (l) to (k);
        %         \node (k') at (-2,-1) {K'};
        %         \node (r') at (-3.5,-1.5) {R'};
        %         \node (c') at (-1.5,-2.2) {C'};
        %         \node (x) at (-3,-3) {X};
        %         \draw [>->] (c') -- (x);
        %         \draw [>->] (r') -- (x); 
        %         \draw [>->] (k') -- (r');
        %         \draw [>->] (k') -- (c');
        %         \draw [>->] (c') -- (c);
        %         \draw [>->] (r') -- (l);
        %         \draw [>->] (k') -- (k);
    
        %         \node (k'') at (-0.5,-1.5) {K''};
        %         \node (r'') at (-2.5,-2) {R''};
        %         \node (c'') at (-0.5,-2.5) {C''};
        %         \draw [>->] (c'') -- (x);
        %         \draw [>->] (r'') -- (x);
        %         \draw [>->] (k'') -- (r'');
        %         \draw [>->] (k'') -- (c'');
        %         \draw [>->] (k'') -- (k);
        %         \draw [>->] (c'') -- (c);
        %         \draw [>->] (r'') -- (l);
    
        %         \draw [>->] (x) edge[bend left] node[pos=0.3,below]{!g''} (g);
        %         \draw [>->] (x) edge[bend right] node[midway,above]{!g'} (g);
    
        %     \end{tikzpicture}
        % \end{center}
        \begin{flalign*}
            h_{XG}'(x) &= h_{XG}'(h_{R'X}(r')) & \text{by definition of $r'$}\\
                         &= (h_{R'X} \mathop{\star} h_{XG}')(r')  \\
                         &= (h_{R'L} \mathop{\star} h_{LG})(r') & \text{by commutativity of $R'XGL$} \\
                         &= h_{LG}(h_{R'L}(r')) &  \\
                         &\mathop{\neq} h_{LG}(h_{R''L}(r'')) &  \text{By injectivity of $h_{LG}$ and Equation~\eqref{rplrprpplrpp}}\\
                         &= (h_{R''L} \mathop{\star} h_{LG})(r'') &  \\
                         &= (h_{R''X} \mathop{\star} h_{XG}'')(r'') & \text{by commutativity of $R''XGL$} \\
                         &= h_{XG}''(h_{R''X}(r'')) \\
                         &= h_{XG}''(x) & \text{by definition of $r''$}
        \end{flalign*}  
        
        % \begin{center}
        %     \begin{tikzpicture} 
        %         \node (k) at (0,0) {K};
        %         \node (c) at (0,-4) {C};
        %         \node (l) at (-4,0) {$L$};
        %         \node (g) at (-4,-4) {$G$};
        %         \draw[>->] (l) -- (g) node [midway,above] {};
        %         \draw[>->] (c) -- (g) node [midway,above] {};
        %         \draw[>->] (k) -- (c) node[midway, left] {};
        %         \draw[<-<] (l) to (k);
        %         \node (k') at (-2,-1) {K'};
        %         \node (r') at (-3.5,-1.5) {R'};
        %         \node (c') at (-1.5,-2.2) {C'};
        %         \node (x) at (-3,-3) {X};
        %         \draw [>->] (c') -- (x);
        %         \draw [>->] (r') -- (x); 
        %         \draw [>->] (k') -- (r');
        %         \draw [>->] (k') -- (c');
        %         \draw [>->] (c') -- (c);
        %         \draw [>->] (r') -- (l);
        %         \draw [>->] (k') -- (k);
    
        %         \node (k'') at (-0.5,-1.5) {K''};
        %         \node (r'') at (-2.5,-2) {R''};
        %         \node (c'') at (-0.5,-2.5) {C''};
        %         \draw [>->] (c'') -- (x);
        %         \draw [>->] (r'') -- (x);
        %         \draw [>->] (k'') -- (r'');
        %         \draw [>->] (k'') -- (c'');
        %         \draw [>->] (k'') -- (k);
        %         \draw [>->] (c'') -- (c);
        %         \draw [>->] (r'') -- (l);
    
        %         \draw [>->] (x) edge[bend left] node[pos=0.3,below]{!g''} (g);
        %         \draw [>->] (x) edge[bend right] node[midway,above]{!g'} (g);
    
        %     \end{tikzpicture}
        % \end{center}
        \item[(3)] Suppose, that there are arrows $r' \mathop{\in} R'$ and $c'' \mathop{\in} C''$ such that 
        \begin{flalign*}
            h_{C''X}(c'') \mathop{=} x
            \\
            h_{R'X}(r') \mathop{=} x
            \\
            \nexists c' \mathop{\in} C'. h_{C'X}(c') \mathop{=} x 
        \end{flalign*}
        \begin{flalign}
            \nexists r'' \mathop{\in} R''. h_{R''X}(r'') \mathop{=} x  \label{assumption_norpp}
        \end{flalign}

        Suppose, by contradiction, that we have \begin{flalign}
            h_{C''C}(c'') \mathop{\in} \operatorname{Im}(h_{KC})  \label{assump_c_in_imhkc}
        \end{flalign} 
        Then,
        there are $k_2 \mathop{\in} K$ such that $h_{KC}(k_2) \mathop{=} h_{C''C}(c'')$.

        There is $k'' \mathop{\in} K''$ such that $h_{K''K}(k'') \mathop{=} k_2$ and $h_{K''C''}(k'') \mathop{=} c''$ becase $K''C''CK$ is a pullback square and $h_{KC}(k_2) \mathop{=} h_{C''C}(c'')$.

        We have $h_{K''R''}(k'') \mathop{\in} R''$ and 
        \begin{flalign*}
          &h_{R''X}(h_{K''R''}(k''))\\
         =&(h_{K''R''} \mathop{\star} h_{R''X})(k'') \\
         =&(h_{K''C''} \mathop{\star} h_{C''X})(k'') & \text{by commutativity of $K''R''XC''$}\\
         =&h_{C''X}(h_{K''C''}(k''))\\
         =&h_{C''X}(c'')\\
         =&x
        \end{flalign*}
        which contradicts the assumption Equation~\eqref{assumption_norpp}.
        
        Thus, the following holds
        \begin{flalign}
            h_{C''C}(c'') \notin \operatorname{Im}(h_{KC})  \label{assump_c_notin_imhkc}
        \end{flalign} 

        We deduce
        \begin{flalign*}
            h_{XG}'(x) &= h_{XG}'(h_{C''X}(c'')) & \text{by definition of $c''$}\\
                         &= (h_{C''X} \mathop{\star} h_{XG}')(c'') \\
                         &= (h_{C''C} \mathop{\star} h_{CG})(c'') & \text{by commutativity of $C''CGX$} \\
                         &= h_{CG}(h_{C''C}(c'')) &  \\
                         &\mathop{\neq}  h_{LG}(h_{R'L}(r')) & \text{by Lemma~\ref{lem:b_c_same_img_exist_a} and the assumption Equation~\eqref{assump_c_notin_imhkc}} \\
                         &= (h_{R'L} \mathop{\star} h_{LG})(r') \\
                         &= (h_{R'X} \mathop{\star} h_{XG}'')(r') & \text{by commutativity of $R'XGL$} \\
                         &= h_{XG}''(h_{R'X}(r'))\\
                         &= h_{XG}''(x) & \text{by definition of $r'$}
        \end{flalign*} 

    %     We analyse the two cases where $h_{C''C}(c'') \notin \operatorname{Im}(h_{KC}) $ and 
    %         \begin{itemize}
    %             \item[(3.1)] Suppose, that the following holds
    %                 \begin{flalign}
    %                     h_{C''C}(c'') \notin \operatorname{Im}(h_{KC})  \label{assump_c_notin_imhkc}
    %                 \end{flalign} then, we have 
    %                 \begin{flalign*}
    %                     h_{XG}'(x) &= h_{XG}'(h_{C''X}(c'')) & \text{by definition of $c''$}\\
    %                                  &= (h_{C''X} \mathop{\star} h_{XG}')(c'') \\
    %                                  &= (h_{C''C} \mathop{\star} h_{CG})(c'') & \text{by commutativity of $C''CGX$} \\
    %                                  &= h_{CG}(h_{C''C}(c'')) &  \\
    %                                  &\mathop{\neq}  h_{LG}(h_{R'L}(r')) & \text{by Lemma~\ref{lem:b_c_same_img_exist_a} and the assumption Equation~\eqref{assump_c_notin_imhkc}} \\
    %                                  &= (h_{R'L} \mathop{\star} h_{LG})(r') \\
    %                                  &= (h_{R'X} \mathop{\star} h_{XG}'')(r') & \text{by commutativity of $R'XGL$} \\
    %                                  &= h_{XG}''(h_{R'X}(r'))\\
    %                                  &= h_{XG}''(x) & \text{by definition of $r'$}
    %                 \end{flalign*} 
    %             \item[(3.2)] Suppose, that the following holds
    %             \begin{flalign}
    %                 h_{R'R}(r') \notin \operatorname{Im}(h_{KR}) \label{rprrp_im_kr}
    %             \end{flalign} 
    %             Since $\rho$ is $X$-non-increasing, by the assumption Equation~\eqref{rprrp_im_kr}, (Definition~\ref{def:creates_more_x_on_the_left})$h_{R'L}$ does not identify non-interface elements with interface elements. Therefore, the following holds 
    %             \begin{flalign}
    %                 h_{R'L}(r') \notin \operatorname{Im}(h_{KL}) \label{lnotinhkl}
    %             \end{flalign} 
    %             \todo{assumption: does not identify non-interface elements with interface elements} 
    %             Therefore, 
    %     
    %     % \begin{center}
    %     %     \begin{tikzpicture} 
    %     %         \node (k) at (0,0) {K};
    %     %         \node (c) at (0,-4) {C};
    %     %         \node (l) at (-4,0) {$L$};
    %     %         \node (g) at (-4,-4) {$G$};
    %     %         \draw[>->] (l) -- (g) node [midway,above] {};
    %     %         \draw[>->] (c) -- (g) node [midway,above] {};
    %     %         \draw[>->] (k) -- (c) node[midway, left] {};
    %     %         \draw[<-<] (l) to (k);
    %     %         \node (k') at (-2,-1) {K'};
    %     %         \node (r') at (-3.5,-1.5) {R'};
    %     %         \node (c') at (-1.5,-2.2) {C'};
    %     %         \node (x) at (-3,-3) {X};
    %     %         \draw [>->] (c') -- (x);
    %     %         \draw [>->] (r') -- (x); 
    %     %         \draw [>->] (k') -- (r');
    %     %         \draw [>->] (k') -- (c');
    %     %         \draw [>->] (c') -- (c);
    %     %         \draw [>->] (r') -- (l);
    %     %         \draw [>->] (k') -- (k);
    
    %     %         \node (k'') at (-0.5,-1.5) {K''};
    %     %         \node (r'') at (-2.5,-2) {R''};
    %     %         \node (c'') at (-0.5,-2.5) {C''};
    %     %         \draw [>->] (c'') -- (x);
    %     %         \draw [>->] (r'') -- (x);
    %     %         \draw [>->] (k'') -- (r'');
    %     %         \draw [>->] (k'') -- (c'');
    %     %         \draw [>->] (k'') -- (k);
    %     %         \draw [>->] (c'') -- (c);
    %     %         \draw [>->] (r'') -- (l);
    
    %     %         \draw [>->] (x) edge[bend left] node[pos=0.3,below]{!g''} (g);
    %     %         \draw [>->] (x) edge[bend right] node[midway,above]{!g'} (g);
    
    %     %     \end{tikzpicture}
    %     % \end{center}
    %                 \begin{flalign*}
    %                     h_{XG}'(x) &= h_{XG}'(h_{R'X}(r')) & \text{by definition of $r'$}\\
    %                                  &= (h_{R'X} \mathop{\star} h_{XG}')(r') \\
    %                                  &= (h_{R'L} \mathop{\star} h_{LG})(r') & \text{by commutativity of $R'LGX$} \\
    %                                  &= h_{LG}(h_{R'L}(r')) &  \\
    %                                  &= h_{CG}(h_{C''C}(c'')) & \text{by Lemma~\ref{lem:b_c_same_img_exist_a} and the assumption Equation~\eqref{lnotinhkl}}  \\
    %                                  &= (h_{C''C} \mathop{\star} h_{CG})(c'') \\
    %                                  &= (h_{C''X} \mathop{\star} h_{XG}'')(c'') & \text{by commutativity of $C''XGC$} \\
    %                                  &= h_{XG}''(h_{C''X}(c''))\\
    %                                  &= h_{XG}''(x) & \text{by definition of $c''$}
    %                 \end{flalign*}
    %                  \newpage  
    %          
    % \begin{center}
    %         \begin{tikzpicture} 
    %             \node (k) at (0,0) {K};
    %             \node (c) at (0,-4) {C};
    %             \node (l) at (-4,0) {$L$};
    %             \node (g) at (-4,-4) {$G$};
    %             \draw[>->] (l) -- (g) node [midway,above] {};
    %             \draw[>->] (c) -- (g) node [midway,above] {};
    %             \draw[>->] (k) -- (c) node[midway, left] {};
    %             \draw[<-<] (l) to (k);
    %             \node (k') at (-2,-1) {K'};
    %             \node (r') at (-3.5,-1.5) {R'};
    %             \node (c') at (-1.5,-2.2) {C'};
    %             \node (x) at (-3,-3) {X};
    %             \draw [>->] (c') -- (x);
    %             \draw [>->] (r') -- (x); 
    %             \draw [>->] (k') -- (r');
    %             \draw [>->] (k') -- (c');
    %             \draw [>->] (c') -- (c);
    %             \draw [>->] (r') -- (l);
    %             \draw [>->] (k') -- (k);
    
    %             \node (k'') at (-0.5,-1.5) {K''};
    %             \node (r'') at (-2.5,-2) {R''};
    %             \node (c'') at (-0.5,-2.5) {C''};
    %             \draw [>->] (c'') -- (x);
    %             \draw [>->] (r'') -- (x);
    %             \draw [>->] (k'') -- (r'');
    %             \draw [>->] (k'') -- (c'');
    %             \draw [>->] (k'') -- (k);
    %             \draw [>->] (c'') -- (c);
    %             \draw [>->] (r'') -- (l);
     
    %             \draw [>->] (x) edge[bend left] node[pos=0.3,below]{!g''} (g);
    %             \draw [>->] (x) edge[bend right] node[midway,above]{!g'} (g);
    
    %         \end{tikzpicture}
    %     \end{center}

    %             \item[(3.3)]  
    %             Suppose, 
    %             \color{red}
    %             that \begin{flalign}
    %                 h_{R'R}(r') \mathop{\in} h_{KR}(K) 
    %                 \\
    %                 h_{C''C}(c'') \mathop{\in} h_{KC}(K)
    %             \end{flalign}
    %             Thus,
    %             there are $k_1, k_2 \mathop{\in} K$ such that $h_{KR}(k_1) \mathop{=} h_{R'R}(r')$ and $h_{KC}(k_2) \mathop{=} h_{C''C}(c'')$.
    %             \color{black}

    %             There is $k'' \mathop{\in} K''$ such that $h_{K''K}(k'') \mathop{=} k_2$ and $h_{K''C''}(k'') \mathop{=} c''$ becase $K''C''CK$ is a pullback square and $h_{KC}(k_2) \mathop{=} c \mathop{=} h_{C''C}(c'')$.

    %             We cannot have $k_1=k_2$. Suppose, by contradiction, that we have $k_1=k_2$. 
    %             \color{red} 
    %             Let $r'' \mathop{=} h_{K''R''}(k'')$. We have 
    %             \begin{flalign}
    %                 &h_{R''R}(r'')\\
    %                =&h_{R''R}(h_{K''R''}(k''))\\
    %                =&(h_{K''R''} \mathop{\star} h_{R''R})(k'') & \text{by definition of $r''$}\\
    %                =&(h_{K''K} \mathop{\star} h_{KR})(k'') &\text{by commutativity of $K''R''RK$}\\
    %                =&h_{KR}(h_{K''K}(k''))  \\
    %                =&h_{KR}(k_1) & \text{by definition of $k_1$} \\
    %                =&r
    %             \end{flalign}
    %             \color{black}
    %             % We have therefore $h_{K''C''}(k'') \mathop{=} c''$ because $h_{C''C}$ is injective, $h_{C''C}(c'') \mathop{=} c$ and 
    %             % \begin{flalign*}
    %             %    &h_{C''C}(h_{K''C''}(k'')) \\
    %             %   =&(h_{K''C''} \mathop{\star} h_{C''C})(k'')\\
    %             %   =&(h_{K''K} \mathop{\star} h_{KC})(k'')\\
    %             %   =&h_{KC}(h_{K''K}(k''))\\
    %             %   =&h_{KC}(k_2)\\
    %             %   =&c
    %             % \end{flalign*}
    %             and 
    %             \begin{flalign*}
    %               &h_{R''X}(r'')\\
    %              =&h_{R''X}(h_{K''R''}(k'')) \\
    %              =&(h_{K''R''} \mathop{\star} h_{R''X})(k'') \\
    %              =&(h_{K''C''} \mathop{\star} h_{C''X})(k'') & \text{by commutativity of $K''R''XC''$}\\
    %              =&h_{C''X}(h_{K''C''}(k''))\\
    %              =&h_{C''X}(c'')\\
    %              =&x
    %             \end{flalign*}


    %             \begin{flalign*}
    %                =&h_{R''X}(h_{K''R''}(k'')) \\
    %                =&(h_{K''R''} \mathop{\star} h_{R''X})(k'') \\
    %                =&(h_{K''C''} \mathop{\star} h_{C''X})(k'') & \text{by commutativity of $K''R''XC''$}\\
    %                =&h_{C''X}(h_{K''C''}(k''))\\
    %                =&h_{C''X}(c'')\\
    %                =&x
    %               \end{flalign*}

    %                     \begin{flalign*}
    %                         x &= h_{C''X}(c'')  & \text{by definition of $r''$}\\
    %                             &= h_{C''X}(h_{K'C''}(k')) & \text{by definition of $k'$}\\
    %                             &= (h_{K'C''} \mathop{\star} h_{C''X})(k') &  \\
    %                             &= (h_{K'R'} \mathop{\star} h_{R'X})(k') & \text{by the commutativity of $K'C'XR'$} \\
    %                             &= h_{R'X}(h_{K'R'}(k')) &  \\
    %                             &= h_{R'X}(r') &  \text{by definition of $k'$}\\
    %                             &= x& \text{by definition of $r'$}
    %                     \end{flalign*}
    %             which contradic the assumption Equation~\eqref{assumption_norpp}.

    %             Hence \begin{flalign}
    %                 k_1 \mathop{\neq} k_2 \label{k1_neq_k2}
    %             \end{flalign}

    %             From Equation~\eqref{k1_neq_k2}, we have 
    %             \begin{flalign}
    %                 h_{KR}(k_1) \mathop{\neq} h_{KR}(k_2) \label{hkrk1_neq_hkrk2}
    %             \end{flalign} 

    %             Since Equation~\eqref{hkrk1_neq_hkrk2} holds and $\rho$ is $X$-non-increasing, we have 
    %             \begin{flalign}
    %                 h_{KL}(k_1) \mathop{\neq} h_{KR}(k_2) \label{hkrk1_neq_hkrk2}
    %             \end{flalign} 

    %             By assumption on $\rho$ \todo{assumption: non-increasing}, we have $h_{KL}(k_2) \mathop{=} h_{R'L}(r') \mathop{=} l$ for some $k_2 \mathop{\in} K$.
                
    %             \begin{flalign*}
    %                 h_{XG}(x) &= h_{XG}(h_{C''X}(c'')) & \text{by definition of $c''$} \\
    %                         &= (h_{C''X} \mathop{\star} h_{XG})(c') &  \\
    %                         &= (h_{C''C} \mathop{\star} h_{CG})(c'') & \text{by commutativity of $C''CGX$} \\
    %                         &= h_{CG}(h_{C''C}(c'')) &  \\
    %                         &= h_{CG}(c) & \text{by definition of $c$} \\
    %                         &= h_{CG}(h_{KC}(k_1)) & \text{by definition of $k_1$} 
    %                         \\
    %                         &= (h_{KC} \mathop{\star} h_{CG})(k_1) & 
    %                         \\
    %                         &\mathop{\neq} (h_{KC} \mathop{\star} h_{CG})(k_2) & \text{by Equation~\eqref{k1_neq_k2} and injectivity of $h_{KC} \mathop{\star} h_{CG}$} 
    %                         \\
    %                         &= (h_{KL} \mathop{\star} h_{LG})(k_2) & 
    %                         \\
    %                         &= h_{LG}(h_{KL}(k_2)) & 
    %                         \\
    %                         &= h_{LG}(h_{R'L}(r')) & 
    %                         \\
    %                         &= (h_{R'L} \mathop{\star} h_{LG})(r') & 
    %                         \\
    %                         &= (h_{R'X} \mathop{\star} h_{XG})(r') & \text{by commutativity of $R'LGX$} \\
    %                         &= h_{XG}(h_{R'X}(r'))\\
    %                             &= h_{XG}(x) & \text{by definition of $r'$}
    %             \end{flalign*}
              
    %         \end{itemize}            
    \end{itemize}
\end{proof}
% \newpage
% \section{Proofs of \autoref{lem:decomp_w_u}, \autoref{lem:xlnlmxrnr}, \autoref{lem:w_u_l_not_geq_r_not} and \autoref{lem:w_g_geq_w_h_leq}}
% \label{sec:appendix:a}
% \begin*{\textbf{ Lemma~\ref{subgraph_counting:lem:decomp_w_u}}}
    Let $X$ be a ruler-graph. For a pushout square as shown below:
        \begin{center}
        % \resizebox{0.2\textwidth}{!}{
            \begin{tikzpicture}
            \node (A) at (0,0) {$A$};
            \node (B) at (0,-2) {$B$}; 
            \node (C) at (-2,0) {$C$}; 
            \node (D) at (-2,-2) {$D$}; 
            \draw [>->] (A) to node [right,label,pos=0.5] {$\alpha$} (B);
            \draw [>->] (A) to node [above,label,pos=0.5] {$\beta$} (C);
            \draw [>->] (B) to node [below,label,pos=0.45] {$\beta'$} (D); 
            \draw [>->] (C) to node [left,label,pos=0.45] {$\alpha'$} (D);
        \end{tikzpicture}
        % }
        \end{center}
         the following equalities hold:
        % $\card{\operatorname{Mono}(X, B)} \mathop{=} \card{\operatorname{Mono}(X, D, \beta')}$ and $\card{\operatorname{Mono}(X, C, \lnot \beta)} \mathop{=} \card{\operatorname{Mono}(X, D, \lnot \beta', \alpha')}$.
        \begin{flalign*}
            \card{\operatorname{Mono}(X, B)} &= \card{\operatorname{Mono}(X, D, \beta')},
            \\
            \card{\operatorname{Mono}(X, C, \lnot \beta)} &= \card{\operatorname{Mono}(X, D, \lnot \beta', \alpha')}.
        \end{flalign*}
\end*{}
\begin{proof}
    \label{subgraph_counting:proof:dcomp_w_u}
%    Since $\operatorname{Mono}(X, B) \mathop{=} \operatorname{Mono}(X, D, \beta')$ by definition of $\operatorname{Mono}(X, B)$ and $\operatorname{Mono}(X, D, \beta')$, the equality $\card{\operatorname{Mono}(X, B)} \mathop{=} \card{\operatorname{Mono}(X, D, \beta')}$ holds.
\begin{claim}
    $\card{\operatorname{Mono}(X, B)} \mathop{=} \card{\operatorname{Mono}(X, D, \beta')}$ holds.
\end{claim}
\begin{itemize} 
    \item The inclusion \(\operatorname{Mono}(X, B) \mathop{\star} \beta' \mathop{\subseteq} \operatorname{Mono}(X, D, \beta')\) holds by the definitions of $\operatorname{Mono}(X, B)$ and $\operatorname{Mono}(X, D, \beta')$. 
    In fact, if \(\iota  \mathop{\in} \operatorname{Mono}(X, B) \mathop{\star} \beta' \), then there is a monomorphism \(\zeta : X \rightarrowtail B\) satisfying \(\iota \mathop{=} \zeta \mathop{\star} \beta'\). Since $\zeta \mathop{\star} \beta'$ is also a monomorphism\todo{monicity of $\beta'$}, \(\iota\) is also an element of \(\operatorname{Mono}(X, D, \beta')\).
    \item Inclusion \(\operatorname{Mono}(X, B) \mathop{\star} \beta' \supseteq \operatorname{Mono}(X, D, \beta')\) holds by the definition of $\operatorname{Mono}(X, B)$ and $\operatorname{Mono}(X, D, \beta')$.
    Suppose \(\iota \mathop{\in} \operatorname{Mono}(X, D, \beta')\). There is a monomorphism \(\zeta : X \rightarrowtail B\) satisfying \(\iota \mathop{=} \zeta \mathop{\star} \beta'\). This shows that \(\iota \) is an element of \(\operatorname{Mono}(X, B) \mathop{\star} \beta'\).
    \item Hence, we have \(\operatorname{Mono}(X, B) \mathop{\star} \beta' \mathop{=} \operatorname{Mono}(X, D, \beta')\).
    \item $\card{\operatorname{Mono}(X, B)} \leq \card{\operatorname{Mono}(X, D, \beta')}$ follows by the injectivity of $\beta'$\todo{monicity of $\beta'$}.
\end{itemize}
\begin{claim}
     $\card{\operatorname{Mono}(X, C, \lnot \beta)} \mathop{=} \card{\operatorname{Mono}(X, D, \lnot \beta', \alpha')}$ holds.
   \end{claim}
    \begin{itemize}
        % \item The inclusion \(\operatorname{Mono}(X, C, \lnot \beta) \mathop{\star} \alpha'  \mathop{\subseteq} \operatorname{Mono}(X, D, \lnot \beta', \alpha')\) can be justified as follows. Suppose that \(\iota : X \mathop{\to} D\) is an element of \(\operatorname{Mono}(X, C, \lnot \beta) \mathop{\star} \alpha'\). According to the definition of \(\operatorname{Mono}(X, C, \lnot \beta) \mathop{\star} \alpha'\), there is \(\eta : X \mathop{\to} C\) in \(\operatorname{Mono}(X, C, \lnot \beta)\) such that \(\iota \mathop{=} \eta \mathop{\star} \alpha'\). We need to show that no \(\zeta : X \mathop{\to} B\) exists such that \(\iota \mathop{=} \zeta \mathop{\star} \beta'\). Assuming the contrary, that such a \(\zeta\) exists. Under this assumption, the following commutative diagram holds:
        \item The inclusion \(\operatorname{Mono}(X, C, \lnot \beta) \mathop{\star} \alpha'  \mathop{\subseteq} \operatorname{Mono}(X, D, \lnot \beta', \alpha')\) can be justified as follows: Suppose \(
            \eta \mathop{\star} \alpha' \mathop{\in} \operatorname{Mono}(X, C, \lnot \beta) \mathop{\star} \alpha'\).
        %  By the definition of \(\operatorname{Mono}(X, C, \lnot \beta) \mathop{\star} \alpha'\), there is a morphism \(\eta : X \mathop{\to} C\) in \(\operatorname{Mono}(X, C, \lnot \beta)\) such that \(\iota \mathop{=} \). 
         Since $\eta \mathop{\star} \alpha'$ is a monomorphism\todo{monicity of $\alpha'$}, it suffices to show that there is no \(\zeta : X \rightarrowtail B\) such that \(\eta \mathop{\star} \alpha' \mathop{=} \zeta \mathop{\star} \beta'\). Suppose, by contradiction, that such a \(\zeta\) exists, then the following commutative diagram holds:
        \begin{center}
            \begin{tikzpicture}
                \node (A) {A}; 
                \node (B) [above right=of A] {B};
                \node (X) [left  = of A] {X};
                \node (D) [below right=of B] {D};
                \node (C) [below right=of A] {C};
                \draw[>->] (A) to node[pos=.7, below] {$\alpha$} (B) ;
                \draw[>->] (C) to  node[pos=0.7, below] {$\alpha'$} (D);
                \draw[>->] (A) -- (C) node[pos=.4, right] {$\beta$};
                \draw[>->] (B) -- (D) node[pos=.4, right] {$\beta'$};
                \draw[>->] (X) -- node[above] {$\zeta$} (B);
                \draw[>->] (X) -- node[below] {$\eta$} (C);
            \end{tikzpicture}
        \end{center} 
        The pushout square \(\square ABDC\) is also a pullback square, by Proposition~\ref{prop:pb_eq_po}. The universal property of the pullback provides a morphism \(\gamma : X \mathop{\rightarrow} A\) such that \(\eta \mathop{=} \gamma \mathop{\star} \beta\). \(\gamma\) is a monomorphism, because \(\eta \mathop{=} \gamma \mathop{\star} \beta\) and $\eta$ is a monomorphism. Therefore, the existence of $\gamma$ contradicts the assumption that \(\eta \mathop{\in} \operatorname{Mono}(X, C, \lnot \beta)\). Thus, \(\iota\) is also an element of \(\operatorname{Mono}(X, D, \lnot \beta', \alpha')\). 
        \item The inclusion \(\operatorname{Mono}(X, C, \lnot \beta) \mathop{\star} \alpha'  \supseteq \operatorname{Mono}(X, D, \lnot \beta', \alpha')\) can be justified as follows. Suppose that \(\iota : X \rightarrowtail D\) is an element of \(\operatorname{Mono}(X, D, \lnot \beta', \alpha')\). According to the definition of \(\operatorname{Mono}(X, D, \lnot \beta', \alpha')\), there is \(\eta : X \rightarrowtail C\) making 
            \begin{flalign}
                \iota \mathop{=} \eta \mathop{\star} \alpha'. \label{eq:etastaralphap}
            \end{flalign}
        We show that \(\eta\) is an element of 
        \(\operatorname{Mono}(X, C, \lnot \beta)\) by contradiction.
        
        Suppose that there is a monomorphism \(\zeta : X \rightarrowtail A\) such that 
        \begin{flalign}
            \eta \mathop{=} \zeta \mathop{\star} \beta. \label{eq:zetastarbeta}
        \end{flalign}
        We have:
        \begin{flalign*}
           \hspace{2cm} \iota &\overset{\operatorname{def}}{=} \eta \mathop{\star} \alpha' &\text{by \eqref{eq:etastaralphap}} 
           \\
                  &\overset{\operatorname{def}}{=} (\zeta \mathop{\star} \beta) \mathop{\star} \alpha' & \text{by \eqref{eq:zetastarbeta}}
                  \\
                  &= \zeta \mathop{\star} (\beta \mathop{\star} \alpha') & \text{by associativity} 
                  \\
                  &= \zeta \mathop{\star} (\alpha \mathop{\star} \beta') &\text{by commutativity of $ABDC$}\\
                  &= (\zeta \mathop{\star} \alpha) \mathop{\star} \beta'.
        \end{flalign*}
          There is a contradiction because no such factorization of \(\iota\) should exist as \(\iota \mathop{\in} \operatorname{Mono}(X, D, \lnot \beta', \alpha')\).
        %    Therefore, there is no \(\zeta : X \mathop{\rightarrow} A\) such that \(\eta \mathop{=} \zeta \mathop{\star} \beta\), confirming \(\eta\) as an element of \(\operatorname{Mono}(X, C, \lnot \beta)\). Consequently, \(\iota \mathop{=} \eta \mathop{\star} \alpha'\) is an element of \(\operatorname{Mono}(X, C, \lnot \beta) \mathop{\star} \alpha'\).
        \item We conclude by the monicity of \(\alpha'\). \todo{monicity of $\alpha'$}
        % From \(\operatorname{Mono}(X, C, \lnot \beta) \mathop{\star} \alpha'  \mathop{\subseteq} \operatorname{Mono}(X, D, \lnot \beta', \alpha')\) and \(\operatorname{Mono}(X, C, \lnot \beta) \mathop{\star} \alpha'  \supseteq \operatorname{Mono}(X, D, \lnot \beta', \alpha')\), we conclude \(\operatorname{Mono}(X, C, \lnot \beta) \mathop{\star} \alpha' \mathop{=} \operatorname{Mono}(X, D, \lnot \beta', \alpha')\) and \(\card{\operatorname{Mono}(X, C, \lnot \beta) \mathop{\star} \alpha'} \mathop{=} \card{\operatorname{Mono}(X, D, \lnot \beta', \alpha')}\).
        % Since \(\alpha'\) is monic, \(\card{\operatorname{Mono}(X, C, \lnot \beta)} \mathop{=} \card{\operatorname{Mono}(X, C, \lnot \beta) \mathop{\star} \alpha'}\). Therefore, \(\card{\operatorname{Mono}(X, C, \lnot \beta)} \mathop{=} \card{\operatorname{Mono}(X, D, \lnot \beta', \alpha')}\).
    \end{itemize}
\end{proof} 

\noindent
\begin*{\textbf{ Lemma~\ref{subgraph_counting:lem:xlnlmxrnr}}}
Let $X$ be a graph. Let $L \overset{l}{\leftarrow} K \overset{r}{\rightarrow} R$ be an injective DPO graph rewriting rule. We have 
\[
   \card{\operatorname{Mono}(X, L, \lnot l)}  - \card{\operatorname{Mono}(X, R, \lnot r)} 
   \mathop{=} 
   \card{\operatorname{Mono}(X, L)}  - \card{\operatorname{Mono}(X, R)}.
    \]
\end*{}
\begin{proof}
    \label{subgraph_counting:proof:lem:xlnlmxrnr}
    \begin{claim}
       $\card{\operatorname{Mono}(X, L, l)} \mathop{=} \card{\operatorname{Mono}(X, R, r)}$ holds.
    \end{claim}
    \begin{itemize}
        \item We prove $\card{\operatorname{Mono}(X, L, l)} \leq \card{\operatorname{Mono}(X, R, r)}$ by constructing an injection from $\operatorname{Mono}(X, L, l)$ to $\operatorname{Mono}(X, R, r)$. Let $h \mathop{\in} \operatorname{Mono}(X, L, l)$. By definition, we have $h \mathop{=} g \mathop{\star} l$ for some $g: X \rightarrowtail K$. Thus, $g \mathop{\star} r \mathop{\in} \operatorname{Mono}(X, L, r)$. Let $h' \mathop{\in} \operatorname{Mono}(X, L, l)$ such that $h \mathop{\neq} h'$. We have $h' \mathop{=} g' \mathop{\star} l$ for some $g':X \rightarrowtail K$. We have $g \mathop{\neq} g'$ because otherwise we would have $h \mathop{=} g \mathop{\star} l \mathop{=} g' \mathop{\star} l \mathop{=} h'$. From the monicity of $r$, we conclude $g \mathop{\star} r \mathop{\neq} g' \mathop{\star} r$.
        \item Analoguously, we can prove $\card{\operatorname{Mono}(X, L, l)} \mathop{\geq} \card{\operatorname{Mono}(X, R, r)}$ by constructing an injection from $\operatorname{Mono}(X, R, r)$ to $\operatorname{Mono}(X, L, l)$.
        %  Let $h \mathop{\in} \operatorname{Mono}(X, R, r)$. By definition, we have $h \mathop{=} g \mathop{\star} r$ for some $g: X \rightarrowtail K$. Thus, $g \mathop{\star} l \mathop{\in} \operatorname{Mono}(X, L, l)$.
        % Let $h' \mathop{\in} \operatorname{Mono}(X, R, r)$ such that $h \mathop{\neq} h'$. We have $h' \mathop{=} g' \mathop{\star} l$ for some $g':X \rightarrowtail K$. Since $r$ is monic, we deduce $g \mathop{\neq} g'$. From the monicity of $l$, we conclude $g \mathop{\star} r \mathop{\neq} g' \mathop{\star} r$.
    \end{itemize}


    % we have \[\card{\operatorname{Mono}(X, L, l)} - \card{\operatorname{Mono}(X, R, r)} \mathop{=} 0 \] becase 
    
    Hence, the following equality holds
    \begin{flalign*}
         &\card{\operatorname{Mono}(X, L, \lnot l)}  - \card{\operatorname{Mono}(X, R, \lnot r)}\\
        =&\card{\operatorname{Mono}(X, L) \mathop{\setminus} \operatorname{Mono}(X, L, l)}  - 
            \card{\operatorname{Mono}(X, R) \mathop{\setminus} \operatorname{Mono}(X, R, r)}\\
        =&(\card{\operatorname{Mono}(X, L)} - \card{\operatorname{Mono}(X, L, l)}) - (\card{\operatorname{Mono}(X, R)} - \card{\operatorname{Mono}(X, R, r)}) \\
        =&(\card{\operatorname{Mono}(X, L)} - \card{\operatorname{Mono}(X, R)})
         - (\card{\operatorname{Mono}(X, L, l)} - \card{\operatorname{Mono}(X, R, r)})\\
        \mathop{=} &\card{\operatorname{Mono}(X, L)} - \card{\operatorname{Mono}(X, R)}.
    \end{flalign*}
\end{proof}

\noindent
\begin*{\textbf{ Lemma~\ref{subgraph_counting:lem:w_u_l_not_geq_r_not}}}
  Let $X$ be a ruler-graph and $\rho \mathop{=} (L \overset{l}{\leftarrowtail} K \overset{r}{\rightarrowtail} R)$ an injective DPO rewriting rule.
        Suppose that $\rho$ is $X$-non-increasing. For every rewriting step induced by the following DPO diagram:
        \begin{center}
        % \resizebox{0.3\textwidth}{!}{ 
                \begin{tikzpicture}
            \node (k) at (0,1) {K};
            \node (l) at (-2,1) {L};
            \node (r) at (2,1) {R};
            \node (c) at (0,-1) {C};
            \node (g) at (-2,-1) {G};
            \node (h) at (2,-1) {H};
            \draw[<-<]  (l) -- (k) node [midway,below] {$l$};
            \draw[>->]  (k) -- (r) node [midway,below] {$r$};
            \draw[>->] (c) -- (g) node [midway, above] {$l'$};
            \draw[>->] (c) -- (h) node [midway,above] {$r'$};
            \draw[>->] (l) -- (g) node[midway, right] {$m$};
            \draw[>->] (r) -- (h) node[midway, left] {$m'$};
            \draw[>->] (k) -- (c) node[midway, left] {};
            \node () [at=($(l)!0.5!(c)$)] {$PO$};
            \node () [at=($(r)!0.5!(c)$)] {$PO$};
        \end{tikzpicture}
        % }
        \end{center}
       The following inequality holds:
        \[
            |\operatorname{Mono}(X, G, \lnot m, \lnot l')| \mathop{\geq} |\operatorname{Mono}(X, H, \lnot m', \lnot r')|
        \]
\end*{}
\begin{proof}
    \label{proof:lem:w_u_l_not_geq_r_not}
    We are going to show that for two arbitrary distinct monomorphisms $h_{XH}', h_{XH}'' \mathop{\in} \operatorname{Mono}(X, H, \lnot m', \lnot r')$, there are two distinct monomorphisms $h_{XG}'$ and $h_{XG}'' \mathop{\in} \operatorname{Mono}(X, G, \lnot m, \lnot l')$.

    Let $h_{XH}'\in \operatorname{Mono}(X, H, \lnot m', \lnot r')$ be a monomorphism. In the category \textbf{Graph}, we can construct the following commutative diagram
    \begin{center}
        % \resizebox{6cm}{!}{
            \begin{tikzpicture}[scale=1.5]
                \node (k) at (0,0) {K};
                \node (r) at (4,0) {R};
                \node (c) at (0,-3) {C};
                \node (h) at (4,-3) {H};
                \draw[<-<]  (r) -- (k) node [midway,above] {};
                \draw[>->] (c) -- (h) node [midway, below] {};
                \draw[>->] (r) -- (h) node[midway, left] {};
                \draw[>->] (k) -- (c) node[midway, left] {};
                \node (k') at (1,-1) {K'};
                \node (r') at (2,-1) {R'};
                \node (c') at (1,-2) {C'};
                \node () at (1.5,-1.5) {$PB$};
                \node () at (3,-1.5) {$PB$};
                \node () at (1.5,-2.5) {$PB$};
                \node (x) at (2,-2) {X};
                \draw [>->] (c') -- (x);
                \draw [>->] (r') -- (x);
                \draw [>->] (k') -- (r');
                \draw [>->] (k') -- (c');
                \draw [>->] (c') -- (c);
                \draw[>->] (r') -- (r);
                \draw[>->] (x) -- (h) node[midway,right] {$h$};
            \end{tikzpicture}
        % }
        \end{center} 
    
    $KCHR$ is also pullback, by Proposition~\ref{prop:pb_eq_po}, because it is pushout. 
    Therefore, since $KCHR$ is also pullback and \(
        h_{K'C'} \mathop{\star} h_{C'C} \mathop{\star} h_{CH} =h_{K'R'} \mathop{\star} h_{R'R} \mathop{\star} h_{R'R}
    \) holds, there is $h_{K'K}:K' \rightarrowtail K$ such that the following equalities hold:
    \begin{flalign}
        h_{K'C'} \mathop{\star} h_{C'C} &= h_{K'K} \mathop{\star} h_{KC}, \label{kpcpcpckpkkc}
        \\
        h_{K'R'} \mathop{\star} h_{R'R} &= h_{K'K} \mathop{\star} h_{KR}. \label{kprprprkpk}
    \end{flalign} 
    
    The square $K'KCC'$ and $K'R'RK$ are pullbacks, by Lemma~\ref{kpcpck_pullback}. Hence, the following commutative diagram holds
    \begin{center}
        % \resizebox{6cm}{!}{
            \begin{tikzpicture}[scale=1.5]
                \node (k) at (0,0) {K};
                \node (r) at (4,0) {R};
                \node (c) at (0,-3) {C};
                \node (h) at (4,-3) {H};
                \draw[<-<]  (r) -- (k) node [midway,above] {};
                \draw[>->] (c) -- (h) node [midway, below] {};
                \draw[>->] (r) -- (h) node[midway, left] {};
                \draw[>->] (k) -- (c) node[midway, left] {};
                \node (k') at (1,-1) {K'};
                \node (r') at (2,-1) {R'};
                \node (c') at (1,-2) {C'};
                \node () at (1.5,-1.5) {$PB$};
                \node () at (3,-1.5) {$PB$};
                \node () at (1.5,-2.5) {$PB$};
                \node () at (1.5,-0.5) {$PB$};
                \node () at (0.5,-1.5) {$PB$};
                \node (x) at (2,-2) {X};
                \draw [>->] (c') -- (x);
                \draw [>->] (r') -- (x);
                \draw [>->] (k') -- (r');
                \draw [>->] (k') -- (c');
                \draw [>->] (c') -- (c);
                \draw[>->] (r') -- (r);
                \draw[>->] (x) -- (h) node[midway,right] {$h$};
                \draw[>->] (k') -- (k) ;
            \end{tikzpicture}
        % }
        \end{center} 

    \todo{use: assumption: non increasing}
    Since the rule is by assumption $X$-non-increasing, there is a morphism $h_{R'L}$ such that $K'KLR'$ is commutative.
    
    $K'C'XR'$ is a pushout by Lemma~\ref{kpcpxrp_po}. 
    
    Thus, the following commutative diagram holds
    \begin{center}
            \begin{tikzpicture}[scale=1.5]
                \node (k) at (0,0) {K};
                \node (r) at (4,0) {L};
                \node (c) at (0,-3) {C};
                \node (h) at (4,-3) {G};
                \draw[<-<]  (r) -- (k) node [midway,above] {};
                \draw[>->] (c) -- (h) node [midway, below] {};
                \draw[>->] (r) -- (h) node[midway, left] {};
                \draw[>->] (k) -- (c) node[midway, left] {};
                \node (k') at (1,-1) {K'};
                \node (r') at (2,-1) {R'};
                \node (c') at (1,-2) {C'};
                \node () at (1.5,-1.5) {$PO$};
                \node () at (0.5,-1.5) {$PB$};
                \node (x) at (2,-2) {X};
                \draw [>->] (c') -- (x);
                \draw [>->] (r') -- (x);
                \draw [>->] (k') -- (r');
                \draw [>->] (k') -- (c');
                \draw [>->] (c') -- (c);
                \draw [->] (k') -- (k);
                \draw[>->] (r') -- (r);
            \end{tikzpicture}
        \end{center} 

    By Lemma~\ref{lem:g_monic}, there is a unique monomorphism $h_{XG}':X \rightarrowtail G$ such that $h_{C'X} \mathop{\star} h_{XG}' \mathop{=} h_{C'C} \mathop{\star} h_{CG}$ and $h_{R'X} \mathop{\star} h_{XG}' \mathop{=} h_{R'L} \mathop{\star} h_{LG}$.

    By $h_{XH}'\in \operatorname{Mono}(X, H, \lnot m', \lnot r')$ and Definition~\ref{subgraph_counting:def:creates_more_x_on_the_left}, the monomorphism $h_{XG}'$ is in the set $\operatorname{Mono}(X, G, \lnot m, \lnot l')$. 
    Intuitively, some elements in $X$ are mapped onto $h_{RH}(R) \mathop{\setminus} (r \mathop{\star} h_{RH})(K)$ if $h_{XH}'$ is an element of $\operatorname{Mono}(X, H, \lnot m', \lnot r')$. By the first assumption of Definition~\ref{subgraph_counting:def:creates_more_x_on_the_left}, these elements are mapped onto $h_{LG}(L) \mathop{\setminus} (l \mathop{\star} h_{LG})(K)$.

    
    The monomorphism \( h_{XG}' \) belongs to the set \( \operatorname{Mono}(X, G, \lnot m, \lnot l') \), due to \( h_{XH}' \mathop{\in} \operatorname{Mono}(X, H, \lnot m', \lnot r') \) and Definition~\ref{subgraph_counting:def:creates_more_x_on_the_left}. Intuitively, \( h_{XH}'$ is an element of $\operatorname{Mono}(X, H, \lnot m', \lnot r') \) implies that certain elements of \( X \) are mapped into \( h_{RH}(R) \mathop{\setminus} (r \mathop{\star} h_{RH})(K) \). By the second assumption of Definition~\ref{subgraph_counting:def:creates_more_x_on_the_left}, these elements are also mapped into \( h_{LG}(L) \mathop{\setminus} (l \mathop{\star} h_{LG})(K) \).

     \todo{use; def non-increase; condition 1}

    Let $h_{XH}'':X \rightarrowtail H$ be a monomorphism such that 
    \begin{flalign}
        h_{XH}' \mathop{\neq} h_{XH}''. \label{hpneqh}
    \end{flalign}
    
    In the category \textbf{Graph}, we can construct the following commutative diagram:
    \begin{center}
            \begin{tikzpicture}[scale=1.5]
                \node (k) at (0,0) {K};
                \node (r) at (4,0) {L};
                \node (c) at (0,-3) {C};
                \node (h) at (4,-3) {G};
                \draw[<-<]  (r) -- (k) node [midway,above] {};
                \draw[>->] (c) -- (h) node [midway, below] {};
                \draw[>->] (r) -- (h) node[midway, left] {};
                \draw[>->] (k) -- (c) node[midway, left] {};
                \node (k') at (1,-1) {K'};
                \node (r') at (2,-1) {R'};
                \node (c') at (1,-2) {C'};
                \node () at (1.5,-1.5) {$PO$};
                \node () at (0.5,-1.5) {$PB$};
                \node (x) at (2,-2) {X};
                \draw [>->] (c') -- (x);
                \draw [>->] (r') -- (x);
                \draw [>->] (k') -- (r');
                \draw [>->] (k') -- (c');
                \draw [>->] (c') -- (c);
                \draw [->] (k') -- (k);
                \draw[>->] (r') -- (r);
            \end{tikzpicture}
        \end{center} 

    Analoguously, there is a unique monomorphism $h_{XG}'':X \rightarrowtail G$ such that $h_{C''X} \mathop{\star} h_{XG} ''= h_{C''C} \mathop{\star} h_{CG}$ and $h_{R''X} \mathop{\star} h_{XG}'' \mathop{=} h_{R''L} \mathop{\star} h_{LG}$.
    
    By \eqref{hpneqh} and Lemma~\ref{lem:h_hp_diff_g_gp_diff}, we have $h_{XG} \mathop{\neq} h_{XG}'$.
\end{proof}



\noindent
\begin*{\textbf{ Lemma~\ref{subgraph_counting:lem:w_g_geq_w_h_leq}}}[Decreasing step]
Let $\rho \mathop{=} (L \overset{l}{\leftarrowtail} K \overset{r}{\rightarrowtail} R)$ be an injective DPO rewriting rule,
\( \mathbb{X} \) a set of ruler-graphs,
\( s_{\mathbb{X}} \mathop{\colon} \mathbb{X} \mathop{\to} \mathbb{N} \) a weight function,
and \( G \mathop{\Rightarrow}_{\rho,\mathfrak{M}} H \) a rewriting step. 
If $\rho$ is \( X \)-non-increasing for every ruler-graph \( X \mathop{\in} \mathbb{X} \), then:\[
    w_{s_\mathbb{X}}(G) - w_{s_\mathbb{X}}(H) 
    \mathop{\geq} 
    w_{s_\mathbb{X}}(L) - w_{s_\mathbb{X}}(R).
\]
\end*{}
\begin{proof}
    \label{subgraph_counting:proof:lem:w_g_geq_w_h_leq}
    Let the following diagram be the witness diagram of the rewriting step
    \begin{center}
        \begin{tikzpicture}
            \node (k) at (0,1) {K};
            \node (l) at (-2,1) {L};
            \node (r) at (2,1) {R};
            \node (c) at (0,-1) {C};
            \node (g) at (-2,-1) {G};
            \node (h) at (2,-1) {H};
            \draw[<-<]  (l) -- (k) node [midway,below] {$l$};
            \draw[>->]  (k) -- (r) node [midway,below] {$r$};
            \draw[>->] (c) -- (g) node [midway, above] {$l'$};
            \draw[>->] (c) -- (h) node [midway,above] {$r'$};
            \draw[>->] (l) -- (g) node[midway, right] {$m$};
            \draw[>->] (r) -- (h) node[midway, left] {$m'$};
            \draw[>->] (k) -- (c) node[midway, left] {};
            \node () [at=($(l)!0.5!(c)$)] {$PO$};
            \node () [at=($(r)!0.5!(c)$)] {$PO$};
        \end{tikzpicture}
    \end{center}
    Let $X \mathop{\in} \mathbb{X}$.
    We have 
    \begin{flalign*}
        & \card{\operatorname{Mono}(X,G)}
            % \overset{\operatorname{def}}{=} 
            % w_{T_\Sigma}(t_G) 
        \\
        % =& \Psi_{G}\\
        =& 
        \card{\operatorname{Mono}(X,G,l')
        \uplus
         \operatorname{Mono}(X,G,\lnot l',m)
        \uplus
         \operatorname{Mono}(X,G,\lnot l',\lnot m)}\\
        =& 
        \card{\operatorname{Mono}(X,G,l')}
        +
         \card{\operatorname{Mono}(X,G,\lnot l',m)}
        +
        \card{\operatorname{Mono}(X,G,\lnot l',\lnot m)}
        \\
    =& \card{\operatorname{Mono}(X,C)}
            +
            \card{\operatorname{Mono}(X, L, \lnot l)} 
            +
            \card{\operatorname{Mono}(X, G,\lnot l', \lnot m)} 
        &  \text{by Lemma~\ref{subgraph_counting:lem:decomp_w_u}}
    \end{flalign*}
    and
    \begin{flalign*}
        % ligne 1
        &\card{\operatorname{Mono}(X,H)}
        % \overset{\operatorname{def}}{=}  
        %     w_{T_\Sigma}(t_H )
        \\
        =& 
        \card{\operatorname{Mono}(X,H,r')
        \uplus
            \operatorname{Mono}(X,H,\lnot r',m')
        \uplus
            \operatorname{Mono}(X,H,\lnot r',\lnot m')}
        \\
        =& 
        \card{\operatorname{Mono}(X,H,r')}
        +
         \card{\operatorname{Mono}(X,H,\lnot r',m')}
        +
        \card{\operatorname{Mono}(X,H,\lnot r',\lnot m')}
        \\
        % \mathop{=} & \Psi_{H} \\
        % ligne 2
        \mathop{=} &
            \card{\operatorname{Mono}(X,C)}
            +
            \card{\operatorname{Mono}(X, R, \lnot r)} 
            +
            \card{\operatorname{Mono}(X, H, \lnot r', \lnot m')}.
        % ( 
        %             w_{T_\Sigma} (h_{RH} \mathop{\star} t_H )\mathop{+}w_{T_\Sigma} (h_{CH} \mathop{\star} t_H  -h_{KC}) 
        % )
        %        \mathop{+} w_{T_\Sigma}(t_H  - \{h_{RH},h_{CH} \})  
        &  \text{by Lemma~\ref{subgraph_counting:lem:decomp_w_u}}
    \end{flalign*}
    Therefore, the following inequality holds
     \begin{flalign*}
           &\card{\operatorname{Mono}(X,G)} - \card{\operatorname{Mono}(X,H)} &\\
           =&(\card{\operatorname{Mono}(X, L, \lnot l)} - \card{\operatorname{Mono}(X, R, \lnot r)})\mathop{+}\\
            &(\card{\operatorname{Mono}(X, G, \lnot l', \lnot m)} - \card{\operatorname{Mono}(X, H, \lnot r', \lnot m')}) \\
       \geq& \card{\operatorname{Mono}(X, L, \lnot l)} - \card{\operatorname{Mono}(X, R, \lnot r)} \hspace{3cm} 
       \text{by Lemma~\ref{subgraph_counting:lem:w_u_l_not_geq_r_not}} \\
          =& \card{\operatorname{Mono}(X, L)} - \card{\operatorname{Mono}(X, R)}. \hspace{3cm} 
          \text{by Lemma~\ref{subgraph_counting:lem:xlnlmxrnr}}
     \end{flalign*}
     Thus, the following inequality holds
     \begin{flalign*}
          &w_{s_\mathbb{X}}(G) - w_{s_\mathbb{X}}(H)
          \\
         =&\sum_{X \mathop{\in} \mathbb{X}}^{}w(X) * |\operatorname{Mono}(X,G)| - \sum_{X \mathop{\in} \mathbb{X}}^{}w(X) * |\operatorname{Mono}(X,H)|
         \\
         =&\sum_{X \mathop{\in} \mathbb{X}}^{}w(X) * \left( \card{\operatorname{Mono}(X,G)} -  \card{\operatorname{Mono}(X,H)} \right)
         \\
         \geq&\sum_{X \mathop{\in} \mathbb{X}}^{}w(X) * \left(  \card{\operatorname{Mono}(X,L)}- \card{\operatorname{Mono}(X,R)} \right)
         \\
         \mathop{=} &\sum_{X \mathop{\in} \mathbb{X}}^{}w(X) * \card{\operatorname{Mono}(X,L)} -  \sum_{X \mathop{\in} \mathbb{X}}^{}w(X) * \card{\operatorname{Mono}(X,R)} 
         \\
         \mathop{=} & w_{s_\mathbb{X}}(L) - w_{s_\mathbb{X}}(R).
     \end{flalign*}
\end{proof}

\noindent
\begin*{\textbf{Theorem\ref{subgraph_counting:thm:termination_grs}}(Termination)}
    Let \(\mathcal{A}\) and \(\mathcal{B}\) be sets of injective DPO rewriting rules, $\mathbb{X}$ a set of ruler-graphs and $s_\mathbb{X}$ a weight function. If the following conditions hold:
    \begin{enumerate}
        \item\label{subgraph_counting:thm:termination_grs:assump:1} $\rho$ is $X$-non-increasing for every rule $\rho \mathop{\in} \mathcal{A} \mathop{\cup} \mathcal{B}$ and for every ruler-graph $X \mathop{\in} \mathbb{X}$,
        \item\label{thm:termination_grs:assump:2} \( w_{s_\mathbb{X}}(lhs(\rho)) \mathop{>} w_{s_\mathbb{X}}(rhs(\rho)) \) for every rule \(\rho \mathop{\in} \mathcal{A}\),
        \item\label{thm:termination_grs:assump:3} \( w_{s_\mathbb{X}}(lhs(\rho)) \mathop{\geq} w_{s_\mathbb{X}}(rhs(\rho)) \) for every rule \(\rho \mathop{\in} \mathcal{B}\).
    \end{enumerate}
    Then \(\mathop{\Rightarrow}_{\mathcal{A},\mathcal{M}}\) terminates relative to \(\mathop{\Rightarrow}_{\mathcal{B},\mathcal{M}}\).
\end*{}
\begin{proof}
    \label{subgraph_counting:proof:thm:termination_grs}
    
    By assumption \eqref{subgraph_counting:thm:termination_grs:assump:1} and  Lemma~\ref{subgraph_counting:lem:w_g_geq_w_h_leq}, the following inequality holds for all rules $\rho \mathop{\in} \mathcal{A} \mathop{\cup} \mathcal{B}$ and for all rewriting steps $G \mathop{\Rightarrow}_{\rho, \mathcal{M}} H$:
     \[
        w_{s_\mathbb{X}}(G) - w_{s_\mathbb{X}}(H) 
        \mathop{\geq} 
        w_{s_\mathbb{X}}(\operatorname{lhs}(\rho)) - w_{s_\mathbb{X}}(\operatorname{rhs}(\rho)).
    \]
    
    \noindent From assumptions \eqref{thm:termination_grs:assump:2} and \eqref{thm:termination_grs:assump:3}, we deduce 
    \begin{itemize}
        \item \( w_{s_\mathbb{X}}(G) \mathop{>} w_{s_\mathbb{X}}(H) \) for every rule \(\rho \mathop{\in} \mathcal{A}\);
        \item  \( w_{s_\mathbb{X}}(G) \mathop{\geq} w_{s_\mathbb{X}}(H) \) for every rule \(\rho \mathop{\in} \mathcal{B}\).
    \end{itemize}
    Since $w_{s_\mathbb{X}}(G) \mathop{\in} \mathbb{N}$ for all graph $G$. Every rewriting chain can only have a finite number of rewriting steps using rules in $\mathcal{A}$.
\end{proof} 
% \section{Additional Examples}
% % % \begin{table}[H]
% %   \begin{NiceTabular}{ccccccccc}[vlines] % <-- 9 columns now (was 7)
% %    \Hline
% %    \Block{1-2}{\diagbox{\enskip \textbf{Examples}}{\textbf{Techniques}}} & &
% %    \RowStyle{\rotate}
% %     \makecell{Forward\\ closure \\~\cite{Plump1995}} % NEW column #1
% %    & \RowStyle{\rotate}
% %     \makecell{Modular \\ criterion \\~\cite{plump2018modular}} % NEW column #2
% %    & \RowStyle{\rotate}
% %     \makecell{Type graph \\~\cite{bruggink2014termination}}  
% %    & \RowStyle{\rotate}
% %     \makecell{Type graph \\~\cite{bruggink2015proving}} 
% %    & \RowStyle{\rotate}
% %     \makecell{Type graph \\~\cite{endrullis2024generalized_arxiv_v2}} 
% %    & \RowStyle{\rotate}
% %     \makecell{Subgraph \\counting \\~\cite{overbeek2024termination_lmcs}}
% %    & \RowStyle{\rotate}
% %     \makecell{Our proposal} \\
% %    \Hline
% %    \Hline
% %    % ----- from plump 1995 -----
% %    \Block{2-1}{\cite{Plump1995}} 
% %     & \hyperref[ex:overbeek_5d8_plump1995_3d8_plump2018_3_overbeek_5d8]{Example 3.8} 
% %                  & \ding{51} & -- & -- & -- & -- &\ding{51} & \ding{51}\\ 
% %     \Hline
% %     & Example 4.1 & \ding{51} & -- & -- & -- & -- & \ding{51}& \ding{55}\\ 
% %     \hline
% %     % ----- from plump 2018 -----
% %    \Block{4-1}{\cite{plump2018modular}} 
% %    & \hyperref[ex:overbeek_5d8_plump1995_3d8_plump2018_3_overbeek_5d8]{Example 3} 
% %               & -- & \ding{51} &  -- & -- & -- & \ding{51} & \ding{51}\\ 
% %    \Hline
% %    & \hyperref[ex:plump_ex4]{Example 4} &  -- &  \ding{51} &  -- & -- & -- & \ding{55} &\ding{55}\\ 
% %    \Hline
% %    & Example 5 &  -- &  \ding{51} &   -- & -- & -- &  \ding{51} & \ding{51}\\ 
% %    \Hline
% %    & Example 6 s->s&  -- & \ding{51} & -- & -- & -- & \ding{51} & \ding{55}\\ 
% %    \Hline
% %    % ----- from bruggink2014 -----
% %    \Block{4-1}{\cite{bruggink2014termination}} 
% %     & Example 1 
% %       & -- & -- & \ding{51} & -- & -- & \ding{55} & \ding{55}\\ \Hline
% %    & \hyperref[ex:plump_ex4]{Example 5}
% %       & -- & -- & \ding{51} & -- & -- & -- &  \ding{55}\\ \Hline
% %    & \hyperref[ex:termination:grsaa]{Example 4 and 6}  grsaa
% %       & -- & -- & \ding{51} & -- & -- & ? & \ding{51} \\ \Hline
% %    & Routing Protocol
% %       & -- & -- & \ding{51} & -- & -- & \ding{55} &  \ding{55}\\ \Hline
% %    %~\autoref{ex_contrib_variant} 
% %    %   & \ding{55} & -- & -- & -- & \ding{51}\\ \Hline
   
% %    % ----- from bruggink2015 -----
% %    \Block{4-1}{\cite{bruggink2015proving}}
% %     &\hyperref[ex:termination:grsaa]{Example 2} grsaa
% %       & -- & -- & -- & \ding{51} & -- & ? &  \ding{51}\\ \Hline
% %    & Example 4 
% %       & -- & -- & -- & \ding{51} & -- & \ding{51} & \ding{51} \\ \Hline
% %    & Example 5 
% %       & -- & -- & -- & \ding{51} & -- & \ding{55} & \ding{55}\\ \Hline
% %    & Example 6 
% %       & -- & -- & -- & \ding{51} & -- & \ding{55} & \ding{55}\\ \Hline
% %    %~\autoref{ex_contrib_variant} 
% %    %   & -- & \ding{55} & -- & -- & \ding{51}\\ \Hline
   
% %    % ----- from endrullis2024generalized_arxiv_v2 -----
% %    \Block{8-1}{\cite{endrullis2024generalized_arxiv_v2}}
% %     & Example 6.2  
% %       & -- & -- & -- & -- & \ding{51} & -- & \ding{51}\\ \Hline
% %    & \hyperref[ex_endrullis_5d8_overbeek]{Example 6.3}
% %       & -- & -- & -- & -- & \ding{51} & \ding{55} & \ding{51}\\ \Hline
% %    & Example 6.4 SGraph
% %       & -- & -- & -- & -- & -- & -- & -- \\ \Hline
% %    & Example 6.5  SGraph
% %       & -- & -- & -- & -- &  -- & -- & -- \\ \Hline
% %    & \hyperref[ex:overbeek_5d8_plump1995_3d8_plump2018_3_overbeek_5d8]{Example D.1}
% %       & -- & -- & -- & -- & \ding{51} & -- & \ding{51}\\ \Hline
% %    & Example D.2 
% %       & -- & -- & -- & -- & \ding{51} & -- & \ding{55}\\ \Hline
% %    & Example D.3 
% %       & -- & -- & -- & -- & \ding{51} & -- & \ding{55}\\ \Hline
% %    & Example D.4 
% %       & -- & -- & -- & -- & \ding{51} & -- & \ding{55}\\ \Hline
% %    % &~\autoref{ex_contrib_variant}
% %    %   & -- & -- & \ding{55} & -- & \ding{51}\\ \Hline
 
% %    % ----- from overbeek2024termination_lmcs -----
% %    \Block{7-1}{\cite{overbeek2024termination_lmcs}}
% %    & Example 5.2
% %       & -- & -- & -- & -- & -- & \ding{51} & \ding{55}\\ \Hline
% %    & \hyperref[ex:overbeek_5d3]{Example 5.3}
% %       & -- & -- & -- & -- & -- & \ding{51} & \ding{51}\\ \Hline
% %  %   & \hyperref[ex:overbeek_5d3]{Example 5.3 monic matches}
% %  %      & -- & -- & -- & -- & -- & \ding{51} & \ding{51}\\ \Hline
% %    & \hyperref[ex:overbeek_5d5]{Example 5.5} 
% %       & -- & -- & -- & -- & -- & \ding{51} & \ding{51}\\ \Hline
% %    & \hyperref[ex:overbeek_5d6]{Example 5.6}
% %       & -- & -- & -- & -- & -- & \ding{51} & \ding{51} \\ \Hline
% %  %   & \hyperref[ex:overbeek_5d6]{Example 5.6 bis}
% %  %      & -- & -- & -- & -- & -- & \ding{51} & -- \\ \Hline
% %    & Example 5.7 
% %       & -- & -- & -- & -- & -- & -- & -- \\ \Hline
% %    & Example 5.9 
% %       & -- & -- & -- & -- & -- & \ding{51} & \ding{55}\\ \Hline
% %    & \hyperref[ex:overbeek_5d8_plump1995_3d8_plump2018_3_overbeek_5d8]{Example 5.8}
% %       & -- & -- & \ding{55} & \ding{55} & -- & \ding{51} & \ding{51}\\ \Hline
% %    Current paper 
% %     &~\autoref{ex_contrib_variant}
% %       & \ding{55} & \ding{55} & \ding{55} & \ding{55} & \ding{55} & \ding{55} & \ding{51} \\ \Hline
% %  \end{NiceTabular}
% %  \caption{Applicability of the termination techniques to the examples with DPO rewriting with monic matches.
% %    \ding{51} indicates that the method can be applied to the example,
% %          \ding{55} indicates that the method cannot be applied,
% %          and the symbol $-$ of a symbol signifies that its applicability
% %          is not relevant to this discussion.
% %         }
% %  \label{tab:comparaison}
% %  \end{table}
 

% \begin{example} 
%   \label{ex:endrullis2024_6d2}  
%   Consider the following rewriting rule presented in~\cite[Example 6.2]{endrullis2024generalized_arxiv_v2}. 

%   \begin{tikzpicture}  
%      \graphbox{$L$}{0mm}{-11mm}{32mm}{15mm}{2mm}{-8mm}{  
%          \node[draw,circle]  (x) at (-6mm,0mm) {1};  
%         \draw[->] (x) edge[loop above] node  {} (x);

%          \node[draw,circle]  (y) at (6mm,0mm) {2};  
%        }  
%        \graphbox{$K$}{33mm}{-11mm}{32mm}{15mm}{2mm}{-8mm}{  
%         \node[draw,circle]  (x) at (-6mm,0mm) {1};  
%         \node[draw,circle]  (y) at (6mm,0mm) {2};  
%        }  
%        \graphbox{$R$}{66mm}{-11mm}{32mm}{15mm}{1mm}{-8mm}{  
%         \node[draw,circle]  (x) at (-6mm,0mm) {1};  
%         \node[draw,circle]  (y) at (6mm,0mm) {2};  
%         \draw[->]  (x) to (y);  
%        }    
%  \end{tikzpicture}  
%  Our method can prove its termination with the ruler-graph  \tikz[baseline=-0.5ex]{
%   \node (x) at (0,0) {$\bullet$};
%   \draw[->] (x) edge[loop above] node  {} (x);
% } of weight $1$.
% \end{example} 

% \begin{example}
%     Consider a rewriting rule given by Overbeek and Endrullis in~\cite[Example 5.2]{overbeek2024termination_lmcs} depicted in the bottom span of the following diagram

%     \begin{tikzpicture}
%         \graphbox{$L$}{0mm}{-11mm}{32mm}{15mm}{2mm}{-8mm}{
%             \node[draw,circle]  (x) at (-6mm,0mm) {1};
%             \node[draw,circle]   (y) at (6mm,0mm) {2};
%             \draw[->]  (x) to (y);
%           }
%           \graphbox{$K$}{33mm}{-11mm}{32mm}{15mm}{2mm}{-8mm}{
%             \node[draw,circle]  (x) at (-6mm,0mm) {1};
%             \node[draw,circle]  (y) at (6mm,0mm) {2};
%             \draw[->]  (x) to (y);
%           }
%           \graphbox{$R$}{66mm}{-11mm}{32mm}{15mm}{1mm}{-8mm}{
%             \node[draw,circle]  (x) at (0mm,0mm) {1 2};
%             \draw[->]  (x) edge [loop right] (x);
%           }  
%     \end{tikzpicture}  

%     The termination of this rule is not in the scope of our method due to its non-node-injective right-hand side morphism. It can be proved using the method proposed by Overbeek et al. in~\cite{overbeek2024termination_lmcs}.
% \end{example}


% \begin{example}
%   \label{ex:overbeek_5d3}
%   Consider the rewriting rules presented in~\cite[Example 5.3]{overbeek2024termination_lmcs} that can be depicted as the bottom spans of

%   \begin{tikzpicture} 
%       \graphbox{$L$}{0mm}{0mm}{23mm}{12mm}{2mm}{-5mm}{
%           \node[draw,circle] (x) at (0mm,0mm) {1};
%           \draw[->] (x) edge [loop right] node {} (x);
%       }
%       \graphbox{$K$}{24mm}{0mm}{23mm}{12mm}{2mm}{-5mm}{
%         \node[draw,circle] (x) at (0mm,0mm) {1};
%       }
%       \graphbox{$R_X$}{24mm}{13mm}{23mm}{12mm}{2mm}{-5mm}{
%           % \node[draw,circle] (x) at (0mm,0mm) {1};
%         }
%       \begin{scope}[opacity=1]        
%       \graphbox{$R$}{48mm}{0mm}{30mm}{12mm}{-1mm}{-5mm}{
%         \node[draw,circle] (x) at (0mm,0mm) {1};
%         \node[draw,circle] (y) at (10mm,0mm) {2};
%       }
%       \end{scope}
%     \end{tikzpicture}

%   \begin{tikzpicture}
%       \graphbox{$L$}{0mm}{-11mm}{32mm}{12mm}{2mm}{-8mm}{
%       \node[draw,circle]  (x) at (-6mm,0mm) {1};
%       \node[draw,circle] (y) at (6mm,0mm) {2};
%       \draw[->]  (x) to (y);
%       }
%       \graphbox{$K$}{33mm}{-11mm}{32mm}{12mm}{2.5mm}{-8mm}{
%           \node[draw,circle]  (x) at (-6mm,0mm) {1};
%           \node[draw,circle]  (y) at (6mm,0mm) {2};
%       }
%       \graphbox{$R_X$}{33mm}{2mm}{32mm}{12mm}{2.5mm}{-8mm}{
%           % \node[draw,circle]  (x) at (-6mm,0mm) {1};
%           % \node[draw,circle]  (y) at (6mm,0mm) {2};
%       }
%       \begin{scope}[opacity=1]        
%       \graphbox{$R$}{66mm}{-11mm}{40mm}{12mm}{-2mm}{-8mm}{
%           \node[draw,circle]  (x) at (-6mm,0mm) {1};
%           \node[draw,circle]  (y) at (6mm,0mm) {2};
%           \node[draw,circle]  (z) at (17mm,0mm) {3};
%       }
%       \end{scope}
%     \end{tikzpicture}

%   Let $X$ be 
%   \tikz[baseline=-0.5ex]{
%       \node (x) at (0,0) {$\bullet$};
%       \node (y) at (1,0) {$\bullet$};
%       \draw[->] (x) to (y);
%   }.
%   We have $|\operatorname{Mono}(X,L)| = 1 > 0 = |\operatorname{Mono}(X,R)|$ for both rules. Thus, this rewriting systems terminates by~\autoref{thm:termination_grs}.
% \end{example}



% \begin{example}
%   \label{ex:overbeek_5d6_bis} 
%   Consider the DPO rewriting system with monic matches presented in~\cite[Example 5.6]{overbeek2024termination_lmcs} with $\rho$ replaced by $\rho'$ as depicted by the span 
%   \begin{center}
%     $\rho' = $\scalebox{0.9} { {
\begin{tikzpicture}[baseline=-10mm]
    \graphbox{$L$}{0mm}{0mm}{38mm}{20mm}{2mm}{-13mm}{
      \node [draw, circle] (x) at (-7mm,0mm) {1};
      \node [draw, circle] (y) at (3mm,0mm) {2 3};
      \draw[->] (x) edge[loop above] node  {$a$} (x);
      \draw[->] (y) edge [loop above] node {$a$} (y);
    }
    \graphbox{$K$}{42mm}{-0mm}{38mm}{20mm}{0mm}{-10mm}{
        \node [draw, circle] (x) at (-7mm,0mm) {1};
        \node [draw, circle] (y) at (0mm,0mm) {2};
        \node [draw, circle] (z) at (7mm,0mm) {3};    
    }
    \begin{scope}[opacity=1]        
    \graphbox{$R$}{85mm}{-0mm}{38mm}{20mm}{2mm}{-13mm}{
      \node [draw, circle] (x) at (-7mm,0mm) {1};
      \node [draw, circle] (y) at (0mm,0mm) {2};
      \node [draw, circle] (z) at (7mm,0mm) {3};
      \draw[->] (x) edge[loop above] node  {$b$} (x);
      \draw[->] (y) edge[loop above] node  {$b$} (y);
      \draw[->] (z) edge[loop above] node  {$b$} (z);
    }
    \end{scope}
    \node () at (40mm,-10mm) {$\leftarrow$};
    \node () at (83mm,-10mm) {$\rightarrow$};
\end{tikzpicture}
}

}
%   \end{center} 

%   Its termination is not in the scope of our method due to its non-node-injective left-hand side morphism. 
% \end{example}

% \begin{example}
%   \cite[Example 5.7]{overbeek2024termination_lmcs} has conditions on the host graph, which cannot be modeled by DPO rewriting rules. 
% \end{example}
% \begin{example}
%   The POPO+ rewriting system presented in~\cite[Example 5.7]{overbeek2024termination_lmcs} cannot be modeled by DPO rewriting rules due to the application condition on the context.
%   % Consider Example 5.7  without application condition.

%   % \begin{tikzpicture}
%   %     \graphbox{L}{0mm}{-9mm}{35mm}{8mm}{1mm}{-4mm}{
%   %         \node [draw,circle] (y) at (0mm,0mm) {1};
%   %         \draw [->] (y) edge [loop right] node {} (y);
%   %     }
%   %     \graphbox{K}{36mm}{-9mm}{35mm}{8mm}{1mm}{-4mm}{
%   %         \node [draw,circle] (y) at (0mm,-0mm) {1};
%   %     }
%   %     \begin{scope}[opacity=1]        
%   %         \graphbox{R}{72mm}{-9mm}{35mm}{8mm}{1mm}{-4mm}{
%   %         \node [draw,circle] (y) at (0mm,-0mm) {1};
%   %     }
%   %     \end{scope}
%   %   \end{tikzpicture}

%   % \begin{tikzpicture}
%   %     \graphbox{L}{0mm}{-9mm}{35mm}{8mm}{1mm}{-4mm}{
%   %       \node [draw,circle] (y) at (0mm,0mm) {1};
%   %     }
%   %     \graphbox{K}{36mm}{-9mm}{35mm}{8mm}{1mm}{-4mm}{
%   %     }
%   %     \begin{scope}[opacity=1]        
%   %     \graphbox{R}{72mm}{-9mm}{35mm}{8mm}{1mm}{-4mm}{
%   %     }
%   %     \end{scope}
%   %   \end{tikzpicture}
% \end{example}



% \begin{example}
%   \label{ex:overbeek_5d8_plump1995_3d8_plump2018_3_overbeek_5d8}
%   Consider the DPO rewriting system presented in~\cite[Example 5.8]{overbeek2024termination_lmcs},~\cite[Example 3]{plump2018modular} and~\cite[Example 3.8]{plump1995ontermination}. It rewriting rules are depicted by the bottom spans of the following diagram

% \begin{center}
%  $\rho = ${ 
%   \begin{tikzpicture}[baseline=-20mm]
%         \graphbox{$L$}{0mm}{-11mm}{35mm}{15mm}{0.5mm}{-9mm}{
%           \node [draw, circle] (x) at (-10mm,0mm) {1};
%           \node [draw, circle] (y) at (0mm,0mm) {2};
%           \node [draw, circle] (z) at (10mm,0mm) {3};
%           \draw[->] (x) to node [above] {$a$} (y);
%           \draw[->] (y) to node [above] {$b$} (z);
%         }
%         \graphbox{$K$}{36mm}{-11mm}{35mm}{15mm}{0.5mm}{-9mm}{
%           \node [draw, circle] (x) at (-10mm,0mm) {1};
%           %\node [draw, circle] (y) at (0mm,0mm) {2};
%           \node [draw, circle] (z) at (10mm,0mm) {3};
%         }
%         \begin{scope}[opacity=1]        
%         \graphbox{$R$}{72mm}{-11mm}{35mm}{15mm}{0.5mm}{-9mm}{
%           \node [draw, circle] (x) at (-10mm,0mm) {1};
%           \node [draw, circle] (y) at (0mm,0mm) {2};
%           \node [draw, circle] (z) at (10mm,0mm) {3};
%           \draw[->] (x) to node [above] {$a$} (y);
%           \draw[->] (y) to node [above] {$c$} (z);
%         }
%         \end{scope}
%       \end{tikzpicture}
%  }
% \end{center}

% \begin{center}
%   $\tau = ${ 
%    \begin{tikzpicture}[baseline=-20mm]
%       \graphbox{$L$}{0mm}{-11mm}{35mm}{15mm}{0.5mm}{-9mm}{
%           \node [draw, circle] (x) at (-10mm,0mm) {1};
%           \node [draw, circle] (y) at (0mm,0mm) {2};
%           \node [draw, circle] (z) at (10mm,0mm) {3};
%           \draw[->] (x) to node [above] {$c$} (y);
%           \draw[->] (y) to node [above] {$d$} (z);
%       }
%       \graphbox{$K$}{36mm}{-11mm}{35mm}{15mm}{0.5mm}{-9mm}{
%           \node [draw, circle] (x) at (-10mm,0mm) {1};
%           %\node [draw, circle] (y) at (0mm,0mm) {2};
%           \node [draw, circle] (z) at (10mm,0mm) {3};
%       }
%       \begin{scope}[opacity=1]        
%       \graphbox{$R$}{72mm}{-11mm}{35mm}{15mm}{0.5mm}{-9mm}{
%           \node [draw, circle] (x) at (-10mm,0mm) {1};
%           \node [draw, circle] (y) at (0mm,0mm) {2};
%           \node [draw, circle] (z) at (10mm,0mm) {3};
%           \draw[->] (x) to node [above] {$d$} (y);
%           \draw[->] (y) to node [above] {$b$} (z);
%       }
%       \end{scope}
%       \end{tikzpicture}
%   }
%   \end{center}

%   Let $X$ be  \tikz[baseline=-0.5ex]{
%       \node (x) at (0,0) {$\bullet$};
%       \node (y) at (1,0) {$\bullet$ };
%       \node (z) at (2,0) { $\bullet$};
%       \draw[->] (x) -- (y) node[midway, above] {$a$};
%       \draw[<-] (z) -- (y) node[midway, above] {$b$};
%   } 

%   For both rules, $R_X$ is 

%   \begin{tikzpicture}
%     \graphbox{$R_X$}{36mm}{6mm}{35mm}{15mm}{0.5mm}{-9mm}{
%       % \node [draw, circle] (x) at (-10mm,0mm) {1};
%       % %\node [draw, circle] (y) at (0mm,0mm) {2};
%       % \node [draw, circle] (z) at (10mm,0mm) {3};
%     }
%   \end{tikzpicture}

%   We have 
%   $|\operatorname{Mono}(X,L)| = 1 > 0 = |\operatorname{Mono}(X,R)|$ for $\rho$, and 
%   $|\operatorname{Mono}(X,L)| = 0 = 0 = |\operatorname{Mono}(X,R)|$ for $\tau$, therefore $\rho$ is eliminated.
  
%   Let $Y$ be  \tikz[baseline=-0.5ex]{
%       \node (x) at (0,0) {$\bullet$};
%       \node (y) at (1,0) {$\bullet$ };
%       \draw[->] (x) -- (y) node[midway, above] {$c$};
%   }. For $\tau$, we have

%   \begin{tikzpicture}
%     \graphbox{$R_X$}{36mm}{6mm}{35mm}{15mm}{0.5mm}{-9mm}{
%       % \node [draw, circle] (x) at (-10mm,0mm) {1};
%       % %\node [draw, circle] (y) at (0mm,0mm) {2};
%       % \node [draw, circle] (z) at (10mm,0mm) {3};
%     }
%   \end{tikzpicture} 
  
%   and $|\operatorname{Mono}(Y,L)| = 1 > 0 = |\operatorname{Mono}(Y,R)|$.

%   Thus, this rewriting systems terminates by~\autoref{thm:termination_grs}.
% \end{example}

% \begin{example}
%   \label{ex:bruggink2015_ex5}
%   Our method cannot prove the termination of the DPO rewriting system with monic matches presented in~\cite[Example 5]{bruggink2015proving}. 
% \end{example}
% \begin{example}
%   \label{ex:bruggink2015_ex6_endrullis2024_d2}
%   Our method cannot prove the termination of the DPO rewriting system with monic matches presented in~\cite[Example 6]{bruggink2015proving}. 
% \end{example}

% \begin{example}
%   \label{ex:plump2018_ex6_endrullis_d4}
%   The DPO rewriting system with monic matches presented in~\cite[Example 6]{plump2018modular} is not in the scope of our method due to the non-injective rule.
% \end{example}


% \begin{example} 
%   \label{ex:termination:contrib} 
%   The rewriting rule illustrated below is from~\cite[Example 6]{plump2018modular}.
%   % \begin{figure}[H] 
%   \begin{center}
%     $\tau = ${ \resizebox{0.6\textwidth}{!}{
%       \begin{tikzpicture}[baseline=-17mm]
%           \graphbox{$L$}{0mm}{0mm}{35mm}{35mm}{2mm}{-5mm}{
%               \coordinate (delta) at (0,-18mm);
%               \node[draw,circle] (l1) at ($(delta) + (-1,1.5)$) {1};
%               \node[draw,circle] (l2) at ($(delta) + (1,1.5)$) {2};
%               \node[draw,circle] (l3) at ($(delta) + (0,0)$) {3};
%               \draw[->] (l1) -- (l3) node[midway,left] {s};
%               \draw[->] (l2) -- (l3) node[midway,right] {s};
%               \draw[->] (l3) edge [loop below] node {0} (l3);
%           }
%           \graphbox{$K$}{40mm}{0mm}{35mm}{35mm}{2mm}{-5mm}{
%               \coordinate (delta) at (0,-18mm);
%               \coordinate (interfaceorigin) at ($(delta) +(5,0)$);
%               \node[draw,circle] (r1) at ($(delta) +(-1,1.5)$) {1};
%               \node[draw,circle] (r2) at ($(delta) +(0.5,1.5)$) {2};
%               \node[draw,circle] (r3) at ($(delta) + (0,0)$) {3};
%               \draw[->] (r1) -- (r3) node[midway,left] {s};
%               \draw[->] (r3) edge [loop below] node {0} (r3);
%           }
%           \graphbox{$R$}{80mm}{0mm}{35mm}{35mm}{2mm}{-5mm}{
%               \coordinate (delta) at (0,-18mm);
%               \node[draw,circle] (r1) at ($(delta) + (-1,1.5)$) {1};
%               \node[draw,circle] (r2) at ($(delta) + (0.5,1.5)$) {2};
%               \node[draw,circle] (r3) at ($(delta) + (0,0)$) {3};
%               \node[draw,circle] (r4) at ($(delta) + (1,0)$) {};
%               \draw[->] (r1) -- (r3) node[midway,left] {s};
%               \draw[->] (r2) -- (r4) node[midway,right] {s};
%               \draw[->] (r4) edge [loop below] node {0} (r4);
%               \draw[->] (r3) edge [loop below] node {0} (r3);
%           }
%           % \graphbox{$R_x$}{40mm}{40mm}{35mm}{35mm}{2mm}{-5mm}{
%           %     \coordinate (delta) at (0,-18mm);
%           %     \coordinate (rxorigin) at ($(interfaceorigin)+(0,6)$);
%           %     \node[draw,circle] (r1) at ($(delta) + (-1,1.5)$) {1};
%           %     \node[draw,circle] (r2) at ($(delta) +  (0.5,1.5)$) {2};
%           %     \node[draw,circle] (r3) at ($(delta) +  (0,0)$) {3};
%           %     \draw[->] (r1) -- (r3) node[midway,left] {s};
%           %     \draw[->] (r3) edge [loop below] node {0} (r3);
%           % }
%           \node () at (37mm,-18mm) {$\leftarrowtail$};
%           \node () at (78mm,-18mm) {$\rightarrowtail$};
%           % \node () at (57mm,2mm) {$\uparrowtail$};
%           % \node () at (38mm,2mm) {$\swarrowtail$};
%           % \node () at (79mm,2mm) {$\searrowtail$};
%       \end{tikzpicture}
%       }
%     }
%   \end{center}
%   Its termination can be proved by our method.
% \end{example}  


% % \begin{example}
% %   \label{ex:termination:grsaa}
% %   Consider the rewriting rule in~\autoref{ex:grsaa_rx}. Let $X$ be the graph \tikz[baseline=-0.5ex]{
% %       \node (x) at (0,0) {$\bullet$};
% %       \node (y) at (1,0) {$\bullet$ }; 
% %       \node (z) at (2,0) { $\bullet$};
% %       \draw[->] (x) -- (y) node[midway, above] {$a$};
% %       \draw[<-] (z) -- (y) node[midway, above] {$a$};
% %   } with weight $1$ and $\mathbb{X} = \{X\}$. The rule is $X$-non-increasing by~\autoref{example:grs_aa:has_more_left}. 
% %   % Let $s_\mathbb{X}$ be the weight function which associates the weight of $1$ to $X$. 
% %   % Since \(w_{s_\mathbb{X}}(L) = 1 > 0 = w_{s_\mathbb{X}}(R)\), 
% %   Since \(w(L) = 1 > 0 = w(R)\),
% %   it terminates by~\autoref{thm:termination_grs}.
% % \end{example}

% % We present an example where $\mathbb{X}$ is not a singleton.
% % \begin{example} 
% %   \label{ex:overbeek_5d6}
% %   Consider the rewriting rules presented in~\cite[Example 5.6]{overbeek2024termination_lmcs} illustrated below:
% %   \begin{center}
% %     $\rho = $\scalebox{0.7} { {
\begin{tikzpicture}[baseline=-15mm,scale=1.2]
    \graphbox{$L$}{0mm}{0mm}{31mm}{20mm}{2mm}{-13mm}{
      \node [draw, circle] (x) at (-7mm,0mm) {1};
      \node [draw, circle] (y) at (0mm,0mm) {2};
      \draw[->] (x) edge[loop above] node  {$a$} (x);
      \draw[->] (y) edge [loop above] node {$a$} (y);
    }
    \graphbox{$K$}{35mm}{-0mm}{18mm}{20mm}{0mm}{-10mm}{
    }
    \begin{scope}  
    \graphbox{$R$}{57mm}{-0mm}{38mm}{20mm}{2mm}{-13mm}{
      \node [draw, circle] (x) at (-7mm,0mm) {3};
      \node [draw, circle] (y) at (0mm,0mm) {4};
      \node [draw, circle] (z) at (7mm,0mm) {5};
      \draw[->] (x) edge[loop above] node  {$b$} (x);
      \draw[->] (y) edge[loop above] node  {$b$} (y);
      \draw[->] (z) edge[loop above] node  {$b$} (z);
    }
    \end{scope}
    \node () at (33mm,-12mm) {$\leftarrowtail$};
    \node () at (55mm,-12mm) {$\rightarrowtail$};
\end{tikzpicture}
}}
% %   \end{center}
% %   \begin{center}
% %   $\tau = $\scalebox{0.7}{ {
  \begin{tikzpicture}[baseline=-15mm,scale=1.5,scale=0.9]
    \graphbox{L}{0mm}{0}{31mm}{20mm}{2mm}{-13mm}{
        \node[draw,circle] (x) at (-7mm,0mm) {1};
        \node[draw,circle] (y) at (0mm,0mm) {2};
        \draw[->] (x) edge[loop above] node {$b$} (x);
        \draw[->] (y) edge [loop above] node {$b$} (y);
      }
      \graphbox{K}{35mm}{0}{18mm}{20mm}{2mm}{-13mm}{
      }
      \begin{scope} 
      \graphbox{R}{57mm}{0}{25mm}{20mm}{1mm}{-13mm}{
        \node[draw,circle] (x) at (-3.5mm,0mm) {3};
        \draw[->] (x) edge[loop above] node {$a$} (x);
      }
      \end{scope}
      \node () at (33mm,-12mm) {$\leftarrowtail$};
     \node () at (55mm,-12mm) {$\rightarrowtail$};
  \end{tikzpicture}}
% %   }
% %   \end{center}
% %   Let $X$ be 
% %   \tikz[baseline=-0.5ex]{
% %       \node (x) at (0,0) {$\bullet$};
% %       \draw[->] (x) edge [loop right] node {a} (x);
% %   } with weight $5$, let $Y$ be
% %   \tikz[baseline=-0.5ex]{
% %       \node (x) at (0,0) {$\bullet$};
% %       \draw[->] (x) edge [loop right] node {b} (x);
% %   } with weight $3$ and $\mathbb{X} = \{X, Y\}$.
% %   The rules $\rho$ and $\tau$ are both $X$- and $Y$-non-increasing with $D(rhs(\rho), X) = D(rhs(\rho), Y) = D(rhs(\tau), X) = D(rhs(\tau), Y) = \emptyset$.   
% %   % Let $s_\mathbb{X}$ be the weight function associating the weight of $5$ to $X$ and the weight of $3$ to $Y$.
% %   % We have $
% %   %     w_{s_\mathbb{X}}(\operatorname{lhs}(\rho)) = 10 > 9 = w_{s_\mathbb{X}}(\operatorname{rhs}(\rho))
% %   %     $ and $
% %   %     w_{s_\mathbb{X}}(\operatorname{lhs}(\tau)) = 6 > 5 = w_{s_\mathbb{X}}(\operatorname{rhs}(\tau))$.
% %   We have $
% %   w(\operatorname{lhs}(\rho)) = 10 > 9 = w(\operatorname{rhs}(\rho))
% %   $ and $
% %   w(\operatorname{lhs}(\tau)) = 6 > 5 = w(\operatorname{rhs}(\tau))$.
% %   Thus, this rewriting systems terminates by~\autoref{thm:termination_grs}.
% % \end{example}
 

\printbibliography
\end{document}  