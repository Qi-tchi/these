Introduced in the early seventies, the double-pushout (DPO) approach [EPS73] is the best studied and most popular approach to graph transformation. There are many varieties of graphs that may be of interest in different contexts. For example, one may work with directed or undirected graphs; they may be typed or untyped; they may be labeled, unlabeled, or include attributes that represent values stored in vertices or edges; they may be graphical structures like Petri nets or state-transition diagrams, or they even may be drags. It should be clear that studying graph rewriting separately for each kind of graph is a waste of time since there is almost no difference between rewriting a directed or an undirected graph or any other kind of graphical structure. Using categorical constructions allows the DPO approach to describe and study at one and the same time rewriting for all manner of structures that satisfy some given properties. The DPO approach is currently defined for any category of objects that is adhesive [EEPT06, LS06] (or M-adhesive [EGH+14, EGH+12], or even M N -adhesive [HP12]). This includes most graph categories as well as other graphical structures, and other categories of objects, like sets, bags, or algebraic specifications. 



Several other algebraic approaches, like Agree [HMP98b], PBPO [CDE+19], and PBPO+ [OER21], have been defined to overcome some limitations of the DPO approach, such as the ability to erase or clone nodes.


