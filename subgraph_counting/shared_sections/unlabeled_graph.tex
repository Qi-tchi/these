% \begin{definition}
%     \label{def:graph:unlabeled}
%     An \textbf{directed unlabeled multigraph} \( G \) consists of 
%     \begin{itemize}
%         \item a collection $V(G)$ of elements, which are called \textbf{nodes} (or \textbf{objects}),
%         \item a collection $E(G)$ of elements different from nodes, which are called \textbf{edges} (or \textbf{arrows}),
%         \item a function $\opn{dom}: E \to V$ assigning to each edge its \textbf{source} (or \textbf{domain}) node,
%         \item a function $\opn{cod}: E \to V$ assigning to each edge its \textbf{target} (or \textbf{codomain}) node.
%     \end{itemize} 
%     An directed unlabeled multigraph $G$ is \textbf{finite} if $V(G)$ and $E(G)$ have both finite elements.
%     Throughout, the term \enquote{unlabeled graph} denotes a directed unlabeled multigraph.
%     We write \( a: s \to t \) to indicate that \( a \) is an edge from \( s \) to \( t \).
% \end{definition}   

% Note that an unlabeled graph can have multiple edges from one node to another, as well as loops (edges from a node to itself), because different edges $u,v$ of the unlabeled graph can have the same source and target nodes, i.e. $\opn{dom}(u) = \opn{dom}(v)$ and $\opn{cod}(u) = \opn{cod}(v)$.


% \begin{example}
%     An unlabeled graph is shown in Figure~\ref{fig:preliminaries:unlabeled_graph}.

%     In an unlabeled graph, edges are directed and will be visualized as arrows; between any two nodes, there can be multiple edges, and loops are allowed. An unlabeled graph is depicted in Figure~\ref{fig:preliminaries:unlabeled_graph} where nodes are marked with numbers to facilitate discussion, but in general, nodes are not labeled.
%     \begin{figure}[H]
%         \centering
%         \begin{tikzpicture}
%             \graphbox{}{0mm}{0mm}{32mm}{28mm}{-10mm}{-14mm}{
%                 \node[draw,circle] (1) at (0,0) {1};
%                 \node[draw,circle] (2) at (2,0) {2};
%                 \draw[->] (1) edge[loop above] (1) ;
%                 \draw[->] (1) edge[loop below] (1) ;
%                 \draw[->] (1) edge[bend left] (2)  ;
%                 \draw[->] (2) edge[bend left] (1)   ;
%             }
%         \end{tikzpicture}
%     \caption{Unlabeled graph}
%     \label{fig:preliminaries:unlabeled_graph}
%     \end{figure}
% \end{example}

% \begin{example}
%     In an unlabeled graph, edges are directed and will be visualized as arrows; between any two nodes, there can be multiple edges, and loops are allowed. An unlabeled graph is depicted in Figure~\ref{fig:preliminaries:unlabeled_graph} where nodes are marked with numbers to facilitate discussion, but in general, nodes are not labeled.
%     \begin{figure}[H]
%         \centering
%         \begin{tikzpicture}
%             \graphbox{}{0mm}{0mm}{32mm}{28mm}{-10mm}{-14mm}{
%                 \node[draw,circle] (1) at (0,0) {1};
%                 \node[draw,circle] (2) at (2,0) {2};
%                 \draw[->] (1) edge[loop above] (1) ;
%                 \draw[->] (1) edge[loop below] (1) ;
%                 \draw[->] (1) edge[bend left] (2)  ;
%                 \draw[->] (2) edge[bend left] (1)   ;
%             }
%         \end{tikzpicture}
%     \caption{Unlabeled graph}
%     \label{fig:preliminaries:unlabeled_graph}
%     \end{figure}
% \end{example}