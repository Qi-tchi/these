
In this chap, we present an automated method for relative termination~\cite{geser1990relative} of double-pushout (DPO) graph rewriting systems with injective rules~\cite{corradini1997algebraic,habel2001double,konig2018atutorial}.
% In this paper, we present a machine-checkable sufficient condition for relative termination~\cite{geser1990relative} of double-pushout (DPO) graph rewriting systems with injective rules~\cite{corradini1997algebraic,habel2001double, konig2018tutorial}. 
DPO graph rewriting provides an intuitive yet formal framework for modeling distributed systems, enabling rigorous analysis and automated verification of  their properties.
Relative termination ensures that, given two rule sets \( \mathcal{A} \) and \( \mathcal{B} \), rules in $\mathcal{A}$ can only be applied finitely many times in any sequence of rule applications from $\mathcal{A} \cup \mathcal{B}$.

Our method works as follows. Let \( \mathbb{X} \) denote a set of graphs. We assign weights (natural numbers) to occurrences of graphs in \( \mathbb{X} \), and the weight of a graph $G$ is defined as the sum of the weights of all occurrences of graphs in \( \mathbb{X} \) within $G$. Let $\mathcal{A}$ and $\mathcal{B}$ be sets of injective DPO rewriting rules. Our technique requires that, for every graph $X$ in $\mathbb{X}$, and for every rewriting step $G \Rightarrow H$ using a rule from $\mathcal{A}$ or $\mathcal{B}$, there exists an injective mapping from the set of \( X \)-occurrences in the result graph $H$
(those neither fully included in the right-hand side graph nor fully included in the context)
 to the set of \( X \)-occurrences in the host graph $G$ (those neither fully included in the left-hand side graph nor fully included in the context).   

Suppose the following conditions hold: (i) For every rule in \( \mathcal{A} \), the left-hand side graph's weight is strictly greater than the right-hand side graph's weight; (ii) For every rule in \( \mathcal{B} \), the left-hand side graph's weight is greater than or equal to the right-hand side graph's weight. 
Under these conditions, rewriting steps using rules in \( \mathcal{A} \) strictly decrease the weights of the host graphs, while rewriting steps using rules in \( \mathcal{B} \) do not increase them.
Consequently, the rewriting rules in \( \mathcal{A} \) can only be applied finitely many times: since the weight of a finite graph is a natural number and strictly decreases with each rewriting step using a rule in \( \mathcal{A} \), the process must terminate after finitely many iterations.  

Our method proves termination for systems such as the one in \autoref{ex_contrib_variant}, which prior interpretation-based approaches~\cite{zantema2014termination,bruggink2014termination,bruggink2015proving,
endrullis2024generalized_arxiv_v2,
overbeek2024termination_lmcs} cannot handle. 
We also provide an implementation of our technique~\footnote{The source code is available at \url{https://github.com/Qi-tchi/LyonParallel/tree/icgt2025}}.  
   
The remainder of this chapter is organized as follows.
\autoref{sec:pre} recalls some essential preliminary definitions. 
\autoref{sec:termination_criterion} presents a sufficient condition for termination of DPO graph rewriting systems.
\autoref{sec:examples} illustrates the method with some examples.
\autoref{sec:related_work} compares our approach with some existing methods.
Finally, \autoref{sec:conclusion} concludes with remarks and future research directions. The missing proofs are available in 
\iflongversion
the appendix.
\else
the preprint version \cite{qiu2025termination}.
\fi