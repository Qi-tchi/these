\noindent Our technique is described for DPO rewriting with injective matches due to their expressiveness, but it also extends to non-injective matches~\cite{habel2001double}.

% \begin{definition}[\cite{habel2001double}] 
%     \ \newline 
%     \noindent
%     \begin{minipage}{0.7\textwidth}
%         Given a rule $\rho = L \leftarrow K \rightarrow R$, a rule $\rho' = (L' \leftarrow K' \rightarrow R')$ is called a \textbf{quotient rule} of $\rho$ if there exists a DPO diagram (shown on the right)
%     where all vertical graph morphisms are surjective. The set of quotient rules of $\rho$ is denoted by $Q(\rho)$, and 
%     for a rule set $\mathcal{A}$, we write $Q(\mathcal{A}) = \bigcup_{\rho\in\mathcal{A}} Q(\rho)$.
%     \end{minipage}
%     \hfill
%     \begin{minipage}{0.29\textwidth}
%         \hfill
%         \begin{tikzpicture}[scale=0.7]
%             \node (k) at (0,0) {K};
%             \node (l) at (-2,0) {L};
%             \node (r) at (2,0) {R};
%             \node (k') at (0,-2) {K'};
%             \node (l') at (-2,-2) {L'};
%             \node (r') at (2,-2)  {R'};
%             \draw[->] (k) -> (l);
%             \draw[->] (k) -> (r); 
%             \draw[->] (k') -> (l'); 
%             \draw[->] (k') -> (r'); 
%             \draw[->] (k) -> (k'); 
%             \draw[->] (l) -> (l'); 
%             \draw[->] (r) -> (r'); 
%             \node () [at=($(l)!0.5!(k')$)] {$PO$};
%             \node () [at=($(r)!0.5!(k')$)] {$PO$};
%         \end{tikzpicture}
%     \end{minipage}
% \end{definition}

% \begin{lemma}[\cite{habel2001double}]
%     Let $\rho$ be a DPO rewriting rule. For any graphs $G$ and $H$,
%     $G \Rightarrow_{\rho,\mathfrak{F}} H$ if and only if $G \Rightarrow_{Q(\rho),\mathfrak{M}} H$.
% \end{lemma}
% \begin{corollary}
%     \label{cor:termination}
%     Let \(\mathcal{A}\) and \(\mathcal{B}\) be sets of injective DPO rewriting rules. 
%     The rewriting relation $\Rightarrow_{\mathcal{A},\mathfrak{F}}$ terminates relative to $\Rightarrow_{\mathcal{B},\mathfrak{F}}$ 
%     if 
%     $\Rightarrow_{Q(\mathcal{A}),\mathfrak{M}}$ terminates relative to $\Rightarrow_{Q(\mathcal{B}),\mathfrak{M}}$.
% \end{corollary}