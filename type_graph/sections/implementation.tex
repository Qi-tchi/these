To the best of our knowledge, no method exists to decide whether a weighted type graph can be constructed over the existing concrete semirings in ~\autoref{example:real_semirings} in general. 
This problem is inherently hard because it requires quantifying over all edge-labeled multigraphs with weighted edges.

Existing implementations presented in \cite[\textsection 6]{bruggink2015proving}, ~\cite[\textsection 6]{zantema2014termination}, \cite[\textsection E]{endrullis2024generalized} employ Z3, a satisfiable modulo theories (SMT) solver that can solve first order theories over the nautural numbers, to searching a suitable type graph.
Furthermore, they constrain the search space by fixing a natural number
\( k \in \mathbb{N} \) and constructing a weighted type graph \((T, \mathbb{E}, S, w)\) where the underlying graph \( T \) is a simple graph with \( k \) nodes.

We adopt the same method, employing the Z3 and constraining the search space by fixing a natural number \( k \in \mathbb{N} \). Specifically, we construct a weighted type graph \((T, \mathbb{E}, S, w)\), where the underlying graph \( T \) is a simple graph with \( k \) nodes.

To model a type graph in Z3, we adopt an approach similar to that proposed by Bruggink in \cite[\textsection 6]{bruggink2015proving}.
Specifically, we fix a complete simple graph $T$ with $k$ nodes, i.e. a graph with an edge for every pair $i,j\in\{1,...,k\}$ of nodes and every edge label $l \in \Sigma$. 
Every edge $e$ in this graph is associated with two variables:
a binary variable \( x_e \in \{0,1\} \) indicating the presence/absence of \( e \) in the type graph;
a numeric variable \( y_e \) (integer or real, depending on the semiring) representing the edge weight when \( e \) exists.  
We fix additionally a real variable $\delta \in \mathbb{R}_{>0}$.
The objective is to assign values to $\delta$, \( (x_e)_{e \in E(T)} \) and \( (y_e)_{e \in E(T)} \) such that all rules are either: decreasing, $\delta$-uniformly decreasing (for tropical or arctic semirings) or \(\delta\)-closure decreasing (for arithmetic semirings).  
These constraints are formalized as first-order formulas over the natural numbers (or real numbers) and encoded in the Z3. The resulting satisfiability problem is then resolved using Z3.

% Existing implementations of the type graph method~\cite{TORPAcyc,grez}~\footnote{There exists an implementation in GraphTT~\cite{endrullis2024generalized}, but to the best of our knowledge, this implementation is not publicly available} 
% constrain the search space by fixing a natural number
% \( k \in \mathbb{N} \) and constructing a weighted type graph \((T, \mathbb{E}, S, w)\) where the underlying graph \( T \) is a simple graph with \( k \) nodes (see~\cite[\textsection 6]{bruggink2015proving}, ~\cite[\textsection 6]{zantema2014termination}, \cite[\textsection E]{endrullis2024generalized}).
However, even under these constraints, \textcolor{red}{constructing weighted type graphs over the natural semirings remains challenging.}\todo{It is (still) not clear that you mean by "constructing". Are you searching for a type graph with specific properties?} Specifically, constructing a weighted type graph over the natural tropical semiring $\mathfrak{T}$ or the natural arctic semiring $\mathfrak{A}$ requires solving first-order formulas in Presburger arithmetic. Any algorithm that decides this problem has at least double-exponential time complexity with respect to $k^2 * | \Sigma |$ where \( \Sigma \) denotes the set of edge labels~\cite{fischer1998super}. For the natural arithmetic semiring $\mathfrak{N}$, the task reduces to solving first-order formulas in Peano arithmetic\textemdash a task that is semi-decidable~\cite{matiyasevivc2003enumerable}.

Consequently, existing tools rely on user-specified sets of weights to constrain the size of the search space. However, determining these sets in advance is hard if not impossible, limiting the tools' accessibility and practical applicability.
 
However, constructing weighted type graphs over the real semirings is significantly easier under these constraints.
Specifically, constructing a weighted type graph whose underlying graph is a \textcolor{red}{simple}\todo{what makes a graph simple? Do you mean that here you are restricting to a specific category? Which one?} graph of size \( k \) over $\mathfrak{T}'$ and $\mathfrak{A}'$ involves solving a mixed-integer linear programming problem with $k^2 * | \Sigma |$ binary variables and $k^2 * | \Sigma |$ real variables.
State-of-the-art SMT solvers such as Z3~\cite{de2008z3} can efficiently handle this task.
 \todo{complexity}
For the real arithmetic semiring $\mathfrak{N}'$, the task reduces to solving first-order formulas of the real numbers. This problem is decidable in double-exponential time
relative to $k^2 * | \Sigma |$, where
\( \Sigma \) denotes the set of edge labels~\cite{collins1974quantifier}, in contrast to the undecidable problem for the natural arithmetic semiring $\mathfrak{N}$.

By avoiding undecidable Peano arithmetic, manual weight selection, and reducing computational complexity, a more automated, accessible, practical and efficient tool can be implemented. Besides, it would be interesting to
combine the complementary strengths of our approach and prior methods, as constructing weighted type graphs over the real arithmetic semiring remains computationally challenging.

We implemented both our approach and the approach proposed by Endrullis et al. \cite{endrullis2024generalized} for edge-labeled directed graph rewriting in a tool in OCaml to leverage their complementary strengths.

% Following prior work, we constrain the search space by fixing a natural number
% \( k \in \mathbb{N} \) and constructing a weighted type graph \((T, \mathbb{E}, S, w)\) where the underlying graph \( T \) is a simple graph with \( k \) nodes. 
Given a graph rewriting system, our tool searches in parallel weighted type graphs over the six existing concrete semirings using an automated strategy: the graph size incrementally increases from 1 to 4; for the natural semirings, the max weight incrementally increases from 1 to 3. Decision problems are first translated into first-order theories in Z3 and then solved using Z3~\cite{de2008z3}.
This fully automated approach requires only the problem definition and a timeout, removing the need for users to specify details such as the semiring to be used or the size of the type graph. 
It accurately reflects real-world scenarios, as it is generally hard to determine parameters such as graph size, semiring, and weight bounds in advance.

Finally, if our tool terminates before timeout and reports that no suitable weighted type graph of size \( k \in \mathbb{N} \) has been found, this implies that no such weighted type graph. Specifically, no weighted type graph with a simple underlying graph of size \( k \) exists for any of the existing six concrete semirings analyzed. \todo{how; why}
For the tropical and arctic semirings, current tools are constrained to constructing type graphs with user-specified finite weight sets, a restriction that inherently precludes them from proving the absence of solutions for all possible weights. In the case of the natural arithmetic semiring, where the decision problem is semi-decidable, existing tools are fundamentally \textcolor{red}{incapable of providing negative results.}\todo{Not true; but you can't be complete in this respect}