Type graphs are weighted over semirings which are algebraic structures as defined below.
\begin{definition} 
    A \textbf{monoid}\index{Monoid} $(S, \mathop{\cdot}, e)$ is a set $S$ equipped with a binary operation $\mathop{\cdot} : S \mathop{\times} S \mathop{\rightarrow} S$ and having a special element $e$ such that:
    \begin{itemize}
        \item {$\mathop{\cdot}$ is associative:} $ (x \mathop{\cdot} y) \mathop{\cdot} z \mathop{=} x \mathop{\cdot} (y \mathop{\cdot} z)$,
        \item {$e$ is an identity element:} $  e \mathop{\cdot} x \mathop{=} x \mathop{=} x \mathop{\cdot} e$.
    \end{itemize}
    A \textbf{commutative monoid} is a monoid where $\mathop{\cdot}$ is commutative: $x \mathop{\cdot} y \mathop{=} y \mathop{\cdot} x$.
\end{definition}

\begin{definition} 
    \label{def_semiring}
        A \textbf{semiring}\index{Semiring} is a 5-tuple $(S,\mathop{\oplus},\mathop{\odot},0,1)$ satisfying the following conditions:
         \begin{itemize}
            \item {$(S,\mathop{\oplus},0)$ is a commutative monoid},
            \item {$(S,\mathop{\odot},1)$ is a monoid},
            \item {$\mathop{\odot}$ is distributive over $\mathop{\oplus}$:}
            \begin{itemize}
                \item $a \mathop{\odot} ( b \mathop{\oplus} c) \mathop{=} (a \mathop{\odot} b) \mathop{\oplus} (a \mathop{\odot} c)$, 
                \item $(b \mathop{\oplus} c) \mathop{\odot} a \mathop{=} (b \mathop{\odot} a) \mathop{\oplus} (c \mathop{\odot} a)$,
            \end{itemize} 
            \item {0 is an annihilator for $\mathop{\odot}$:} $x \mathop{\odot} 0= 0 \mathop{=}  0 \mathop{\odot} x $.
        \end{itemize}
        A semiring is said to be commutative if $(S,\mathop{\odot},1)$ is a commutative monoid.
\end{definition}
The following definition of well-founded semirings is from~\cite{endrullis2024generalized_icgt}.
\begin{definition}
    \label{def:well_founded_semiring}
    A \textbf{well-founded semiring}\index{Semiring!well-founded} $(S, \mathop{\oplus}, \mathop{\odot}, 0, 1,\prec, \mathop{\preceq})$ consists of
    \begin{itemize}
        \item a semiring $(S, \mathop{\oplus}, \mathop{\odot}, 0, 1)$, and
        \item non-empty orders $\prec, \mathop{\preceq} \mathop{\subseteq} S \mathop{\times} S$ for which $\mathop{\prec} \mathop{\subseteq} \mathop{\preceq}$ and $\mathop{\preceq}$ is reflexive,
    \end{itemize}
    such that $\mathop{\succ}$ is terminating relative to $\mathop{\succeq}$, $0 \mathop{\neq} 1$ and for all $x,y,z,w \mathop{\in} S$ we have
        \begin{align*}
            x \mathop{\preceq} x' \mathop{\land} y \mathop{\preceq} y' 
            &\mathop{\Rightarrow}
            x \mathop{\oplus} y \mathop{\preceq} x' \mathop{\oplus} y',
            &\tag{S1} \label{wfs:ax:s1} 
            \\   
            % x < y  
            % &\mathop{\Rightarrow}
            % x \mathop{\oplus} z \mathop{\preceq} y \mathop{\oplus} z 
            % \tag{S1} \label{eq:ordered_semiring_plus_monotonic} 
            % \\ w
            x \mathop{\prec} x' \mathop{\land} y \mathop{\prec} y'  
            &\mathop{\Rightarrow}
            x \mathop{\oplus} y \mathop{\prec} x' \mathop{\oplus} y',
            &\tag{S2} \label{wfs:ax:s2} 
            \\
            x \mathop{\preceq} x' \mathop{\land} 1 \mathop{\preceq} y
            &\mathop{\Rightarrow} 
            x \mathop{\odot} y \mathop{\preceq} x' \mathop{\odot} y \mathop{\land} y \mathop{\odot} x \mathop{\preceq} y \mathop{\odot} x',
            &\tag{S3} \label{wfs:ax:s3} 
            \\
            x \mathop{\prec} x' \mathop{\land} 1 \mathop{\preceq} y \mathop{\neq} 0 
            &\mathop{\Rightarrow}
            x \mathop{\odot} y \mathop{\prec} x' \mathop{\odot} y \mathop{\land} y \mathop{\odot} x \mathop{\prec} y \mathop{\odot} x'.
            &\tag{S4} \label{wfs:ax:s4}
        \end{align*}
      The semiring is \textbf{strictly monotonic}\index{Semiring!strictly monotonic} if it additionally satisfies 
        \begin{flalign*}
                \hspace{5cm}x \mathop{\prec} x'  \mathop{\land} y \mathop{\preceq} y'
                &\mathop{\Rightarrow}
                x \mathop{\oplus} y \mathop{\prec} x' \mathop{\oplus} y. 
                &\tag{S5} \label{wfs:ax:s5} 
        \end{flalign*}
\end{definition}
\begin{example}
    % The natural tropical semiring $\mathfrak{T} \mathop{=} (\mathbb{N} \mathop{\cup} \{\mathop{+\infty}\},\min,+,\mathop{+\infty}, 0, <, \leq)$ has domain $\mathbb{N} \mathop{\cup} \{\mathop{+\infty}\}$, the binary function symbol $\mathop{\oplus}$ interpreted by $\min$ and the binary function symbol $\mathop{\odot}$ interpreted by $+$, the constant symbols $0_s$ and $1_s$ interpreted by $\mathop{+\infty}$ and $0_\mathbb{N}$, respectively, the binary relation symbols $\prec$ and $\leq$ interpreted by the canonical \enquote{strictly less than} and \enquote{less than or equal to} on $\mathbb{N} \mathop{\cup} \{\mathop{+\infty}\}$. It is a well-founded semiring. It is not strictly monotonic because $2 < 3$ but $2 \mathop{\oplus} 2 \mathop{=} \min(2,2) \mathop{=} 2 \not < 2 \mathop{=} \min(3,2) \mathop{=} 3 \mathop{\oplus} 2$.

    The \textbf{natural tropical semiring}\index{Semiring!natural tropical} 
    $$
    \mathfrak{T} \mathop{=} 
    (\mathbb{N} \mathop{\cup} \{\mathop{+\infty}\},
    \opn{min}_{\mathbb{N} \mathop{\cup} \{\mathop{+\infty}\}}, 
    \mathop{+_{\mathbb{N} \mathop{\cup} \{\mathop{+\infty}\}}},
     \mathop{+\infty}, 
     0_{\mathbb{N} \mathop{\cup} \{\mathop{+\infty}\}}, 
     <_{\mathbb{N} \mathop{\cup} \{\mathop{+\infty}\}}, 
     \leq_{\mathbb{N} \mathop{\cup} \{\mathop{+\infty}\}}
     )$$ is an instance of the well-founded semiring where
    \begin{flalign*}
        S & \mathop{\longmapsto} \mathbb{N} \mathop{\cup} \{\mathop{+\infty}\},
        \\
        \mathop{\oplus} & \mathop{\longmapsto} \opn{min}_{\mathbb{N} \mathop{\cup} \{\mathop{+\infty}\}},
        \\
        \mathop{\odot} & \mathop{\longmapsto} \mathop{+_{\mathbb{N} \mathop{\cup} \{\mathop{+\infty}\}}},
        \\
        0_s & \mathop{\longmapsto} \mathop{\mathop{+\infty}},
        \\
        1_s & \mathop{\longmapsto} 0_{\mathbb{N} \mathop{\cup} \{\mathop{+\infty}\}},
        \\
        \mathop{\prec} & \mathop{\longmapsto} <_{\mathbb{N} \mathop{\cup} \{\mathop{+\infty}\}},
        \\
        \mathop{\preceq} & \mathop{\longmapsto} \mathop{\leq_{\mathbb{N} \mathop{\cup} \{\mathop{+\infty}\}}}.
    \end{flalign*}
    It is a well-founded semiring but not strictly monotonic because $2 \mathop{<}_\mathbb{N} 3$ but $2 \mathop{\oplus} 2 \isdef \min(2,2) \mathop{=} 2 \not <_\mathbb{N} 2 \mathop{=} \min(3,2) \isdef 3 \mathop{\oplus} 2$.
\end{example}
    
\begin{example}
    % The natural arctic semiring $\mathfrak{A} \mathop{=} (\mathbb{N} \mathop{\cup} \{\mathop{-\infty}\},\max,+,\mathop{-\infty}, 0,<,\mathop{\preceq})$ has domain $\mathbb{N} \mathop{\cup} \{\mathop{-\infty}\}$, the binary function symbol $\mathop{\oplus}$ interpreted by $\max$ and the binary function symbol $\mathop{\odot}$ interpreted by $+$, the constant symbols $0_s$ and $1_s$ interpreted by $\mathop{-\infty}$ and $0_\mathbb{N}$, respectively, the binary relation symbols $\prec$ and $\mathop{\preceq}$ interpreted by the canonical \enquote{strictly less than} and \enquote{less than or equal to} on $\mathbb{N} \mathop{\cup} \{\mathop{-\infty}\}$. It is a well-founded semiring. It is not strictly monotonic because $2 < 3$ but $2 \mathop{\oplus} 3 \mathop{=} \max(2,3) \mathop{=} 3 \not < 3 \mathop{=} \max(3,3) \mathop{=} 3 \mathop{\oplus} 3$.

    The \textbf{natural arctic semiring}\index{Semiring!natural arctic} 
    $$\mathfrak{A} \mathop{=} 
    (\mathbb{N} \mathop{\cup} \{\mathop{-\infty}\},
    \opn{max}_{\mathbb{N} \mathop{\cup} \{\mathop{-\infty}\}},
    +_{\mathbb{N} \mathop{\cup} \{\mathop{-\infty}\}},
    \mathop{-\infty}, 
    0_{\mathbb{N} \mathop{\cup} \{\mathop{-\infty}\}},
    <_{\mathbb{N} \mathop{\cup} \{\mathop{-\infty}\}},
    \leq_{\mathbb{N} \mathop{\cup} \{\mathop{-\infty}\}})$$ is an instance of the well-founded semiring where
    \begin{flalign*}
        S & \mathop{\longmapsto} \mathbb{N} \mathop{\cup} \{\mathop{-\infty}\},
        \\
        \mathop{\oplus} & \mathop{\longmapsto} \opn{\max}_{\mathbb{N} \mathop{\cup} \{\mathop{-\infty}\}},
        \\
        \mathop{\odot} & \mathop{\longmapsto} \mathop{+_{\mathbb{N} \mathop{\cup} \{\mathop{-\infty}\}}},
        \\
        0_s & \mathop{\longmapsto} \mathop{\mathop{-\infty}},
        \\
        1_s & \mathop{\longmapsto} 0_{\mathbb{N} \mathop{\cup} \{\mathop{-\infty}\}},
        \\
        \mathop{\prec} & \mathop{\longmapsto} \mathop{<_{\mathbb{N} \mathop{\cup} \{\mathop{-\infty}\}}},
        \\
        \mathop{\preceq} & \mathop{\longmapsto} \mathop{\leq_{\mathbb{N} \mathop{\cup} \{\mathop{-\infty}\}}}.
    \end{flalign*}
    It is a well-founded semiring but not strictly monotonic because $2 \mathop{<}_\mathbb{N} 3$ but $2 \mathop{\oplus} 3 \isdef \max(2,3) \mathop{=} 3 \not <_\mathbb{N} 3 \mathop{=} \max(3,3) \isdef 3 \mathop{\oplus} 3$.
\end{example}

\begin{example}
    % The natural arithmetic semiring $\mathfrak{N} \mathop{=} (\mathbb{N},+,*,0,1,<,\leq)$ has domain $\mathbb{N}$, the binary function symbol $\mathop{\oplus}$ interpreted by $+$ and the binary function symbol $\mathop{\odot}$ interpreted by $*$, the constant symbols $0_s$ and $1_s$ interpreted by $0_\mathbb{N}$ and $1_\mathbb{N}$, respectively, the binary relation symbols $\prec$ and $\leq$ interpreted by the canonical \enquote{strictly less than} and \enquote{less than or equal to} on $\mathbb{N}$. It is a strictly monotonic semiring.
    The \textbf{natural arithmetic semiring}\index{Semiring!natural arithmetic} $\mathfrak{N} \mathop{=} (\mathbb{N},+_\mathbb{N},*_\mathbb{N},0_\mathbb{N},1_\mathbb{N},\mathop{<}_\mathbb{N},\leq_\mathbb{N}) $ is an instance of the well-founded semiring where
    \begin{flalign*}
        S & \mathop{\longmapsto} \mathbb{N},
        \\
        \mathop{\oplus} & \mathop{\longmapsto} \mathop{+_\mathbb{N}},
        \\
        \mathop{\odot} & \mathop{\longmapsto} \mathop{*_\mathbb{N}},
        \\
        0_s & \mathop{\longmapsto} 0_\mathbb{N},
        \\
        1_s & \mathop{\longmapsto} 1_\mathbb{N},
        \\
        \mathop{\prec} & \mathop{\longmapsto} \mathop{<_\mathbb{N}},
        \\
        \mathop{\preceq} & \mathop{\longmapsto} \mathop{\leq_\mathbb{N}}.
    \end{flalign*}
    It is strictly monotonic.
\end{example}

\begin{notation} 
    \label{def:bigodot}
Let $(S, \mathop{\oplus}, \mathop{\odot}, 0, 1)$ be a semiring. We extend naturally the binary operations $\mathop{\oplus}$ and $\mathop{\odot}$ to finite sets $E \mathop{\subseteq} S$ by letting
    \begin{itemize}
        \item $\mathop{\bigodot} \emptyset \isdef 1$ and $\mathop{\bigodot} \left( E \mathop{\cup} \{x\} \right) \isdef \left( \mathop{\bigodot} E \right) \mathop{\odot} x$,
        \item $\mathop{\bigoplus} \emptyset \isdef 0$ and $\mathop{\bigoplus} \left( E \mathop{\cup} \{x\} \right) \isdef \left( \mathop{\bigoplus} E \right) \mathop{\oplus} x$.
    \end{itemize}
\end{notation}
 We define the exponentiation operation for elements of the semiring before defining formally the morphism weight.
\begin{notation} 
    \label{wfs:def:exponentiation}
Let $(S, \mathop{\oplus}, \mathop{\odot}, 0_S, 1_S)$ be a semiring. We define the \textbf{exponentiation operation}\index{Semiring!exponentiation operation} for all $x \mathop{\in} S$ and $n \mathop{\in} \mathbb{N}$ by
   \begin{itemize}
        \item $ x^0 \isdef 1_S$,
        \item $x^{n+1} \isdef x^n \mathop{\odot} x$.
   \end{itemize}
\end{notation}
% \begin{notation}
%     Let $S$ be a strongly monotonic measurable semiring. We write $0_s$, $1_s$ and $\mu_s$ to denote the additive and multiplicative neutral elements of $S$ and the homomorphism, respectively.
% \end{notation}