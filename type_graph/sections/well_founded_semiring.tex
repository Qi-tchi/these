
\begin{definition}[Monoid]
    A \textbf{monoid} $(S, \cdot, e)$ is a set $S$ equipped with a binary operation $\cdot : S \times S \rightarrow S$ and having a spacial element $e$ such that:
    \begin{itemize}
        \item {$\cdot$ is associative:} $ (x \cdot y) \cdot z = x \cdot (y \cdot z)$
        \item {$e$ is an identity element:} $  e \cdot x = x = x \cdot e$
    \end{itemize}
    A \textbf{commutative monoid} is a monoid where $\cdot$ is commutative: $x \cdot y = y \cdot x$.
\end{definition}

\begin{definition}[Semiring]
    \label{def_semiring}
        A \textbf{semiring} is a 5-tuple $(S,\oplus,\odot,0,1)$ satisfying the following conditions:
         \begin{itemize}
            \item {$(S,\oplus,0)$ is a commutative monoid} 
            \item {$(S,\odot,1)$ is a monoid} 
            \item {$\odot$ is distributive over $\oplus$:}
            \begin{itemize}
                \item $a \odot ( b \oplus c) = (a \odot b) \oplus (a \odot c)$ 
                \item $(b \oplus c) \odot a = (b \odot a) \oplus (c \odot a)$
            \end{itemize} 
            \item {0 is an annihilator for $\odot$:} $x \odot 0= 0 =  0 \odot x $
        \end{itemize}
        A \textbf{commutative semiring} is a semiring where $\odot$ is commutative.
\end{definition}

\begin{definition}[Well-founded Semiring~\cite{endrullis2024generalized_icgt}]
    \label{def:well_founded_semiring}
    A \textbf{well-founded semiring} $(S, \oplus, \odot, 0, 1,\prec, \leq)$ consists of
    \begin{itemize}
        \item A semiring $(S, \oplus, \odot, 0, 1)$,
        \item non-empty orders $\prec, \leq \subseteq S \times S$,
    \end{itemize}
    such that the equivalent class $\prec / \leq$ is well-founded, $0 \neq 1$ and for all $x,y,z,w \in S$ we have
        \begin{align*}
            x \preceq x' \land y \preceq y' 
            &\Rightarrow
            x \oplus y \preceq x' \oplus y'
            &\tag{S1} \label{wfs:ax:s1} 
            \\   
            % x < y  
            % &\Rightarrow
            % x \oplus z \leq y \oplus z 
            % \tag{S1} \label{eq:ordered_semiring_plus_monotonic} 
            % \\ w
            x \prec x' \land y \prec y'  
            &\Rightarrow
            x \oplus y \prec x' \oplus y'
            &\tag{S2} \label{wfs:ax:s2} 
            \\
            x \preceq x' \land 1 \leq y
            &\Rightarrow 
            x \odot y \preceq x' \odot y \land y \odot x \preceq y \odot x' 
            &\tag{S3} \label{wfs:ax:s3} 
            \\
            x \prec x' \land 1 \leq y \neq 0 
            &\Rightarrow
            x \odot y \prec x' \odot y \land y \odot x \prec y \odot x'
            &\tag{S4} \label{wfs:ax:s4}
        \end{align*}
      The semiring is \textbf{strictly monotonic} if it additionally satisfies 
        \begin{flalign*}
                \hspace{5cm}x \prec x'  \land y \leq y'
                &\Rightarrow
                x \oplus y \prec x' \oplus y 
                &\tag{S5} \label{wfs:ax:s5} 
        \end{flalign*}
\end{definition}
\autoref{ax:s5}
% \begin{example} 
%     \label{example:real_semirings}
%     The natural tropical semiring: $\mathfrak{T} = (\mathbb{N} \cup \{+\infty\},\min,+,+\infty, 0, <, \leq)$
%     and the natural arctic semiring: $\mathfrak{A} = (\mathbb{N} \cup \{-\infty\},\max,+,-\infty, 0,<, \leq)$ are well-founded semirings.
%     The natural arithmetic semiring $\mathfrak{N} = (\mathbb{N},+,*,0,1,<, \leq)$ is a strictly monotonic.
% \end{example}
\begin{example}
    % The natural tropical semiring $\mathfrak{T} = (\mathbb{N} \cup \{+\infty\},\min,+,+\infty, 0, <, \leq)$ has domain $\mathbb{N} \cup \{+\infty\}$, the binary function symbol $\oplus$ interpreted by $\min$ and the binary function symbol $\odot$ interpreted by $+$, the constant symbols $0_s$ and $1_s$ interpreted by $+\infty$ and $0_\mathbb{N}$, respectively, the binary relation symbols $\prec$ and $\leq$ interpreted by the canonical \enquote{strictly less than} and \enquote{less than or equal to} on $\mathbb{N} \cup \{+\infty\}$. It is a well-founded semiring. It is not strictly monotonic because $2 < 3$ but $2 \oplus 2 = \min(2,2) = 2 \not < 2 = \min(3,2) = 3 \oplus 2$.

    The natural tropical semiring $\mathfrak{T} = (\mathbb{N} \cup \{+\infty\},\min, \mathop{+_\mathbb{N}}, +\infty, 0_\mathbb{N}, <_\mathbb{N}, \leq_\mathbb{N})$ is an instance of the well-founded semiring where
    \begin{flalign*}
        S & \mapsto \mathbb{N} \cup \{+\infty\}
        \\
        \mathop{\oplus} & \mapsto \mathop{\min}
        \\
        \mathop{\odot} & \mapsto \mathop{+_\mathbb{N}}
        \\
        0_s & \mapsto \mathop{+\infty}
        \\
        1_s & \mapsto 0_\mathbb{N}
        \\
        \mathop{\prec} & \mapsto \mathop{<_\mathbb{N}}
        \\
        \mathop{\leq} & \mapsto \mathop{\leq_\mathbb{N}}
    \end{flalign*}
    It is a well-founded semiring but not strictly monotonic because $2 <_\mathbb{N} 3$ but $2 \oplus 2 = \min(2,2) = 2 \not <_\mathbb{N} 2 = \min(3,2) = 3 \oplus 2$.
\end{example}
    
\begin{example}
    % The natural arctic semiring $\mathfrak{A} = (\mathbb{N} \cup \{-\infty\},\max,+,-\infty, 0,<,\leq)$ has domain $\mathbb{N} \cup \{-\infty\}$, the binary function symbol $\oplus$ interpreted by $\max$ and the binary function symbol $\odot$ interpreted by $+$, the constant symbols $0_s$ and $1_s$ interpreted by $-\infty$ and $0_\mathbb{N}$, respectively, the binary relation symbols $\prec$ and $\leq$ interpreted by the canonical \enquote{strictly less than} and \enquote{less than or equal to} on $\mathbb{N} \cup \{-\infty\}$. It is a well-founded semiring. It is not strictly monotonic because $2 < 3$ but $2 \oplus 3 = \max(2,3) = 3 \not < 3 = \max(3,3) = 3 \oplus 3$.

    The natural arctic semiring $\mathfrak{A} = (\mathbb{N} \cup \{-\infty\},\max,+_\mathbb{N},-\infty, 0_\mathbb{N},<_\mathbb{N},\leq_\mathbb{N})$ is an instance of the well-founded semiring where
    \begin{flalign*}
        S & \mapsto \mathbb{N} \cup \{-\infty\}
        \\
        \oplus & \mapsto \mathop{\max}
        \\
        \odot & \mapsto \mathop{+_\mathbb{N}}
        \\
        0_s & \mapsto \mathop{-\infty}
        \\
        1_s & \mapsto 0_\mathbb{N}
        \\
        \prec & \mapsto \mathop{<_\mathbb{N}}
        \\
        \leq & \mapsto \mathop{\leq_\mathbb{N}}
    \end{flalign*}
    It is a well-founded semiring but not strictly monotonic because $2 <_\mathbb{N} 3$ but $2 \oplus 3 = \max(2,3) = 3 \not <_\mathbb{N} 3 = \max(3,3) = 3 \oplus 3$.
\end{example}

\begin{example}
    % The natural arithmetic semiring $\mathfrak{N} = (\mathbb{N},+,*,0,1,<,\leq)$ has domain $\mathbb{N}$, the binary function symbol $\oplus$ interpreted by $+$ and the binary function symbol $\odot$ interpreted by $*$, the constant symbols $0_s$ and $1_s$ interpreted by $0_\mathbb{N}$ and $1_\mathbb{N}$, respectively, the binary relation symbols $\prec$ and $\leq$ interpreted by the canonical \enquote{strictly less than} and \enquote{less than or equal to} on $\mathbb{N}$. It is a strictly monotonic semiring.
    The natural arithmetic semiring $\mathfrak{N} = (\mathbb{N},+_\mathbb{N},*_\mathbb{N},0_\mathbb{N},1_\mathbb{N},<_\mathbb{N},\leq_\mathbb{N}) $ is an instance of the strictly monotonic semiring where
    \begin{flalign*}
        S & \mapsto \mathbb{N}
        \\
        \oplus & \mapsto \mathop{+_\mathbb{N}}
        \\
        \odot & \mapsto \mathop{*_\mathbb{N}}
        \\
        0_s & \mapsto 0_\mathbb{N}
        \\
        1_s & \mapsto 1_\mathbb{N}
        \\
        \prec & \mapsto \mathop{<_\mathbb{N}}
        \\
        \leq & \mapsto \mathop{\leq_\mathbb{N}}
    \end{flalign*}
    It is a strictly monotonic semiring.
\end{example}

\begin{notation} 
    \label{def:bigodot}
Let $(S, \oplus, \odot, 0_s, 1_s)$ be a semiring. We extend naturally the binary operations $\oplus$ and $\odot$ to finite sets $E \subseteq S$ by letting
    \begin{itemize}
        \item $\bigodot \emptyset \overset{\operatorname{def}}{=} 1_s$ and $\bigodot \left( E \cup \{x\} \right) \overset{\operatorname{def}}{=} \left( \bigodot E \right) \odot x$;
        \item $\bigoplus \emptyset \overset{\operatorname{def}}{=} 0_s$ and $\bigoplus \left( E \cup \{x\} \right) \overset{\operatorname{def}}{=} \left( \bigoplus E \right) \oplus x$.
    \end{itemize}
%  \begin{flalign*}
%     \bigodot \emptyset &\overset{\operatorname{def}}{=} 1_s
% \\
%     \bigodot \left( E \cup \{x\} \right) &\overset{\operatorname{def}}{=} \left( \bigodot E \right) \odot x
%     \\
%     \bigoplus \emptyset &\overset{\operatorname{def}}{=} 0_s
%     \\
%         \bigoplus \left( E \cup \{x\} \right) &\overset{\operatorname{def}}{=} \left( \bigoplus E \right) \oplus x
% \end{flalign*}
\end{notation}

% \begin{notation}
%     Let $S$ be a strongly monotonic measurable semiring. We write $0_s$, $1_s$ and $\mu_s$ to denote the additive and multiplicative neutral elements of $S$ and the homomorphism, respectively.
% \end{notation}