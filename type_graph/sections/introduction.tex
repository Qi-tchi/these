Modern computing environments increasingly rely on distributed systems to enhance performance, scalability, and efficiency. 
However, it is challenging to ensure their correctness due to the inherent complexity of these systems~\cite{lamport2019thebyzantine}. 
Double pushout (DPO) graph rewriting~\cite{corradini1997algebraic,habel2001double, konig2018atutorial} offers a powerful yet intuitive formal framework for modeling such systems, enabling formal verification of their properties through interactive theorem proving~\cite{harrison2014history}.
While interactive theorem proving is a promising approach for formal verification of properties of distributed systems~\cite{plump2024formalisingDPO, potop2019formal,courtieu2016certified},  
 it requires proficiency in proof assistants and is often time-consuming. Automated methods aim to mitigate these barriers by synthesizing machine-verifiable proofs, which can be independently validated using established theorem provers~\cite{contejean2011automated}. Such techniques streamline verification workflows while maintaining formal guarantees.

 
The weighted type graph method is a technique for proving relative termination~\cite{geser1990relative} of DPO graph rewriting systems. 
Relative termination of a set of rules $\mathcal{A}$ with respect to another set of rules $\mathcal{B}$ ensures that in every infinite sequence of rule applications using rules from  $\mathcal{A} \cup \mathcal{B}$, rules from $\mathcal{A}$ are applied only finitely often. The method was first introduced in~\cite{zantema2014termination} for cycle rewriting. Subsequent work generalized it for DPO rewriting on edge-labeled directed multigraphs with injective rules and injective matches~\cite{bruggink2014termination}, later extending it to general DPO rewriting on edge-labeled multigraphs~\cite{bruggink2015proving} and adapting it to broader categories and DPO variants~\cite{endrullis2024generalized_arxiv_v2}.
The method assigns weights to morphisms targeting a weighted type graph over a well-founded semiring, and the weight of a graph is defined
 as the sum of the weights of all morphisms from that graph to the type graph. Relative termination of $\mathcal{A}$ with respect to $\mathcal{B}$ is proven by ensuring that rewriting steps using rules in \( \mathcal{A} \) strictly decrease the weights of the host graphs, while rewriting steps using rules in \( \mathcal{B} \) do not increase them.
   
    In the previous work~\cite{zantema2014termination,bruggink2014termination,bruggink2015proving}, three concrete semirings on natural numbers were proposed: the natural tropical semiring $\mathfrak{T}$, the natural arctic semiring $\mathfrak{A}$, and the natural arithmetic semiring $\mathfrak{N}$.
    However, constructing weighted type graphs over these concrete semirings for DPO rewriting on edge-labeled directed multigraphs is undecidable in general. This is because it requires quantifying over all edge-labeled directed multigraphs with weighted edges over $\mathbb{N}$. 

    In response to this challenge, the authors in \cite{zantema2014termination,bruggink2014termination,bruggink2015proving} proposed reducing the search space by fixing the number of nodes \( k \in \mathbb{N} \) and the maximum weight of edges of the weighted type graph, and then constructing a weighted type graph as follows: (1) construct a graph with \( k \) nodes, with a directed edge per ordered pair of nodes and label from the finite set of edge labels $\Sigma$ of the rewriting system; (2) for each edge, decide if it exists in the weighted type graph; (3) if an edge exists, assign a weight (a natural number) to it; (4) check if the weighted type graph satisfies the required conditions.
    
    Despite these constraints, searching for weighted type graphs that witness termination of a DPO rewriting system on edge-labeled directed multigraphs
    remains challenging. 
    Let $n = k^2 \cdot | \Sigma |$.
    For a weighted type graph over the natural tropical semiring $\mathfrak{T}$ or the natural arctic semiring $\mathfrak{A}$, the problem can be reduced to checking the satisfiability of an existential Presburger arithmetic formula with $n$ binary variables and $n$ integer variables.
    While modern SMT solvers, such as Z3~\cite{arithmetic2024z3} (which incorporates the dedicated CutSat solver~\cite{z3ilp_cutsat}), can solve practical instances of this problem, its worst-case complexity remains exponential \( O(2^{2n}) \).
     For a weighted type graph over the natural arithmetic semiring $\mathfrak{N}$, the task can be reduced to checking the satisfiability of an existential Peano arithmetic formula with addition and multiplication, involving $n$ binary variables and $n$ integer variables. Though modern solvers like Z3 can tackle practical instances, it is a semi-decidable problem~\cite{matiyasevivc2003enumerable}.

    There are automated tools that implement the weighted type graph method for proving termination of DPO rewriting systems on edge-labeled directed multigraphs: 
    \texttt{TORPAcyc}~\cite{TORPAcyc} implements the weighted type graph method for cycle rewriting and uses the SMT solver \texttt{Yices}~\cite{yices} to solve constraint systems;
    \texttt{Grez}~\cite{grez} implements the weighted type graph method for DPO rewriting on edge-labeled directed multigraphs; \texttt{GraphTT-wtg}~\footnote{To the best of our knowledge, no implementation is publicly available}~\cite{endrullis2024generalized_arxiv_v3} implements the weighted type graph method for DPO rewriting on many categories.
    \texttt{Grez} and \texttt{GraphTT-wtg}
    use  the SMT solver \texttt{Z3}~\cite{de2008z3} to solve constraint systems.


    In spite of the theoretical power of the weighted type graph method, its practical applicability is limited by the high computational complexity of searching for suitable weighted type graphs over well-founded semirings on 
    natural numbers. In fact, to reduce the search space, users of the above-mentioned tools are required
     to fix the size of the weighted type graph, as well as the maximum weight of edges in the weighted type graph, which are challenging tasks if not impossible to determine a priori even for experts in the field. 
    Furthermore, our experiments with the two publicly available ones \texttt{TORPAcyc} and \texttt{Grez} show that when the number of edge labels is 2, they struggle to search for weighted type graphs with 4 nodes and maximum edge weight of 2, or to search for weighted type graphs with 3 nodes when the maximum edge weight is larger than 3. 
    This article addresses this limitation to make the weighted type graph method more accessible to non-expert users and to enhance its practical applicability.

To reduce the complexity of searching for weighted type graphs that witness termination of a DPO rewriting system on edge-labeled directed multigraphs, we propose extending the weighted type graph method to non-well-founded semirings and introduce three concrete semirings over the real numbers: the real tropical semiring $\mathfrak{T}'$, the real arctic semiring $\mathfrak{A}'$, and the real arithmetic semiring $\mathfrak{N}'$. 


Under the same constraints, searching for weighted type graphs over these semirings on real numbers is computationally more tractable. Specifically,
for the real tropical semiring $\mathfrak{T}'$ or the real arctic semiring $\mathfrak{A}'$, the problem reduces 
to checking the satisfiability of an 
existential linear real arithmetic formula with $n$ binary variables and $n$ real variables for the real tropical semiring $\mathfrak{T}'$ or the real arctic semiring $\mathfrak{A}'$. 
The SMT solver Z3 can solve practical instances of this problem more easily using CutSat solver, because there are only $n$ integer variables. The worst-case complexity for solving this problem is $O(2^n)$.
For a weighted type graph over the real arithmetic semiring $\mathfrak{N}'$, the task reduces to checking the satisfiability of 
an existential non-linear real arithmetic formula\textemdash a task that is decidable~\cite{collins1974quantifier,z3realarithmetic}.


We have implemented both our approach and the approach proposed by Endrullis et al. \cite{endrullis2024generalized_arxiv_v2} for DPO rewriting on edge-labeled directed multigraphs in a unified tool~\footnote{The source code can be accessed at \url{https://github.com/Qi-tchi/LyonParallel/tree/icgt2025}} written in OCaml and employs Z3~\cite{de2008z3} to solve the constraint systems.

The remainder of this paper is structured as follows.  
\autoref{sec:type_graph:preliminaries} reviews essential preliminary definitions. 
\autoref{sec:strongly_monotonic_measurable_semiring} introduces the non-well-founded semirings. 
\autoref{sec:type_graph:measuring_graphs} introduces weighted type graphs.  
\autoref{sec:type_graph:weighing_pushout} introduces definitions used to estimate pushout-object weights.   
\autoref{sec:type_graph:termination} presents a termination criterion. 
\autoref{sec:type_graph:implementation} presents the implementation of our tool. 
\autoref{sec:type_graph:result} presents the experimental results.
 \autoref{sec:type_graph:related_work} discusses empirical results obtained using the tool. 
\autoref{sec:type_graph:conclusion} presents brief remarks and outlines future research directions. The missing proofs can be found in the Appendix.
