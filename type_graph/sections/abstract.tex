% \begin{abstract}
%     \begin{itemize}
%         \item what do we do?
%         \item what do we do exactly
%     \end{itemize}
%     We improve a termination technique initially developed by Zantema et al. for cycle rewriting, later extended to graph rewriting by Bruggink et al., and generalized by Endrullis et al., which uses weighted type graphs, to prove the relative termination of arrow-labeled multigraph rewriting systems with the double pushout approach.
    
%     This method proves the termination of a graph rewriting system, by constructing a type graph, which satisfies certain property, of a given order with arrows weighted by elements from a semiring. 
    
%     Under the existing formulation of this method, one can only have implementations that, at every execution, check whether it is possible to construct a weighted type graph satisfying the required properties by assigning weights from a finite set of elements of a semiring to arrows. 

%     We generalizes the technique to non-well-founded semirings which can be embedded in semirings on extended real numbers. As a consequence, an implementation of this method with which, for one execution, a weighted type graph can be constructed when 

%     \begin{itemize}
%         \item Weighted type graph method is a technique for proving termination of algebraic double-pushout (DPO) graph rewriting system.
%         \item This method proves the termination of a graph rewriting system, by constructing a graph, called type graph, with arrows weighted by elements from a well-founded semiring such that certain properties are satisfied.
%         \item To construct such a weighted type graph in practice, 
%         one has to solve a first-order theory of the integer numbers, which is undecidable. 
%         % either guess a reasonable size of weights sufficients for a suitable type graph, or iteratively check whether with a ever-size-increasing set of finite weights a suitable weighted type graph can be construct.
%         \item This an important drawback of this technique, because on the one hand a termination tool is supposed to be used by specialistes as well as by non specialistes, so we can not let users guess a reasonable size of weights sufficients for a suitable type graph, and on the other hand to much redundant work need to be done with a iterative strategie.
%         \item To overcome this problem, we generalizes the technique to non-well-founded semirings which can be embeded in semirings on extended real numbers. As a consequence, a termination problem can be translated into a first-order theory of the real numbers with addition and multiplication is decidable, which is decidable. 
%     \end{itemize}

% \keywords{Automated reasoning \and Program verification \and Termination \and Graph transformation \and weighted type graph}
% \end{abstract}
 
\begin{abstract}
    \begin{itemize}
        \item Weighted type graph method is a technique for proving termination of algebraic double-pushout (DPO) graph rewriting system.
        \item This method proves the termination of a graph rewriting system, by constructing a graph, called type graph, with arrows weighted by elements from a well-founded semiring such that certain properties are satisfied.
        \item In the previous work, three well-founded semirings on extended natural numbers was proposed. To construct a suitable weighted type graph over these semirings, 
        one is lead to solve a first-order theory of the integer numbers, which is undecidable. 
        % either guess a reasonable size of weights sufficients for a suitable type graph, or iteratively check whether with a ever-size-increasing set of finite weights a suitable weighted type graph can be construct.
        % \item This a major drawback of this technique, because on the one hand a termination tool is supposed to be used by specialistes as well as non specialistes, therefore we can not let users guess a reasonable size of weights sufficients for a suitable type graph, and on the other hand to much redundant work need to be done with a iterative strategie.
        \item To overcome this problem, we generalizes the technique to non-well-founded semirings which can be embeded \todo{ to be verified: can be embeded} in semirings on extended real numbers. 
        \item Constructing a suitable weighted type graph over these semirings is solving a first-order theory of the real numbers with addition and multiplication, which is decidable. 
    \end{itemize}

\keywords{Automated reasoning \and Program verification \and Termination \and Graph transformation \and weighted type graph}
\end{abstract}