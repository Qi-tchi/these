% \todo{plus generalement, c'est defini sur des relations, pas forcement de reecriture}
% \begin{definition} 
%     Let \(\mathcal{R}\) be a set of rewriting rules and let $\mathfrak{F}$ be a DPO rewriting framework.
%     An \textbf{$(\mathcal{R},\mathfrak{F})$-rewriting sequence} is either  
%     \begin{itemize}
%         \item a finite sequence \(s_0,s_1,\hdots, s_m\) of objects such that \( s_n \Rightarrow_{\mathcal{R},\mathfrak{F}} s_{n+1}\text{ for each } 0 \leq n \leq m-1\), or
%         \item an infinite sequence \(s_0,s_1,\hdots\) of objects such that \(s_n \Rightarrow_{\mathcal{R},\mathfrak{F}} s_{n+1}\) for each \(n \in \mathbb{N}\).
%     \end{itemize}
%     An $(\mathcal{R},\mathfrak{F})$-rewriting sequence from \( s_0 \) will be denoted by:
%     \[
%     s_0 \Rightarrow_{\mathcal{R},\mathfrak{F}} s_1 \Rightarrow_{\mathcal{R},\mathfrak{F}} s_2 \Rightarrow_{\mathcal{R},\mathfrak{F}} \cdots 
%     \]
% \end{definition}
Given a DPO rewriting framework \(\mathfrak{F}\), a rule set \(\mathcal{R}\) defines the binary relation \enquote{object $X$ can be rewritten to object $Y$ using rules from the rule set} on the objects of the category $\mathcal{C}$.
\begin{definition}\label{def:rewriting-chain}
Let \(\mathcal{R}\) be a rule set and let \(\mathfrak{F}\) be a DPO rewriting framework.
An \textbf{\((\mathcal{R},\mathfrak{F})\)-rewriting chain} is a \(\Rightarrow_{\mathcal{R},\mathfrak{F}}\)-chain. Hence, when the context is clear, we simply call it a \textbf{rewriting chain}.
\end{definition}

\begin{example}
    Consider the rewriting rules in~\autoref{fig:preliminaries:graph_transformation_rule_nonterminating}, which
    replaces an occurrence of the graph 
\raisebox{2pt}{
            \scalebox{0.7}{\tikz[baseline=-0.5ex]{
            \node [draw,circle] (z) at (-1,0) {};
            \node [draw,circle] (x) at (0,0) {};
            \node[draw,circle] (y) at (1,0) {};
            \draw[->] (z)--(x) node[midway, above] {$a$};
            \draw[->] (x)--(y) node[midway, above] {$b$};
        }}} with an occurrence of the graph \raisebox{2pt}{
            \scalebox{0.7}{\tikz[baseline=-0.5ex]{
            \node [draw,circle] (z) at (-1,0) {};
            \node [draw,circle] (x) at (0,0) {};
            \node[draw,circle] (y) at (1,0) {};
            \draw[->] (z)--(x) node[midway, above] {$b$};
            \draw[->] (x)--(y) node[midway, above] {$a$};
        }}}, keeping the extreme nodes unchanged.
    \begin{figure}[H]
        \centering
            \resizebox{0.85\textwidth}{!}{
                \begin{tikzpicture}[baseline=-3ex]
                    \graphbox{\( L \)}{0mm}{-3mm}{34mm}{15mm}{2mm}{2mm}{
                        \coordinate (o) at (0mm,-11mm); 
                        \node[draw,circle] (l1) at ($(o)+(-10mm,0mm)$) {1};
                        \node[draw,circle] (l2) at ($(l1)+(2,0)$) {2};
                        \node[draw,circle] (l3) at ($(l1) + (1,0)$) {3};
                        \draw[->] (l1) -- (l3) node[midway,above] {$a$};
                        \draw[->] (l3) -- (l2) node[midway,above] {$b$};
                    } 
            
                    \graphbox{\( K \)}{40mm}{-3mm}{34mm}{15mm}{2mm}{2mm}{
                        \coordinate (o) at (0mm,-11mm); 
                        \node[draw,circle] (l1) at ($(o)+(-10mm,0mm)$) {1};
                        \node[draw,circle] (l2) at ($(l1)+(2,0)$) {2};
                    }  
            
                    \graphbox{\( R \)}{80mm}{-3mm}{35mm}{15mm}{2mm}{2mm}{
                        \coordinate (o) at (-5mm,-11mm); 
                        \node[draw,circle] (l1) at ($(o)+(-10mm,0mm)$) {1};
                        % \node[draw,circle] (l2) at ($(l1)+(3,0)$) {2};
                        \node[draw,circle] (l3) at ($(l1) + (1,0)$) {4};
                        \node[draw,circle] (l4) at ($(l1) + (2,0)$) {2};
                        \draw[->] (l1) -- (l3) node[midway,above] {$b$};
                        \draw[->] (l3) -- (l4) node[midway,above] {$a$};
                        % \draw[->] (l4) -- (l2) node[midway,above] {$a$};
                    }    
                    \node () at (37mm,-10mm) {\( \leftarrowtail \)}; % K -> L
                    \node () at (77mm,-10mm) {\( \rightarrowtail \)}; % K -> R
                \end{tikzpicture}
                }
        \caption{}
        \label{fig:preliminaries:graph_transformation_rule_nonterminating}
    \end{figure} 
  
    A looping rewriting chain using this rule is shown in~\autoref{fig:preliminaries:sequence_of_transformation_infinite}, in which the subgraph to be replaced at each transformation step is highlighted in red.
       
        \begin{figure}[H]
           \centering
          \resizebox{0.85\textwidth}{!}{
            \tikz
            [baseline=-0.5ex]
            { 
                \node[draw,circle] (x) at (0,0) {};
                \node[draw,circle] (y) at (1,0) {};
                \node[draw,circle] (z) at (0.5,0.86) {};
                \draw[->,red] (x) -- node[midway,below] {$a$} (y) ;
                \draw[->,red] (y) -- node[midway,right] {$b$} (z) ;
                \draw[->] (z) -- node[midway,left] {$b$} (x) ;
            } 
            $\Rightarrow$ 
            \tikz[baseline=-0.5ex]{ 
                \node[draw,circle] (x) at (0,0) {};  
                \node[draw,circle] (y) at (1,0) {};
                \node[draw,circle] (z) at (0.5,0.86) {};
                \draw[->] (x) -- node[midway,below] {$b$} (y) ;
                \draw[->,red] (y) -- node[midway,right] {$a$} (z) ;
                \draw[->,red] (z) -- node[midway,left] {$b$} (x) ;
            }
            $\Rightarrow$ 
            \tikz[baseline=-0.5ex]{ 
                \node[draw,circle] (x) at (0,0) {};  
                \node[draw,circle] (y) at (1,0) {};
                \node[draw,circle] (z) at (0.5,0.86) {};
                \draw[->,red] (x) -- node[midway,below] {$b$} (y) ;
                \draw[->] (y) -- node[midway,right] {$b$} (z) ;
                \draw[->,red] (z) -- node[midway,left] {$a$} (x) ;
            }
            $\Rightarrow$
            \tikz[baseline=-0.5ex]{ 
                \node[draw,circle] (x) at (0,0) {};   
                \node[draw,circle] (y) at (1,0) {};
                \node[draw,circle] (z) at (0.5,0.86) {};
                \draw[->,red] (x) -- node[midway,below] {$a$} (y) ;
                \draw[->,red] (y) -- node[midway,right] {$b$} (z) ;
                \draw[->] (z) -- node[midway,left] {$b$} (x) ;
            }
          }
          \caption{}
          \label{fig:preliminaries:sequence_of_transformation_infinite}
        \end{figure}
\end{example}

For a set of rewriting rules \(\mathcal{R}\) (and a DPO rewriting framework \(\mathfrak{F}\)), the impossibility of transforming any object infinitely with the non-deterministic strategy \enquote{apply rules as long as possible} using rules from \(\mathcal{R}\) (in the framework \(\mathfrak{F}\)) is called \emph{termination} in the literature~\cite{middeldorp1997simple}. This property corresponds to program termination on all inputs in conventional programming languages, and is undecidable in general~\cite{plump1998terminationundecidable}.

For example of a non-terminating rule set, consider the set with the unique rule shown in~\autoref{fig:preliminaries:graph_transformation_rule_nonterminating} and the infinite rewriting chain shown in~\autoref{fig:preliminaries:sequence_of_transformation_infinite}. 
For example of a terminating rule set, consider the set with the unique rule shown in~\autoref{fig:intro:edge_deletion_andffsfjsssdkdsglkadjl}, which deletes an edge. Any rewriting chain using this rule is finite, because each rewriting step decreases the number of edges by one and a graph has only finitely many edges by definition.
 \begin{figure}[H]
    \centering
     \resizebox{0.85\textwidth}{!}{ 
                \begin{tikzpicture}[baseline=-3ex]
                    \graphbox{\( L \)}{0mm}{-3mm}{34mm}{15mm}{2mm}{2mm}{
                        \coordinate (o) at (0mm,-11mm); 
                        \node[draw,circle] (l1) at ($(o)+(-10mm,0mm)$) {1};
                        \node[draw,circle] (l2) at ($(l1)+(2,0)$) {2};
                        % \node[draw,circle] (l3) at ($(l1) + (1,0)$) {3};
                        \draw[->] (l1) -- (l2) node[midway,above] {$a$};
                    } 
            
                    \graphbox{\( K \)}{40mm}{-3mm}{34mm}{15mm}{2mm}{2mm}{
                        \coordinate (o) at (0mm,-11mm); 
                        \node[draw,circle] (l1) at ($(o)+(-10mm,0mm)$) {1};
                        \node[draw,circle] (l2) at ($(l1)+(2,0)$) {2};
                    }  
            
                    \graphbox{\( R \)}{80mm}{-3mm}{35mm}{15mm}{2mm}{2mm}{
                        \coordinate (o) at (-5mm,-11mm); 
                        \node[draw,circle] (l1) at ($(o)+(-10mm,0mm)$) {1};
                        % \node[draw,circle] (l2) at ($(l1)+(3,0)$) {2};
                        % \node[draw,circle] (l3) at ($(l1) + (1,0)$) {4};
                        \node[draw,circle] (l4) at ($(l1) + (2,0)$) {2};
                        \draw[->] (l1) -- (l4) node[midway,above] {$b$};
                        % \draw[->] (l4) -- (l2) node[midway,above] {$a$};
                    }    
                    \node () at (37mm,-10mm) {\( \leftarrowtail \)}; % K -> L
                    \node () at (77mm,-10mm) {\( \rightarrowtail \)}; % K -> R
                \end{tikzpicture}
                }
                \caption{}
                \label{fig:intro:edge_deletion_andffsfjsssdkdsglkadjl}
        \end{figure}

However, in many cases, an interesting property can be proved even if the whole rule set is not terminating. For example, consider the graph transformation system with two rules shown in~\autoref{fig:intro:edge_deletion_and_node_addition_ruledfakdsjflsdaj}: rule $\alpha$ deletes an arbitrary edge labeled by $A$, and rule $\beta$ introduces a fresh node.
  \begin{figure}[H]
        \centering
%   \begin{subfigure}{0.3\textwidth}
%         % \centering
        $\alpha$ = {
             \resizebox{0.7\textwidth}{!}{
             \begin{tikzpicture}[baseline=-7ex]
                    \graphbox{\( \mathcal{L} \)}{0mm}{-3mm}{34mm}{15mm}{2mm}{2mm}{
                        \coordinate (o) at (0mm,-11mm); 
                        \node[draw,circle] (l1) at ($(o)+(-10mm,0mm)$) {$1$};
                        \node[draw,circle] (l2) at ($(l1)+(2,0)$) {$2$};
                        \draw[->] (l1) -- (l2) node[midway,above] {$A$};
                    } 
            
                    \graphbox{\( \mathcal{K} \)}{40mm}{-3mm}{34mm}{15mm}{2mm}{2mm}{
                        \coordinate (o) at (0mm,-11mm); 
                        \node[draw,circle] (l1) at ($(o)+(-10mm,0mm)$) {$1$};
                        \node[draw,circle] (l2) at ($(l1)+(2,0)$) {$2$};
                    }  
            
                    \graphbox{\( \mathcal{R} \)}{80mm}{-3mm}{35mm}{15mm}{5mm}{2mm}{
                        \coordinate (o) at (-5mm,-11mm); 
                        \node[draw,circle] (l1) at ($(o)+(-10mm,0mm)$) {$1$};
                        \node[draw,circle] (l4) at ($(l1) + (2,0)$) {$2$};
                    }    
                    \node () at (37mm,-10mm) {\( \leftarrowtail \)}; % K -> L
                    \node () at (77mm,-10mm) {\( \rightarrowtail \)}; % K -> R
                \end{tikzpicture}
            }
        }
    %     \caption{A graph transformation rule for edge deletion}
    % \label{fig:intro:edge_deletion_rule}
    % \end{subfigure}
    
    % \begin{subfigure}{0.3\textwidth}
    %     % \centering

        $\beta$ ={
             \resizebox{0.7\textwidth}{!}{
             \begin{tikzpicture}[baseline=-7ex]
                    \graphbox{\( \mathcal{L} \)}{0mm}{-3mm}{34mm}{15mm}{2mm}{2mm}{
                        
                    } 
            
                    \graphbox{\( \mathcal{K} \)}{40mm}{-3mm}{34mm}{15mm}{2mm}{2mm}{
                       
                    }  
            
                    \graphbox{\( \mathcal{R} \)}{80mm}{-3mm}{35mm}{15mm}{5mm}{2mm}{
                        \coordinate (o) at (0mm,-11mm); 
                        \node[draw,circle] (l1) at ($(o)+(-10mm,0mm)$) {};
                    }    
                    \node () at (37mm,-10mm) {\( \leftarrowtail \)}; % K -> L
                    \node () at (77mm,-10mm) {\( \rightarrowtail \)}; % K -> R
                \end{tikzpicture}
            }
        }
    %     \caption{A graph transformation rule for node addition}
    %     \label{fig:intro:node_addition_rule}
    % \end{subfigure} 
    \caption{}
    \label{fig:intro:edge_deletion_and_node_addition_ruledfakdsjflsdaj}
  \end{figure}
The system does not terminate because the node-adding rule $\beta$ can be applied indefinitely. However, the edge-deleting rule $\alpha$ can be applied a finite number of times only: it deletes an edge on each application, and since no rule increases the edge count and the initial graph is finite, only finitely many deletions are possible. Therefore, termination of the full system depends solely on the node-adding rule $\beta$. This observation motivates the more general property called \emph{relative termination}, which was originally introduced by Klop in~\cite{klop1987term} for binary relations, and has been studied and employed in the context of rewriting systems~\cite{geser1990relative,kassing2024dependency,endrullis2024generalized_icgt,zantema2014termination,bruggink2014termination,bruggink2015proving}. 

\begin{definition}
Let $A$ be a collection of objects and let $R$ and $S$ be binary relations on $A$. 
We say that $R$ is \textbf{terminating relative to} $S$ (or that $R$ \textbf{terminates relative to} $S$) if 
any $(R \cup S)$-chain contains only finitely many $R$-steps.
In particular, $R$ is \textbf{terminating} (or terminates) if it is terminating relative to the empty relation $\emptyset$.
\end{definition}

For example, the relation $>$ on $\mathbb{N}$ is terminating relative to the empty relation, since any $(> \cup \emptyset)$-chain contains only finitely many $>$-steps. The relation $>$ is also terminating relative to $\geq$ on $\mathbb{N}$, since any $(> \cup \geq)$-chain can only contain finitely many $>$-steps. 

Note that termination and well-foundedness are the same property of a binary relation $\to$ described from two perspectives. Operationally, we read $x \to y$ as \enquote{$x$ can be transformed to $y$}; the absence of infinite $\rightarrow$-chains therefore means that any transformation sequence starting from an initial object $x$ is finite, i.e. every computation or rewrite sequence eventually terminates. Structurally, we read $x \to y$ as \enquote{$x$ is constructed from $y$}; the absence of infinite $\to$-chains then implies that every object $x$ can be traced back along a finite chain to an element that is not constructed from any other, so the structure is well-founded. In this thesis we adopt the operational viewpoint and use the term termination.

Relative termination 
carries over to rewriting systems in a straightforward way via their associated rewriting relations.
\begin{definition}
    \label{termination:def:relative_termination}
     Let $\mathcal{R}$ and $\mathcal{S}$ be sets of rewriting rules and let $\mathfrak{F}$ be a DPO rewriting framework. 
     We say that $\Rightarrow_{\mathcal{R},\mathfrak{F}}$ is \textbf{terminating relative to} $\Rightarrow_{\mathcal{S}, \mathfrak{F}}$ if 
     $\Rightarrow_{\mathcal{R},\mathfrak{F}}$ is \textbf{terminating relative to} $\Rightarrow_{\mathcal{S}, \mathfrak{F}}$.
\end{definition}
In practice, to prove termination of a rewriting system $\mathcal{R}$, one partitions the set of rules into two disjoint subsets \( \mathcal{B} \) and \( \mathcal{A} \) with non-empty $\mathcal{A}$ such that \( \mathcal{A} \) terminates relative to \( \mathcal{B} \), if $\mathcal{B}$ is not empty then the termination of $\mathcal{R}$ is established, otherwise, a new iteration starts with the strictly smaller rule set $\mathcal{B}$.
 