A commutative semiring is an algebraic structure (see~\cite{bruggink2015proving}\cite{endrullis2024generalized_arxiv_v2}). In this work, we adapt the concept of an ordered semiring from~\cite{endrullis2024generalized_arxiv_v2}. 
The key difference is that the semiring is required to be equipped with a homomorphism to the extended real numbers instead of being well-founded.
Throughout the remainder of this article, $<$ and $\leq$ denote the canonical irreflexive and reflexive orders on the set of extended real numbers $\overline{\mathbb{R}} = \mathbb{R} \cup \{-\infty, +\infty\}$.
\begin{definition}[Strongly monotonic measurable semiring]
    \label{def:real_strongly_monotonic_semiring}
    A \textbf{strongly monotonic measurable semiring} $(S, \oplus, \odot, 0, 1, \prec, \mu)$ consists of
    \begin{itemize} 
        \item A commutative semiring $(S, \oplus, \odot, 0, 1)$,
        \item A non-empty irreflexive order $\prec$ on $S$,
        \item A homomorphism $\mu : (S, \prec) \to ( \overline{\mathbb{R}}, < )$,
    \end{itemize}
    such that $0 \neq 1$ and for all $x,y,z,w \in S$, for all $\delta \in \mathbb{R}_{\geq 0}$, we have
        \begin{align*}
            1 \preceq x \land 1 \preceq y 
            &\Rightarrow
            1 \preceq x \oplus y
            &\tag{S0} \label{ax:s0} 
            \\ 
            x \preceq x' \land y \preceq y' 
            &\Rightarrow
            x \oplus y \preceq x' \oplus y'
            &\tag{S1} \label{ax:s1} 
            \\   
            % x < y  
            % &\Rightarrow
            % x \oplus z \leq y \oplus z 
            % \tag{S1} \label{eq:ordered_semiring_plus_monotonic} 
            % \\ w
            x \prec x' \land y \prec y'  
            &\Rightarrow
            x \oplus y \prec x' \oplus y'
            &\tag{S2} \label{ax:s2} 
            \\
            \delta + \mu(x) < \mu(y) \land \delta + \mu(z) < \mu(w)
            &\Rightarrow
            \delta + \mu(x \oplus z) < \mu(y \oplus w)
            &\tag{S3} \label{ax:s2'}
            \\
            x \preceq x'
            &\Rightarrow 
            x \odot y \preceq x' \odot y 
            &\tag{S4} \label{ax:s3} 
            \\
            x \prec x' \land y \neq 0 
            &\Rightarrow
            x \odot y \prec x' \odot y
            &\tag{S5} \label{ax:s4}
            \\ 
            \delta + \mu(x) < \mu(y) \land 1 \preceq z \neq 0
            &\Rightarrow
            \delta + \mu(x \odot z) < \mu(y \odot z)
            &\tag{S6} \label{ax:s4'}
            \\
            \delta+ \mu(x) < \mu(x') \land y \neq 0
            &\Rightarrow
            \mu(x \odot y) < \mu(x' \odot y)
            &\tag{S7} \label{ax:s4''}
        %    \\
            % \\
            % 1 \leq z \neq 0 \land X < Y  
            % &\Rightarrow
            % \exists \mu(x * z) < \mu( y * z)
            % \tag{S101} \label{eq:strongly_ordered_measurable_semiring_lt_preserved_neq0_geq1}  
        %      \\     
        %     a + X < Y \land z \neq 0 
        %    &\Rightarrow
        %    \exists b> 0. b + \mu(x* z) < \mu(y * z) 
        %    \tag{S3} \label{eq:ordered_semiring_times_stable_under_mesure} 
        \end{align*}
        where $\preceq$ denotes the reflexive closure of $\prec$. The semiring is a \textbf{strictly monotonic measurable semiring} if it additionally satisfies 
    \begin{flalign*}
        \hspace{4.5cm} x \prec x' 
        &\Rightarrow
        x \oplus y \prec x' \oplus y 
        &\tag{S8} \label{ax:s5} 
        \\
        \delta + \mu(x) < \mu(x')
        &\Rightarrow
        \delta + \mu(x \oplus y) < \mu(x' \oplus y)
        &\tag{S9} \label{ax:s5'}
    \end{flalign*}
\end{definition} 
\begin{example} 
    The real tropical semiring $\mathfrak{T}' = (\mathbb{R} \cup \{+\infty\}, \min,+,+\infty, 0_\mathbb{R},<,\operatorname{id}_{\mathbb{R} \cup \{+\infty\}})$ has domain $\mathbb{R} \cup \{+\infty\}$, the binary function symbol $\oplus$ interpreted by $\min$ and the binary function symbol $\odot$ interpreted by $+$, the constant symbols $0_s$ and $1_s$ interpreted by $+\infty$ and $0_\mathbb{R}$, respectively, the binary relation symbol $\prec$ interpreted by the canonical order $<$ on $\mathbb{R} \cup \{+\infty\}$, and the unary function symbol $\mu$ interpreted by the identity function on $\mathbb{R} \cup \{+\infty\}$. The real tropical semiring is a strongly monotonic measurable semiring. It is not strictly monotonic measurable because $2 < 3$ but $2 \oplus 2 = \min(2,2) = 2 \not < 2 = \min(3,2) = 3 \oplus 2$.
\end{example}
\begin{example}
    The real arctic semiring $\mathfrak{A}' = (\mathbb{R} \cup \{-\infty\},\max,+,-\infty, 0_\mathbb{R},<,\operatorname{id}_{\mathbb{R} \cup \{-\infty\}})$ has domain $\mathbb{R} \cup \{-\infty\}$, the binary function symbol $\oplus$ interpreted by $\max$ and the binary function symbol $\odot$ interpreted by $+$, the constant symbols $0_s$ and $1_s$ interpreted by $-\infty$ and $0_\mathbb{R}$, respectively, the binary relation symbol $\prec$ interpreted by the canonical order $<$ on $\mathbb{R} \cup \{-\infty\}$, and the unary function symbol $\mu$ interpreted by the identity function on $\mathbb{R} \cup \{-\infty\}$. The real arctic semiring is a strongly monotonic measurable semiring. It is not strictly monotonic measurable because $2 < 3$ but $2 \oplus 3 = \max(2,3) = 3 \not < 3 = \max(3,3) = 3 \oplus 3$.
%    The real arctic semiring: $\mathfrak{A}' = (\mathbb{R} \cup \{-\infty\},\max,+,-\infty, 0,<,\operatorname{id}_{\mathbb{R} \cup \{-\infty\}})$.
\end{example}
\begin{example}
    The real arithmetic semiring $\mathfrak{N}' = (\mathbb{R}^+,+,*,0_\mathbb{R},1_\mathbb{R},<,\operatorname{id}_{\mathbb{R}^+})$ has as domain the set $\mathbb{R}^+$ of positive real numbers, the binary function symbol $\oplus$ interpreted by $+$ and the binary function symbol $\odot$ interpreted by $*$, the constant symbols $0_s$ and $1_s$ interpreted by $0_\mathbb{R}$ and $1_\mathbb{R}$, respectively, the binary relation symbol $\prec$ interpreted by the canonical order $<$ on $\mathbb{R}^+$, and the unary function symbol $\mu$ interpreted by the identity function on $\mathbb{R}^+$. The real arithmetic semiring is a strictly monotonic measurable semiring. 
\end{example}
\begin{example} 
    \label{example:real_semirings}
    The natural tropical semiring: $\mathfrak{T} = (\mathbb{N} \cup \{+\infty\},\min,+,+\infty, 0, < , \operatorname{id}_{\mathbb{N} \cup \{+\infty\}})$
    and the natural arctic semiring: $\mathfrak{A} = (\mathbb{N} \cup \{-\infty\},\max,+,-\infty, 0,<, \operatorname{id}_{\mathbb{N} \cup \{-\infty\}})$ are strongly monotonic measurable semirings.
    The natural arithmetic semiring $\mathfrak{N} = (\mathbb{N},+,*,0,1,<,\operatorname{id}_\mathbb{N})$ is a strictly monotonic measurable semiring.
\end{example}

% \begin{notation} 
%     \label{def:bigodot}
% Let $(S, \oplus, \odot, 0_s, 1_s)$ be a semiring. We extend naturally the binary operations $\oplus$ and $\odot$ to finite sets $E \subseteq S$ by letting
%     \begin{itemize}
%         \item $\bigodot \emptyset \overset{\operatorname{def}}{=} 1_s$ and $\bigodot \left( E \cup \{x\} \right) \overset{\operatorname{def}}{=} \left( \bigodot E \right) \odot x$;
%         \item $\bigoplus \emptyset \overset{\operatorname{def}}{=} 0_s$ and $\bigoplus \left( E \cup \{x\} \right) \overset{\operatorname{def}}{=} \left( \bigoplus E \right) \oplus x$.
%     \end{itemize}
% %  \begin{flalign*}
% %     \bigodot \emptyset &\overset{\operatorname{def}}{=} 1_s
% % \\
% %     \bigodot \left( E \cup \{x\} \right) &\overset{\operatorname{def}}{=} \left( \bigodot E \right) \odot x
% %     \\
% %     \bigoplus \emptyset &\overset{\operatorname{def}}{=} 0_s
% %     \\
% %         \bigoplus \left( E \cup \{x\} \right) &\overset{\operatorname{def}}{=} \left( \bigoplus E \right) \oplus x
% % \end{flalign*}
% \end{notation}

% \textcolor{blue}{\begin{remark}
%     \label{remark:diff_measurable_semiring}
% A strongly monotonic measurable semiring differs from a well-founded strongly monotonic semiring in~\cite{endrullis2024generalized} in four ways:
% \begin{enumerate}[label=(\arabic*),noitemsep]
%     \item Replacing well-foundedness with the homomorphism~$\mu$ and introducing Axioms \eqref{ax:s2'}, \eqref{ax:s4'}, and \eqref{ax:s4''}. This modification enables the use of non-well-founded semirings (e.g., as in~\autoref{example:real_semirings}).
%     % \item Removing the condition $1_S \preceq y$ from the original Axioms~(S3) and~(S4)\todo{This is not helpful}, resulting in our Axioms~(S4) and~(S5). This relaxation allows inclusion of elements smaller than $1_S$, as motivated in~\autoref{remark:greater_than_1}.
%     \item Adding Axiom~\eqref{ax:s0}. This technical adjustment ensures that if every $\mathcal{T}$-valued element of a type graph (see~\autoref{def:weighted_type_graph}) has a weight greater than $1_S$, then all objects subject to rewriting inherit this property.
%     \item Defining $\preceq$ as the reflexive closure of $\prec$ to simplify the theory. This is motivated by the fact that concrete semirings proposed in prior work and our paper satisfy this property.
% \end{enumerate} 
% \end{remark}}