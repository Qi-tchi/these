It is crucial to ensure that the weight of any object subject to rewriting is not $0_S$, because \(0_S\) behaves unpredictably in strongly monotonic measurable semirings. For instance, in the natural and real tropical semirings \((\mathbb{N} \cup \{+\infty\}, \min, +)\), \(0_S\) is the greatest element \(+\infty\), while in the natural and real arctic semirings \((\mathbb{N} \cup \{-\infty\}, \max, +)\), \(0_S\) is the smallest element \(-\infty\).
The existence of a context closure does the job. Further details can be found in~\cite[\textsection 5.3]{endrullis2024generalized}.

% \begin{remark} 
%   \label{remark:semiring_0_unpredictable}
%   The requirement \textquote{for all \(e \in \mathbb{E}, w(e) \neq 0_S\)} is necessary because \(0_S\) behaves unpredictably in strongly monotonic measurable semirings. For instance, in the natural and real tropical semirings \((\mathbb{N} \cup \{+\infty\}, \min, +)\), \(0_S\) is the greatest element \(+\infty\), while in the natural and real arctic semirings \((\mathbb{N} \cup \{-\infty\}, \max, +)\), \(0_S\) is the smallest element \(-\infty\).
% \end{remark} 

\begin{definition}[Context Closure~\text{\cite[\textdef~5.3]{endrullis2024generalized}}]
    \label{def:context_closure}  
    \ \newline 
\begin{minipage}{0.65\textwidth}
    Let $\mathcal{T} = (T,\mathbb{E}, S, w)$ be a type graph, \(\rho = (L \overset{l}{\leftarrow} K \overset{r}{\rightarrow} R ) \) a DPO rewriting rule and $\mathfrak{F}$ a rewriting framework. 
    A \textbf{context closure} for $\rho$ and $\mathcal{T}$ in $\mathfrak{F}$ is a morphism $c:L \rightarrow T$ such that for every DPO diagram in $\mathfrak{F}(\rho)$ (shown on the right) 
    there exists $\alpha : G \rightarrow T$ such that $m \star \alpha = c$.
\end{minipage}
\begin{minipage}{0.35\textwidth}
    \begin{center}
        \begin{tikzpicture}[rotate=90]
          \node (I) {$K$}; 
          \node (L) [left of=I] {$L$};
          \node (R) [right of=I] {$R$};
          \node (G) [below of=L] {$G$};
          \node (C) [below of=I] {$C$};
          \node (H) [below of=R] {$H$};
          \node (T) [left=of $(L)!0.5!(G)$] {$T$};
          \draw [->] (L) to  node [label, above] {$c$}  (T);
          \draw [->] (G) to  node [label, below] {$\alpha$} (T);
          \draw [->] (I) to node [label, above] {$l$} (L);
          \draw [->] (I) to node [label,above] {$r$} (R);
          \draw [->] (L) to node [label, right] {$m$} (G);
          \draw [->] (I) to (C);
          \draw [->] (R) to (H);
          \draw [->] (C) to (G);
          \draw [->] (C) to (H);
        \end{tikzpicture}
      \end{center}
\end{minipage}
\end{definition}
\begin{example}
    \label{example:context_closure}
   Consider the DPO rule in \autoref{ex:grsaa}, the weighted type graph in \autoref{example:weighted_type_graph}.
   and the morphism $c$ illustrated below. Since for every match $m : L \to G$, there exists a morphism $\alpha : G \to T$ such that $m \star \alpha = c$, the morphism $c$ is a context closure for the DPO rule in the type graph.
  \begin{center}
    \begin{tikzpicture}
      \graphbox{\( L \)}{-50mm}{0mm}{40mm}{30mm}{2mm}{-6mm}{
        \coordinate (o) at (0mm,-10mm); 
        \node[draw,circle] (l1) at ($(o)+(-10mm,0mm)$) {1};
        \node[draw,circle] (l2) at ($(l1)+(2,0)$) {2};
        \node[draw,circle] (l3) at ($(l1) + (1,0)$) {3};
        \draw[] (l1) -- (l3) node[midway,above] {a};
        \draw[] (l3) -- (l2) node[midway,above] {a};
    } 
        \graphbox{$T$}{0mm}{0mm}{40mm}{30mm}{-10mm}{-15mm}{
            \node[draw,circle] (1) at (0,0) {$1\ 2$};
            \node[draw,circle] (2) at (2,0) {3};
            \draw[->] (1) edge[loop above] node[midway, above] {$a^{1.0}$} (1) ;
            \draw[->] (1) edge[loop below] node[midway, below] {$b^{1.0}$} (1) ;
            \draw[->] (1) edge[bend left] node[midway, above] {$a^{1.0}$}  (2)  ;
            \draw[->] (2) edge[bend left] node[midway, below] {$a^{1.0}$} (1)   ;
        }
        \node () at (-5mm,-15mm) {$\overset{c}{\to}$};
    \end{tikzpicture}
  \end{center}
  
\end{example}
