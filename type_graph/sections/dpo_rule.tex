\begin{definition}[Rewriting rule and match~\cite{corradini1997algebraic}]
  \label{def:grs:dpo_rule}
A \textbf{DPO rewriting rule} $\rho$ is a span \( L \overset{l}{\leftarrow} K \overset{r}{\rightarrow} R \), where \( K \) is the \textbf{interface}, \( L \) is the \textbf{left-hand-side graph}, denoted \( \operatorname{lhs}(\rho) \), and \( R \) is the \textbf{right-hand-side graph}, denoted \( \operatorname{rhs}(\rho) \). The rule is \textbf{monic} if $l$ and $r$ are both monic.
A match of the rule in an graph \( G \) is a morphism \( m: L \rightarrow G \).   
\end{definition}
   In this paper, we use examples from the category \textbf{Graph} of edge-labeled directed multigraphs (see~\cite{konig2018atutorial}) to illustrate the discussed concepts. 
\begin{example}
  \label{ex:grsaa}
  The injective DPO rule from \cite[Example 6]{bruggink2014termination} will be used to illustrate the concepts discussed throughout this paper.
  The rule can be visualized as follows:
  \begin{center} 
      \resizebox{0.7\textwidth}{!}{
      \begin{tikzpicture}
          \graphbox{$L$}{0mm}{0mm}{34mm}{15mm}{2mm}{-5mm}{
              \coordinate (o) at (0mm,-3mm); 
              \node[draw,circle] (l1) at ($(o)+(-10mm,0mm)$) {1};
              \node[draw,circle] (l2) at ($(l1)+(2,0)$) {2};
              \node[draw,circle] (l3) at ($(l1) + (1,0)$) {3};
              \draw[->] (l1) -- (l3) node[midway,above] {a};
              \draw[->] (l3) -- (l2) node[midway,above] {a};
          }     
          \graphbox{$K$}{40mm}{0mm}{24mm}{15mm}{2mm}{-5mm}{
              \coordinate (o) at (5mm,-3mm); 
              \node[draw,circle] (l1) at ($(o)+(-10mm,0mm)$) {1};
              \node[draw,circle] (l2) at ($(l1)+(1,0)$) {2};
              % \node[draw,circle] (l3) at ($(l1) + (1,0)$) {$\ $};
              % \draw[->] (l1) -- (l3) node[midway,above] {a};
              % \draw[->] (l3) -- (l2) node[midway,above] {a};
          }    
          \graphbox{$R$}{70mm}{0mm}{45mm}{15mm}{2mm}{-5mm}{
              \coordinate (o) at (-5mm,-3mm); 
              \node[draw,circle] (l1) at ($(o)+(-10mm,0mm)$) {1};
              \node[draw,circle] (l2) at ($(l1)+(3,0)$) {2};
              \node[draw,circle] (l3) at ($(l1) + (1,0)$) {4};
              \node[draw,circle] (l4) at ($(l1) + (2,0)$) {5};
              \draw[->] (l1) -- (l3) node[midway,above] {a};
              \draw[->] (l3) -- (l4) node[midway,above] {b};
              \draw[->] (l4) -- (l2) node[midway,above] {a};
          }    
          \node () at (37mm,-8mm) {$\overset{l}{\leftarrowtail}$};
          \node () at (67mm,-8mm) {$\overset{r}{\rightarrowtail}$};
          % \draw[>->] (51mm,2mm) -- (52mm,3mm);
      \end{tikzpicture}
      }
  \end{center}
\end{example}