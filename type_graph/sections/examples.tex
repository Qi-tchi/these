\begin{notation}
    We use the notation from~\cite[Notation 1]{overbeek2023apbpo+} to visualize edge-labeled graph homomorphisms. Labeled graphs are enclosed in boxes with their names displayed in the top-left corner. Nodes and edges are assigned subsets of \(\mathbb{N}\) as identifiers, and these identifiers are chosen such that: (i) Each node or edge \( y \) in the codomain graph is assigned the union of the identifiers of all nodes or edges in the domain graph that are mapped to \( y \); (ii) The graph homomorphism is uniquely determined by this assignment.
     
    \noindent To further improve readability, we represent sets by listing their elements. Additionally, we omit identifiers when doing so does not cause confusion. This is illustrated in the following representation of a homomorphism \( h: G \mathop{\to} H \).
    
    \begin{center}
        \resizebox{0.45\textwidth}{!}{
        \begin{tikzpicture}
            \graphbox{\( G \)}{00mm}{-20mm}{45mm}{20mm}{2mm}{-5mm}{
                \coordinate (o) at (-5mm,-8mm); 
                \node[draw,circle] (l1) at ($(o)+(-10mm,0mm)$) {1};
                \node[draw,circle] (l2) at ($(l1)+(3,0)$) {2};
                \node[draw,circle] (l3) at ($(l1)+(1,0)$) {3};
                \node[draw,circle] (l4) at ($(l1)+(2,0)$) {4};
                \draw[->] (l1) -- (l3) node[midway,above] {$a$};
                \draw[->] (l3) -- (l4) node[midway,above] {$b$};
                \draw[->] (l4) -- (l2) node[midway,above] {$a$};
            }  
            \graphbox{\( H \)}{50mm}{-20mm}{34mm}{20mm}{2mm}{-5mm}{
                \coordinate (o) at (0mm,-8mm); 
                \node[draw,circle] (l1) at ($(o)+(-10mm,0mm)$) {1};
                \node[draw,circle] (l2) at ($(l1)+(2,0)$) {2};
                \node[draw,circle] (l3) at ($(l1)+(1,0)$) {3\ 4};
                \draw[->] (l1) -- (l3) node[midway,above] {$a$};
                \draw[->] (l3) edge[loop above] (l3) node[midway,above] {$b$};
                \draw[->] (l3) -- (l2) node[midway,above] {$a$};
            }      
            % \node () at (53mm,-30mm) {$\rightarrow$};
        \end{tikzpicture}
    }
    \end{center} 
    In this example, the sets \(\{1\}\), \(\{2\}\), \(\{3\}\), \(\{4\}\), and \(\{3,4\}\) are represented as \(1\), \(2\), \(3\), \(4\), and \(3\ 4\), respectively. Edge identifiers are omitted.
\end{notation} 
\begin{example}
    \label{nonwf}
    Let $n \in \mathbb{N}_{\geq 2}$. Consider the graph rewriting system with three rules shown below. The right-hand-side graph of the first rule contains \( n \) arrows labeled \( b \) from node $1$ to node $2$, and the right-hand-side graph of the second rule contains \( n \) arrows labeled \( c \) from node $1$ to node $2$. The left-hand-side graph for the third rule contains \( n^2+1\) arrows labeled \( c \) from node $1$ to node $2$.

    \begin{center}
        % \resizebox{0.7\textwidth}{!}{
            \begin{tikzpicture}
                    \graphbox{\( L  \)}{0mm}{5mm}{34mm}{19mm}{2mm}{-5mm}{
                        \coordinate (o) at (0mm,-7mm); 
                        \node[draw,circle] (l1) at ($(o)+(-10mm,2mm)$) {1};
                        \node[draw,circle] (l2) at ($(l1)+(2,0)$) {2};
                        \draw[->] (l1) -- (l2) node[midway,above] {a};
                    } 
      
                    \graphbox{\( K \)}{40mm}{5mm}{34mm}{19mm}{2mm}{-5mm}{
                        \coordinate (o) at (0mm,-7mm); 
                        \node[draw,circle] (l1) at ($(o)+(-10mm,2mm)$) {1};
                        \node[draw,circle] (l2) at ($(l1)+(2,0)$) {2};
                    }  
      
                    \graphbox{\( R  \)}{80mm}{5mm}{45mm}{19mm}{2mm}{-5mm}{
                        \coordinate (o) at (-5mm,-7mm); 
                        \node[draw,circle] (l1) at ($(o)+(-10mm,2mm)$) {1};
                        \node[draw,circle] (l2) at ($(l1)+(2,0)$) {2};
                        \node () at ($(l1)+(1,0)$) {$\vdots$};
                        \draw[->] (l1) to[bend left] node[midway,above] {b} (l2) ;
                        \draw[->] (l1) to[bend right]  node[midway,below] {b}(l2) ;
                    }    
                    \node () at (37mm,-8mm) {\( \leftarrowtail \)}; % K -> L
                    \node () at (77mm,-8mm) {\( \rightarrowtail \)}; % K -> R
            \end{tikzpicture}
    
            \begin{tikzpicture}
                \graphbox{\( L  \)}{0mm}{5mm}{34mm}{19mm}{2mm}{-5mm}{
                    \coordinate (o) at (0mm,-7mm); 
                    \node[draw,circle] (l1) at ($(o)+(-10mm,2mm)$) {1};
                    \node[draw,circle] (l2) at ($(l1)+(2,0)$) {2};
                    \draw[->] (l1) -- (l2) node[midway,above] {b};
                } 
  
                \graphbox{\( K \)}{40mm}{5mm}{34mm}{19mm}{2mm}{-5mm}{
                    \coordinate (o) at (0mm,-7mm); 
                    \node[draw,circle] (l1) at ($(o)+(-10mm,2mm)$) {1};
                    \node[draw,circle] (l2) at ($(l1)+(2,0)$) {2};
                }  
  
                \graphbox{\( R  \)}{80mm}{5mm}{45mm}{19mm}{2mm}{-5mm}{
                    \coordinate (o) at (-5mm,-7mm); 
                    \node[draw,circle] (l1) at ($(o)+(-10mm,2mm)$) {1};
                    \node[draw,circle] (l2) at ($(l1)+(2,0)$) {2};
                    \node () at ($(l1)+(1,0)$) {$\vdots$};
                    \draw[->] (l1) to[bend left] node[midway,above] {c} (l2) ;
                    \draw[->] (l1) to[bend right]  node[midway,below] {c}(l2) ;
                }    
                \node () at (37mm,-8mm) {\( \leftarrowtail \)}; % K -> L
                \node () at (77mm,-8mm) {\( \rightarrowtail \)}; % K -> R
        \end{tikzpicture}


        \begin{tikzpicture}
            \graphbox{\( L  \)}{0mm}{5mm}{34mm}{19mm}{2mm}{-5mm}{
                \coordinate (o) at (0mm,-7mm); 
                \node[draw,circle] (l1) at ($(o)+(-10mm,2mm)$) {1};
                \node[draw,circle] (l2) at ($(l1)+(2,0)$) {2};
                \draw[->] (l1) to[bend left] node[midway,above] {c} (l2);
                % \draw[->] (l1) to[out = 60, in=120] node[midway,above] {c} (l2);
                % \draw[->] (l1) to[out = 45, in=135] node[midway,above] {c} (l2);
                % \draw[->] (l1) to[out = 30, in=150] node[midway,above] {c} (l2);
                \draw[->] (l1) to[bend right] node[midway,below] {c} (l2);
                \node () at ($(l1)+(1,0)$) {$\vdots$};
            } 

            \graphbox{\( K \)}{40mm}{5mm}{34mm}{19mm}{2mm}{-5mm}{
                \coordinate (o) at (0mm,-7mm); 
                \node[draw,circle] (l1) at ($(o)+(-10mm,2mm)$) {1};
                \node[draw,circle] (l2) at ($(l1)+(2,0)$) {2};
            }  

            \graphbox{\( R  \)}{80mm}{5mm}{45mm}{19mm}{2mm}{-5mm}{
                \coordinate (o) at (-5mm,-7mm); 
                \node[draw,circle] (l1) at ($(o)+(-10mm,2mm)$) {1};
                \node[draw,circle] (l2) at ($(l1)+(2,0)$) {2};
                \draw[->] (l1) -- (l2) node[midway,above] {a};
            }    
            \node () at (37mm,-8mm) {\( \leftarrowtail \)}; % K -> L
            \node () at (77mm,-8mm) {\( \rightarrowtail \)}; % K -> R
    \end{tikzpicture}
        \end{center}

    The termination of this rewriting system can be proved using weighted type graphs of size $1$ over the natural tropical semiring, the natural arctic semiring or the arithmetic semiring. Let $w_a$,$w_c$ and $w_c$ be the weights of edges labeled by $a$, $b$ and $c$ respectively.
    For any feasible weighted type graph over the natural tropical semiring or the natural arctic semiring, we have $w_a \geq n*2$ the following conditions must be satisfied
    \begin{flalign}
        w_a \geq n*w_b\\
        w_b \geq n*w_c\\
        (n^2+1) * w_c \geq w_a \\
        w_a > n*w_b \lor 
        w_b > n*w_c \lor 
        (n^2+1) * w_c > w_a
    \end{flalign}
    For any feasible weighted type graph over the natural arithmetic semiring, we have $w_a \geq n*2$ as the following conditions must be satisfied
    \begin{flalign}
        w_a \geq w_b^n\\
        w_b \geq w_c^n\\
        w_c^{n^2+1} \geq w_a \\
        w_a \geq w_b^n \lor 
        w_b \geq w_c^n \lor
        w_c^{n^2+1} \geq w_a
    \end{flalign}

    Let $\mathcal{R}$ be a systems including this system. The constraints on weights will included too. If proving termination of $R$ requires a type graph with dimension greater than $1$, then constructing a suitable weighted type graph over the well-founded semiring on extended natural numbers proposed in the prior work becomes extremely difficult in practice, due the high computational complexity.
\end{example}