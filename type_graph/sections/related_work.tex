\paragraph{Acceleration with the real tropical~/~arctic semiring.} 
Consider the examples in Table~\ref{tabular:benchmarks} whose termination is provable using the tropical or arctic semirings. For all such examples except \cite[Example 5]{plump2018modular} and \cite[Example 4]{bruggink2015proving}, the real-number semiring approach achieves better runtime performance than the natural-number semiring approach. The acceleration rate varies across examples due to Z3's internal heuristics. This result is notable because termination proofs for these systems only require weighted type graphs with small edge weights (e.g., values less than 2 for most examples and at most 3 for all), which theoretically favors the implementation of the natural-number semiring approach.

\paragraph{Timeout with the real arithmetic semiring.}
The systems whose termination is not proven by our tool are \cite[Example 6]{plump2018modular}, \cite[Examples 5 and 6]{bruggink2015proving}, \cite[Example 4]{plump2018modular}, and \cite[Example 5]{bruggink2014termination}.
The first case (\cite[Example 6]{plump2018modular}) is due to a limitation of the type graph method in general, as the system has a rule whose right-hand side graph can be embedded into its left-hand side graph, which is problematic for the type graph method as pointed out in \cite[Example D.4]{endrullis2024generalized_arxiv_v2}. For the remaining cases,
 the failure stems from the double-exponential time complexity~\cite{collins1974quantifier,z3realarithmetic} in the number of variables \( (x_{u,v,l})_{u,v \mathop{\in} \{1,...,k\}, l \mathop{\in} \Sigma} \) and \( (y_{u,v,l})_{u,v \mathop{\in} \{1,...,k\}, l \mathop{\in} \Sigma} \), which is also influenced by the length of the conditions of Theorem~\ref{nwf:thm:termination_grs} encoded in Z3.
\paragraph{Acceleration with the real arithmetic semiring.} For some examples, the approach with semirings on real numbers can achieve a better runtime performance than the approach with semirings on natural numbers. These are (1) examples which need weighted type graphs with 1 or 2 nodes, and the constraints can be expressed in Z3 with a short formula, or (2) examples which can first be simplified by eliminating some rules using weighted type graphs with 1 node, and then the remaining rules can be 
easily shown to terminate
as the first case.
 \paragraph{Interest of the approach with semiring on natural numbers.}
For~\cite[Example 4]{plump2018modular},~\cite[Example 5 and 6]{bruggink2015proving} and~\cite[Example 5]{bruggink2014termination}, the semiring on natural numbers is more efficient, as it allows one to restrict the search space to an extremely small size by bounding the maximum weight of edges, which is not possible for the semiring on real numbers. However, this advantage can be achieved only if the termination can be proven with a weighted type graph with whose weights are in a extremely small set.
\paragraph{Limitations of our experiments.} 
While our search strategy is very close to the real-world scenario that a non-expert user would use to prove termination of DPO rewriting systems, 
it makes it difficult to estimate the runtime performance gain of our approach over the one proposed by Endrullis and Overbeek~\cite{endrullis2024generalized_arxiv_v2}.  
This is because the reported runtimes include (1) translation of constraints into Z3's input format, (2) the time spent in Z3 to solve the constraint system, and (3) the parallelism overhead. 
Another problem is that the examples in our experiments are limited to those whose termination can be proven with a weighted type graph whose edge weights are small (at most 3). These examples favor the approach with semirings on natural numbers, as the search space is extremely small when the maximum edge weight is bounded by 3.
\paragraph{Comparison with \texttt{GraphTT-wtg}.}
A comparison of our implementation of the approach with semirings on real numbers with the one proposed in \texttt{GraphTT-wtg} by Endrullis and Overbeek~\cite{endrullis2024generalized_arxiv_v3} is desired, but this tool is not publicly available and we do not think their article provides enough information of the implementation to allow us to do so.