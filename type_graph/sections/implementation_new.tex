% We have implemented both our approach and the one proposed by Endrullis and Overbeek~\cite{endrullis2024generalized_arxiv_v2} for edge-labeled DPO rewriting systems into a unified tool. These two approaches can be launched in parallel and collaborate to establish termination of a DPO rewriting system.
Chapter~\ref{chap:lyonparallel} provides a description of the implementation of our approach and the one proposed by Endrullis and Overbeek~\cite{endrullis2024generalized_arxiv_v2} in our tool, called LyonParallel.

\paragraph{Searching Strategy.}
Consider a DPO rewriting system $\mathcal{R}$. Given a fixed semiring $S$, a processor is spawned to iteratively search for a suitable weighted type graph over $S$.
The search starts with a weighted type graph containing 1 node, incrementally increasing the node count by 1 until a maximum of 4 nodes is reached.
For a fixed number of nodes,
 if $S$ is a semiring over the natural numbers, the maximum edge weight is initialized to 1 and incremented by 1 (up to a limit of 3) if no suitable weighted type graph is found;
if $S$ is a semiring over the real numbers, weights are constrained to be positive real numbers, and no upper bound is imposed.
Processors targeting different semirings can run in parallel to analyze the same DPO rewriting system.
When a processor discovers a weighted type graph that witnesses relative termination of a subset of rules $\mathcal{A}$ with respect to another subset of rules $\mathcal{B}$ such that $\mathcal{R} \mathop{=} \mathcal{A} \mathop{\cup} \mathcal{B}$, it broadcasts this result to all processors and waits for them to terminate. If $\mathcal{B} \mathop{=} \emptyset$, the system's termination is proven. Otherwise, we initiate a new search for the rules in $\mathcal{B}$ unless a timeout is reached.
 
For a fixed number of nodes $k$, we adopt the approach proposed in \cite{bruggink2015proving, bruggink2014termination,zantema2014termination} to reduce the search space. Specifically, we construct a weighted type graph with
% \( \mathcal{T}=(T, \mathbb{E}, S, w) \) where: (1) the type graph $T$ has 
$k$ nodes and no parallel edges of the same label.
%  and (2) the set $\mathbb{E}$ of morphism-rulers consists of identity morphisms from subgraphs of the type graph $T$ (each is a labeled edge with its incident nodes) to the type graph $T$.
The search proceeds as follows: 
\begin{enumerate}
    \item decide if $s \overset{l}{\to} t$ exists for every pair of nodes \( s, t\) and label \( l\);
    \item assign a weight to every existing edge;
    \item verify the existence of a constant $\delta >0$ and a partition of the rule set $\mathcal{R}$ into a non-empty subset $\mathcal{A}$ and a subset $\mathcal{B}$
 such that:
    \begin{itemize}
        \item either all rules in $\mathcal{A}$ are $\delta$-uniformly decreasing or all rules in $\mathcal{A}$ are $\delta$-closure decreasing, and
        \item all rules in $\mathcal{B}$ are weakly decreasing.
    \end{itemize}
\end{enumerate}

%  Check if the weighted type graph satisfy requirements
% (1) construct a graph with \( k \) nodes and no parallel edges of the same label, which has a directed edge from node $s$ to node $t$ labeled by $l$ for each ordered pair $(s,t)$ of nodes and label $l$ from the finite set of edge labels $\Sigma$ of the rewriting system; (2) decide whether each edge exists in the weighted type graph; (3) assign a weight (a natural number or a real number according to $S$) to every existing edge; (4) 

This procedure amounts to checking the satisfiability of an existential Presburger arithmetic, Peano arithmetic, linear real arithmetic, or non-linear real arithmetic formula depending on the semiring $S$ considered.

\paragraph{Z3 Modeling.}
The type graph $T$ is modeled in Z3 by defining 
a boolean variable $x_{u,v,l} \mathop{\in} \mathbb{B}$ for every directed labeled edge $u\overset{l}{\to} v$ in the type graph, where $u,v\in\{1,...,k\}$ are nodes and $l \mathop{\in} \Sigma$ is an edge label. 
The variable $x_{u,v,l}$ has the value \textit{true} if the directed edge $u\overset{l}{\to} v$ exists in the result type graph, \textit{false} otherwise. 

The weight function $w$ and the set $\mathbb{E}$ of morphism-rulers are modeled by defining 
a variable $y_{u,v,l}$ of Z3's \textit{Real} sort or \textit{Int} sort (the solver's default theory for real numbers and integers), depending on the semiring $\mathcal{S}$ considered,
 for every pair $u,v\in\{1,...,k\}$ of nodes and edge label $l \mathop{\in} \Sigma$. 
The variable $y_{u,v,l}$ represents the weight of the directed labeled edge $u\overset{l}{\to} v$ in the resulting weighted type graph, but only if edge $u\overset{l}{\to} v$ exists (i.e. $x_{u,v,l}$ has the value \textit{true}). 
They are constrained to be non-negative.

A variable $\delta \mathop{\in} \mathbb{R}_{>0}$ is defined to ensure that there is
a partition of the rule set $\mathcal{R}$ into a non-empty subset $\mathcal{A}$ and a subset $\mathcal{B}$ such that all rules in $\mathcal{A}$ are either $\delta$-uniformly decreasing or $\delta$-closure decreasing, and all rules in $\mathcal{B}$ are weakly decreasing.

For convenience, the following auxiliary variables are also defined:
\begin{itemize}
    \item a boolean variable $v_h \mathop{\in} \mathbb{B}$ for every morphism $h$ from $L$ or $R$ to $T$;
    \item a real-valued variable $v_{h'} \mathop{\in} \mathbb{R}_{\geq 0}$ for every morphism $h$ from $L$ or $R$ to $T$;
    \item a boolean variable $v_c \mathop{\in} \mathbb{B}$ for every morphism $c$ from $L$ to $T$.
\end{itemize}  
The value of $v_h$ has the value \textit{true} if the morphism $h$ exists in the result type graph (i.e. all edges in its image exist), and \textit{false} otherwise.
The variable $v_{h'}$ holds the weight of the morphism $h$, provided the morphism $h$ exists (i.e., $v_h$ has the value \textit{true}).
The variable $v_c$ has the value \textit{true} if the morphism $c$ exists in the result type graph and can serve as a context closure for the rule, and \textit{false} otherwise.

If there is an assignment of values to 
    \begin{itemize}
        \item $(x_{u,v,l})$, for all $u,v \mathop{\in} \{1,...,k\}$ and for all $l \mathop{\in} \Sigma$, and
        \item $(y_{u,v,l})$, for all $u,v \mathop{\in} \{1,...,k\}$ and for all $l \mathop{\in} \Sigma$, and
        \item $\delta$,
    \end{itemize} 
    such that all conditions of Theorem~\ref{nwf:thm:termination_grs} are satisfied, then a suitable weighted type graph that witnesses termination of the DPO rewriting system can be constructed, because
    \begin{itemize}
        \item the values of \( (x_{u,v,l})_{u,v \mathop{\in} \{1,...,k\}, l \mathop{\in} \Sigma} \) define the type graph $T$, and
        \item the values of \( (y_{u,v,l})_{u,v \mathop{\in} \{1,...,k\}, l \mathop{\in} \Sigma} \) define the weight function $w$ and the set $\mathbb{E}$ of morphism-rulers.
    \end{itemize} 