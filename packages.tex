
\usepackage[colorinlistoftodos,draft]{todonotes}

\usepackage{marginnote}
\setlength{\marginparwidth}{2cm}
\usepackage{pgfplots, xcolor,diagbox,colortbl,tikz,graphicx,algorithm2e,cancel,verbatim,
listings,float,amsmath,amssymb,enumitem,array,subfiles,bussproofs,
rotating,MnSymbol,mathtools,subcaption,caption}
\pgfplotsset{compat=1.18}
\usetikzlibrary{automata, positioning,graphs,shapes, arrows, calc}
\usepackage{csquotes}

\hyphenation{ma-chine-check-a-ble}
\hyphenation{dou-ble-push-out}
%%%%%%% careful : order is important begin
\usepackage{amsthm}
\usepackage{hyperref}
\renewcommand{\sectionautorefname}{\textsection}
\newtheorem{theorem}{Theorem}[section]
\newtheorem{corollary}{Corollary}[theorem]
\newtheorem{lemma}{Lemma}[theorem]
\newtheorem{idea}{Idea}[theorem]
\newtheorem{definition}{Definition}[section]
\newtheorem{proposition}{Proposition}[section]
\newtheorem{remark}{Remark}[section]
\newtheorem{example}{Example}[section]
\newtheorem{Plan}{Plan}[section]
\newtheorem{claim}{Claim}[theorem]
\newtheorem{notation}[theorem]{Notation} 
\def\theoremautorefname{Theorem}                % Define autoref name for 'theorem'
\def\corollaryautorefname{Corollary}            % Define autoref name for 'corollary'
\def\lemmaautorefname{Lemma}                    % Define autoref name for 'lemma'
\def\ideaautorefname{Idea}                      % Define autoref name for 'idea'
\def\definitionautorefname{Definition}          % Define autoref name for 'definition'
\def\propositionautorefname{Proposition}        % Define autoref name for 'proposition'
\def\remarkautorefname{Remark}                  % Define autoref name for 'remark'
\def\exampleautorefname{Example}                % Define autoref name for 'example'
\def\planautorefname{Plan}                      % Define autoref name for 'plan'
\def\claimautorefname{Claim}                    % Define autoref name for 'claim'
\def\notationautorefname{Notation}              % Define autoref name for 'notation'
%%%%%%% careful : order is important end


\newcommand{\set}[1]{\{#1\}}
\newcommand{\graphbox}[8]{
  \begin{scope}[xshift=#2,yshift=#3]
    \draw [rounded corners=2mm] (0,0) rectangle (#4,-#5);
    \node at (0,0mm) [anchor=north west,inner sep=1mm] {#1};
    \begin{scope}[xshift=#4/2+#6,yshift=#7] 
    #8
    \end{scope}
  \end{scope}
}
\newcommand{\card}[1]{|#1|}
\newcommand{\homset}[2]{\operatorname{Hom}(#1,#2)} 

\newcommand{\isdef}{\overset{\operatorname{def}}{=}}
\newcommand{\cat}[1]{\mathcal{#1}}
\newcommand{\f}[1]{\mathfrak{#1}}
\newcommand{\rs}[1]{\mathcal{#1}}
\newcommand{\opn}[1]{\operatorname{#1}}
%arrow
\newcommand{\itrs}{\overset{i}{\to}} 

\usepackage{tocloft}

% % Customizing the table of contents
% % \setlength{\cftpartindent}{0pt} %  space before part titles
\setlength{\cftchapindent}{15pt} %  space before chapter titles
\setlength{\cftsecindent}{30pt} %  space before sec titles
\setlength{\cftsubsecindent}{40pt} %  space before subsec titles
\setlength{\cftsubsubsecindent}{50pt} %  space before subsubsec titles


\newcommand{\textdef}{\text{$\operatorname{Def.}$}}
\setcounter{secnumdepth}{3}
\newcommand{\pbpop}{$\operatorname{PBPO}^+$}
\addtolength{\cftbeforechapskip}{-10pt} % Reduce spacing by 10pt
\addtolength{\cftbeforepartskip}{-20pt} % Reduce spacing by 10pt
\newif\iflongversion
\longversiontrue
\newif\iftrackChange
\trackChangetrue
\newcommand{\trackedtext}[1]{%
    \iftrackChange
        \textcolor{red}{#1}% Apply red color if trackChange is true
    \else
        #1% No color change if trackChange is false
    \fi
} 


\usepackage[style=alphabetic]{biblatex}
\addbibresource{bib/these.bib} % Add your bibliography file here

\usepackage{titlesec} 
 
\usepackage{tikz-cd}

\usepackage{pifont}      
\usepackage{makecell}    
\usepackage{nicematrix}  
\usepackage{wrapfig}     
\usepackage{lipsum}      